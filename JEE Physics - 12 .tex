\documentclass{article}
                    \usepackage{amsmath}
                    \usepackage{amssymb}
                    \usepackage{graphicx}
                    \usepackage{enumitem}
                    \usepackage{longtable}
                    \title{JEE Physics - 12 }
                    \begin{document}
                    \maketitle
                    \section*{Question 1}
A current of \(2~ A\) flows in conductors as shown. The potential difference \(V_A-V_B\) will be:\newline\includegraphics[width=\textwidth]{https://testseries.edugorilla.com/static/media/wl_client/1/qdump/e15a7351bd3aa55f4a06d4d7dd6f96fc/af8244670ba631de9614f9ff5c8f691c.jpeg}
\begin{enumerate}[label=(\alph*)]
\item \(+4 V\)
\item \(-1 {~V}\)
\item \(+1 {~V}\)
\item \(+2 {~V}\)
\end{enumerate}
\newpage
\section*{Question 2}
In a practical Wheatstone bridge circuit (Figure), when one more resistance of \(100 \Omega\) is connected in parallel with unknown resistance \(x\), then the ratio \(\frac{l_1}{l_2}\) becomes \(2 . l\), is the balance length. AB is a uniform wire. Then the value of \(x\) must be:\includegraphics[width=\textwidth]{static/media/wl_client/1/qdump/1009b535936fbcc80544f09c19b1b72b/01765626dc484041e0ff492ce8dd1045.png}
\begin{enumerate}[label=(\alph*)]
\item \(50 \Omega\)
\item \(100 \Omega\)
\item \(200 \Omega\)
\item \(400 \Omega\)
\end{enumerate}
\newpage
\section*{Question 3}
A parallel combination of two resistors, of \(1 \Omega\) each, is connected in series with a \(1.5 \Omega\) resistor. The total combination is connected across a \(10 {~V}\) battery. The current flowing in the circuit is:
\begin{enumerate}[label=(\alph*)]
\item \(5 {~A}\)
\item \(20 {~A}\)
\item \(0.2 {~A}\)
\item \(0.4 {~A}\)
\end{enumerate}
\newpage
\section*{Question 4}
What is the equivalent resistance across A and B?<font face="Arial" size="2">\newline\newline\includegraphics[width=\textwidth]{https://testseries.edugorilla.com/static/media/qdump/ef1dffb30ac783918f7220f2a7e9984c/e9a967fa9b6b508a08f80b286830adcf.png}</font>
\begin{enumerate}[label=(\alph*)]
\item \(2 \Omega\)
\item \(3 \Omega\)
\item \(4 \Omega\)
\item \(5 \Omega\)
\end{enumerate}
\newpage
\section*{Question 5}
In the electric circuit shown, each cell has an e.m.f of \(2 {~V}\) and internal resistance of \(1 \Omega\). The external resistance is \(2 \Omega\). The value of the current is I is (in A):\includegraphics[width=\textwidth]{https://testseries.edugorilla.com/static/media/wl_client/1/qdump/50272f977f0f492a38d073f6f7507847/d1477266d9535091026aa3766ce241d9.png}\newline
\begin{enumerate}[label=(\alph*)]
\item \(2\)
\item \(1.2\)
\item \(0.4\)
\item \(3\)
\end{enumerate}
\newpage
\section*{Question 6}
<style>.fm-math,fmath{font-family:STIXGeneral,'DejaVu Serif','DejaVu Sans',Times,OpenSymbol,'Standard Symbols L',serif;line-height:1.2}.fm-math mtext,fmath mtext{line-height:normal}.fm-mo,.ma-sans-serif,fmath mi[mathvariant*=sans-serif],fmath mn[mathvariant*=sans-serif],fmath mo,fmath ms[mathvariant*=sans-serif],fmath mtext[mathvariant*=sans-serif]{font-family:STIXGeneral,'DejaVu Sans','DejaVu Serif','Arial Unicode MS','Lucida Grande',Times,OpenSymbol,'Standard Symbols L',sans-serif}.fm-mo-Luc{font-family:STIXGeneral,'DejaVu Sans','DejaVu Serif','Lucida Grande','Arial Unicode MS',Times,OpenSymbol,'Standard Symbols L',sans-serif}.questionsfont{font-weight:200;font-family:Arial, sans-serif, STIXGeneral,'DejaVu Sans','DejaVu Serif','Lucida Grande','Arial Unicode MS',Times,OpenSymbol,'Standard Symbols L',sans-serif!important}.fm-separator{padding:0 .56ex 0 0}.fm-infix-loose{padding:0 .56ex}.fm-infix{padding:0 .44ex}.fm-prefix{padding:0 .33ex 0 0}.fm-postfix{padding:0 0 0 .33ex}.fm-prefix-tight{padding:0 .11ex 0 0}.fm-postfix-tight{padding:0 0 0 .11ex}.fm-quantifier{padding:0 .11ex 0 .22ex}.ma-non-marking{display:none}.fm-vert,fmath menclose,menclose.fm-menclose{display:inline-block}.fm-large-op{font-size:1.3em}.fm-inline .fm-large-op{font-size:1em}fmath mrow{white-space:nowrap}.fm-vert{vertical-align:middle}fmath table,fmath tbody,fmath td,fmath tr{border:0!important;padding:0!important;margin:0!important;outline:0!important}fmath table{border-collapse:collapse!important;text-align:center!important;table-layout:auto!important;float:none!important}.fm-frac{padding:0 1px!important}td.fm-den-frac{border-top:solid thin!important}.fm-root{font-size:.6em}.fm-radicand{padding:0 1px 0 0;border-top:solid;margin-top:.1em}.fm-script{font-size:.71em}.fm-script .fm-script .fm-script{font-size:1em}td.fm-underover-base{line-height:1!important}td.fm-mtd{padding:.5ex .4em!important;vertical-align:baseline!important}fmath mphantom{visibility:hidden}fmath menclose[notation=top],menclose.fm-menclose[notation=top]{border-top:solid thin}fmath menclose[notation=right],menclose.fm-menclose[notation=right]{border-right:solid thin}fmath menclose[notation=bottom],menclose.fm-menclose[notation=bottom]{border-bottom:solid thin}fmath menclose[notation=left],menclose.fm-menclose[notation=left]{border-left:solid thin}fmath menclose[notation=box],menclose.fm-menclose[notation=box]{border:thin solid}fmath none{display:none}</style> A resistance wire connected in the left gap of a meter bridge balances a <fmath class="fm-inline"><mrow><mn>10</mn><mi class="fm-mi-length-1" mathvariant="italic">Ω</mi></mrow></fmath> resistance in the right gap at a point which divides the bridge wire in the ratio 3 : 2. If the length of the resistance wire is 1.5 m, then the length of <fmath class="fm-inline"><mrow><mn>1</mn><mi class="fm-mi-length-1" mathvariant="italic">Ω</mi></mrow></fmath> of the resistance wire is :  
\begin{enumerate}[label=(\alph*)]
\item  <fmath class="fm-inline"><mrow><mn>1.0</mn><mo class="fm-infix" lspace=".22em" rspace=".22em">×</mo><mrow><msup><mn>10</mn><mrow><mo class="fm-prefix-tight">−</mo><mn>1</mn></mrow></msup><mi class="fm-mi-length-1" mathvariant="italic">m</mi></mrow></mrow></fmath> 
\item  <fmath class="fm-inline"><mrow><mn>1.5</mn><mo class="fm-infix" lspace=".22em" rspace=".22em">×</mo><mrow><msup><mn>10</mn><mrow><mo class="fm-prefix-tight">−</mo><mn>1</mn></mrow></msup><mi class="fm-mi-length-1" mathvariant="italic">m</mi></mrow></mrow></fmath> 
\item  <fmath class="fm-inline"><mrow><mn>1.5</mn><mo class="fm-infix" lspace=".22em" rspace=".22em">×</mo><mrow><msup><mn>10</mn><mrow><mo class="fm-prefix-tight">−</mo><mn>2</mn></mrow></msup><mi class="fm-mi-length-1" mathvariant="italic">m</mi></mrow></mrow></fmath> 
\item  <fmath class="fm-inline"><mrow><mn>1.0</mn><mo class="fm-infix" lspace=".22em" rspace=".22em">×</mo><mrow><msup><mn>10</mn><mrow><mo class="fm-prefix-tight">−</mo><mn>2</mn></mrow></msup><mi class="fm-mi-length-1" mathvariant="italic">m</mi></mrow></mrow></fmath> 
\end{enumerate}
\newpage
\section*{Question 7}
A potentiometer has a uniform wire of length 10m and resistance 5 ohms. The potentiometer is connected to an external battery of emf of 10V and negligible internal resistance and a resistance of 995 ohms in series. The potential gradient along the wire is:
\begin{enumerate}[label=(\alph*)]
\item 1 mV/m
\item 1 mV/cm
\item 5 mV/cm
\item 5 mV/m
\end{enumerate}
\newpage
\section*{Question 8}
A Wheatstone bridge has the resistances 10Ω, 10Ω, 10Ω and 30Ω in its four arms. What resistance joined in parallel to the 30Ω resistance will bring it to be balanced condition?
\begin{enumerate}[label=(\alph*)]
\item 2Ω
\item 5Ω
\item 10Ω
\item 15Ω
\end{enumerate}
\newpage
\section*{Question 9}
The current in the rectangular loop of area \(A\) is I. If this loop is placed in the uniform magnetic field of intensity \(B\) and the angle between the magnetic field and the area is \(\theta\), then how much torque on loop will be?
\begin{enumerate}[label=(\alph*)]
\item \(NIAB \sin \theta\)
\item \(\frac{IB \sin \theta}{A}\)
\item \(\frac{NI}{B A \sin \theta}\)
\item \(\frac{I A B \sin \theta}{2}\)
\end{enumerate}
\newpage
\section*{Question 10}
A rectangular coil of length \(40\) cm and width \(10\) cm consists of \(10\) turns and carries a current of \(16\) A. The coil is suspended such that the normal to the plane of the coil makes an angle of \(60^{\circ}\) with the direction of a uniform magnetic field of magnitude \(0.60\) T. Find the magnitude of the torque experienced by the coil.
\begin{enumerate}[label=(\alph*)]
\item \(1.92\) N-m
\item \(1.92 \sqrt{3}\) N-m
\item \(1.62 \sqrt{3}\) N-m
\item \(0.64 \sqrt{3}\) N-m
\end{enumerate}
\newpage
\section*{Question 11}
A circular coil A of radius ‘a’ carries current ‘I’. Another circular coil B of radius ‘2a’ also carries the same current of ‘I’. The magnetic fields at the centers of the circular coils are in the ratio of:
\begin{enumerate}[label=(\alph*)]
\item 2 ∶ 1
\item 4 ∶ 1
\item 3 ∶ 1
\item 1 ∶ 1
\end{enumerate}
\newpage
\section*{Question 12}
The earth's magnetic induction at a certain point is \(7 \pi \times 10^{-5}\) Wb/m\(^{2}\). This is to be annulled by the magnetic induction at the center of a circular conducting loop of radius \(5 cm\). The required current in the loop is: \((\mu_{0}=4 \pi \times 10^{-7}\) TA\(^{-1}\) m\()\)
\begin{enumerate}[label=(\alph*)]
\item  0.56 A
\item 17.5 A
\item 0.17 A
\item  2.8 A
\end{enumerate}
\newpage
\section*{Question 13}
The force per unit length is 10-3 N on the two current-carrying wires of equal length that are separated by a distance of 2 m and placed parallel to each other. If the current in both the wires is doubled and the distance between the wires is halved, then what will be the force per unit length on the wire?
\begin{enumerate}[label=(\alph*)]
\item 2×10$^{-3}$ N
\item 4×10$^{-3}$ N
\item 8×10$^{-3}$ N
\item 16×10$^{-3}$ N
\end{enumerate}
\newpage
\section*{Question 14}
An electron is moving in a circular orbit in a magnetic field of \(2 \times 10^{-4}\) weber/m\(^{2}\). Its time period of revolution is:
\begin{enumerate}[label=(\alph*)]
\item \(1.79 \times 10^{-7}\) sec
\item \(3.5 \times 10^{-7}\) sec
\item \(7 \times 10^{-7}\) sec
\item \(2.75 \times 10^{-7}\) sec
\end{enumerate}
\newpage
\section*{Question 15}
Consider an infinitely long cylindrical straight conductor of radius \(r =2\) mm carrying current \(i=5\) A having uniform current density. At what distance \(d\) from the center of the wire is the value of the magnetic field \(1\) mT? (use \(\mu_{0}=4 \pi \times 10^{-7}\) H/m and assume r \(>\) d)
\begin{enumerate}[label=(\alph*)]
\item 0.25 mm
\item 0.5 mm
\item 1 mm
\item 1.5 mm
\end{enumerate}
\newpage
\section*{Question 16}
A power line lies along the East-West direction and carries a current of 10 A. The force per unit length due to the earth's magnetic field of 10$^{–4}$ T is :
\begin{enumerate}[label=(\alph*)]
\item 10$^{–5}$ Nm$^{–1}$
\item 10$^{–4}$ Nm$^{–1}$
\item 10$^{–3}$ Nm$^{–1}$
\item 10$^{–2}$ Nm$^{–1}$
\end{enumerate}
\newpage
\section*{Question 17}
A paramagnetic substance of susceptibility \(3 \times 10^{-4}\) is placed in a magnetic field of \(4 \times 10^{-4} Am ^{-1}\). Then the intensity of magnetization in the units of \(A m^{-1}\) is:
\begin{enumerate}[label=(\alph*)]
\item \(1.33 \times 10^8\)
\item \(0.75 \times 10^{-8}\)
\item \(12 \times 10^{-8}\)
\item \(14 \times 10^{-8}\)
\end{enumerate}
\newpage
\section*{Question 18}
A square loop of wire, side length \(10~ cm\) is placed at angle of \(45^{\circ}\) with a magnetic field that changes uniformly from \(0.1 ~T\) to zero in \(0.7\) seconds. The induced current in the loop (its resistance is \(1 ~\Omega\) ) is:
\begin{enumerate}[label=(\alph*)]
\item \(1.0 ~mA\)
\item \(2.5~ mA\)
\item \(3.5~ mA\)
\item \(4. 0 ~mA\)
\end{enumerate}
\newpage
\section*{Question 19}
A short bar magnet placed with its axis at \(30^{\circ}\) with a uniform external magnetic field of \(0.16 \mathrm{~T}\) experiences a torque of magnitude \(0.032 \mathrm{~Nm}\). If the bar magnet is free to rotate, then calculate its potential energies when it is in stable and unstable equilibrium state?
\begin{enumerate}[label=(\alph*)]
\item \(-0.064 \mathrm{~J},+0.064 \mathrm{~J}\)
\item \(-0.032 \mathrm{~J},+0.032 \mathrm{~J}\)
\item \(+0.064 \mathrm{~J},-0.128 \mathrm{~J}\)
\item \(0.032 \mathrm{~J},-0.032 \mathrm{~J}\)
\end{enumerate}
\newpage
\section*{Question 20}
A bar magnet having a magnetic moment of 2 × 10$^{4}$ JT$^{-1}$ is free to rotate in a horizontal plane. A horizontal magnetic field B = 6 × 10$^{-4}$ T exists in the space. The work done in taking the magnet slowly from a direction parallel to the field to a direction 60° from the field is:
\begin{enumerate}[label=(\alph*)]
\item 12 J
\item 6 J
\item 2 J
\item 0.6 J
\end{enumerate}
\newpage
\end{document}