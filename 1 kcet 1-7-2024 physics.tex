\documentclass{article}
                    \usepackage{amsmath}
                    \usepackage{amssymb}
                    \usepackage{graphicx}
                    \usepackage{enumitem}
                    \usepackage{longtable}
                    \title{1 kcet 1-7-2024 physics}
                    \begin{document}
                    \maketitle
                    \section*{Question 1}
The magnitude of the difference between the individual measurement and true value of the quantity is called:\newline
\begin{enumerate}[label=(\alph*)]
\item Absolute error
\item Relative error
\item Percentage error
\item None of these
\end{enumerate}
\newpage
\section*{Question 2}
The velocity of a freely falling body depends on \(g^{p} h^{q}\), where \(g\) is acceleration due to gravity and \(h\) is the height. The values of \(p\) and \(q\) are:
\begin{enumerate}[label=(\alph*)]
\item \(\frac{1}{2}, \frac{1}{2}\)
\item \(\frac{1}{2},-\frac{1}{2}\)
\item \(\frac{1}{2}, 1\)
\item 1,1
\end{enumerate}
\newpage
\section*{Question 3}
Which of the following pairs has the same dimensions?
\begin{enumerate}[label=(\alph*)]
\item Specific heat and latent heat
\item Impulse and momentum
\item Surface tension and force
\item Moment of Inertia and torque
\end{enumerate}
\newpage
\section*{Question 4}
Which of the following physical quantity is dimensionless?\newline
\begin{enumerate}[label=(\alph*)]
\item Angle
\item Strain
\item Specific gravity
\item All of these
\end{enumerate}
\newpage
\section*{Question 5}
Find the dimensional formula for magnetic field \(\vec{B}\) from the given formula \(\vec{F}=q \vec{B}|\vec{v} \sin \theta|\).
\begin{enumerate}[label=(\alph*)]
\item \(\left[\mathrm{M} \mathrm{L}^{1} \mathrm{~T}^{-2} \mathrm{~A}^{-1}\right]\)
\item \(\left[\mathrm{M} \mathrm{T}^{-2} \mathrm{~A}^{-1}\right]\)
\item \(\left[\mathrm{M} \mathrm{T}^{-3} \mathrm{~A}^{-1}\right]\)
\item \(\left[\mathrm{M} \mathrm{T}^{-2} \mathrm{~A}^{-2}\right]\)
\end{enumerate}
\newpage
\section*{Question 6}
A physical quantity Q is found to depend on observables \(\mathrm{x}, \mathrm{y}\) and \(\mathrm{z}\), obeying relation \(\mathrm{Q}=\frac{\mathrm{x}^{2 / 5} \mathrm{z}^{3}}{\mathrm{y}}\) The percentage error in the measurements of \(\mathrm{x}, \mathrm{y}\) and \(\mathrm{z}\) are \(1 \%, 2 \%\) and \(4 \%\) respectively. What is percentage error in the quantity Q will be:
\begin{enumerate}[label=(\alph*)]
\item 9 %
\item 10 %
\item 11 %
\item 12 %
\end{enumerate}
\newpage
\section*{Question 7}
Random error can be eliminated by:
\begin{enumerate}[label=(\alph*)]
\item Careful observation
\item Eliminating the cause
\item Measuring the quantity with more than one instrument
\item Taking large number of observations and then their mean
\end{enumerate}
\newpage
\section*{Question 8}
If the fundamental quantities are energy \((E)\), velocity \((v)\) and force \((F)\), then what will the dimensions of mass?
\begin{enumerate}[label=(\alph*)]
\item \({Ev}^{2}\)
\item \({Ev}^{-2}\)
\item \({Fv}^{-1}\)
\item \({Fv}^{-2}\)
\end{enumerate}
\newpage
\section*{Question 9}
The dimensional formula of Planck's constant is:
\begin{enumerate}[label=(\alph*)]
\item \(\left[M L^{2} T^{-1}\right]\)
\item \(\left[M L^{2} T^{-2}\right]\)
\item \(\left[M L^{0} T^{2}\right]\)
\item \(\left[M L T^{-2}\right]\)
\end{enumerate}
\newpage
\section*{Question 10}
An Odometer is an instrument used to measure ________ in automobiles.
\begin{enumerate}[label=(\alph*)]
\item speed
\item odour
\item direction
\item distance
\end{enumerate}
\newpage
\section*{Question 11}
In SI system the fundamental units are:
\begin{enumerate}[label=(\alph*)]
\item meter, kilogram, second, ampere, Kelvin, mole and candela
\item meter, kilogram, second, coulomb, Kelvin, mole and candela
\item meter, Newton, second, ampere, Kelvin, mole and candela
\item meter, kilogram, second, ampere, Kelvin, mole and lux
\end{enumerate}
\newpage
\section*{Question 12}
If \(\mathrm{M}\) denotes angular momentum and \(\mathrm{p}\) denotes linear momentum, the dimensions of \(\frac{\mathrm M}{\mathrm p}\) is:
\begin{enumerate}[label=(\alph*)]
\item \([\mathrm L^2]\)
\item \([\mathrm L^0]\)
\item \([\mathrm L^1]\)
\item \([\mathrm L^3]\)
\end{enumerate}
\newpage
\section*{Question 13}
The unit of momentum is:
\begin{enumerate}[label=(\alph*)]
\item \(\mathrm{Kgms}^{2}\)
\item \(\mathrm{Kgms}^{-2}\)
\item \(\mathrm{Kgms}\)
\item \(\mathrm{Kgms}^{-1}\)
\end{enumerate}
\newpage
\section*{Question 14}
What is the dimensional formula of strain?
\begin{enumerate}[label=(\alph*)]
\item \(\mathrm{M}^{1} \mathrm{~L}^{2} \mathrm{~T}^{-2}\)
\item \(\mathrm{M}^{1} \mathrm{~L}^{-1} \mathrm{~T}^{-2}\)
\item \(\mathrm{M}^{0} \mathrm{~L}^{0} \mathrm{~T}^{-1}\)
\item None of the above
\end{enumerate}
\newpage
\section*{Question 15}
Which of following is the dimensional formula of Density?
\begin{enumerate}[label=(\alph*)]
\item \(\left[\mathrm{M}^{0} \mathrm{LT}^{-1}\right]\)
\item \(\left[\mathrm{MLT}^{-2}\right]\)
\item \(\left[\mathrm{ML}^{-3} \mathrm{~T}^{0}\right]\)
\item \(\left[\mathrm{MLT}^{-1}\right]\)
\end{enumerate}
\newpage
\section*{Question 16}
Which of the physics quantity has the same unit in both C.G.S and M.K.S system?
\begin{enumerate}[label=(\alph*)]
\item Velocity
\item Distance
\item Time
\item Mass
\end{enumerate}
\newpage
\section*{Question 17}
Which one of the following is not a derived unit?
\begin{enumerate}[label=(\alph*)]
\item Joule
\item  Watt
\item Newton
\item Kilogram
\end{enumerate}
\newpage
\section*{Question 18}
A meter reads \(125 \mathrm{~V}\) and the true value of the voltage is \(125.5 \mathrm{~V}\). Find the absolute error of the instrument.
\begin{enumerate}[label=(\alph*)]
\item \(\frac{125 }{0.5 \mathrm{~V}}\)
\item \(125 \mathrm{~V}\)
\item \(0.5 \mathrm{~V}\)
\item \(\frac{0.5 }{125 \mathrm{~V}}\)
\end{enumerate}
\newpage
\section*{Question 19}
A wattmeter reads \(25.34 \mathrm{~W}\). The absolute error in measurement is \(-0.11 \mathrm{~W}\). What is the true value of power:
\begin{enumerate}[label=(\alph*)]
\item \(25.23 \mathrm{~W}\)
\item \(25.45 \mathrm{~W}\)
\item \(-25.23 \mathrm{~W}\)
\item \(-25.45 \mathrm{~W}\)
\end{enumerate}
\newpage
\section*{Question 20}
Which of the following quantity has dimensional formula as that of \(\frac{\text { Energy }}{\text { Mass } \times \text { Length }}\) is:
\begin{enumerate}[label=(\alph*)]
\item Force
\item Power
\item Pressure
\item Acceleration
\end{enumerate}
\newpage
\section*{Question 21}
The dimensions of 'resistance' are same as those of __________ where \(h\) is the Planck's constant.
\begin{enumerate}[label=(\alph*)]
\item \(\frac{h}{e^2}\)
\item \(\frac{h}{e}\)
\item \(\frac{h^2}{e^2}\)\newline
\item \(\frac{h^2}{e}\)
\end{enumerate}
\newpage
\section*{Question 22}
If the unit of length, time and mass are increased by 10 times, then the numerical value of "g" (acceleration due to gravity) will:
\begin{enumerate}[label=(\alph*)]
\item Decreased by 10 times
\item Increased by 10 times
\item Increased by 20 times
\item Remains the same
\end{enumerate}
\newpage
\section*{Question 23}
The magnitude of any physical quantity:
\begin{enumerate}[label=(\alph*)]
\item Depends on the method of measurement
\item Does not depend on the method of measurement
\item Is more in SI system than in CGS system
\item Directly proportional to the fundamental units of mass, length and time
\end{enumerate}
\newpage
\section*{Question 24}
If the unit of length and force be increased four times, then the unit of energy is:
\begin{enumerate}[label=(\alph*)]
\item Increased 4 times
\item Increased 8 times
\item Increased 16 times
\item Decreased 16 times
\end{enumerate}
\newpage
\section*{Question 25}
The dimensional formula for impulse is same as the dimensional formula for:
\begin{enumerate}[label=(\alph*)]
\item Momentum
\item Force
\item Rate of change of momentum
\item Torque
\end{enumerate}
\newpage
\section*{Question 26}
Planck's constant has the dimensions (unit) of:
\begin{enumerate}[label=(\alph*)]
\item Energy
\item Linear momentum
\item Work
\item Angular momentum
\end{enumerate}
\newpage
\section*{Question 27}
Which one of the following quantities has dimensions different from the remaining three?
\begin{enumerate}[label=(\alph*)]
\item Energy per unit volume
\item Force per unit area
\item Product of voltage and charge per unit volume
\item Angular momentum per unit mass
\end{enumerate}
\newpage
\section*{Question 28}
Which of the following quantities does not have Nm$^{–2}$ as the unit?
\begin{enumerate}[label=(\alph*)]
\item Pressure
\item Stress
\item Strain
\item Young’s modulus
\end{enumerate}
\newpage
\section*{Question 29}
________ is the international unit of measuring Energy.
\begin{enumerate}[label=(\alph*)]
\item Calorie
\item Joule
\item Watt
\item Kilowatt
\end{enumerate}
\newpage
\section*{Question 30}
Lux is unit of which physical quantity?
\begin{enumerate}[label=(\alph*)]
\item Luminance
\item Luminous intensity
\item Illumination
\item None of these
\end{enumerate}
\newpage
\section*{Question 31}
The unit Joule/coulomb is the same as:
\begin{enumerate}[label=(\alph*)]
\item 1 ampere
\item 1 kWh
\item 1 KW
\item 1 Volt
\end{enumerate}
\newpage
\section*{Question 32}
The unit of surface tension may be expressed as:
\begin{enumerate}[label=(\alph*)]
\item Joule metre
\item Newton metre\({ }^{-2}\)
\item Joule metre\({ }^{-2}\)
\item Newton metre
\end{enumerate}
\newpage
\section*{Question 33}
The dimension of \(\frac{\mathrm{h}}{\mathrm{e}}\) is the same as of (here \(h\) is Planck's constant and \(e\) is electronic charges):
\begin{enumerate}[label=(\alph*)]
\item Voltage
\item Magnetic flux
\item Current
\item Angular
\end{enumerate}
\newpage
\section*{Question 34}
If energy of photon is \(\mathrm{E} \propto \mathrm{h}^{\mathrm{a}} \mathrm{c}^{\mathrm{b}} \lambda^{\mathrm{d}}\). Here, \(h=\) plank's constant, \(c=\) speed of light, \(\lambda=\) wavelength of photon. Then the values of \(a, b\) and \(d\) are:
\begin{enumerate}[label=(\alph*)]
\item \(1,1,1\)
\item \(1,-1,1\)
\item \(1,1,-1\)
\item None of these
\end{enumerate}
\newpage
\section*{Question 35}
Which of the following pairs does not have similar dimensions?
\begin{enumerate}[label=(\alph*)]
\item Tension and surface tension
\item Stress and pressure
\item Angle and strain
\item Planck's constant and angular momentum
\end{enumerate}
\newpage
\section*{Question 36}
What is an international (SI) unit of electric current?
\begin{enumerate}[label=(\alph*)]
\item Ohm
\item Volt
\item Ohm-meter
\item Ampere
\end{enumerate}
\newpage
\section*{Question 37}
Dimensional analysis can be applied to:
\begin{enumerate}[label=(\alph*)]
\item To change units from one system to another
\item To check the consistency of a dimensional equation
\item To derive the relation between physical quantities in physical phenomena
\item All of the above
\end{enumerate}
\newpage
\section*{Question 38}
In SI unit system, pascal is the unit of:
\begin{enumerate}[label=(\alph*)]
\item Pressure
\item Work
\item Energy
\item Power
\end{enumerate}
\newpage
\section*{Question 39}
Which of the following is not a physical quantity?
\begin{enumerate}[label=(\alph*)]
\item Length
\item Time
\item Electric current
\item Kilogram (kg)
\end{enumerate}
\newpage
\section*{Question 40}
A force \(F\) is given by \(F=a t+b t^2\), where \(t\) is time. What are the dimensions of \(a\) and \(b\)?
\begin{enumerate}[label=(\alph*)]
\item \([M L T^{-3}]\) and \([M L^2 T^{-4}]\)
\item \([M L T^{-3}]\) and \([M L T^{-4}]\)
\item \([M L T^{-1}]\) and \([M L T^0]\)
\item \([M L T^{-4}]\) and \([M L T^1]\)
\end{enumerate}
\newpage
\section*{Question 41}
Which one of the following cannot be the unit of 'Pressure'?
\begin{enumerate}[label=(\alph*)]
\item \(\mathrm{N} / \mathrm{m}^2\)
\item \(\mathrm{Kg}\)-\(\mathrm{m}/\mathrm{s}^2\)
\item \(\mathrm{Kg}/(\mathrm{m}.\mathrm{s}^2)\)
\item Pascal
\end{enumerate}
\newpage
\section*{Question 42}
Light year is a unit of:
\begin{enumerate}[label=(\alph*)]
\item Time
\item Mass
\item Distance
\item Energy
\end{enumerate}
\newpage
\section*{Question 43}
Which of the following has the unit Candela?
\begin{enumerate}[label=(\alph*)]
\item Electric intensity
\item Luminous intensity
\item Sound intensity
\item None of these
\end{enumerate}
\newpage
\section*{Question 44}
Which of the following pair has same dimensions?
\begin{enumerate}[label=(\alph*)]
\item Pressure and stress
\item Stress and strain
\item Pressure and force
\item Power and force
\end{enumerate}
\newpage
\section*{Question 45}
The dimensional formula for the modulus of rigidity is:
\begin{enumerate}[label=(\alph*)]
\item \([\mathrm{ML}^2 \mathrm{~T}^{-2}]\)
\item \([\mathrm{ML} \mathrm{~T}^{-3}]\)
\item \([\mathrm{ML}^{-1} \mathrm{~T}^{-3}]\)
\item \([\mathrm{ML}^{-1} \mathrm{~T}^{-2}]\)
\end{enumerate}
\newpage
\section*{Question 46}
The unit of surface tension may be expressed as:
\begin{enumerate}[label=(\alph*)]
\item \(\mathrm{Nm}\)
\item \(\mathrm{Nm}^2\)
\item \(\mathrm{Nm}^2\)
\item \(\mathrm{Nm}^{-1}\)
\end{enumerate}
\newpage
\section*{Question 47}
If \(V\) denotes the potential difference across the plates of a capacitor of capacitance \(c\), the dimensions of \(\mathrm{CV}^2\) are:
\begin{enumerate}[label=(\alph*)]
\item Not expressible in [MLT]
\item  \([\mathrm{MLT}^{-2}]\)
\item  \([\mathrm{M}^2 \mathrm{LT}^{-1}]\)
\item \([\mathrm{ML}^2 \mathrm{~T}^{-2}]\)
\end{enumerate}
\newpage
\section*{Question 48}
A system of units uses force (F), acceleration (A) and time (T) as their fundamental physical quantities. The dimension of length in the system is:
\begin{enumerate}[label=(\alph*)]
\item \([\mathrm{FT}^2]\)
\item \([\mathrm{F}^{-1} \mathrm{~T}^2]\)
\item \([\mathrm{F}^{-1} \mathrm{~A}^2 \mathrm{~T}^{-1}]\)
\item \([\mathrm{AT}^2]\)
\end{enumerate}
\newpage
\section*{Question 49}
The pair having the same dimensions is:
\begin{enumerate}[label=(\alph*)]
\item Angular momentum, work
\item Work, torque
\item Potential energy, linear momentum
\item Kinetic energy, velocity
\end{enumerate}
\newpage
\section*{Question 50}
Dimensions of the following three quantities are the same:
\begin{enumerate}[label=(\alph*)]
\item Work, energy, force
\item Velocity, momentum, impulse
\item Potential energy, kinetic energy, momentum
\item Pressure, stress, coefficient of elasticity
\end{enumerate}
\newpage
\section*{Question 51}
The fundamental physical quantities that have same dimensions in the dimensional formulae of torque and angular momentum are:
\begin{enumerate}[label=(\alph*)]
\item Mass, time
\item Time, length
\item Mass, length
\item Time, mole
\end{enumerate}
\newpage
\section*{Question 52}
If pressure \(P\), velocity \(v\) and time \(T\) are taken as fundamental physical quantities, the dimensional formula of force is:
\begin{enumerate}[label=(\alph*)]
\item \(P v^2 T^2\)
\item \(P^{-1} v^2 T^{-2}\)
\item \(P v T^2\)
\item \(P^{-1} v T^2\)
\end{enumerate}
\newpage
\section*{Question 53}
The dimensions of \(\mathrm{B}^2 \mathrm{~L}^2 \mathrm{C}\) (where \(\mathrm{B}\) is magnetic field, \(\mathrm{L}\) is length and \(\mathrm{C}\) is capacitance) is same as that of:
\begin{enumerate}[label=(\alph*)]
\item Mass
\item Length
\item Time
\item Force
\end{enumerate}
\newpage
\section*{Question 54}
The dimensional formula for strain is same as that for:
\begin{enumerate}[label=(\alph*)]
\item Stress
\item Modulus of elasticity
\item Thrust
\item Angle
\end{enumerate}
\newpage
\section*{Question 55}
Dimensions of velocity gradient are same as that of:\newline
\begin{enumerate}[label=(\alph*)]
\item Time period
\item Frequency
\item Angular acceleration
\item Acceleration
\end{enumerate}
\newpage
\section*{Question 56}
Temperature can be expressed as derived quantity in terms of:
\begin{enumerate}[label=(\alph*)]
\item Mass and time
\item Length and mass
\item Length, mass and time
\item None of these
\end{enumerate}
\newpage
\section*{Question 57}
Which of the following is a dimensionless quantity?
\begin{enumerate}[label=(\alph*)]
\item \(\frac{\text {Force}}{\text {Acceleration}}\)
\item \(\frac{\text {Velocity}}{\text {Acceleration}}\)\newline
\item \(\frac{\text { Volume}}{\text {Area}}\)
\item \(\frac{\text { Energy }}{\text { Work }}\)
\end{enumerate}
\newpage
\section*{Question 58}
Which of the following is a dimensional constant?
\begin{enumerate}[label=(\alph*)]
\item Gravitational constant
\item Dielectric constant
\item Refractive index
\item Relative density
\end{enumerate}
\newpage
\section*{Question 59}
Given below are two statements:
Statement I : Astronomical unit \((Au)\), Parsec \((Pc)\) and Light year \((ly)\) are units for measuring astronomical distances.
Statement II : \(Au <\) Parsec \(( Pc )< ly\)
In the light of the above statements, choose the most appropriate answer from the options given below: 
\begin{enumerate}[label=(\alph*)]
\item Both Statements I and Statements II are incorrect
\item Statements I is correct but Statements II is incorrect
\item Both Statements I and Statements II are correct
\item Statements I is incorrect but Statements II is correct
\end{enumerate}
\newpage
\section*{Question 60}
Identify the pair of physical quantities which have different dimensions: 
\begin{enumerate}[label=(\alph*)]
\item Wave number and Rydberg's constant
\item Stress and Coefficient of elasticity
\item Coercivity and Magnetisation
\item Specific heat capacity and Latent heat
\end{enumerate}
\newpage
\end{document}