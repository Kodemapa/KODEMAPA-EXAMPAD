\documentclass{article}
                    \usepackage{amsmath}
                    \usepackage{amssymb}
                    \usepackage{graphicx}
                    \usepackage{enumitem}
                    \usepackage{longtable}
                    \title{Chem - KCET - 12}
                    \begin{document}
                    \maketitle
                    \section*{Question 1}
If a solution prepared by dissolving 1.0 g of polymer of molar mass 185,000 in 450 mL of water at 37°C, calculate the osmotic pressure in Pascal exerted by it?
\begin{enumerate}[label=(\alph*)]
\item 31 \(Pa\)
\item 30 \(Pa\)
\item 32 \(Pa\)
\item 33 \(Pa\)
\end{enumerate}
\newpage
\section*{Question 2}
Heptane and octane form an ideal solution. At 373 K, the vapour pressures of the two liquid components are 105.2 kPa and 46.8 kPa respectively. What will be the vapour pressure of a mixture of 26.0 g of heptane and 35 g of octane?
\begin{enumerate}[label=(\alph*)]
\item 74.3 kPa
\item 73.43 kPa
\item 76.42 kPa
\item 79.50 kPa
\end{enumerate}
\newpage
\section*{Question 3}
At 300 K, 36 g of glucose present in a litre of its solution has an osmotic pressure of 4.98 bar. If the osmotic pressure of the solution is 1.52 bars at the same temperature, what would be its concentration?
\begin{enumerate}[label=(\alph*)]
\item 0.061 mol
\item 0.063 mol
\item 0.065 mol
\item 0.070 mol
\end{enumerate}
\newpage
\section*{Question 4}
100 g of liquid A (molar mass 140 g mol$^{–1}$) was dissolved in 1000 g of liquid B (molar mass 180 g mol–1). The vapour pressure of pure liquid B was found to be 500 torrs. Calculate the vapour pressure of pure liquid A and its vapour pressure in the solution if the total vapour pressure of the solution is 475 Torr.
\begin{enumerate}[label=(\alph*)]
\item 281.7 torr
\item 290.8 torr
\item 260.7 torr
\item 280.7 torr
\end{enumerate}
\newpage
\section*{Question 5}
An aqueous solution of hydrochloric acid:
\begin{enumerate}[label=(\alph*)]
\item Obeys Raoult's law
\item Shows negative deviation from Raoult's law
\item Shows positive deviation from Raoult's law
\item Obeys Henry's law at all compositions
\end{enumerate}
\newpage
\section*{Question 6}
\(1.00 {~g}\) of a non- electrolyte solute (molar mass \(250 {~g} {~mol}^{-1}\) ) was dissolved in \(51.2 {~g}\) of benzene. If the freezing point depression constant, \({K}_{{f}}\) of benzene is \(5.12 {~k} {~kg} {mol}^{-1}\), the freezing point of benzene will be lowered by:
\begin{enumerate}[label=(\alph*)]
\item \(0.4 {~K}\)
\item \(0.3 {~K}\)
\item \(0.5 {~K}\)
\item \(0.2 {~K}\)
\end{enumerate}
\newpage
\section*{Question 7}
The solubility of the gas in a liquid solution _____ with increase in temperature.
\begin{enumerate}[label=(\alph*)]
\item Decreases
\item Increases
\item Remains same
\item None of these
\end{enumerate}
\newpage
\section*{Question 8}
When a non volatile solute is added to a pure solvent, the :
\begin{enumerate}[label=(\alph*)]
\item Vapour pressure of the solution becomes lower then that of the pure solvent
\item Rate of evaporation of the pure solvent is reduced
\item Solute does not effect the rate of condensation
\item Rate of the evaporation of the solution is equal to the rate of condensation of the solution at a lower vapour pressure than that in the case of the pure solvent
\end{enumerate}
\newpage
\section*{Question 9}
Effect of adding a non-volatile solute to a solvent is:
\begin{enumerate}[label=(\alph*)]
\item To lower the vapour pressure
\item To increase its boiling point
\item Both (A) and (B)
\item To decrease its osmotic pressure
\end{enumerate}
\newpage
\section*{Question 10}
Which of the following is 'NOT' a colligative property ?
\begin{enumerate}[label=(\alph*)]
\item Elevation in boiling point
\item Depression in freezing point
\item Osmotic pressure
\item Lowering of vapour pressure
\end{enumerate}
\newpage
\section*{Question 11}
When a solute is present in trace quantities. It is convenient to express concentration in:
\begin{enumerate}[label=(\alph*)]
\item Mass by volume percentage
\item Mole fraction
\item Mass percentage
\item Parts per million
\end{enumerate}
\newpage
\section*{Question 12}
What happens when a solute crystal is added to a supersaturated solution?
\begin{enumerate}[label=(\alph*)]
\item It becomes a colloidal solution
\item The solute dissolves in the solution
\item The solution desaturates
\item The solute precipitates out of the solution
\end{enumerate}
\newpage
\section*{Question 13}
Which of the following is a true solution?
\begin{enumerate}[label=(\alph*)]
\item Salt solution
\item Ink
\item Blood
\item Starch solution
\end{enumerate}
\newpage
\section*{Question 14}
The expression relating molality \((\mathrm{m})\) and mole fraction \(\left(\mathrm{x}_2\right)\) of solute in a solution is:
\begin{enumerate}[label=(\alph*)]
\item \(\mathrm{x}_2=\frac{\mathrm{mM}_1}{1+\mathrm{mM}_1}\)
\item \(\mathrm{x}_2=\frac{\mathrm{mM}_1}{1-\mathrm{mM}_1}\)
\item \(\mathrm{x}_2=\frac{1+\mathrm{mM}_1}{\mathrm{mM}_1}\)
\item \(\mathrm{x}_2=\frac{1-\mathrm{mM}_1}{\mathrm{mM}_1}\)
\end{enumerate}
\newpage
\section*{Question 15}
On increasing temperature, vapour pressure of a substance ______________.
\begin{enumerate}[label=(\alph*)]
\item always increases.
\item decreases.
\item does not depend on temperature.
\item partially depends on temperature.
\end{enumerate}
\newpage
\section*{Question 16}
\(60 {~gm}\) of Urea (Mol. wt 60) was dissolved in \(9.9\) moles of water. If the vapour pressure of pure water is \({P}^{\circ}\), the vapour pressure of solution is:\newline
\begin{enumerate}[label=(\alph*)]
\item \(0.10 \ {P}^{\circ}\)
\item \(1.10 \ {P}^{\circ}\)\newline
\item \(0.90 \ {P}^{\circ}\)
\item \(0.99 \ {P}^{\circ}\)
\end{enumerate}
\newpage
\section*{Question 17}
Which of the following colligative properties can provide molar mass of proteins (or polymers or colloids) with greater precision:
\begin{enumerate}[label=(\alph*)]
\item Relative lowering of vapour pressure
\item Elevation of boiling point
\item Depression in freezing point
\item Osmotic pressure
\end{enumerate}
\newpage
\section*{Question 18}
Cryoscopic constant of a liquid is:
\begin{enumerate}[label=(\alph*)]
\item Decrease in freezing point when 1 gram of solute is dissolved per kg of the solvent.
\item Decrease in the freezing point when 1 mole of solute is dissolved per kg of the solvent.
\item The elevation for 1 molar solution
\item factor used for calculation of elevation in boiling point
\end{enumerate}
\newpage
\section*{Question 19}
Van't Hoff factor, when benzoic acid is dissolved in benzene, will be:
\begin{enumerate}[label=(\alph*)]
\item \(2\)
\item \(1\)
\item \(0.5\)
\item \(1.5\)
\end{enumerate}
\newpage
\section*{Question 20}
Which of the following solvents would most likely dissolve 3-Aminopropan-1-ol?
\begin{enumerate}[label=(\alph*)]
\item C$_{6}$H$_{5}$OH
\item C$_{2}$H$_{5}$OH
\item H$_{2}$O
\item CH$_{3}$COCH$_{3}$
\end{enumerate}
\newpage
\section*{Question 21}
Given below are the half-cell reactions:\(M n^{2+}+2 e^{-} \rightarrow M n ; E^o=-1.18 V\)\(2\left(M n^{3+}+e^{-} \rightarrow M n^{2+}\right) ; E^o=+1.51 V\)The \(E^o\) for \(3 Mn ^{2+} \rightarrow M n+2 M n^{3+}\) will be:
\begin{enumerate}[label=(\alph*)]
\item \(-0.33 V\); the reaction will not occur
\item \(-0.33 V\); the reaction will occur
\item \(-2.69 V\); the reaction will not occur
\item \(-2.69 V\); the reaction will occur
\end{enumerate}
\newpage
\section*{Question 22}
Silver is uniformly electro-deposited on a metallic vessel of surface area of \(900 \mathrm{~cm}^{2}\) by passing a current of \(0.5\) ampere for 2 hours. Calculate the thickness of silver deposited. (Given : the density of silver is \(10.5 \mathrm{~g} / \mathrm{cm}^{3}\) and atomic mass of \(\mathrm{Ag}=180 \mathrm{~amu}, \mathrm{F}= 96,500\mathrm{~C ~mol^{-1}}\) )
\begin{enumerate}[label=(\alph*)]
\item \(4.26 \times 10^{-4} \mathrm{~cm}\)\newline
\item \(2.46 \times 10^{-4} \mathrm{~cm}\)\newline
\item \(3.29 \times 10^{-4} \mathrm{~cm}\)\newline
\item \(1.69 \times 10^{-4} \mathrm{~cm}\)\newline
\end{enumerate}
\newpage
\section*{Question 23}
\(E_{\text {cell }}^{o}\) for the reaction, \(2 {H}_{2} {O} \rightarrow {H}_{3} {O}^{+}+{OH}^{-}\)at \(25^{\circ} {C}\) is \(-0.8277 {~V}\). The equilibrium constant for the reaction is:
\begin{enumerate}[label=(\alph*)]
\item \(10^{-14}\)
\item \(10^{-23}\)
\item \(10^{-7}\)
\item \(10^{-21}\)
\end{enumerate}
\newpage
\section*{Question 24}
How much electricity in terms of Faraday is required to produce \(100 {~g}\) of Ca from molten \(C a C l_{2} ?\)
\begin{enumerate}[label=(\alph*)]
\item 1F
\item 2F
\item 3F
\item 5F
\end{enumerate}
\newpage
\section*{Question 25}
Current in an electrolyte is carried by ________.
\begin{enumerate}[label=(\alph*)]
\item only electrons
\item only -ve ions
\item only +ve ions
\item both +ve and -ve ions
\end{enumerate}
\newpage
\section*{Question 26}
\(Cu ^{+}\)is not stable and undergoes disproportionation. \(E ^0\) for \(Cu ^{+}\)disproportionation is :Given standard reduction potentials,\(E _{ Cu ^{2+} / Cu ^{+}}^0=+0.15 V \)\newline\(E _{ Cu ^{+} / Cu }^0=0.53 V\)
\begin{enumerate}[label=(\alph*)]
\item \(+0.68 V\)
\item \(+0.19 V\)
\item \(-0.19 V\)
\item \(+0.38 V\)
\end{enumerate}
\newpage
\section*{Question 27}
_______ is an application of electrolysis.
\begin{enumerate}[label=(\alph*)]
\item Oxidation
\item Electrotyping
\item Electric shielding
\item Electric polishing
\end{enumerate}
\newpage
\section*{Question 28}
The concentration of potassium ions inside a biological cell is at least twenty times higher than the outside. The resulting potential difference across the cell is important in several processes such as the transmission of nerve impulses and maintaining the ion balance. A simple model for such a concentration cell involving a metal \(M\) is:\(M ( s ) \mid M ^{+}( aq ; 0.05\) molar \() \| M ^{+}( aq ), 1\) molar \() \mid M ( s )\)For the above electrolytic cell the magnitude of the cell potential \(\left| E _{\text {cell }}\right|=70 mV\).For the above cell:
\begin{enumerate}[label=(\alph*)]
\item \(E _{\text {cell }}<0 ; \Delta G >0\)
\item \(E _{\text {cell }}>0 ; \Delta G <0\)
\item \(E_{\text {cell }}<0 ; \Delta G ^{\circ}>0\)
\item \(E _{ cell }>0 ; \Delta G ^0>0\)
\end{enumerate}
\newpage
\section*{Question 29}
Standard electrode potentials of \(Fe ^{2+}+2 e ^{-} \rightarrow Fe\) and \(Fe ^{3+}+3 e ^{-} \rightarrow Fe\) are \(-0.440 V\) and \(-0.036 V\) respectively. The standard electrode potential \(\left( E ^{\circ}\right)\) for \(Fe ^{3+}+ e ^{-} \rightarrow Fe ^{2+}\) is:
\begin{enumerate}[label=(\alph*)]
\item \(-0.476 V\)
\item \(-0.404 V\)
\item \(+0.404 V\)
\item \(+0.772 V\)
\end{enumerate}
\newpage
\section*{Question 30}
Maintenance-free batteries, now in use, in place of common batteries, have:
\begin{enumerate}[label=(\alph*)]
\item Electrodes made of lead-lead oxide
\item Electrodes made of calcium-containing lead alloy
\item Non-aqueous solvents as medium
\item Platinum electrodes
\end{enumerate}
\newpage
\section*{Question 31}
An acid is a substance that produces _________ ions in a water solution.
\begin{enumerate}[label=(\alph*)]
\item oxygen
\item nitrogen
\item carbon
\item hydrogen
\end{enumerate}
\newpage
\section*{Question 32}
The filament resistance of the bulb is ________ to its resistance when it is not glowing.
\begin{enumerate}[label=(\alph*)]
\item greater
\item lower
\item equal
\item none of above
\end{enumerate}
\newpage
\section*{Question 33}
An example of the secondary battery cell is:
\begin{enumerate}[label=(\alph*)]
\item Edison Alkaline cell
\item Daniel cell
\item Lachanche cell
\item Bunsencell
\end{enumerate}
\newpage
\section*{Question 34}
What happens when the lead storage battery is discharged?
\begin{enumerate}[label=(\alph*)]
\item \(SO _2\) is evolved
\item Lead sulphate is consumed
\item Lead is formed
\item \(H _2 SO _4\) is consumed
\end{enumerate}
\newpage
\section*{Question 35}
The tarnishing of silver ornaments in the atmosphere is due to:
\begin{enumerate}[label=(\alph*)]
\item \(Ag _2 Cl\)
\item \(Ag _2 S\)
\item \(Ag _2 CO _3\)
\item \(Ag _2 SO _4\)
\end{enumerate}
\newpage
\section*{Question 36}
The approximate time duration in hours to electroplate \(30 g\) of calcium from molten calcium chloride using a current of \(5 amp\) is:[Atomic mass of \(Ca =40\)]
\begin{enumerate}[label=(\alph*)]
\item 8
\item 80
\item 10
\item 16
\end{enumerate}
\newpage
\section*{Question 37}
The charge carriers in metallic conductors and in electrolytes are respectively:
\begin{enumerate}[label=(\alph*)]
\item Both ions
\item Both electrons
\item Electrons and ions
\item Ions and electrons
\end{enumerate}
\newpage
\section*{Question 38}
Which of the following is not a characteristic feature of a salt bridge?
\begin{enumerate}[label=(\alph*)]
\item Salt bridge joins the two halves of an electrochemical cell
\item It completes the inner circuit
\item It is filled with a salt solution (or gel)
\item It does not maintain electrical neutrality of the electrolytic solutions of the half-cells
\end{enumerate}
\newpage
\section*{Question 39}
What is the direction of flow of electrons in an electrolytic cell?
\begin{enumerate}[label=(\alph*)]
\item Anode to cathode externally
\item Anode to cathode internally
\item Cathode to anode externally
\item Cathode to anode in the solution
\end{enumerate}
\newpage
\section*{Question 40}
Which of the following statements regarding primary cells is false?
\begin{enumerate}[label=(\alph*)]
\item Primary cells cannot be recharged
\item They have low internal resistance
\item They have an irreversible chemical reaction
\item Their initial cost is cheap
\end{enumerate}
\newpage
\section*{Question 41}
Which of the following compounds  has the highest boiling point?
\begin{enumerate}[label=(\alph*)]
\item RF
\item RCl
\item RBr
\item RI
\end{enumerate}
\newpage
\section*{Question 42}
Which of the following compounds is an allyl bromide?
\begin{enumerate}[label=(\alph*)]
\item C$_{3}$H$_{5}$Br\newline
\item CH$_{3}$CH$_{2}$CH$_{2}$Br
\item CH$_{2}$=CHCH$_{2}$CH$_{2}$Br
\item CH$_{2}$=CHCH$_{2}$CH$_{2}$CH$_{2}$Br
\end{enumerate}
\newpage
\section*{Question 43}
Groove's method is used for the preparation of:\newline
\begin{enumerate}[label=(\alph*)]
\item \(\mathrm{C}_2 \mathrm{H}_5 \mathrm{Cl}\)\newline
\item \(\mathrm{CH}_3 \mathrm{COOH}\)\newline
\item \(\mathrm{C}_2 \mathrm{H}_5 \mathrm{OH}\)\newline
\item All of these\newline
\end{enumerate}
\newpage
\section*{Question 44}
Flourobenzene \(\left(\mathrm{C}_6 \mathrm{H}_5 \mathrm{~F}\right)\) can be synthesised in the laboratory:
\begin{enumerate}[label=(\alph*)]
\item A by heating phenol with \(\mathrm{HF}\) and \(\mathrm{KF}\)
\item from aniline by diazotisation followed by heating the diazonium salt with \(\mathrm{HBF}_4\)
\item by direct fluorination of benzene with \(\mathrm{F}_2\) gas
\item by reacting \(\mathrm{PhBr}\) with \(\mathrm{NaF}\) solution
\end{enumerate}
\newpage
\section*{Question 45}
When alkyl halide is heated with dry \(\mathrm{Ag}_2 \mathrm{O}\), it produces ______.
\begin{enumerate}[label=(\alph*)]
\item ester
\item ether
\item ketone
\item alcohol
\end{enumerate}
\newpage
\section*{Question 46}
The process of converting alkyl halides to alcohols involve:
\begin{enumerate}[label=(\alph*)]
\item Addition reaction
\item Rearrangement reaction
\item Substitution reaction
\item Dehydrohalogenation reaction
\end{enumerate}
\newpage
\section*{Question 47}
Which of the following is an example of aryl alkyl halide?
\begin{enumerate}[label=(\alph*)]
\item P-chlorotoluene
\item Chlorobenzene
\item Allyl chloride
\item Benzyl chloride
\end{enumerate}
\newpage
\section*{Question 48}
 What is the IUPAC name of the following compound?<a href="https://www.sanfoundry.com/wp-content/uploads/2020/02/chemistry-questions-answers-nomenclature-q2.png">\includegraphics[width=\textwidth]{https://www.sanfoundry.com/wp-content/uploads/2020/02/chemistry-questions-answers-nomenclature-q2.png}</a>
\begin{enumerate}[label=(\alph*)]
\item 1-Bromo-3-methylprop-2-ene
\item 3-Bromo-1-methylpropene
\item 1-Bromobut-2-ene
\item 4-Bromobut-2-ene
\end{enumerate}
\newpage
\section*{Question 49}
In the common naming system, the prefix sym- is used for haloarenes with _____ halogen atoms.
\begin{enumerate}[label=(\alph*)]
\item 1
\item 2
\item 3
\item 4
\end{enumerate}
\newpage
\section*{Question 50}
How many carbon atoms does Isobutyl chloride have in its parent carbon chain?
\begin{enumerate}[label=(\alph*)]
\item 2
\item 3
\item 4
\item 5
\end{enumerate}
\newpage
\section*{Question 51}
Addition of \(\mathrm{K}\mathrm{I}\) accelerates the hydrolysis of primary alkyl halides because:
\begin{enumerate}[label=(\alph*)]
\item \(\mathrm{K}\mathrm{I}\) is soluble in organic solvents
\item The iodide ion is a weak base and a poor leaving group
\item The iodide ion is a strong base
\item The iodide ion is a powerful nucleophile as well as a good leaving group
\end{enumerate}
\newpage
\section*{Question 52}
In the following pair of compounds, which of the following relation is correct for nucleophilicity in a polar-protic solvent?
\begin{enumerate}[label=(\alph*)]
\item \(CH _3 SCH _3> CH _3 OCH _3\)
\item \(CH _3 SCH _3< CH _3 OCH _3\)
\item \(CH _3 SCH _3= CH _3 OCH _3\)
\item None of these
\end{enumerate}
\newpage
\section*{Question 53}
The correct order of reactivity of following alkyl halides for \(S_N 1\) reaction is:
\begin{enumerate}[label=(\alph*)]
\item \(R - I > R - Br > R - Cl > R - F\)
\item \(R - F > R - Cl > R - Br > R - I\)
\item \(R - F > R - Br > R - Cl > R - I\)
\item \(R - Cl > R - I > R - Br > R - F\)
\end{enumerate}
\newpage
\section*{Question 54}
A mono haloarene is an example of __________.
\begin{enumerate}[label=(\alph*)]
\item aliphatic halogen compound
\item side-chain substituted aryl halide
\item alkyl halide
\item aromatic halogen compound
\end{enumerate}
\newpage
\section*{Question 55}
What do you get by heating a mixture of hexanol and concentrated aqueous hydrogen chloride?
\begin{enumerate}[label=(\alph*)]
\item Cyclochlorohexane
\item Chlorohexane
\item Phosphorus acid
\item No reaction
\end{enumerate}
\newpage
\section*{Question 56}
Which of the following haloalkanes is most reactive?\newline
\begin{enumerate}[label=(\alph*)]
\item 1-chloropropane \newline
\item 1-bromopropane
\item 2-chloropropane
\item 2-bromopropane
\end{enumerate}
\newpage
\section*{Question 57}
The reaction of benzene with chlorine in the presence of iron gives:
\begin{enumerate}[label=(\alph*)]
\item Benzene hexachloride
\item Chlorobenzene\newline
\item Benzyl chloride
\item Benzoyl chloride
\end{enumerate}
\newpage
\section*{Question 58}
\(\mathbf{C H}_{3}-\mathbf{I}+2 \mathbf{N} \mathbf{a}+\mathbf{I}-\mathbf{C H}_{3} \stackrel{\text { dry ether }}{\longrightarrow}\) ____________.
\begin{enumerate}[label=(\alph*)]
\item \(\mathrm{C}_{2} \mathrm{H}_{5}-\mathrm{C}_{2} \mathrm{H}_{5}\)
\item \(\mathrm{CH}_{4}\)
\item \(\mathrm{C}_{3} \mathrm{H}_{8}\)
\item \(\mathrm{CH}_{3}-\mathrm{CH}_{3}\)
\end{enumerate}
\newpage
\section*{Question 59}
Which of the following has the highest boiling point?
\begin{enumerate}[label=(\alph*)]
\item \(\mathrm{C}_{3} \mathrm{H}_{7} \mathrm{Cl}\)
\item \(\mathrm{C}_{4} \mathrm{H}_{9} \mathrm{Cl}\)
\item \(\mathrm{CH}_{3} \mathrm{CH}\left(\mathrm{CH}_{3}\right) \mathrm{CH}_{2} \mathrm{Cl}\)
\item \(\left(\mathrm{CH}_{3}\right)_{3} \mathrm{C}-\mathrm{Cl}\)
\end{enumerate}
\newpage
\section*{Question 60}
In isomeric alkyl halides:
\begin{enumerate}[label=(\alph*)]
\item The branch chain isomer have relatively high boiling point as compared to its straight chain isomer
\item The branch chain isomer have relatively low boiling point as compared to its straight chain isomer
\item The branch chain isomer boiling point equal to its straight chain isomer
\item None of these
\end{enumerate}
\newpage
\end{document}