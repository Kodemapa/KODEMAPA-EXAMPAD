\documentclass{article}
                    \usepackage{amsmath}
                    \usepackage{amssymb}
                    \usepackage{graphicx}
                    \usepackage{enumitem}
                    \usepackage{longtable}
                    \usepackage{array}
                    \usepackage{booktabs}
                    \title{check2}
                    \begin{document}
                    \maketitle
                    \section*{Question 1}
what is A in the following reaction ?



\includegraphics{https://kodemapa.com/static/media/wl_client/1/qdump/dd962b43da3e663bef2c213d7dbe3f88/236e4ffa2b286d5be56b69d34825b864.png}\\


\begin{enumerate}[label=(\alph*)]
\item \includegraphics{https://kodemapa.com/static/media/wl_client/1/qdump/dd962b43da3e663bef2c213d7dbe3f88/7e913335697edc435c318cd8f4c74c35.png}


\item \includegraphics{https://kodemapa.com/static/media/wl_client/1/qdump/dd962b43da3e663bef2c213d7dbe3f88/af241ff17a69316d69e677cbe7bee714.png}


\item \includegraphics{https://kodemapa.com/static/media/wl_client/1/qdump/dd962b43da3e663bef2c213d7dbe3f88/96390fc856177859ff1bc0820852c540.png}


\item \includegraphics{https://kodemapa.com/static/media/wl_client/1/qdump/dd962b43da3e663bef2c213d7dbe3f88/502f4bfc33a7ea5f5446b31091aa8c80.png}


\end{enumerate}
\newpage
\section*{Question 2}
An organic compound "A" on treatment with benzene sulphonyl chloride

gives compound \textbackslash(B . B\textbackslash) is soluble in dil.

\textbackslash(\textbackslash mathrm\{NaOH\}\textbackslash) solution.~



Compound \textbackslash(A\textbackslash) is\_\_\_\_\_\_\_\_\_\_\_\_\_\_.


\begin{enumerate}[label=(\alph*)]
\item \textbackslash(\textbackslash mathrm\{C\}\_6

\textbackslash mathrm\{H\}\_5-\textbackslash mathrm\{N\}-\textbackslash left(\textbackslash mathrm\{CH\}\_3\textbackslash right)\_2\textbackslash)


\item \textbackslash(\textbackslash mathrm\{C\}\_6

\textbackslash mathrm\{H\}\_5-\textbackslash mathrm\{NHCH\}\_2

\textbackslash mathrm\{CH\}\_3\textbackslash)


\item \textbackslash(\textbackslash mathrm\{C\}\_6

\textbackslash mathrm\{H\}\_5-\textbackslash mathrm\{CH\}\_2

\textbackslash mathrm\{NHCH\}\_3\textbackslash)


\item \textbackslash(\textbackslash mathrm\{C\}\_6

\textbackslash mathrm\{H\}\_5-\textbackslash mathrm\{CH\}-\textbackslash mathrm\{NH\}\_2\textbackslash)


\end{enumerate}
\newpage
\section*{Question 3}
The total number of reagents from those given below, that can convert

nitrobenzene into aniline is \_\_\_\_\_\_\_\_\_\_\_\_\_\_\_ (Integer

answer)



\begin{longtable}[]{@{}ll@{}}

\toprule\noalign{}

\endhead

\bottomrule\noalign{}

\endlastfoot

\textbackslash(I .

\textbackslash mathrm\{Sn\}-\textbackslash mathrm\{HCI\}\textbackslash)~

& \textbackslash(\textbackslash mathrm\{II\} \textbackslash cdot

\textbackslash mathrm\{Sn\}-\textbackslash mathrm\{NH\}\_4

\textbackslash mathrm\{OH\}\textbackslash)~ \\

\textbackslash(I I I \textbackslash cdot

\textbackslash mathrm\{Fe\}-\textbackslash mathrm\{HCl\}\textbackslash)~

& \textbackslash(I V \textbackslash cdot

\textbackslash mathrm\{Zn\}-\textbackslash mathrm\{HCI\}\textbackslash)~ \\

\textbackslash(V \textbackslash cdot

\textbackslash mathrm\{H\}\_2-\textbackslash mathrm\{Pd\}\textbackslash)~

& ~\textbackslash(V I \textbackslash cdot

\textbackslash mathrm\{H\}\_2-\textbackslash) Raney nickel \\

\end{longtable}


\begin{enumerate}[label=(\alph*)]
\end{enumerate}
\newpage
\section*{Question 4}
Match List I with List II.



\begin{longtable}[]{@{}ll@{}}

\toprule\noalign{}

\endhead

\bottomrule\noalign{}

\endlastfoot

List-I~~ & ~List-II~ \\

A. Benzenesulphonyl Chloride & ~I. Test for primary amines~ \\

B. Hoffmann bromamide reaction~~ & II. Anti Saytzeff~~ \\

C. Carbylamine reaction~ & III. Hinsberg reagent~ \\

D. Hoffmann orientation~ & IV. Known reaction of Isocyanates~ \\

\end{longtable}



Choose the correct answer from the options given below:\\


\begin{enumerate}[label=(\alph*)]
\item A-IV, B-III, C-II, D-I


\item A-IV, B-II, C-I, D-II


\item A-III, B-IV, C-I, D-II


\item A-IV, B-III, C-I, D-II


\end{enumerate}
\newpage
\section*{Question 5}
Primary, secondary and tertiary amines can be separated using.


\begin{enumerate}[label=(\alph*)]
\item para-toluene sulphonyl chloride


\item chloroform and

\textbackslash(\textbackslash mathrm\{KOH\}\textbackslash)


\item benzene sulphonic acid


\item acetyl amide


\end{enumerate}
\newpage
\section*{Question 6}
Compound \textbackslash(\textbackslash mathrm\{A\}\textbackslash) is

converted to \textbackslash(\textbackslash mathrm\{B\}\textbackslash) on

reaction with

\textbackslash(\textbackslash mathrm\{CHCl\}\_3\textbackslash) and

\textbackslash(\textbackslash mathrm\{KOH\}\textbackslash). The compound

\textbackslash(\textbackslash mathrm\{B\}\textbackslash) is toxic and

can be decomposed by C. A, B and \textbackslash(C\textbackslash)

respectively are :


\begin{enumerate}[label=(\alph*)]
\item primary amine, nitrile compound, conc.

\textbackslash(\textbackslash mathrm\{HCl\}\textbackslash)


\item secondary amine, isonitrile compound, conc.

\textbackslash(\textbackslash mathrm\{NaOH\}\textbackslash)


\item primary amine, isonitrile compound, conc.

\textbackslash(\textbackslash mathrm\{HCl\}\textbackslash)


\item secondary amine, nitrile compound, conc.

\textbackslash(\textbackslash mathrm\{NaOH\}\textbackslash)


\end{enumerate}
\newpage
\section*{Question 7}
The major product formed in the following reaction is.



\includegraphics{https://kodemapa.com/static/media/wl_client/1/qdump/dd962b43da3e663bef2c213d7dbe3f88/cd501f5bf6a7ba9867212e69f5da1961.png}\\


\begin{enumerate}[label=(\alph*)]
\item \includegraphics{https://kodemapa.com/static/media/wl_client/1/qdump/dd962b43da3e663bef2c213d7dbe3f88/ea66b26d652be69c42f279d0ca208c29.png}


\item \includegraphics{https://kodemapa.com/static/media/wl_client/1/qdump/dd962b43da3e663bef2c213d7dbe3f88/2952e31a186e1acc29cc186fdfb5dd6b.png}


\item \includegraphics{https://kodemapa.com/static/media/wl_client/1/qdump/dd962b43da3e663bef2c213d7dbe3f88/d4fa77cbadb82c63c120985ab1e82614.png}


\item \includegraphics{https://kodemapa.com/static/media/wl_client/1/qdump/dd962b43da3e663bef2c213d7dbe3f88/59084e148fc2d739417759e72917593f.png}


\end{enumerate}
\newpage
\section*{Question 8}
Choose the correct colour of the product for the following reaction.



\includegraphics{https://kodemapa.com/static/media/wl_client/1/qdump/dd962b43da3e663bef2c213d7dbe3f88/40efa76b9a92d9d9f7f9a8bc33bd53c0.png}\\


\begin{enumerate}[label=(\alph*)]
\item Yellow


\item White


\item Red


\item Blue


\end{enumerate}
\newpage
\section*{Question 9}
The correct order in aqueous medium of basic strength in case of methyl

substituted amines is :


\begin{enumerate}[label=(\alph*)]
\item \textbackslash(\textbackslash mathrm\{Me\}\_2

\textbackslash mathrm\{NH\}\textgreater\textbackslash mathrm\{MeNH\}\_2\textgreater\textbackslash mathrm\{Me\}\_3

\textbackslash mathrm\{\textasciitilde N\}\textgreater\textbackslash mathrm\{NH\}\_3\textbackslash)


\item \textbackslash(\textbackslash mathrm\{Me\}\_2

\textbackslash mathrm\{NH\}\textgreater\textbackslash mathrm\{Me\}\_3

\textbackslash mathrm\{\textasciitilde N\}\textgreater\textbackslash mathrm\{MeNH\}\_2\textgreater\textbackslash mathrm\{NH\}\_3\textbackslash)


\item \textbackslash(\textbackslash mathrm\{NH\}\_3\textgreater\textbackslash mathrm\{Me\}\_3

\textbackslash mathrm\{\textasciitilde N\}\textgreater\textbackslash mathrm\{MeNH\}\_2\textgreater\textbackslash mathrm\{Me\}\_2

\textbackslash mathrm\{NH\}\textbackslash)


\item \textbackslash(\textbackslash mathrm\{Me\}\_3

\textbackslash mathrm\{\textasciitilde N\}\textgreater\textbackslash mathrm\{Me\}\_2

\textbackslash mathrm\{NH\}\textgreater\textbackslash mathrm\{MeNH\}\_2\textgreater\textbackslash mathrm\{NH\}\_3\textbackslash)


\end{enumerate}
\newpage
\section*{Question 10}
Number of isomeric aromatic amines with molecular formula

\textbackslash(\textbackslash mathrm\{C\}\_8

\textbackslash mathrm\{H\}\_\{11\}

\textbackslash mathrm\{\textasciitilde N\}\textbackslash), which can be

synthesized by Gabriel Phthalimide synthesis is \_\_\_\_\_\_\_\_\_\_\_.~

{[}6-Apr-2023{]}


\begin{enumerate}[label=(\alph*)]
\end{enumerate}
\newpage
\section*{Question 11}
Given below are two statements :



Statement I : In Hofmann degradation reaction, the migration of only an

alkyl group takes place from carbonyl carbon of the amide to the

nitrogen atom.



Statement II : The group is migrated in Hofmann degradation reaction to

electron deficient atom.



In the light of the above statements, choose the most appropriate answer

from the options given below:


\begin{enumerate}[label=(\alph*)]
\item Both Statement I and Statement II are correct.


\item Both Statement I and Statement II are incorrect.


\item Statement I is correct but Statement II is incorrect.


\item Statement I is incorrect but Statement II is correct.


\end{enumerate}
\newpage
\section*{Question 12}
Given below are two statements :



Statement I : Aniline is less basic than acetamide.



Statement II : In aniline, the lone pair of electrons on nitrogen atom

is delocalised over benzene ring due to resonance and hence less

available to a proton.



Choose the most appropriate option ;


\begin{enumerate}[label=(\alph*)]
\item Statement I is true but statement II is false.


\item Statement I is false but statement II is true.


\item Both statement I and statement II are true.


\item Both statement I and statement II are false.


\end{enumerate}
\newpage
\section*{Question 13}
The total number of electrons around the nitrogen atom in amines are,


\begin{enumerate}[label=(\alph*)]
\end{enumerate}
\newpage
\section*{Question 14}
\includegraphics{https://kodemapa.com/static/media/wl_client/1/qdump/dd962b43da3e663bef2c213d7dbe3f88/4ec04ea353b6f65a5cf7b06ed0289d5e.png}\\



Consider the given reaction, percentage yield of,~


\begin{enumerate}[label=(\alph*)]
\item \textbackslash((C)\textgreater(A)\textgreater(B)\textbackslash)


\item \textbackslash((B)\textgreater(C)\textgreater(A)\textbackslash)


\item \textbackslash((A)\textgreater(C)\textgreater(B)\textbackslash)


\item \textbackslash((C)\textgreater(B)\textgreater(A)\textbackslash)


\end{enumerate}
\newpage
\section*{Question 15}
The product A formed in the following reaction is:



\includegraphics{https://kodemapa.com/static/media/wl_client/1/qdump/dd962b43da3e663bef2c213d7dbe3f88/1520d487acea3e12471d7be5f928b59f.png}\\



{}\strut \\


\begin{enumerate}[label=(\alph*)]
\item \includegraphics{https://kodemapa.com/static/media/wl_client/1/qdump/dd962b43da3e663bef2c213d7dbe3f88/2ca5365cc0784172da4f25597d9ad12a.png}


\item \includegraphics{https://kodemapa.com/static/media/wl_client/1/qdump/dd962b43da3e663bef2c213d7dbe3f88/9fbdb1561f5c01c9175d265eb6391c64.png}


\item \includegraphics{https://kodemapa.com/static/media/wl_client/1/qdump/dd962b43da3e663bef2c213d7dbe3f88/3dc05a6f09a27296e56e9f0ecfac5216.png}


\item \includegraphics{https://kodemapa.com/static/media/wl_client/1/qdump/dd962b43da3e663bef2c213d7dbe3f88/567f0d4eba6072aab1e7657d9bd4afdd.png}


\end{enumerate}
\newpage
\section*{Question 16}
The products A and B formed in the following reaction scheme are

respectively



\includegraphics{https://kodemapa.com/static/media/wl_client/1/qdump/dd962b43da3e663bef2c213d7dbe3f88/4d6bf3841a1af092d1876901f8f5ca11.png}\\



{}\strut \\


\begin{enumerate}[label=(\alph*)]
\item \includegraphics{https://kodemapa.com/static/media/wl_client/1/qdump/dd962b43da3e663bef2c213d7dbe3f88/2a3246b15ff319d98764dd66e03e52ca.png}


\item \includegraphics{https://kodemapa.com/static/media/wl_client/1/qdump/dd962b43da3e663bef2c213d7dbe3f88/eee70e6d6582dc36823640835c13ac70.png}


\item \includegraphics{https://kodemapa.com/static/media/wl_client/1/qdump/dd962b43da3e663bef2c213d7dbe3f88/5c0894ccccc6ea4d63ce98ae8ae80869.png}


\item \includegraphics{https://kodemapa.com/static/media/wl_client/1/qdump/dd962b43da3e663bef2c213d7dbe3f88/347277ab3c9f0b5b788ab0c182916da4.png}


\end{enumerate}
\newpage
\section*{Question 17}
Number of compounds which give reaction with Hinsberg\textquotesingle s

reagent is \_\_\_\_\_\_\_.



\includegraphics{https://kodemapa.com/static/media/wl_client/1/qdump/dd962b43da3e663bef2c213d7dbe3f88/cd8470ec36b43456e313180dcf6fbeba.png}\\



\hfill\break


\begin{enumerate}[label=(\alph*)]
\end{enumerate}
\newpage
\section*{Question 18}
Given below are two statements :



Statement I: Aniline reacts with con.

\textbackslash(\textbackslash mathrm\{H\}\_2

\textbackslash mathrm\{SO\}\_4\textbackslash) followed by heating at

\textbackslash(453-473

\textbackslash mathrm\{\textasciitilde K\}\textbackslash) gives

p-aminobenzene~sulphonic acid, which gives blood red colour in the

\textquotesingle Lassaigne\textquotesingle s test\textquotesingle.\\

{Statement II: In Friedel - Craft\textquotesingle s alkylation and

acylation reactions, aniline forms salt with the

\textbackslash(\textbackslash mathrm\{AlCl\}\_3\textbackslash)

catalyst.\\

}{Due to this, nitrogen of aniline aquires a positive charge and acts as

deactivating group.\\

}{In the light of the above statements, choose the correct answer from

the options given below :}


\begin{enumerate}[label=(\alph*)]
\item Statement I is false but statement II is true


\item Both statement I and statement II are false


\item Statement I is true but statement II is false


\item Both statement I and statement II are true


\end{enumerate}
\newpage
\section*{Question 19}
Which one of the products of the following reactions does not react with

Hinsberg reagent to form sulphonamide? {[}25 Jul 2021{]}


\begin{enumerate}[label=(\alph*)]
\item \includegraphics{https://kodemapa.com/static/media/wl_client/1/qdump/dd962b43da3e663bef2c213d7dbe3f88/2c25b5302176976e694d20326b655cbb.png}


\item \includegraphics{https://kodemapa.com/static/media/wl_client/1/qdump/dd962b43da3e663bef2c213d7dbe3f88/77e8483bbf8a5f91aca9e2e40598b611.png}


\item \includegraphics{https://kodemapa.com/static/media/wl_client/1/qdump/dd962b43da3e663bef2c213d7dbe3f88/335c787e1add6d2886cf53b8fe0870d7.png}


\item \includegraphics{https://kodemapa.com/static/media/wl_client/1/qdump/dd962b43da3e663bef2c213d7dbe3f88/07ad49568681c4e55c2b1bfd60d5b384.png}


\end{enumerate}
\newpage
\section*{Question 20}
The total number of reagents from those given below, that can convert

nitrobenzene into aniline is \_\_\_\_\_\_\_\_\_\_\_\_\_\_\_ (Integer

answer)



\begin{longtable}[]{@{}ll@{}}

\toprule\noalign{}

\endhead

\bottomrule\noalign{}

\endlastfoot

\textbackslash(I .

\textbackslash mathrm\{Sn\}-\textbackslash mathrm\{HCI\}\textbackslash)~

& \textbackslash(\textbackslash mathrm\{II\} \textbackslash cdot

\textbackslash mathrm\{Sn\}-\textbackslash mathrm\{NH\}\_4

\textbackslash mathrm\{OH\}\textbackslash)~ \\

\textbackslash(I I I \textbackslash cdot

\textbackslash mathrm\{Fe\}-\textbackslash mathrm\{HCl\}\textbackslash)~

& \textbackslash(I V \textbackslash cdot

\textbackslash mathrm\{Zn\}-\textbackslash mathrm\{HCI\}\textbackslash)~ \\

\textbackslash(V \textbackslash cdot

\textbackslash mathrm\{H\}\_2-\textbackslash mathrm\{Pd\}\textbackslash)~

& ~\textbackslash(V I \textbackslash cdot

\textbackslash mathrm\{H\}\_2-\textbackslash) Raney nickel \\

\end{longtable}


\begin{enumerate}[label=(\alph*)]
\end{enumerate}
\newpage
\section*{Question 21}
Match List I with List II.



\begin{longtable}[]{@{}ll@{}}

\toprule\noalign{}

\endhead

\bottomrule\noalign{}

\endlastfoot

List-I~~ & ~List-II~ \\

A. Benzenesulphonyl Chloride & ~I. Test for primary amines~ \\

B. Hoffmann bromamide reaction~~ & II. Anti Saytzeff~~ \\

C. Carbylamine reaction~ & III. Hinsberg reagent~ \\

D. Hoffmann orientation~ & IV. Known reaction of Isocyanates~ \\

\end{longtable}



Choose the correct answer from the options given below:\\


\begin{enumerate}[label=(\alph*)]
\item A-IV, B-III, C-II, D-I


\item A-IV, B-II, C-I, D-II


\item A-III, B-IV, C-I, D-II


\item A-IV, B-III, C-I, D-II


\end{enumerate}
\newpage
\section*{Question 22}
Compound \textbackslash(\textbackslash mathrm\{A\}\textbackslash) is

converted to \textbackslash(\textbackslash mathrm\{B\}\textbackslash) on

reaction with

\textbackslash(\textbackslash mathrm\{CHCl\}\_3\textbackslash) and

\textbackslash(\textbackslash mathrm\{KOH\}\textbackslash). The compound

\textbackslash(\textbackslash mathrm\{B\}\textbackslash) is toxic and

can be decomposed by C. A, B and \textbackslash(C\textbackslash)

respectively are :


\begin{enumerate}[label=(\alph*)]
\item primary amine, nitrile compound, conc.

\textbackslash(\textbackslash mathrm\{HCl\}\textbackslash)


\item secondary amine, isonitrile compound, conc.

\textbackslash(\textbackslash mathrm\{NaOH\}\textbackslash)


\item primary amine, isonitrile compound, conc.

\textbackslash(\textbackslash mathrm\{HCl\}\textbackslash)


\item secondary amine, nitrile compound, conc.

\textbackslash(\textbackslash mathrm\{NaOH\}\textbackslash)


\end{enumerate}
\newpage
\section*{Question 23}
The major product formed in the following reaction is.



\includegraphics{https://kodemapa.com/static/media/wl_client/1/qdump/dd962b43da3e663bef2c213d7dbe3f88/cd501f5bf6a7ba9867212e69f5da1961.png}\\


\begin{enumerate}[label=(\alph*)]
\item \includegraphics{https://kodemapa.com/static/media/wl_client/1/qdump/dd962b43da3e663bef2c213d7dbe3f88/ea66b26d652be69c42f279d0ca208c29.png}


\item \includegraphics{https://kodemapa.com/static/media/wl_client/1/qdump/dd962b43da3e663bef2c213d7dbe3f88/2952e31a186e1acc29cc186fdfb5dd6b.png}


\item \includegraphics{https://kodemapa.com/static/media/wl_client/1/qdump/dd962b43da3e663bef2c213d7dbe3f88/d4fa77cbadb82c63c120985ab1e82614.png}


\item \includegraphics{https://kodemapa.com/static/media/wl_client/1/qdump/dd962b43da3e663bef2c213d7dbe3f88/59084e148fc2d739417759e72917593f.png}


\end{enumerate}
\newpage
\section*{Question 24}
Consider the following sequence of reaction :



\includegraphics{https://kodemapa.com/static/media/wl_client/1/qdump/dd962b43da3e663bef2c213d7dbe3f88/57dcd9d8a769c8fea41cff9d8e01d464.png}\\



The product \textquotesingle{}

\textbackslash(\textbackslash mathrm\{B\}\textbackslash)

\textquotesingle{} is :


\begin{enumerate}[label=(\alph*)]
\item \includegraphics{https://kodemapa.com/static/media/wl_client/1/qdump/dd962b43da3e663bef2c213d7dbe3f88/424f41a0563c11457d451a9c8548f457.png}


\item \includegraphics{https://kodemapa.com/static/media/wl_client/1/qdump/dd962b43da3e663bef2c213d7dbe3f88/d4b4b4ca1568a3c5ecaffa3c96777568.png}


\item \includegraphics{https://kodemapa.com/static/media/wl_client/1/qdump/dd962b43da3e663bef2c213d7dbe3f88/514d98122f31a4308660c7cf35c5c48c.png}


\item \includegraphics{https://kodemapa.com/static/media/wl_client/1/qdump/dd962b43da3e663bef2c213d7dbe3f88/b8f6fcf1f36be06b6950c90da3323688.png}


\end{enumerate}
\newpage
\section*{Question 25}
A compound with molecular mass 180 is acylated with

\textbackslash(\textbackslash mathrm\{CH\}\_3

\textbackslash mathrm\{COCl\}\textbackslash) to get a compound with

molecular mass 390 . The number of amino groups present per molecule of

the former compound is:


\begin{enumerate}[label=(\alph*)]
\end{enumerate}
\newpage
\section*{Question 26}
The decreasing order of basicity of the following amines is:



\includegraphics{https://kodemapa.com/static/media/wl_client/1/qdump/dd962b43da3e663bef2c213d7dbe3f88/2752f49d87bf0fdd936512818343839f.png}\\


\begin{enumerate}[label=(\alph*)]
\item \textbackslash((A)\textgreater(C)\textgreater(D)\textgreater(B)\textbackslash)


\item \textbackslash((C)\textgreater(A)\textgreater(B)\textgreater(D)\textbackslash)


\item \textbackslash((B)\textgreater(C)\textgreater(D)\textgreater(A)\textbackslash)


\item \textbackslash((C)\textgreater(B)\textgreater(A)\textgreater(D)\textbackslash)


\end{enumerate}
\newpage
\section*{Question 27}
\includegraphics{https://kodemapa.com/static/media/wl_client/1/qdump/dd962b43da3e663bef2c213d7dbe3f88/e017bcaa9a7e8b73dcdedce8d8d5d33d.png}\\



In the chemical reactions given above \textbackslash(A\textbackslash)

and \textbackslash(B\textbackslash) respectively are:


\begin{enumerate}[label=(\alph*)]
\item \textbackslash(\textbackslash mathrm\{H\}\_3

\textbackslash mathrm\{PO\}\_2\textbackslash) and

\textbackslash(\textbackslash mathrm\{CH\}\_3

\textbackslash mathrm\{CH\}\_2

\textbackslash mathrm\{Cl\}\textbackslash)


\item \textbackslash(\textbackslash mathrm\{CH\}\_3

\textbackslash mathrm\{CH\}\_2

\textbackslash mathrm\{OH\}\textbackslash) and

\textbackslash(\textbackslash mathrm\{H\}\_3

\textbackslash mathrm\{PO\}\_2\textbackslash)


\item \textbackslash(\textbackslash mathrm\{H\}\_3

\textbackslash mathrm\{O\}\_2\textbackslash) and

\textbackslash(\textbackslash mathrm\{CH\}\_3

\textbackslash mathrm\{CH\}\_2

\textbackslash mathrm\{OH\}\textbackslash)


\item \textbackslash(\textbackslash mathrm\{CH\}\_3

\textbackslash mathrm\{CH\}\_2

\textbackslash mathrm\{Cl\}\textbackslash) and

\textbackslash(\textbackslash mathrm\{H\}\_3

\textbackslash mathrm\{PO\}\_2\textbackslash)


\end{enumerate}
\newpage
\section*{Question 28}
The total number of electrons around the nitrogen atom in amines are,


\begin{enumerate}[label=(\alph*)]
\end{enumerate}
\newpage
\section*{Question 29}
The number of primary amines of formula

\textbackslash(\textbackslash mathrm\{C\}\_4

\textbackslash mathrm\{H\}\_\{11\}

\textbackslash mathrm\{\textasciitilde N\}\textbackslash) is ?


\begin{enumerate}[label=(\alph*)]
\end{enumerate}
\newpage
\section*{Question 30}
The products A and B formed in the following reaction scheme are

respectively



\includegraphics{https://kodemapa.com/static/media/wl_client/1/qdump/dd962b43da3e663bef2c213d7dbe3f88/4d6bf3841a1af092d1876901f8f5ca11.png}\\



{}\strut \\


\begin{enumerate}[label=(\alph*)]
\item \includegraphics{https://kodemapa.com/static/media/wl_client/1/qdump/dd962b43da3e663bef2c213d7dbe3f88/2a3246b15ff319d98764dd66e03e52ca.png}


\item \includegraphics{https://kodemapa.com/static/media/wl_client/1/qdump/dd962b43da3e663bef2c213d7dbe3f88/eee70e6d6582dc36823640835c13ac70.png}


\item \includegraphics{https://kodemapa.com/static/media/wl_client/1/qdump/dd962b43da3e663bef2c213d7dbe3f88/5c0894ccccc6ea4d63ce98ae8ae80869.png}


\item \includegraphics{https://kodemapa.com/static/media/wl_client/1/qdump/dd962b43da3e663bef2c213d7dbe3f88/347277ab3c9f0b5b788ab0c182916da4.png}


\end{enumerate}
\newpage
\section*{Question 31}
Number of compounds which give reaction with Hinsberg\textquotesingle s

reagent is \_\_\_\_\_\_\_.



\includegraphics{https://kodemapa.com/static/media/wl_client/1/qdump/dd962b43da3e663bef2c213d7dbe3f88/cd8470ec36b43456e313180dcf6fbeba.png}\\



\hfill\break


\begin{enumerate}[label=(\alph*)]
\end{enumerate}
\newpage
\section*{Question 32}
Given below are two statements :



Statement I: Aniline reacts with con.

\textbackslash(\textbackslash mathrm\{H\}\_2

\textbackslash mathrm\{SO\}\_4\textbackslash) followed by heating at

\textbackslash(453-473

\textbackslash mathrm\{\textasciitilde K\}\textbackslash) gives

p-aminobenzene~sulphonic acid, which gives blood red colour in the

\textquotesingle Lassaigne\textquotesingle s test\textquotesingle.\\

{Statement II: In Friedel - Craft\textquotesingle s alkylation and

acylation reactions, aniline forms salt with the

\textbackslash(\textbackslash mathrm\{AlCl\}\_3\textbackslash)

catalyst.\\

}{Due to this, nitrogen of aniline aquires a positive charge and acts as

deactivating group.\\

}{In the light of the above statements, choose the correct answer from

the options given below :}


\begin{enumerate}[label=(\alph*)]
\item Statement I is false but statement II is true


\item Both statement I and statement II are false


\item Statement I is true but statement II is false


\item Both statement I and statement II are true


\end{enumerate}
\newpage
\section*{Question 33}
The number of nitrogen atoms in a semicarbazone molecule of acetone

is\_\_\_\_\_\_\_\_\_\_\_\_\_.


\begin{enumerate}[label=(\alph*)]
\end{enumerate}
\newpage
\section*{Question 34}
Match List I with List II.



\begin{longtable}[]{@{}ll@{}}

\toprule\noalign{}

\endhead

\bottomrule\noalign{}

\endlastfoot

List-I~~ & ~List-II~ \\

A. Benzenesulphonyl Chloride & ~I. Test for primary amines~ \\

B. Hoffmann bromamide reaction~~ & II. Anti Saytzeff~~ \\

C. Carbylamine reaction~ & III. Hinsberg reagent~ \\

D. Hoffmann orientation~ & IV. Known reaction of Isocyanates~ \\

\end{longtable}



Choose the correct answer from the options given below:\\


\begin{enumerate}[label=(\alph*)]
\item A-IV, B-III, C-II, D-I


\item A-IV, B-II, C-I, D-II


\item A-III, B-IV, C-I, D-II


\item A-IV, B-III, C-I, D-II


\end{enumerate}
\newpage
\section*{Question 35}
The major product formed in the following reaction is.



\includegraphics{https://kodemapa.com/static/media/wl_client/1/qdump/dd962b43da3e663bef2c213d7dbe3f88/cd501f5bf6a7ba9867212e69f5da1961.png}\\


\begin{enumerate}[label=(\alph*)]
\item \includegraphics{https://kodemapa.com/static/media/wl_client/1/qdump/dd962b43da3e663bef2c213d7dbe3f88/ea66b26d652be69c42f279d0ca208c29.png}


\item \includegraphics{https://kodemapa.com/static/media/wl_client/1/qdump/dd962b43da3e663bef2c213d7dbe3f88/2952e31a186e1acc29cc186fdfb5dd6b.png}


\item \includegraphics{https://kodemapa.com/static/media/wl_client/1/qdump/dd962b43da3e663bef2c213d7dbe3f88/d4fa77cbadb82c63c120985ab1e82614.png}


\item \includegraphics{https://kodemapa.com/static/media/wl_client/1/qdump/dd962b43da3e663bef2c213d7dbe3f88/59084e148fc2d739417759e72917593f.png}


\end{enumerate}
\newpage
\section*{Question 36}
Choose the correct colour of the product for the following reaction.



\includegraphics{https://kodemapa.com/static/media/wl_client/1/qdump/dd962b43da3e663bef2c213d7dbe3f88/40efa76b9a92d9d9f7f9a8bc33bd53c0.png}\\


\begin{enumerate}[label=(\alph*)]
\item Yellow


\item White


\item Red


\item Blue


\end{enumerate}
\newpage
\section*{Question 37}
Number of isomeric aromatic amines with molecular formula

\textbackslash(\textbackslash mathrm\{C\}\_8

\textbackslash mathrm\{H\}\_\{11\}

\textbackslash mathrm\{\textasciitilde N\}\textbackslash), which can be

synthesized by Gabriel Phthalimide synthesis is \_\_\_\_\_\_\_\_\_\_\_.~

{[}6-Apr-2023{]}


\begin{enumerate}[label=(\alph*)]
\end{enumerate}
\newpage
\section*{Question 38}
During halogen test, sodium fusion extract is boiled with concentrated

\textbackslash(\textbackslash mathrm\{HNO\}\_3\textbackslash) to


\begin{enumerate}[label=(\alph*)]
\item remove unreacted sodium


\item decompose cyanide or sulphide of sodium


\item extract halogen from organic compound


\item maintain the \textbackslash(\textbackslash mathrm\{pH\}\textbackslash)

of extract.


\end{enumerate}
\newpage
\section*{Question 39}
A compound with molecular mass 180 is acylated with

\textbackslash(\textbackslash mathrm\{CH\}\_3

\textbackslash mathrm\{COCl\}\textbackslash) to get a compound with

molecular mass 390 . The number of amino groups present per molecule of

the former compound is:


\begin{enumerate}[label=(\alph*)]
\end{enumerate}
\newpage
\section*{Question 40}
The most appropriate reagent for conversion of

\textbackslash(\textbackslash mathrm\{C\}\_2

\textbackslash mathrm\{H\}\_5 \textbackslash mathrm\{CN\}\textbackslash)

into \textbackslash(\textbackslash mathrm\{CH\}\_3

\textbackslash mathrm\{CH\}\_2 \textbackslash mathrm\{CH\}\_2

\textbackslash mathrm\{NH\}\_2\textbackslash) is:


\begin{enumerate}[label=(\alph*)]
\item \textbackslash(\textbackslash mathrm\{NaBH\}\_4\textbackslash)


\item \textbackslash(\textbackslash mathrm\{CaH\}\_2\textbackslash)


\item LiAl H\textbackslash(\_4\textbackslash)


\item \textbackslash(\textbackslash mathrm\{Na\}(\textbackslash mathrm\{CN\})

\textbackslash mathrm\{BH\}\_3\textbackslash)


\end{enumerate}
\newpage
\section*{Question 41}
The decreasing order of basicity of the following amines is:



\includegraphics{https://kodemapa.com/static/media/wl_client/1/qdump/dd962b43da3e663bef2c213d7dbe3f88/2752f49d87bf0fdd936512818343839f.png}\\


\begin{enumerate}[label=(\alph*)]
\item \textbackslash((A)\textgreater(C)\textgreater(D)\textgreater(B)\textbackslash)


\item \textbackslash((C)\textgreater(A)\textgreater(B)\textgreater(D)\textbackslash)


\item \textbackslash((B)\textgreater(C)\textgreater(D)\textgreater(A)\textbackslash)


\item \textbackslash((C)\textgreater(B)\textgreater(A)\textgreater(D)\textbackslash)


\end{enumerate}
\newpage
\section*{Question 42}
\includegraphics{https://kodemapa.com/static/media/wl_client/1/qdump/dd962b43da3e663bef2c213d7dbe3f88/4ec04ea353b6f65a5cf7b06ed0289d5e.png}\\



Consider the given reaction, percentage yield of,~


\begin{enumerate}[label=(\alph*)]
\item \textbackslash((C)\textgreater(A)\textgreater(B)\textbackslash)


\item \textbackslash((B)\textgreater(C)\textgreater(A)\textbackslash)


\item \textbackslash((A)\textgreater(C)\textgreater(B)\textbackslash)


\item \textbackslash((C)\textgreater(B)\textgreater(A)\textbackslash)


\end{enumerate}
\newpage
\section*{Question 43}
The product A formed in the following reaction is:



\includegraphics{https://kodemapa.com/static/media/wl_client/1/qdump/dd962b43da3e663bef2c213d7dbe3f88/1520d487acea3e12471d7be5f928b59f.png}\\



{}\strut \\


\begin{enumerate}[label=(\alph*)]
\item \includegraphics{https://kodemapa.com/static/media/wl_client/1/qdump/dd962b43da3e663bef2c213d7dbe3f88/2ca5365cc0784172da4f25597d9ad12a.png}


\item \includegraphics{https://kodemapa.com/static/media/wl_client/1/qdump/dd962b43da3e663bef2c213d7dbe3f88/9fbdb1561f5c01c9175d265eb6391c64.png}


\item \includegraphics{https://kodemapa.com/static/media/wl_client/1/qdump/dd962b43da3e663bef2c213d7dbe3f88/3dc05a6f09a27296e56e9f0ecfac5216.png}


\item \includegraphics{https://kodemapa.com/static/media/wl_client/1/qdump/dd962b43da3e663bef2c213d7dbe3f88/567f0d4eba6072aab1e7657d9bd4afdd.png}


\end{enumerate}
\newpage
\section*{Question 44}
In the reaction of hypobromite with amide, the carbonyl carbon is lost

as


\begin{enumerate}[label=(\alph*)]
\item \textbackslash(\textbackslash mathrm\{CO\}\_3\{

\}\^{}\{2-\}\textbackslash)


\item \textbackslash(\textbackslash mathrm\{HCO\}\_3\{

\}\^{}\{-\}\textbackslash)


\item \textbackslash(\textbackslash mathrm\{CO\}\_2\textbackslash)


\item \textbackslash(\textbackslash mathrm\{CO\}\textbackslash)


\end{enumerate}
\newpage
\section*{Question 45}
The number of nitrogen atoms in a semicarbazone molecule of acetone

is\_\_\_\_\_\_\_\_\_\_\_\_\_.


\begin{enumerate}[label=(\alph*)]
\end{enumerate}
\newpage
\section*{Question 46}
The correct order in aqueous medium of basic strength in case of methyl

substituted amines is :


\begin{enumerate}[label=(\alph*)]
\item \textbackslash(\textbackslash mathrm\{Me\}\_2

\textbackslash mathrm\{NH\}\textgreater\textbackslash mathrm\{MeNH\}\_2\textgreater\textbackslash mathrm\{Me\}\_3

\textbackslash mathrm\{\textasciitilde N\}\textgreater\textbackslash mathrm\{NH\}\_3\textbackslash)


\item \textbackslash(\textbackslash mathrm\{Me\}\_2

\textbackslash mathrm\{NH\}\textgreater\textbackslash mathrm\{Me\}\_3

\textbackslash mathrm\{\textasciitilde N\}\textgreater\textbackslash mathrm\{MeNH\}\_2\textgreater\textbackslash mathrm\{NH\}\_3\textbackslash)


\item \textbackslash(\textbackslash mathrm\{NH\}\_3\textgreater\textbackslash mathrm\{Me\}\_3

\textbackslash mathrm\{\textasciitilde N\}\textgreater\textbackslash mathrm\{MeNH\}\_2\textgreater\textbackslash mathrm\{Me\}\_2

\textbackslash mathrm\{NH\}\textbackslash)


\item \textbackslash(\textbackslash mathrm\{Me\}\_3

\textbackslash mathrm\{\textasciitilde N\}\textgreater\textbackslash mathrm\{Me\}\_2

\textbackslash mathrm\{NH\}\textgreater\textbackslash mathrm\{MeNH\}\_2\textgreater\textbackslash mathrm\{NH\}\_3\textbackslash)


\end{enumerate}
\newpage
\section*{Question 47}
Number of isomeric aromatic amines with molecular formula

\textbackslash(\textbackslash mathrm\{C\}\_8

\textbackslash mathrm\{H\}\_\{11\}

\textbackslash mathrm\{\textasciitilde N\}\textbackslash), which can be

synthesized by Gabriel Phthalimide synthesis is \_\_\_\_\_\_\_\_\_\_\_.~

{[}6-Apr-2023{]}


\begin{enumerate}[label=(\alph*)]
\end{enumerate}
\newpage
\section*{Question 48}
During halogen test, sodium fusion extract is boiled with concentrated

\textbackslash(\textbackslash mathrm\{HNO\}\_3\textbackslash) to


\begin{enumerate}[label=(\alph*)]
\item remove unreacted sodium


\item decompose cyanide or sulphide of sodium


\item extract halogen from organic compound


\item maintain the \textbackslash(\textbackslash mathrm\{pH\}\textbackslash)

of extract.


\end{enumerate}
\newpage
\section*{Question 49}
Given below are two statements :



Statement I : Aniline is less basic than acetamide.



Statement II : In aniline, the lone pair of electrons on nitrogen atom

is delocalised over benzene ring due to resonance and hence less

available to a proton.



Choose the most appropriate option ;


\begin{enumerate}[label=(\alph*)]
\item Statement I is true but statement II is false.


\item Statement I is false but statement II is true.


\item Both statement I and statement II are true.


\item Both statement I and statement II are false.


\end{enumerate}
\newpage
\section*{Question 50}
The decreasing order of basicity of the following amines is:



\includegraphics{https://kodemapa.com/static/media/wl_client/1/qdump/dd962b43da3e663bef2c213d7dbe3f88/2752f49d87bf0fdd936512818343839f.png}\\


\begin{enumerate}[label=(\alph*)]
\item \textbackslash((A)\textgreater(C)\textgreater(D)\textgreater(B)\textbackslash)


\item \textbackslash((C)\textgreater(A)\textgreater(B)\textgreater(D)\textbackslash)


\item \textbackslash((B)\textgreater(C)\textgreater(D)\textgreater(A)\textbackslash)


\item \textbackslash((C)\textgreater(B)\textgreater(A)\textgreater(D)\textbackslash)


\end{enumerate}
\newpage
\section*{Question 51}
\includegraphics{https://kodemapa.com/static/media/wl_client/1/qdump/dd962b43da3e663bef2c213d7dbe3f88/e017bcaa9a7e8b73dcdedce8d8d5d33d.png}\\



In the chemical reactions given above \textbackslash(A\textbackslash)

and \textbackslash(B\textbackslash) respectively are:


\begin{enumerate}[label=(\alph*)]
\item \textbackslash(\textbackslash mathrm\{H\}\_3

\textbackslash mathrm\{PO\}\_2\textbackslash) and

\textbackslash(\textbackslash mathrm\{CH\}\_3

\textbackslash mathrm\{CH\}\_2

\textbackslash mathrm\{Cl\}\textbackslash)


\item \textbackslash(\textbackslash mathrm\{CH\}\_3

\textbackslash mathrm\{CH\}\_2

\textbackslash mathrm\{OH\}\textbackslash) and

\textbackslash(\textbackslash mathrm\{H\}\_3

\textbackslash mathrm\{PO\}\_2\textbackslash)


\item \textbackslash(\textbackslash mathrm\{H\}\_3

\textbackslash mathrm\{O\}\_2\textbackslash) and

\textbackslash(\textbackslash mathrm\{CH\}\_3

\textbackslash mathrm\{CH\}\_2

\textbackslash mathrm\{OH\}\textbackslash)


\item \textbackslash(\textbackslash mathrm\{CH\}\_3

\textbackslash mathrm\{CH\}\_2

\textbackslash mathrm\{Cl\}\textbackslash) and

\textbackslash(\textbackslash mathrm\{H\}\_3

\textbackslash mathrm\{PO\}\_2\textbackslash)


\end{enumerate}
\newpage
\section*{Question 52}
\includegraphics{https://kodemapa.com/static/media/wl_client/1/qdump/dd962b43da3e663bef2c213d7dbe3f88/80a0fa0979964d76f700803011f08f52.png}\\



In the above reactions, product \textbackslash(A\textbackslash) and

product \textbackslash(B\textbackslash) respectively are:


\begin{enumerate}[label=(\alph*)]
\item \includegraphics{https://kodemapa.com/static/media/wl_client/1/qdump/dd962b43da3e663bef2c213d7dbe3f88/c0d95ec45caccc885e7a10882ab11dcf.png}


\item \includegraphics{https://kodemapa.com/static/media/wl_client/1/qdump/dd962b43da3e663bef2c213d7dbe3f88/edb2a59275575343522e7f636ee9ead1.png}


\item \includegraphics{https://kodemapa.com/static/media/wl_client/1/qdump/dd962b43da3e663bef2c213d7dbe3f88/eb64f7e901eb6e6ef08159de542d0071.png}


\item \includegraphics{https://kodemapa.com/static/media/wl_client/1/qdump/dd962b43da3e663bef2c213d7dbe3f88/f1418dc91d368959bcf7b2a892c91d36.png}


\end{enumerate}
\newpage
\section*{Question 53}
\includegraphics{https://kodemapa.com/static/media/wl_client/1/qdump/dd962b43da3e663bef2c213d7dbe3f88/4ec04ea353b6f65a5cf7b06ed0289d5e.png}\\



Consider the given reaction, percentage yield of,~


\begin{enumerate}[label=(\alph*)]
\item \textbackslash((C)\textgreater(A)\textgreater(B)\textbackslash)


\item \textbackslash((B)\textgreater(C)\textgreater(A)\textbackslash)


\item \textbackslash((A)\textgreater(C)\textgreater(B)\textbackslash)


\item \textbackslash((C)\textgreater(B)\textgreater(A)\textbackslash)


\end{enumerate}
\newpage
\section*{Question 54}
what is A in the following reaction ?



\includegraphics{https://kodemapa.com/static/media/wl_client/1/qdump/dd962b43da3e663bef2c213d7dbe3f88/236e4ffa2b286d5be56b69d34825b864.png}\\


\begin{enumerate}[label=(\alph*)]
\item \includegraphics{https://kodemapa.com/static/media/wl_client/1/qdump/dd962b43da3e663bef2c213d7dbe3f88/7e913335697edc435c318cd8f4c74c35.png}


\item \includegraphics{https://kodemapa.com/static/media/wl_client/1/qdump/dd962b43da3e663bef2c213d7dbe3f88/af241ff17a69316d69e677cbe7bee714.png}


\item \includegraphics{https://kodemapa.com/static/media/wl_client/1/qdump/dd962b43da3e663bef2c213d7dbe3f88/96390fc856177859ff1bc0820852c540.png}


\item \includegraphics{https://kodemapa.com/static/media/wl_client/1/qdump/dd962b43da3e663bef2c213d7dbe3f88/502f4bfc33a7ea5f5446b31091aa8c80.png}


\end{enumerate}
\newpage
\section*{Question 55}
Which one of the products of the following reactions does not react with

Hinsberg reagent to form sulphonamide? {[}25 Jul 2021{]}


\begin{enumerate}[label=(\alph*)]
\item \includegraphics{https://kodemapa.com/static/media/wl_client/1/qdump/dd962b43da3e663bef2c213d7dbe3f88/2c25b5302176976e694d20326b655cbb.png}


\item \includegraphics{https://kodemapa.com/static/media/wl_client/1/qdump/dd962b43da3e663bef2c213d7dbe3f88/77e8483bbf8a5f91aca9e2e40598b611.png}


\item \includegraphics{https://kodemapa.com/static/media/wl_client/1/qdump/dd962b43da3e663bef2c213d7dbe3f88/335c787e1add6d2886cf53b8fe0870d7.png}


\item \includegraphics{https://kodemapa.com/static/media/wl_client/1/qdump/dd962b43da3e663bef2c213d7dbe3f88/07ad49568681c4e55c2b1bfd60d5b384.png}


\end{enumerate}
\newpage
\section*{Question 56}
Match List I with List II.



\begin{longtable}[]{@{}ll@{}}

\toprule\noalign{}

\endhead

\bottomrule\noalign{}

\endlastfoot

List-I~~ & ~List-II~ \\

A. Benzenesulphonyl Chloride & ~I. Test for primary amines~ \\

B. Hoffmann bromamide reaction~~ & II. Anti Saytzeff~~ \\

C. Carbylamine reaction~ & III. Hinsberg reagent~ \\

D. Hoffmann orientation~ & IV. Known reaction of Isocyanates~ \\

\end{longtable}



Choose the correct answer from the options given below:\\


\begin{enumerate}[label=(\alph*)]
\item A-IV, B-III, C-II, D-I


\item A-IV, B-II, C-I, D-II


\item A-III, B-IV, C-I, D-II


\item A-IV, B-III, C-I, D-II


\end{enumerate}
\newpage
\section*{Question 57}
The major product formed in the following reaction is.



\includegraphics{https://kodemapa.com/static/media/wl_client/1/qdump/dd962b43da3e663bef2c213d7dbe3f88/cd501f5bf6a7ba9867212e69f5da1961.png}\\


\begin{enumerate}[label=(\alph*)]
\item \includegraphics{https://kodemapa.com/static/media/wl_client/1/qdump/dd962b43da3e663bef2c213d7dbe3f88/ea66b26d652be69c42f279d0ca208c29.png}


\item \includegraphics{https://kodemapa.com/static/media/wl_client/1/qdump/dd962b43da3e663bef2c213d7dbe3f88/2952e31a186e1acc29cc186fdfb5dd6b.png}


\item \includegraphics{https://kodemapa.com/static/media/wl_client/1/qdump/dd962b43da3e663bef2c213d7dbe3f88/d4fa77cbadb82c63c120985ab1e82614.png}


\item \includegraphics{https://kodemapa.com/static/media/wl_client/1/qdump/dd962b43da3e663bef2c213d7dbe3f88/59084e148fc2d739417759e72917593f.png}


\end{enumerate}
\newpage
\section*{Question 58}
Given below are two statements :



Statement I : In Hofmann degradation reaction, the migration of only an

alkyl group takes place from carbonyl carbon of the amide to the

nitrogen atom.



Statement II : The group is migrated in Hofmann degradation reaction to

electron deficient atom.



In the light of the above statements, choose the most appropriate answer

from the options given below:


\begin{enumerate}[label=(\alph*)]
\item Both Statement I and Statement II are correct.


\item Both Statement I and Statement II are incorrect.


\item Statement I is correct but Statement II is incorrect.


\item Statement I is incorrect but Statement II is correct.


\end{enumerate}
\newpage
\section*{Question 59}
Hydrolysis of which compound will give carbolic acid?


\begin{enumerate}[label=(\alph*)]
\item Cumene


\item Benzenediazonium chloride


\item Benzal chloride


\item Ethylene glycol ketal


\end{enumerate}
\newpage
\section*{Question 60}
Given below are two statements :



Statement I : Aniline is less basic than acetamide.



Statement II : In aniline, the lone pair of electrons on nitrogen atom

is delocalised over benzene ring due to resonance and hence less

available to a proton.



Choose the most appropriate option ;


\begin{enumerate}[label=(\alph*)]
\item Statement I is true but statement II is false.


\item Statement I is false but statement II is true.


\item Both statement I and statement II are true.


\item Both statement I and statement II are false.


\end{enumerate}
\newpage
\section*{Question 61}
\includegraphics{https://kodemapa.com/static/media/wl_client/1/qdump/dd962b43da3e663bef2c213d7dbe3f88/4ec04ea353b6f65a5cf7b06ed0289d5e.png}\\



Consider the given reaction, percentage yield of,~


\begin{enumerate}[label=(\alph*)]
\item \textbackslash((C)\textgreater(A)\textgreater(B)\textbackslash)


\item \textbackslash((B)\textgreater(C)\textgreater(A)\textbackslash)


\item \textbackslash((A)\textgreater(C)\textgreater(B)\textbackslash)


\item \textbackslash((C)\textgreater(B)\textgreater(A)\textbackslash)


\end{enumerate}
\newpage
\section*{Question 62}
\includegraphics{https://kodemapa.com/static/media/wl_client/1/qdump/dd962b43da3e663bef2c213d7dbe3f88/4db12f12a4cdcdfcd29ad6819c260159.png}\\



In the chemical reactions given above \textbackslash(A\textbackslash)

and \textbackslash(B\textbackslash) respectively are:


\begin{enumerate}[label=(\alph*)]
\item \textbackslash(\textbackslash mathrm\{H\}\_3

\textbackslash mathrm\{PO\}\_2\textbackslash) and

\textbackslash(\textbackslash mathrm\{CH\}\_3

\textbackslash mathrm\{CH\}\_2

\textbackslash mathrm\{Cl\}\textbackslash)


\item \textbackslash(\textbackslash mathrm\{CH\}\_3

\textbackslash mathrm\{CH\}\_2

\textbackslash mathrm\{OH\}\textbackslash) and

\textbackslash(\textbackslash mathrm\{H\}\_3

\textbackslash mathrm\{PO\}\_2\textbackslash)


\item \textbackslash(\textbackslash mathrm\{H\}\_3

\textbackslash mathrm\{O\}\_2\textbackslash) and

\textbackslash(\textbackslash mathrm\{CH\}\_3

\textbackslash mathrm\{CH\}\_2

\textbackslash mathrm\{OH\}\textbackslash)


\item \textbackslash(\textbackslash mathrm\{CH\}\_3

\textbackslash mathrm\{CH\}\_2

\textbackslash mathrm\{Cl\}\textbackslash) and

\textbackslash(\textbackslash mathrm\{H\}\_3

\textbackslash mathrm\{PO\}\_2\textbackslash)


\end{enumerate}
\newpage
\section*{Question 63}
{What is the correct name for a molecule that has two amino groups in

opposing (para) locations around a benzene ring?}


\begin{enumerate}[label=(\alph*)]
\item {~Benzenediamine}


\item {Benzene-1,4-diamine}


\item {p-Aminoaniline}


\item {4-Aminobenzenamine}


\end{enumerate}
\newpage
\end{document}