\documentclass{article}
                    \usepackage{amsmath}
                    \usepackage{amssymb}
                    \usepackage{graphicx}
                    \usepackage{enumitem}
                    \usepackage{longtable}
                    \title{2kcet physics 1-07}
                    \begin{document}
                    \maketitle
                    \section*{Question 1}
If the sizes of charged bodies are very small compared to the distances between them, we treat them as ____________.
\begin{enumerate}[label=(\alph*)]
\item Zero charges
\item Point charges
\item Single charge
\item No charges
\end{enumerate}
\newpage
\section*{Question 2}
The force per unit charge is known as ____________.
\begin{enumerate}[label=(\alph*)]
\item Electric current
\item Electric potential
\item Electric field
\item Electric space
\end{enumerate}
\newpage
\section*{Question 3}
If the charge of 1 C is placed at a distance of 1 m from another charge of the same magnitude in a vacuum, it experiences an electrical force repulsion of magnitude ____________.
\begin{enumerate}[label=(\alph*)]
\item \(9 \times 10^{-9} N\)
\item \(9 \times 10^9 N\)
\item \(10 \times 10^9 N\)
\item \(10 \times 10^{-9} N\)
\end{enumerate}
\newpage
\section*{Question 4}
The quantization of charge indicates that:
\begin{enumerate}[label=(\alph*)]
\item Charge, which is a fraction of charge on an electron, is not possible
\item A charge cannot be destroyed
\item Charge exists on particles
\item There exists a minimum permissible charge on a particle
\end{enumerate}
\newpage
\section*{Question 5}
The property which differentiates two kinds of charges is called ____________.
\begin{enumerate}[label=(\alph*)]
\item Equality of charge
\item Polarity of charge
\item Fraction of charge
\item None of these
\end{enumerate}
\newpage
\section*{Question 6}
What happens when a glass rod is rubbed with silk?
\begin{enumerate}[label=(\alph*)]
\item gains protons from silk
\item gains electrons from silk
\item gives electrons to silk
\item gives protons to silk
\end{enumerate}
\newpage
\section*{Question 7}
If an object is positively charged, theoretically the mass of the object:
\begin{enumerate}[label=(\alph*)]
\item Increases slightly by a factor of \(9.11 \times 10^{-31} kg\)
\item Decreases slightly by a factor of \(9.11 \times 10^{-31} kg\)
\item Remains the same
\item May increase or decrease
\end{enumerate}
\newpage
\section*{Question 8}
If number of neutrons become more than the number of electrons in the element then it will become:
\begin{enumerate}[label=(\alph*)]
\item Positively charged
\item Negatively charged
\item Neutral
\item None of these
\end{enumerate}
\newpage
\section*{Question 9}
When the light pieces of paper which do not have any charge are kept close to a negatively charged comb, then there will be:
\begin{enumerate}[label=(\alph*)]
\item Attraction
\item Repulsion
\item No force
\item None of these
\end{enumerate}
\newpage
\section*{Question 10}
Which of the following is a good conductor?
\begin{enumerate}[label=(\alph*)]
\item Graphite
\item Charcoal
\item Anthracite
\item Diamond
\end{enumerate}
\newpage
\section*{Question 11}
If the charge on four glass rods are 2q, -q, -5q and 3q respectively, then the total charge stored in the system is:
\begin{enumerate}[label=(\alph*)]
\item -q
\item 11q
\item -11q
\item Zero
\end{enumerate}
\newpage
\section*{Question 12}
When a body is charged, then its mass will:
\begin{enumerate}[label=(\alph*)]
\item Increase
\item Decrease
\item May increase or decrease
\item Neither increase nor decrease
\end{enumerate}
\newpage
\section*{Question 13}
The force of attraction or repulsion between any two charges does not depend upon the:
\begin{enumerate}[label=(\alph*)]
\item Magnitude of the charges
\item Distance between the charges
\item Medium
\item None of these
\end{enumerate}
\newpage
\section*{Question 14}
What is the CGS unit of electric dipole moment?
\begin{enumerate}[label=(\alph*)]
\item Coulomb meter
\item Statcoulomb-centimetre
\item Joule-centimeter
\item Newton
\end{enumerate}
\newpage
\section*{Question 15}
The formation of a dipole is due to two equal and dissimilar point charges placed at a ________.
\begin{enumerate}[label=(\alph*)]
\item short distance
\item long distance
\item touching each other
\item None of these
\end{enumerate}
\newpage
\section*{Question 16}
The diploe moment between two equal charges of magnitude 9 μC but opposite in sign is 18× 10-$^{9}$ Cm. Calculate the distance between both the charges:
\begin{enumerate}[label=(\alph*)]
\item 1 mm
\item 2 mm
\item 2.3 mm
\item 2.5 mm
\end{enumerate}
\newpage
\section*{Question 17}
Dipole field is _________.
\begin{enumerate}[label=(\alph*)]
\item diploe momentum per unit volume 
\item electric field produced by the diploe
\item force experienced for unit charge when placed along the axis of diploe 
\item force experienced for unit charge when placed along the equatorial plane of diploe 
\end{enumerate}
\newpage
\section*{Question 18}
A square loop of area 50 cm$^{2}$ is placed in a uniform electric field of intensity 50 N/C such that the angle between the surface and the electric field is 60°, then the electric flux associated with the loop will be:
\begin{enumerate}[label=(\alph*)]
\item 1250 V-m
\item 125 V-m
\item 0.125 V-m
\item 1.25 V-m
\end{enumerate}
\newpage
\section*{Question 19}
There will be no electric field lines due to:
\begin{enumerate}[label=(\alph*)]
\item Neutron
\item Electric dipole
\item Both (A) and (B)
\item None of these
\end{enumerate}
\newpage
\section*{Question 20}
The force experienced by a unit positive test charge placed at a point is called:
\begin{enumerate}[label=(\alph*)]
\item Magnetic field at that point
\item Gravitational field at that point
\item Electrical field at that point
\item Nuclear field at that point
\end{enumerate}
\newpage
\section*{Question 21}
Inside a hollow conducting sphere electric field:
\begin{enumerate}[label=(\alph*)]
\item is zero
\item is a non - zero constant
\item changes with the magnitude of the charge given to the conductor
\item changes with distance from the centre of the sphere
\end{enumerate}
\newpage
\section*{Question 22}
The electric field in the inner region between two charged plates of a parallel plate capacitor is equal to______________. (\(\sigma\) is surface charge density)
\begin{enumerate}[label=(\alph*)]
\item \(\frac{\sigma }{ \varepsilon_0}\)
\item \(\frac{\sigma }{\left(2 \varepsilon_0\right)}\)
\item \(\frac{2 \sigma }{ \varepsilon_0}\)
\item \(\frac{\sigma^2 }{\varepsilon_0}\)
\end{enumerate}
\newpage
\section*{Question 23}
The Gauss law is true for:
\begin{enumerate}[label=(\alph*)]
\item All the surfaces
\item All the closed surfaces
\item Only spherical surfaces
\item None of these
\end{enumerate}
\newpage
\section*{Question 24}
The total flux associated with any closed surface depends on the:
\begin{enumerate}[label=(\alph*)]
\item Net charge enclosed in the surface
\item Surface area of the surface
\item Both (A) and (B)
\item None of these
\end{enumerate}
\newpage
\section*{Question 25}
Which of the following surface can't be a Gaussian surface?
\begin{enumerate}[label=(\alph*)]
\item Sphere
\item Cube
\item Disc
\item None of these
\end{enumerate}
\newpage
\section*{Question 26}
<style>.fm-math,fmath{font-family:STIXGeneral,'DejaVu Serif','DejaVu Sans',Times,OpenSymbol,'Standard Symbols L',serif;line-height:1.2}.fm-math mtext,fmath mtext{line-height:normal}.fm-mo,.ma-sans-serif,fmath mi[mathvariant*=sans-serif],fmath mn[mathvariant*=sans-serif],fmath mo,fmath ms[mathvariant*=sans-serif],fmath mtext[mathvariant*=sans-serif]{font-family:STIXGeneral,'DejaVu Sans','DejaVu Serif','Arial Unicode MS','Lucida Grande',Times,OpenSymbol,'Standard Symbols L',sans-serif}.fm-mo-Luc{font-family:STIXGeneral,'DejaVu Sans','DejaVu Serif','Lucida Grande','Arial Unicode MS',Times,OpenSymbol,'Standard Symbols L',sans-serif}.questionsfont{font-weight:200;font-family:Arial, sans-serif, STIXGeneral,'DejaVu Sans','DejaVu Serif','Lucida Grande','Arial Unicode MS',Times,OpenSymbol,'Standard Symbols L',sans-serif!important}.fm-separator{padding:0 .56ex 0 0}.fm-infix-loose{padding:0 .56ex}.fm-infix{padding:0 .44ex}.fm-prefix{padding:0 .33ex 0 0}.fm-postfix{padding:0 0 0 .33ex}.fm-prefix-tight{padding:0 .11ex 0 0}.fm-postfix-tight{padding:0 0 0 .11ex}.fm-quantifier{padding:0 .11ex 0 .22ex}.ma-non-marking{display:none}.fm-vert,fmath menclose,menclose.fm-menclose{display:inline-block}.fm-large-op{font-size:1.3em}.fm-inline .fm-large-op{font-size:1em}fmath mrow{white-space:nowrap}.fm-vert{vertical-align:middle}fmath table,fmath tbody,fmath td,fmath tr{border:0!important;padding:0!important;margin:0!important;outline:0!important}fmath table{border-collapse:collapse!important;text-align:center!important;table-layout:auto!important;float:none!important}.fm-frac{padding:0 1px!important}td.fm-den-frac{border-top:solid thin!important}.fm-root{font-size:.6em}.fm-radicand{padding:0 1px 0 0;border-top:solid;margin-top:.1em}.fm-script{font-size:.71em}.fm-script .fm-script .fm-script{font-size:1em}td.fm-underover-base{line-height:1!important}td.fm-mtd{padding:.5ex .4em!important;vertical-align:baseline!important}fmath mphantom{visibility:hidden}fmath menclose[notation=top],menclose.fm-menclose[notation=top]{border-top:solid thin}fmath menclose[notation=right],menclose.fm-menclose[notation=right]{border-right:solid thin}fmath menclose[notation=bottom],menclose.fm-menclose[notation=bottom]{border-bottom:solid thin}fmath menclose[notation=left],menclose.fm-menclose[notation=left]{border-left:solid thin}fmath menclose[notation=box],menclose.fm-menclose[notation=box]{border:thin solid}fmath none{display:none}</style> Two identical conducting spheres with negligible volume have <fmath class="fm-inline"><mrow><mn>2.1</mn><mrow><mi class="fm-mi-length-1" mathvariant="italic">n</mi><mi class="fm-mi-length-1" mathvariant="italic">C</mi></mrow></mrow></fmath> and <fmath class="fm-inline"><mrow><mo class="fm-prefix-tight">−</mo><mrow><mn>0.1</mn><mrow><mi class="fm-mi-length-1" mathvariant="italic">n</mi><mi class="fm-mi-length-1" mathvariant="italic">C</mi></mrow></mrow></mrow></fmath> charges, respectively. They are brought into contact and then separated by a distance of <fmath class="fm-inline"><mrow><mn>0.5</mn><mi class="fm-mi-length-1" mathvariant="italic">m</mi></mrow></fmath>. The electrostatic force acting between the spheres is ____ <fmath class="fm-inline"><mrow><mo class="fm-prefix-tight" lspace=".22em" rspace=".22em">×</mo><mrow><msup><mn>10</mn><mrow><mo class="fm-prefix-tight">−</mo><mn>9</mn></mrow></msup><mi class="fm-mi-length-1" mathvariant="italic" style="padding-right: 0.44ex;">N</mi></mrow></mrow></fmath>. \newline [Given, <fmath class="fm-inline"><mrow><mrow><mrow><mn>4</mn><mi class="fm-mi-length-1" mathvariant="italic">π</mi></mrow><msub><mi class="fm-mi-length-1" mathvariant="italic">ε</mi><mn>0</mn></msub></mrow><mo class="fm-infix-loose">=</mo><mrow><mspace style="margin-right: 0.28em; padding-right: 0.001em; visibility: hidden;" width=".28em">‌</mspace>\begin{tabular}{|c|c|}
\hline
<mn>1</mn> \\
\hline
<mrow><mn>9</mn><mo class="fm-infix" lspace=".22em" rspace=".22em">×</mo><msup><mn>10</mn><mn>9</mn></msup></mrow> \\
\hline
\end{tabular}
</mrow></mrow></fmath> SI unit]
\begin{enumerate}[label=(\alph*)]
\end{enumerate}
\newpage
\section*{Question 27}
<style>.fm-math,fmath{font-family:STIXGeneral,'DejaVu Serif','DejaVu Sans',Times,OpenSymbol,'Standard Symbols L',serif;line-height:1.2}.fm-math mtext,fmath mtext{line-height:normal}.fm-mo,.ma-sans-serif,fmath mi[mathvariant*=sans-serif],fmath mn[mathvariant*=sans-serif],fmath mo,fmath ms[mathvariant*=sans-serif],fmath mtext[mathvariant*=sans-serif]{font-family:STIXGeneral,'DejaVu Sans','DejaVu Serif','Arial Unicode MS','Lucida Grande',Times,OpenSymbol,'Standard Symbols L',sans-serif}.fm-mo-Luc{font-family:STIXGeneral,'DejaVu Sans','DejaVu Serif','Lucida Grande','Arial Unicode MS',Times,OpenSymbol,'Standard Symbols L',sans-serif}.questionsfont{font-weight:200;font-family:Arial, sans-serif, STIXGeneral,'DejaVu Sans','DejaVu Serif','Lucida Grande','Arial Unicode MS',Times,OpenSymbol,'Standard Symbols L',sans-serif!important}.fm-separator{padding:0 .56ex 0 0}.fm-infix-loose{padding:0 .56ex}.fm-infix{padding:0 .44ex}.fm-prefix{padding:0 .33ex 0 0}.fm-postfix{padding:0 0 0 .33ex}.fm-prefix-tight{padding:0 .11ex 0 0}.fm-postfix-tight{padding:0 0 0 .11ex}.fm-quantifier{padding:0 .11ex 0 .22ex}.ma-non-marking{display:none}.fm-vert,fmath menclose,menclose.fm-menclose{display:inline-block}.fm-large-op{font-size:1.3em}.fm-inline .fm-large-op{font-size:1em}fmath mrow{white-space:nowrap}.fm-vert{vertical-align:middle}fmath table,fmath tbody,fmath td,fmath tr{border:0!important;padding:0!important;margin:0!important;outline:0!important}fmath table{border-collapse:collapse!important;text-align:center!important;table-layout:auto!important;float:none!important}.fm-frac{padding:0 1px!important}td.fm-den-frac{border-top:solid thin!important}.fm-root{font-size:.6em}.fm-radicand{padding:0 1px 0 0;border-top:solid;margin-top:.1em}.fm-script{font-size:.71em}.fm-script .fm-script .fm-script{font-size:1em}td.fm-underover-base{line-height:1!important}td.fm-mtd{padding:.5ex .4em!important;vertical-align:baseline!important}fmath mphantom{visibility:hidden}fmath menclose[notation=top],menclose.fm-menclose[notation=top]{border-top:solid thin}fmath menclose[notation=right],menclose.fm-menclose[notation=right]{border-right:solid thin}fmath menclose[notation=bottom],menclose.fm-menclose[notation=bottom]{border-bottom:solid thin}fmath menclose[notation=left],menclose.fm-menclose[notation=left]{border-left:solid thin}fmath menclose[notation=box],menclose.fm-menclose[notation=box]{border:thin solid}fmath none{display:none}</style> Consider a sphere of radius <fmath class="fm-inline"><mi class="fm-mi-length-1" mathvariant="italic">R</mi></fmath> which carries a uniform charge density <fmath class="fm-inline"><mi class="fm-mi-length-1 ma-upright" mathvariant="normal" style="padding-right: 0px;">ρ</mi></fmath>. If a sphere of radius <fmath class="fm-inline">\begin{tabular}{|c|c|}
\hline
<mi class="fm-mi-length-1" mathvariant="italic">R</mi> \\
\hline
<mn>2</mn> \\
\hline
\end{tabular}
</fmath> is carved out of it, as shown, the ratio <fmath class="fm-inline">\begin{tabular}{|c|c|c|}
\hline
<mrow><mo class="fm-mo-Luc" style="font-size: 1.718em; vertical-align: 0em; display: inline-block; transform: scaleX(0.5);">|</mo><msub><mo style="display: block; margin-top: -0.25em; margin-bottom: -0.25em;">→</mo> \\
\hline
<mi class="fm-mi-length-1" mathvariant="italic" style="padding-right: 0.44ex;">E</mi> \\
\hline
\end{tabular}
<mi class="fm-mi-length-1" mathvariant="italic">A</mi></msub><mo class="fm-mo-Luc" style="font-size: 1.718em; vertical-align: 0em; display: inline-block; transform: scaleX(0.5);">|</mo></mrow></td></tr><tr><td class="fm-den-frac fm-inline"><mrow><mo class="fm-mo-Luc" style="font-size: 1.718em; vertical-align: 0em; display: inline-block; transform: scaleX(0.5);">|</mo><msub>\begin{tabular}{|c|c|}
\hline
<mo style="display: block; margin-top: -0.25em; margin-bottom: -0.25em;">→</mo> \\
\hline
<mi class="fm-mi-length-1" mathvariant="italic" style="padding-right: 0.44ex;">E</mi> \\
\hline
\end{tabular}
<mi class="fm-mi-length-1" mathvariant="italic">B</mi></msub><mo class="fm-mo-Luc" style="font-size: 1.718em; vertical-align: 0em; display: inline-block; transform: scaleX(0.5);">|</mo></mrow></td></tr></tbody></table></fmath> of magnitude of electric field <fmath class="fm-inline"><msub>\begin{tabular}{|c|c|}
\hline
<mo style="display: block; margin-top: -0.25em; margin-bottom: -0.25em;">→</mo> \\
\hline
<mi class="fm-mi-length-1" mathvariant="italic" style="padding-right: 0.44ex;">E</mi> \\
\hline
\end{tabular}
<mi class="fm-mi-length-1" mathvariant="italic">A</mi></msub></fmath> and <fmath class="fm-inline"><msub>\begin{tabular}{|c|c|}
\hline
<mo style="display: block; margin-top: -0.25em; margin-bottom: -0.25em;">→</mo> \\
\hline
<mi class="fm-mi-length-1" mathvariant="italic" style="padding-right: 0.44ex;">E</mi> \\
\hline
\end{tabular}
<mi class="fm-mi-length-1" mathvariant="italic">B</mi></msub></fmath>, respectively, at points <fmath class="fm-inline"><mi class="fm-mi-length-1" mathvariant="italic">A</mi></fmath> and <fmath class="fm-inline"><mi class="fm-mi-length-1" mathvariant="italic">B</mi></fmath> due to the remaining portion is: } \includegraphics[width=\textwidth]{static/media/wl_client/1/qdump/dd962b43da3e663bef2c213d7dbe3f88/e05a3c368b1a67f62bd162259b6212bc.png}
\begin{enumerate}[label=(\alph*)]
\item  <fmath class="fm-inline">\begin{tabular}{|c|c|}
\hline
<mn>21</mn> \\
\hline
<mn>34</mn> \\
\hline
\end{tabular}
</fmath>
\item  <fmath class="fm-inline">\begin{tabular}{|c|c|}
\hline
<mn>18</mn> \\
\hline
<mn>34</mn> \\
\hline
\end{tabular}
</fmath>
\item  <fmath class="fm-inline">\begin{tabular}{|c|c|}
\hline
<mn>17</mn> \\
\hline
<mn>54</mn> \\
\hline
\end{tabular}
</fmath>
\item  <fmath class="fm-inline">\begin{tabular}{|c|c|}
\hline
<mn>18</mn> \\
\hline
<mn>54</mn> \\
\hline
\end{tabular}
</fmath>
\end{enumerate}
\newpage
\section*{Question 28}
<style>.fm-math,fmath{font-family:STIXGeneral,'DejaVu Serif','DejaVu Sans',Times,OpenSymbol,'Standard Symbols L',serif;line-height:1.2}.fm-math mtext,fmath mtext{line-height:normal}.fm-mo,.ma-sans-serif,fmath mi[mathvariant*=sans-serif],fmath mn[mathvariant*=sans-serif],fmath mo,fmath ms[mathvariant*=sans-serif],fmath mtext[mathvariant*=sans-serif]{font-family:STIXGeneral,'DejaVu Sans','DejaVu Serif','Arial Unicode MS','Lucida Grande',Times,OpenSymbol,'Standard Symbols L',sans-serif}.fm-mo-Luc{font-family:STIXGeneral,'DejaVu Sans','DejaVu Serif','Lucida Grande','Arial Unicode MS',Times,OpenSymbol,'Standard Symbols L',sans-serif}.questionsfont{font-weight:200;font-family:Arial, sans-serif, STIXGeneral,'DejaVu Sans','DejaVu Serif','Lucida Grande','Arial Unicode MS',Times,OpenSymbol,'Standard Symbols L',sans-serif!important}.fm-separator{padding:0 .56ex 0 0}.fm-infix-loose{padding:0 .56ex}.fm-infix{padding:0 .44ex}.fm-prefix{padding:0 .33ex 0 0}.fm-postfix{padding:0 0 0 .33ex}.fm-prefix-tight{padding:0 .11ex 0 0}.fm-postfix-tight{padding:0 0 0 .11ex}.fm-quantifier{padding:0 .11ex 0 .22ex}.ma-non-marking{display:none}.fm-vert,fmath menclose,menclose.fm-menclose{display:inline-block}.fm-large-op{font-size:1.3em}.fm-inline .fm-large-op{font-size:1em}fmath mrow{white-space:nowrap}.fm-vert{vertical-align:middle}fmath table,fmath tbody,fmath td,fmath tr{border:0!important;padding:0!important;margin:0!important;outline:0!important}fmath table{border-collapse:collapse!important;text-align:center!important;table-layout:auto!important;float:none!important}.fm-frac{padding:0 1px!important}td.fm-den-frac{border-top:solid thin!important}.fm-root{font-size:.6em}.fm-radicand{padding:0 1px 0 0;border-top:solid;margin-top:.1em}.fm-script{font-size:.71em}.fm-script .fm-script .fm-script{font-size:1em}td.fm-underover-base{line-height:1!important}td.fm-mtd{padding:.5ex .4em!important;vertical-align:baseline!important}fmath mphantom{visibility:hidden}fmath menclose[notation=top],menclose.fm-menclose[notation=top]{border-top:solid thin}fmath menclose[notation=right],menclose.fm-menclose[notation=right]{border-right:solid thin}fmath menclose[notation=bottom],menclose.fm-menclose[notation=bottom]{border-bottom:solid thin}fmath menclose[notation=left],menclose.fm-menclose[notation=left]{border-left:solid thin}fmath menclose[notation=box],menclose.fm-menclose[notation=box]{border:thin solid}fmath none{display:none}</style>  A particle of mass <fmath class="fm-inline"><mi class="fm-mi-length-1" mathvariant="italic">m</mi></fmath> and charge <fmath class="fm-inline"><mi class="fm-mi-length-1" mathvariant="italic">q</mi></fmath> is released from rest in a uniform electric field. If there is no other force on the particle, the dependence of its speed <fmath class="fm-inline"><mi class="fm-mi-length-1" mathvariant="italic">v</mi></fmath> on the distance <fmath class="fm-inline"><mi class="fm-mi-length-1" mathvariant="italic">x</mi></fmath> travelled by it is correctly given by (graphs are schematic and not drawn to scale) }
\begin{enumerate}[label=(\alph*)]
\item  \includegraphics[width=\textwidth]{static/media/wl_client/1/qdump/dd962b43da3e663bef2c213d7dbe3f88/b72719366995bd04eac8ffd07642bfe9.png}
\item  \includegraphics[width=\textwidth]{static/media/wl_client/1/qdump/dd962b43da3e663bef2c213d7dbe3f88/71c5df525329a486a904cd439e8cee67.png}
\item  \includegraphics[width=\textwidth]{static/media/wl_client/1/qdump/dd962b43da3e663bef2c213d7dbe3f88/6793c84f8db8a9fe2cf11392ac4308ed.png}
\item  \includegraphics[width=\textwidth]{static/media/wl_client/1/qdump/dd962b43da3e663bef2c213d7dbe3f88/a366372a87b3919d3eb78bd0a35eed0c.png}
\end{enumerate}
\newpage
\section*{Question 29}
<style>.fm-math,fmath{font-family:STIXGeneral,'DejaVu Serif','DejaVu Sans',Times,OpenSymbol,'Standard Symbols L',serif;line-height:1.2}.fm-math mtext,fmath mtext{line-height:normal}.fm-mo,.ma-sans-serif,fmath mi[mathvariant*=sans-serif],fmath mn[mathvariant*=sans-serif],fmath mo,fmath ms[mathvariant*=sans-serif],fmath mtext[mathvariant*=sans-serif]{font-family:STIXGeneral,'DejaVu Sans','DejaVu Serif','Arial Unicode MS','Lucida Grande',Times,OpenSymbol,'Standard Symbols L',sans-serif}.fm-mo-Luc{font-family:STIXGeneral,'DejaVu Sans','DejaVu Serif','Lucida Grande','Arial Unicode MS',Times,OpenSymbol,'Standard Symbols L',sans-serif}.questionsfont{font-weight:200;font-family:Arial, sans-serif, STIXGeneral,'DejaVu Sans','DejaVu Serif','Lucida Grande','Arial Unicode MS',Times,OpenSymbol,'Standard Symbols L',sans-serif!important}.fm-separator{padding:0 .56ex 0 0}.fm-infix-loose{padding:0 .56ex}.fm-infix{padding:0 .44ex}.fm-prefix{padding:0 .33ex 0 0}.fm-postfix{padding:0 0 0 .33ex}.fm-prefix-tight{padding:0 .11ex 0 0}.fm-postfix-tight{padding:0 0 0 .11ex}.fm-quantifier{padding:0 .11ex 0 .22ex}.ma-non-marking{display:none}.fm-vert,fmath menclose,menclose.fm-menclose{display:inline-block}.fm-large-op{font-size:1.3em}.fm-inline .fm-large-op{font-size:1em}fmath mrow{white-space:nowrap}.fm-vert{vertical-align:middle}fmath table,fmath tbody,fmath td,fmath tr{border:0!important;padding:0!important;margin:0!important;outline:0!important}fmath table{border-collapse:collapse!important;text-align:center!important;table-layout:auto!important;float:none!important}.fm-frac{padding:0 1px!important}td.fm-den-frac{border-top:solid thin!important}.fm-root{font-size:.6em}.fm-radicand{padding:0 1px 0 0;border-top:solid;margin-top:.1em}.fm-script{font-size:.71em}.fm-script .fm-script .fm-script{font-size:1em}td.fm-underover-base{line-height:1!important}td.fm-mtd{padding:.5ex .4em!important;vertical-align:baseline!important}fmath mphantom{visibility:hidden}fmath menclose[notation=top],menclose.fm-menclose[notation=top]{border-top:solid thin}fmath menclose[notation=right],menclose.fm-menclose[notation=right]{border-right:solid thin}fmath menclose[notation=bottom],menclose.fm-menclose[notation=bottom]{border-bottom:solid thin}fmath menclose[notation=left],menclose.fm-menclose[notation=left]{border-left:solid thin}fmath menclose[notation=box],menclose.fm-menclose[notation=box]{border:thin solid}fmath none{display:none}</style> Shown in the figure is a shell made of a conductor. It has inner radius <fmath class="fm-inline"><mi class="fm-mi-length-1" mathvariant="italic">a</mi></fmath> and outer radius <fmath class="fm-inline"><mrow><mi class="fm-mi-length-1" mathvariant="italic">b</mi><mo class="fm-postfix-tight">,</mo></mrow></fmath> and carries charge <fmath class="fm-inline"><mi class="fm-mi-length-1" mathvariant="italic">Q</mi></fmath>. At its centre is a dipole <fmath class="fm-inline">\begin{tabular}{|c|c|}
\hline
<mo style="display: block; margin-top: -0.25em; margin-bottom: -0.25em;">→</mo> \\
\hline
<mi class="fm-mi-length-1" mathvariant="italic" style="display: block; margin-top: -0.25em;">p</mi> \\
\hline
\end{tabular}
</fmath> as shown. In this case : } \includegraphics[width=\textwidth]{static/media/wl_client/1/qdump/dd962b43da3e663bef2c213d7dbe3f88/c6fb3c018758f441869086b4eae97367.png}
\begin{enumerate}[label=(\alph*)]
\item  surface change density on the inner surface is uniform and equal to <fmath class="fm-inline">\begin{tabular}{|c|c|}
\hline
<mrow><mi class="fm-mi-length-1" mathvariant="italic">Q</mi><mo class="fm-infix-loose">∕</mo><mn>2</mn></mrow> \\
\hline
<mrow><mrow><mn>4</mn><mi class="fm-mi-length-1 ma-upright" mathvariant="normal" style="padding-right: 0px;">π</mi></mrow><msup><mi class="fm-mi-length-1" mathvariant="italic">a</mi><mn>2</mn></msup></mrow> \\
\hline
\end{tabular}
</fmath>
\item  electric field outside the shell is the same as that of a point charge at the centre of the shell. 
\item  surface charge density on the outer surface depends on <fmath class="fm-inline"><mrow><mo class="fm-mo-Luc" style="font-size: 1.508em; vertical-align: 0.118em; display: inline-block; transform: scaleX(0.5);">|</mo>\begin{tabular}{|c|c|}
\hline
<mo style="display: block; margin-top: -0.25em; margin-bottom: -0.25em;">→</mo> \\
\hline
<mi class="fm-mi-length-1" mathvariant="italic">P</mi> \\
\hline
\end{tabular}
<mo class="fm-mo-Luc" style="font-size: 1.508em; vertical-align: 0.118em; display: inline-block; transform: scaleX(0.5);">|</mo></mrow></fmath>
\item  surface charge density on the inner surface of the shell is zero everywhere.-
\end{enumerate}
\newpage
\section*{Question 30}
<style>.fm-math,fmath{font-family:STIXGeneral,'DejaVu Serif','DejaVu Sans',Times,OpenSymbol,'Standard Symbols L',serif;line-height:1.2}.fm-math mtext,fmath mtext{line-height:normal}.fm-mo,.ma-sans-serif,fmath mi[mathvariant*=sans-serif],fmath mn[mathvariant*=sans-serif],fmath mo,fmath ms[mathvariant*=sans-serif],fmath mtext[mathvariant*=sans-serif]{font-family:STIXGeneral,'DejaVu Sans','DejaVu Serif','Arial Unicode MS','Lucida Grande',Times,OpenSymbol,'Standard Symbols L',sans-serif}.fm-mo-Luc{font-family:STIXGeneral,'DejaVu Sans','DejaVu Serif','Lucida Grande','Arial Unicode MS',Times,OpenSymbol,'Standard Symbols L',sans-serif}.questionsfont{font-weight:200;font-family:Arial, sans-serif, STIXGeneral,'DejaVu Sans','DejaVu Serif','Lucida Grande','Arial Unicode MS',Times,OpenSymbol,'Standard Symbols L',sans-serif!important}.fm-separator{padding:0 .56ex 0 0}.fm-infix-loose{padding:0 .56ex}.fm-infix{padding:0 .44ex}.fm-prefix{padding:0 .33ex 0 0}.fm-postfix{padding:0 0 0 .33ex}.fm-prefix-tight{padding:0 .11ex 0 0}.fm-postfix-tight{padding:0 0 0 .11ex}.fm-quantifier{padding:0 .11ex 0 .22ex}.ma-non-marking{display:none}.fm-vert,fmath menclose,menclose.fm-menclose{display:inline-block}.fm-large-op{font-size:1.3em}.fm-inline .fm-large-op{font-size:1em}fmath mrow{white-space:nowrap}.fm-vert{vertical-align:middle}fmath table,fmath tbody,fmath td,fmath tr{border:0!important;padding:0!important;margin:0!important;outline:0!important}fmath table{border-collapse:collapse!important;text-align:center!important;table-layout:auto!important;float:none!important}.fm-frac{padding:0 1px!important}td.fm-den-frac{border-top:solid thin!important}.fm-root{font-size:.6em}.fm-radicand{padding:0 1px 0 0;border-top:solid;margin-top:.1em}.fm-script{font-size:.71em}.fm-script .fm-script .fm-script{font-size:1em}td.fm-underover-base{line-height:1!important}td.fm-mtd{padding:.5ex .4em!important;vertical-align:baseline!important}fmath mphantom{visibility:hidden}fmath menclose[notation=top],menclose.fm-menclose[notation=top]{border-top:solid thin}fmath menclose[notation=right],menclose.fm-menclose[notation=right]{border-right:solid thin}fmath menclose[notation=bottom],menclose.fm-menclose[notation=bottom]{border-bottom:solid thin}fmath menclose[notation=left],menclose.fm-menclose[notation=left]{border-left:solid thin}fmath menclose[notation=box],menclose.fm-menclose[notation=box]{border:thin solid}fmath none{display:none}</style> An electric dipole is formed by two equal and opposite charges <fmath class="fm-inline"><mi class="fm-mi-length-1" mathvariant="italic">q</mi></fmath> with separation <fmath class="fm-inline"><mi class="fm-mi-length-1" mathvariant="italic" style="padding-right: 0.44ex;">d</mi></fmath>. The charges have same mass <fmath class="fm-inline"><mi class="fm-mi-length-1" mathvariant="italic">m</mi></fmath>. It is kept in a uniform electric field <fmath class="fm-inline"><mi class="fm-mi-length-1" mathvariant="italic" style="padding-right: 0.44ex;">E</mi></fmath>. If it is slightly rotated from its equilibrium orientation, then its angular frequency <fmath class="fm-inline"><mi class="fm-mi-length-1 ma-upright" mathvariant="normal" style="padding-right: 0px;">ω</mi></fmath> is : }
\begin{enumerate}[label=(\alph*)]
\item  <fmath class="fm-inline"><mrow mtagname="msqrt"><mo class="fm-radic" style="font-size: 2.256em; vertical-align: -0.219em; display: inline-block; transform: scaleX(0.5);">√</mo>\begin{tabular}{|c|c|}
\hline
<mrow><mi class="fm-mi-length-1" mathvariant="italic">q</mi><mi class="fm-mi-length-1" mathvariant="italic" style="padding-right: 0.44ex;">E</mi></mrow> \\
\hline
<mrow><mi class="fm-mi-length-1" mathvariant="italic">m</mi><mi class="fm-mi-length-1" mathvariant="italic" style="padding-right: 0.44ex;">d</mi></mrow> \\
\hline
\end{tabular}
</mrow></fmath>
\item  <fmath class="fm-inline"><mrow mtagname="msqrt"><mo class="fm-radic" style="font-size: 2.256em; vertical-align: -0.219em; display: inline-block; transform: scaleX(0.5);">√</mo>\begin{tabular}{|c|c|}
\hline
<mrow><mrow><mn>2</mn><mi class="fm-mi-length-1" mathvariant="italic">q</mi></mrow><mi class="fm-mi-length-1" mathvariant="italic" style="padding-right: 0.44ex;">E</mi></mrow> \\
\hline
<mrow><mi class="fm-mi-length-1" mathvariant="italic">m</mi><mi class="fm-mi-length-1" mathvariant="italic" style="padding-right: 0.44ex;">d</mi></mrow> \\
\hline
\end{tabular}
</mrow></fmath>
\item  <fmath class="fm-inline"><mrow><mn>2</mn><mrow mtagname="msqrt"><mo class="fm-radic" style="font-size: 2.256em; vertical-align: -0.219em; display: inline-block; transform: scaleX(0.5);">√</mo>\begin{tabular}{|c|c|}
\hline
<mrow><mi class="fm-mi-length-1" mathvariant="italic">q</mi><mi class="fm-mi-length-1" mathvariant="italic" style="padding-right: 0.44ex;">E</mi></mrow> \\
\hline
<mrow><mi class="fm-mi-length-1" mathvariant="italic">m</mi><mi class="fm-mi-length-1" mathvariant="italic" style="padding-right: 0.44ex;">d</mi></mrow> \\
\hline
\end{tabular}
</mrow></mrow></fmath>
\item  <fmath class="fm-inline"><mrow mtagname="msqrt"><mo class="fm-radic" style="font-size: 2.256em; vertical-align: -0.219em; display: inline-block; transform: scaleX(0.5);">√</mo>\begin{tabular}{|c|c|}
\hline
<mrow><mi class="fm-mi-length-1" mathvariant="italic">q</mi><mi class="fm-mi-length-1" mathvariant="italic" style="padding-right: 0.44ex;">E</mi></mrow> \\
\hline
<mrow><mrow><mn>2</mn><mi class="fm-mi-length-1" mathvariant="italic">m</mi></mrow><mi class="fm-mi-length-1" mathvariant="italic" style="padding-right: 0.44ex;">d</mi></mrow> \\
\hline
\end{tabular}
</mrow></fmath>
\end{enumerate}
\newpage
\section*{Question 31}
<style>.fm-math,fmath{font-family:STIXGeneral,'DejaVu Serif','DejaVu Sans',Times,OpenSymbol,'Standard Symbols L',serif;line-height:1.2}.fm-math mtext,fmath mtext{line-height:normal}.fm-mo,.ma-sans-serif,fmath mi[mathvariant*=sans-serif],fmath mn[mathvariant*=sans-serif],fmath mo,fmath ms[mathvariant*=sans-serif],fmath mtext[mathvariant*=sans-serif]{font-family:STIXGeneral,'DejaVu Sans','DejaVu Serif','Arial Unicode MS','Lucida Grande',Times,OpenSymbol,'Standard Symbols L',sans-serif}.fm-mo-Luc{font-family:STIXGeneral,'DejaVu Sans','DejaVu Serif','Lucida Grande','Arial Unicode MS',Times,OpenSymbol,'Standard Symbols L',sans-serif}.questionsfont{font-weight:200;font-family:Arial, sans-serif, STIXGeneral,'DejaVu Sans','DejaVu Serif','Lucida Grande','Arial Unicode MS',Times,OpenSymbol,'Standard Symbols L',sans-serif!important}.fm-separator{padding:0 .56ex 0 0}.fm-infix-loose{padding:0 .56ex}.fm-infix{padding:0 .44ex}.fm-prefix{padding:0 .33ex 0 0}.fm-postfix{padding:0 0 0 .33ex}.fm-prefix-tight{padding:0 .11ex 0 0}.fm-postfix-tight{padding:0 0 0 .11ex}.fm-quantifier{padding:0 .11ex 0 .22ex}.ma-non-marking{display:none}.fm-vert,fmath menclose,menclose.fm-menclose{display:inline-block}.fm-large-op{font-size:1.3em}.fm-inline .fm-large-op{font-size:1em}fmath mrow{white-space:nowrap}.fm-vert{vertical-align:middle}fmath table,fmath tbody,fmath td,fmath tr{border:0!important;padding:0!important;margin:0!important;outline:0!important}fmath table{border-collapse:collapse!important;text-align:center!important;table-layout:auto!important;float:none!important}.fm-frac{padding:0 1px!important}td.fm-den-frac{border-top:solid thin!important}.fm-root{font-size:.6em}.fm-radicand{padding:0 1px 0 0;border-top:solid;margin-top:.1em}.fm-script{font-size:.71em}.fm-script .fm-script .fm-script{font-size:1em}td.fm-underover-base{line-height:1!important}td.fm-mtd{padding:.5ex .4em!important;vertical-align:baseline!important}fmath mphantom{visibility:hidden}fmath menclose[notation=top],menclose.fm-menclose[notation=top]{border-top:solid thin}fmath menclose[notation=right],menclose.fm-menclose[notation=right]{border-right:solid thin}fmath menclose[notation=bottom],menclose.fm-menclose[notation=bottom]{border-bottom:solid thin}fmath menclose[notation=left],menclose.fm-menclose[notation=left]{border-left:solid thin}fmath menclose[notation=box],menclose.fm-menclose[notation=box]{border:thin solid}fmath none{display:none}</style> A solid ball of radius <fmath class="fm-inline"><mi class="fm-mi-length-1" mathvariant="italic">R</mi></fmath> has a charge density <fmath class="fm-inline"><mi class="fm-mi-length-1 ma-upright" mathvariant="normal" style="padding-right: 0px;">ρ</mi></fmath> given by<fmath class="fm-inline"><mrow><mi class="fm-mi-length-1 ma-upright" mathvariant="normal" style="padding-right: 0px;">ρ</mi><mo class="fm-infix-loose">=</mo><mrow><msub><mi class="fm-mi-length-1 ma-upright" mathvariant="normal" style="padding-right: 0px;">ρ</mi><mn>0</mn></msub><mrow><mo class="fm-mo-Luc" style="font-size: 2.05em; vertical-align: -0.128em; display: inline-block; transform: scaleX(0.5);">(</mo><mrow><mn>1</mn><mo class="fm-infix">−</mo>\begin{tabular}{|c|c|}
\hline
<mi class="fm-mi-length-1" mathvariant="italic">r</mi> \\
\hline
<mi class="fm-mi-length-1" mathvariant="italic">R</mi> \\
\hline
\end{tabular}
</mrow><mo class="fm-mo-Luc" style="font-size: 2.05em; vertical-align: -0.128em; display: inline-block; transform: scaleX(0.5);">)</mo></mrow></mrow></mrow></fmath> for <fmath class="fm-inline"><mrow><mrow><mn>0</mn><mo class="fm-infix-loose">≤</mo><mi class="fm-mi-length-1" mathvariant="italic">r</mi></mrow><mo class="fm-infix-loose">≤</mo><mrow><mi class="fm-mi-length-1" mathvariant="italic">R</mi><mo class="fm-postfix-tight">.</mo></mrow></mrow></fmath> The electric field outside the ball is: }
\begin{enumerate}[label=(\alph*)]
\item  <fmath class="fm-inline">\begin{tabular}{|c|c|}
\hline
<mrow><msub><mi class="fm-mi-length-1 ma-upright" mathvariant="normal" style="padding-right: 0px;">ρ</mi><mn>0</mn></msub><msup><mi class="fm-mi-length-1" mathvariant="italic">R</mi><mn>3</mn></msup></mrow> \\
\hline
<mrow><msub><mi class="fm-mi-length-1 ma-upright" mathvariant="normal" style="padding-right: 0px;">ε</mi><mn>0</mn></msub><msup><mi class="fm-mi-length-1" mathvariant="italic">r</mi><mn>2</mn></msup></mrow> \\
\hline
\end{tabular}
</fmath>
\item  <fmath class="fm-inline">\begin{tabular}{|c|c|}
\hline
<mrow><mrow><mn>4</mn><msub><mi class="fm-mi-length-1 ma-upright" mathvariant="normal" style="padding-right: 0px;">ρ</mi><mn>0</mn></msub></mrow><msup><mi class="fm-mi-length-1" mathvariant="italic">R</mi><mn>3</mn></msup></mrow> \\
\hline
<mrow><mrow><mn>3</mn><msub><mi class="fm-mi-length-1 ma-upright" mathvariant="normal" style="padding-right: 0px;">ε</mi><mn>0</mn></msub></mrow><msup><mi class="fm-mi-length-1" mathvariant="italic">r</mi><mn>2</mn></msup></mrow> \\
\hline
\end{tabular}
</fmath>
\item  <fmath class="fm-inline">\begin{tabular}{|c|c|}
\hline
<mrow><mrow><mn>3</mn><msub><mi class="fm-mi-length-1 ma-upright" mathvariant="normal" style="padding-right: 0px;">ρ</mi><mn>0</mn></msub></mrow><msup><mi class="fm-mi-length-1" mathvariant="italic">R</mi><mn>3</mn></msup></mrow> \\
\hline
<mrow><mrow><mn>4</mn><msub><mi class="fm-mi-length-1 ma-upright" mathvariant="normal" style="padding-right: 0px;">ε</mi><mn>0</mn></msub></mrow><msup><mi class="fm-mi-length-1" mathvariant="italic">r</mi><mn>2</mn></msup></mrow> \\
\hline
\end{tabular}
</fmath>
\item  <fmath class="fm-inline">\begin{tabular}{|c|c|}
\hline
<mrow><msub><mi class="fm-mi-length-1 ma-upright" mathvariant="normal" style="padding-right: 0px;">ρ</mi><mn>0</mn></msub><msup><mi class="fm-mi-length-1" mathvariant="italic">R</mi><mn>3</mn></msup></mrow> \\
\hline
<mrow><mrow><mn>12</mn><msub><mi class="fm-mi-length-1 ma-upright" mathvariant="normal" style="padding-right: 0px;">ε</mi><mn>0</mn></msub></mrow><msup><mi class="fm-mi-length-1" mathvariant="italic">r</mi><mn>2</mn></msup></mrow> \\
\hline
\end{tabular}
</fmath>
\end{enumerate}
\newpage
\section*{Question 32}
<style>.fm-math,fmath{font-family:STIXGeneral,'DejaVu Serif','DejaVu Sans',Times,OpenSymbol,'Standard Symbols L',serif;line-height:1.2}.fm-math mtext,fmath mtext{line-height:normal}.fm-mo,.ma-sans-serif,fmath mi[mathvariant*=sans-serif],fmath mn[mathvariant*=sans-serif],fmath mo,fmath ms[mathvariant*=sans-serif],fmath mtext[mathvariant*=sans-serif]{font-family:STIXGeneral,'DejaVu Sans','DejaVu Serif','Arial Unicode MS','Lucida Grande',Times,OpenSymbol,'Standard Symbols L',sans-serif}.fm-mo-Luc{font-family:STIXGeneral,'DejaVu Sans','DejaVu Serif','Lucida Grande','Arial Unicode MS',Times,OpenSymbol,'Standard Symbols L',sans-serif}.questionsfont{font-weight:200;font-family:Arial, sans-serif, STIXGeneral,'DejaVu Sans','DejaVu Serif','Lucida Grande','Arial Unicode MS',Times,OpenSymbol,'Standard Symbols L',sans-serif!important}.fm-separator{padding:0 .56ex 0 0}.fm-infix-loose{padding:0 .56ex}.fm-infix{padding:0 .44ex}.fm-prefix{padding:0 .33ex 0 0}.fm-postfix{padding:0 0 0 .33ex}.fm-prefix-tight{padding:0 .11ex 0 0}.fm-postfix-tight{padding:0 0 0 .11ex}.fm-quantifier{padding:0 .11ex 0 .22ex}.ma-non-marking{display:none}.fm-vert,fmath menclose,menclose.fm-menclose{display:inline-block}.fm-large-op{font-size:1.3em}.fm-inline .fm-large-op{font-size:1em}fmath mrow{white-space:nowrap}.fm-vert{vertical-align:middle}fmath table,fmath tbody,fmath td,fmath tr{border:0!important;padding:0!important;margin:0!important;outline:0!important}fmath table{border-collapse:collapse!important;text-align:center!important;table-layout:auto!important;float:none!important}.fm-frac{padding:0 1px!important}td.fm-den-frac{border-top:solid thin!important}.fm-root{font-size:.6em}.fm-radicand{padding:0 1px 0 0;border-top:solid;margin-top:.1em}.fm-script{font-size:.71em}.fm-script .fm-script .fm-script{font-size:1em}td.fm-underover-base{line-height:1!important}td.fm-mtd{padding:.5ex .4em!important;vertical-align:baseline!important}fmath mphantom{visibility:hidden}fmath menclose[notation=top],menclose.fm-menclose[notation=top]{border-top:solid thin}fmath menclose[notation=right],menclose.fm-menclose[notation=right]{border-right:solid thin}fmath menclose[notation=bottom],menclose.fm-menclose[notation=bottom]{border-bottom:solid thin}fmath menclose[notation=left],menclose.fm-menclose[notation=left]{border-left:solid thin}fmath menclose[notation=box],menclose.fm-menclose[notation=box]{border:thin solid}fmath none{display:none}</style> A charge <fmath class="fm-inline"><mi class="fm-mi-length-1" mathvariant="italic">Q</mi></fmath> is placed at a distance <fmath class="fm-inline"><mrow><mi class="fm-mi-length-1" mathvariant="italic">a</mi><mo class="fm-infix-loose">∕</mo><mn>2</mn></mrow></fmath> above the centre of the square surface of edge a as shown in the figure. The electric flux through the square surface is:}\includegraphics[width=\textwidth]{static/media/wl_client/1/qdump/dd962b43da3e663bef2c213d7dbe3f88/23e16d8e33ea526225cb96db5d28cefc.png}
\begin{enumerate}[label=(\alph*)]
\item  <fmath class="fm-inline">\begin{tabular}{|c|c|}
\hline
<mi class="fm-mi-length-1" mathvariant="italic">Q</mi> \\
\hline
<mrow><mn>3</mn><msub><mi class="fm-mi-length-1 ma-upright" mathvariant="normal" style="padding-right: 0px;">ε</mi><mn>0</mn></msub></mrow> \\
\hline
\end{tabular}
</fmath>
\item  <fmath class="fm-inline">\begin{tabular}{|c|c|}
\hline
<mi class="fm-mi-length-1" mathvariant="italic">Q</mi> \\
\hline
<mrow><mn>6</mn><msub><mi class="fm-mi-length-1 ma-upright" mathvariant="normal" style="padding-right: 0px;">ε</mi><mn>0</mn></msub></mrow> \\
\hline
\end{tabular}
</fmath>
\item  <fmath class="fm-inline">\begin{tabular}{|c|c|}
\hline
<mi class="fm-mi-length-1" mathvariant="italic">Q</mi> \\
\hline
<mrow><mn>2</mn><msub><mi class="fm-mi-length-1 ma-upright" mathvariant="normal" style="padding-right: 0px;">ε</mi><mn>0</mn></msub></mrow> \\
\hline
\end{tabular}
</fmath>
\item  <fmath class="fm-inline">\begin{tabular}{|c|c|}
\hline
<mi class="fm-mi-length-1" mathvariant="italic">Q</mi> \\
\hline
<msub><mi class="fm-mi-length-1 ma-upright" mathvariant="normal" style="padding-right: 0px;">ε</mi><mn>0</mn></msub> \\
\hline
\end{tabular}
</fmath>
\end{enumerate}
\newpage
\section*{Question 33}
<style>.fm-math,fmath{font-family:STIXGeneral,'DejaVu Serif','DejaVu Sans',Times,OpenSymbol,'Standard Symbols L',serif;line-height:1.2}.fm-math mtext,fmath mtext{line-height:normal}.fm-mo,.ma-sans-serif,fmath mi[mathvariant*=sans-serif],fmath mn[mathvariant*=sans-serif],fmath mo,fmath ms[mathvariant*=sans-serif],fmath mtext[mathvariant*=sans-serif]{font-family:STIXGeneral,'DejaVu Sans','DejaVu Serif','Arial Unicode MS','Lucida Grande',Times,OpenSymbol,'Standard Symbols L',sans-serif}.fm-mo-Luc{font-family:STIXGeneral,'DejaVu Sans','DejaVu Serif','Lucida Grande','Arial Unicode MS',Times,OpenSymbol,'Standard Symbols L',sans-serif}.questionsfont{font-weight:200;font-family:Arial, sans-serif, STIXGeneral,'DejaVu Sans','DejaVu Serif','Lucida Grande','Arial Unicode MS',Times,OpenSymbol,'Standard Symbols L',sans-serif!important}.fm-separator{padding:0 .56ex 0 0}.fm-infix-loose{padding:0 .56ex}.fm-infix{padding:0 .44ex}.fm-prefix{padding:0 .33ex 0 0}.fm-postfix{padding:0 0 0 .33ex}.fm-prefix-tight{padding:0 .11ex 0 0}.fm-postfix-tight{padding:0 0 0 .11ex}.fm-quantifier{padding:0 .11ex 0 .22ex}.ma-non-marking{display:none}.fm-vert,fmath menclose,menclose.fm-menclose{display:inline-block}.fm-large-op{font-size:1.3em}.fm-inline .fm-large-op{font-size:1em}fmath mrow{white-space:nowrap}.fm-vert{vertical-align:middle}fmath table,fmath tbody,fmath td,fmath tr{border:0!important;padding:0!important;margin:0!important;outline:0!important}fmath table{border-collapse:collapse!important;text-align:center!important;table-layout:auto!important;float:none!important}.fm-frac{padding:0 1px!important}td.fm-den-frac{border-top:solid thin!important}.fm-root{font-size:.6em}.fm-radicand{padding:0 1px 0 0;border-top:solid;margin-top:.1em}.fm-script{font-size:.71em}.fm-script .fm-script .fm-script{font-size:1em}td.fm-underover-base{line-height:1!important}td.fm-mtd{padding:.5ex .4em!important;vertical-align:baseline!important}fmath mphantom{visibility:hidden}fmath menclose[notation=top],menclose.fm-menclose[notation=top]{border-top:solid thin}fmath menclose[notation=right],menclose.fm-menclose[notation=right]{border-right:solid thin}fmath menclose[notation=bottom],menclose.fm-menclose[notation=bottom]{border-bottom:solid thin}fmath menclose[notation=left],menclose.fm-menclose[notation=left]{border-left:solid thin}fmath menclose[notation=box],menclose.fm-menclose[notation=box]{border:thin solid}fmath none{display:none}</style> Four closed surfaces and corresponding charge distributions are shown below. \includegraphics[width=\textwidth]{static/media/wl_client/1/qdump/dd962b43da3e663bef2c213d7dbe3f88/e08514f58651b0b338b408356f32c705.png}Let the respective electric fluxes through the surfaces be <fmath class="fm-inline"><mrow><mrow><mrow><msub><mi class="fm-mi-length-1 ma-upright" mathvariant="normal" style="padding-right: 0px;">Φ</mi><mn>1</mn></msub><mo class="fm-separator">,</mo><msub><mi class="fm-mi-length-1 ma-upright" mathvariant="normal" style="padding-right: 0px;">Φ</mi><mn>2</mn></msub></mrow><mo class="fm-separator">,</mo><msub><mi class="fm-mi-length-1 ma-upright" mathvariant="normal" style="padding-right: 0px;">Φ</mi><mn>3</mn></msub></mrow><mo class="fm-postfix-tight">,</mo></mrow></fmath> and <fmath class="fm-inline"><mrow><msub><mi class="fm-mi-length-1 ma-upright" mathvariant="normal" style="padding-right: 0px;">Φ</mi><mn>4</mn></msub><mo class="fm-postfix-tight">.</mo></mrow></fmath> Then : }
\begin{enumerate}[label=(\alph*)]
\item  <fmath class="fm-inline"><mrow><mrow><msub><mi class="fm-mi-length-1 ma-upright" mathvariant="normal" style="padding-right: 0px;">Φ</mi><mn>1</mn></msub><mo class="fm-infix-loose"><</mo><mrow><msub><mi class="fm-mi-length-1 ma-upright" mathvariant="normal" style="padding-right: 0px;">Φ</mi><mn>2</mn></msub><mo class="fm-infix-loose">=</mo><msub><mi class="fm-mi-length-1 ma-upright" mathvariant="normal" style="padding-right: 0px;">Φ</mi><mn>3</mn></msub></mrow></mrow><mo class="fm-infix-loose">></mo><msub><mi class="fm-mi-length-1 ma-upright" mathvariant="normal" style="padding-right: 0px;">Φ</mi><mn>4</mn></msub></mrow></fmath>
\item  <fmath class="fm-inline"><mrow><mrow><mrow><msub><mi class="fm-mi-length-1 ma-upright" mathvariant="normal" style="padding-right: 0px;">Φ</mi><mn>1</mn></msub><mo class="fm-infix-loose">></mo><msub><mi class="fm-mi-length-1 ma-upright" mathvariant="normal" style="padding-right: 0px;">Φ</mi><mn>2</mn></msub></mrow><mo class="fm-infix-loose">></mo><msub><mi class="fm-mi-length-1 ma-upright" mathvariant="normal" style="padding-right: 0px;">Φ</mi><mn>3</mn></msub></mrow><mo class="fm-infix-loose">></mo><msub><mi class="fm-mi-length-1 ma-upright" mathvariant="normal" style="padding-right: 0px;">Φ</mi><mn>4</mn></msub></mrow></fmath>
\item  <fmath class="fm-inline"><mrow><mrow><mrow><msub><mi class="fm-mi-length-1 ma-upright" mathvariant="normal" style="padding-right: 0px;">Φ</mi><mn>1</mn></msub><mo class="fm-infix-loose">=</mo><msub><mi class="fm-mi-length-1 ma-upright" mathvariant="normal" style="padding-right: 0px;">Φ</mi><mn>2</mn></msub></mrow><mo class="fm-infix-loose">=</mo><msub><mi class="fm-mi-length-1 ma-upright" mathvariant="normal" style="padding-right: 0px;">Φ</mi><mn>3</mn></msub></mrow><mo class="fm-infix-loose">=</mo><msub><mi class="fm-mi-length-1 ma-upright" mathvariant="normal" style="padding-right: 0px;">Φ</mi><mn>4</mn></msub></mrow></fmath>
\item  <fmath class="fm-inline"><mrow><mrow><msub><mi class="fm-mi-length-1 ma-upright" mathvariant="normal" style="padding-right: 0px;">Φ</mi><mn>1</mn></msub><mo class="fm-infix-loose">></mo><msub><mi class="fm-mi-length-1 ma-upright" mathvariant="normal" style="padding-right: 0px;">Φ</mi><mn>3</mn></msub></mrow><mo class="fm-separator">;</mo><mrow><msub><mi class="fm-mi-length-1 ma-upright" mathvariant="normal" style="padding-right: 0px;">Φ</mi><mn>2</mn></msub><mo class="fm-infix-loose"><</mo><msub><mi class="fm-mi-length-1 ma-upright" mathvariant="normal" style="padding-right: 0px;">Φ</mi><mn>4</mn></msub></mrow></mrow></fmath>
\end{enumerate}
\newpage
\section*{Question 34}
<style>.fm-math,fmath{font-family:STIXGeneral,'DejaVu Serif','DejaVu Sans',Times,OpenSymbol,'Standard Symbols L',serif;line-height:1.2}.fm-math mtext,fmath mtext{line-height:normal}.fm-mo,.ma-sans-serif,fmath mi[mathvariant*=sans-serif],fmath mn[mathvariant*=sans-serif],fmath mo,fmath ms[mathvariant*=sans-serif],fmath mtext[mathvariant*=sans-serif]{font-family:STIXGeneral,'DejaVu Sans','DejaVu Serif','Arial Unicode MS','Lucida Grande',Times,OpenSymbol,'Standard Symbols L',sans-serif}.fm-mo-Luc{font-family:STIXGeneral,'DejaVu Sans','DejaVu Serif','Lucida Grande','Arial Unicode MS',Times,OpenSymbol,'Standard Symbols L',sans-serif}.questionsfont{font-weight:200;font-family:Arial, sans-serif, STIXGeneral,'DejaVu Sans','DejaVu Serif','Lucida Grande','Arial Unicode MS',Times,OpenSymbol,'Standard Symbols L',sans-serif!important}.fm-separator{padding:0 .56ex 0 0}.fm-infix-loose{padding:0 .56ex}.fm-infix{padding:0 .44ex}.fm-prefix{padding:0 .33ex 0 0}.fm-postfix{padding:0 0 0 .33ex}.fm-prefix-tight{padding:0 .11ex 0 0}.fm-postfix-tight{padding:0 0 0 .11ex}.fm-quantifier{padding:0 .11ex 0 .22ex}.ma-non-marking{display:none}.fm-vert,fmath menclose,menclose.fm-menclose{display:inline-block}.fm-large-op{font-size:1.3em}.fm-inline .fm-large-op{font-size:1em}fmath mrow{white-space:nowrap}.fm-vert{vertical-align:middle}fmath table,fmath tbody,fmath td,fmath tr{border:0!important;padding:0!important;margin:0!important;outline:0!important}fmath table{border-collapse:collapse!important;text-align:center!important;table-layout:auto!important;float:none!important}.fm-frac{padding:0 1px!important}td.fm-den-frac{border-top:solid thin!important}.fm-root{font-size:.6em}.fm-radicand{padding:0 1px 0 0;border-top:solid;margin-top:.1em}.fm-script{font-size:.71em}.fm-script .fm-script .fm-script{font-size:1em}td.fm-underover-base{line-height:1!important}td.fm-mtd{padding:.5ex .4em!important;vertical-align:baseline!important}fmath mphantom{visibility:hidden}fmath menclose[notation=top],menclose.fm-menclose[notation=top]{border-top:solid thin}fmath menclose[notation=right],menclose.fm-menclose[notation=right]{border-right:solid thin}fmath menclose[notation=bottom],menclose.fm-menclose[notation=bottom]{border-bottom:solid thin}fmath menclose[notation=left],menclose.fm-menclose[notation=left]{border-left:solid thin}fmath menclose[notation=box],menclose.fm-menclose[notation=box]{border:thin solid}fmath none{display:none}</style> The electric field in a region of space is given by, <fmath class="fm-inline"><mrow>\begin{tabular}{|c|c|}
\hline
<mo style="display: block; margin-top: -0.25em; margin-bottom: -0.25em;">→</mo> \\
\hline
<mi class="fm-mi-length-1" mathvariant="italic" style="padding-right: 0.44ex;">E</mi> \\
\hline
\end{tabular}
<mo class="fm-infix-loose">=</mo><mrow><mrow><msub><mi class="fm-mi-length-1" mathvariant="italic" style="padding-right: 0.44ex;">E</mi><mi class="fm-mi-length-1" mathvariant="italic">o</mi></msub>\begin{tabular}{|c|c|}
\hline
<mo style="display: block; margin-top: 0em; margin-bottom: -0.5em;">^</mo> \\
\hline
<mi class="fm-mi-length-1" mathvariant="italic">i</mi> \\
\hline
\end{tabular}
</mrow><mo class="fm-infix">+</mo><mrow><mrow><mn>2</mn><msub><mi class="fm-mi-length-1" mathvariant="italic" style="padding-right: 0.44ex;">E</mi><mi class="fm-mi-length-1" mathvariant="italic">o</mi></msub></mrow>\begin{tabular}{|c|c|}
\hline
<mo style="display: block; margin-top: 0em; margin-bottom: -0.5em;">^</mo> \\
\hline
<mi class="fm-mi-length-1" mathvariant="italic">j</mi> \\
\hline
\end{tabular}
</mrow></mrow></mrow></fmath> where <fmath class="fm-inline"><mrow><msub><mi class="fm-mi-length-1" mathvariant="italic" style="padding-right: 0.44ex;">E</mi><mi class="fm-mi-length-1" mathvariant="italic">o</mi></msub><mo class="fm-infix-loose">=</mo><mrow><mrow><mn>100</mn><mi class="fm-mi-length-1" mathvariant="italic" style="padding-right: 0.44ex;">N</mi></mrow><mo class="fm-infix-loose">∕</mo><mi class="fm-mi-length-1" mathvariant="italic">C</mi></mrow></mrow></fmath>. The flux of the field through a circular surface of radius 0.02 m parallel to the YZ plane is nearly: 
\begin{enumerate}[label=(\alph*)]
\item <fmath class="fm-inline"><mrow><mrow><mrow><mn>0.125</mn><mi class="fm-mi-length-1" mathvariant="italic" style="padding-right: 0.44ex;">N</mi></mrow><msup><mi class="fm-mi-length-1" mathvariant="italic">m</mi><mn>2</mn></msup></mrow><mo class="fm-infix-loose">∕</mo><mi class="fm-mi-length-1" mathvariant="italic">C</mi></mrow></fmath> 
\item  <fmath class="fm-inline"><mrow><mrow><mrow><mn>0.02</mn><mi class="fm-mi-length-1" mathvariant="italic" style="padding-right: 0.44ex;">N</mi></mrow><msup><mi class="fm-mi-length-1" mathvariant="italic">m</mi><mn>2</mn></msup></mrow><mo class="fm-infix-loose">∕</mo><mi class="fm-mi-length-1" mathvariant="italic">C</mi></mrow></fmath> 
\item  <fmath class="fm-inline"><mrow><mrow><mrow><mn>0.005</mn><mi class="fm-mi-length-1" mathvariant="italic" style="padding-right: 0.44ex;">N</mi></mrow><msup><mi class="fm-mi-length-1" mathvariant="italic">m</mi><mn>2</mn></msup></mrow><mo class="fm-infix-loose">∕</mo><mi class="fm-mi-length-1" mathvariant="italic">C</mi></mrow></fmath> 
\item  <fmath class="fm-inline"><mrow><mrow><mrow><mn>3.14</mn><mi class="fm-mi-length-1" mathvariant="italic" style="padding-right: 0.44ex;">N</mi></mrow><msup><mi class="fm-mi-length-1" mathvariant="italic">m</mi><mn>2</mn></msup></mrow><mo class="fm-infix-loose">∕</mo><mi class="fm-mi-length-1" mathvariant="italic">C</mi></mrow></fmath> 
\end{enumerate}
\newpage
\section*{Question 35}
<style>.fm-math,fmath{font-family:STIXGeneral,'DejaVu Serif','DejaVu Sans',Times,OpenSymbol,'Standard Symbols L',serif;line-height:1.2}.fm-math mtext,fmath mtext{line-height:normal}.fm-mo,.ma-sans-serif,fmath mi[mathvariant*=sans-serif],fmath mn[mathvariant*=sans-serif],fmath mo,fmath ms[mathvariant*=sans-serif],fmath mtext[mathvariant*=sans-serif]{font-family:STIXGeneral,'DejaVu Sans','DejaVu Serif','Arial Unicode MS','Lucida Grande',Times,OpenSymbol,'Standard Symbols L',sans-serif}.fm-mo-Luc{font-family:STIXGeneral,'DejaVu Sans','DejaVu Serif','Lucida Grande','Arial Unicode MS',Times,OpenSymbol,'Standard Symbols L',sans-serif}.questionsfont{font-weight:200;font-family:Arial, sans-serif, STIXGeneral,'DejaVu Sans','DejaVu Serif','Lucida Grande','Arial Unicode MS',Times,OpenSymbol,'Standard Symbols L',sans-serif!important}.fm-separator{padding:0 .56ex 0 0}.fm-infix-loose{padding:0 .56ex}.fm-infix{padding:0 .44ex}.fm-prefix{padding:0 .33ex 0 0}.fm-postfix{padding:0 0 0 .33ex}.fm-prefix-tight{padding:0 .11ex 0 0}.fm-postfix-tight{padding:0 0 0 .11ex}.fm-quantifier{padding:0 .11ex 0 .22ex}.ma-non-marking{display:none}.fm-vert,fmath menclose,menclose.fm-menclose{display:inline-block}.fm-large-op{font-size:1.3em}.fm-inline .fm-large-op{font-size:1em}fmath mrow{white-space:nowrap}.fm-vert{vertical-align:middle}fmath table,fmath tbody,fmath td,fmath tr{border:0!important;padding:0!important;margin:0!important;outline:0!important}fmath table{border-collapse:collapse!important;text-align:center!important;table-layout:auto!important;float:none!important}.fm-frac{padding:0 1px!important}td.fm-den-frac{border-top:solid thin!important}.fm-root{font-size:.6em}.fm-radicand{padding:0 1px 0 0;border-top:solid;margin-top:.1em}.fm-script{font-size:.71em}.fm-script .fm-script .fm-script{font-size:1em}td.fm-underover-base{line-height:1!important}td.fm-mtd{padding:.5ex .4em!important;vertical-align:baseline!important}fmath mphantom{visibility:hidden}fmath menclose[notation=top],menclose.fm-menclose[notation=top]{border-top:solid thin}fmath menclose[notation=right],menclose.fm-menclose[notation=right]{border-right:solid thin}fmath menclose[notation=bottom],menclose.fm-menclose[notation=bottom]{border-bottom:solid thin}fmath menclose[notation=left],menclose.fm-menclose[notation=left]{border-left:solid thin}fmath menclose[notation=box],menclose.fm-menclose[notation=box]{border:thin solid}fmath none{display:none}</style> The magnitude of the average electric field normally present in the atmosphere just above the surface of the Earth is about <fmath class="fm-inline"><mrow><mrow><mn>150</mn><mi class="fm-mi-length-1" mathvariant="italic" style="padding-right: 0.44ex;">N</mi></mrow><mo class="fm-infix-loose">∕</mo><mi class="fm-mi-length-1" mathvariant="italic">C</mi></mrow></fmath> , directed inward towards the center of the Earth. This gives the total net surface charge carried by the Earth to be: \newline [Given <fmath class="fm-inline"><mrow><mrow><msub><mi class="fm-mi-length-1 ma-upright" mathvariant="normal" style="padding-right: 0px;">ε</mi><mn>0</mn></msub><mo class="fm-infix-loose">=</mo><mrow><mrow><mn>8.85</mn><mo class="fm-infix" lspace=".22em" rspace=".22em">×</mo><mrow><msup><mn>10</mn><mrow><mo class="fm-prefix-tight">−</mo><mn>12</mn></mrow></msup><msup><mi class="fm-mi-length-1" mathvariant="italic">C</mi><mn>2</mn></msup></mrow></mrow><mo class="fm-infix-loose">∕</mo><mrow><mi class="fm-mi-length-1" mathvariant="italic" style="padding-right: 0.44ex;">N</mi><mo class="fm-infix">−</mo><msup><mi class="fm-mi-length-1" mathvariant="italic">m</mi><mn>2</mn></msup></mrow></mrow></mrow><mo class="fm-postfix-tight">,</mo></mrow></fmath> <fmath class="fm-inline"><mrow class="ma-repel-adj"><mrow><msub><mi class="fm-mi-length-1" mathvariant="italic">R</mi><mi class="fm-mi-length-1" mathvariant="italic" style="padding-right: 0.44ex;">E</mi></msub><mo class="fm-infix-loose">=</mo><mrow><mn>6.37</mn><mo class="fm-infix" lspace=".22em" rspace=".22em">×</mo><mrow><msup><mn>10</mn><mn>6</mn></msup><mi class="fm-mi-length-1" mathvariant="italic">m</mi></mrow></mrow></mrow><mo class="fm-mo-Luc fm-postfix">]</mo></mrow></fmath> }
\begin{enumerate}[label=(\alph*)]
\item  <fmath class="fm-inline"><mrow><mo class="fm-prefix-tight">+</mo><mrow><mn>670</mn><mrow><mi class="fm-mi-length-1" mathvariant="italic">k</mi><mi class="fm-mi-length-1" mathvariant="italic">C</mi></mrow></mrow></mrow></fmath>
\item  <fmath class="fm-inline"><mrow><mo class="fm-prefix-tight">−</mo><mrow><mn>670</mn><mrow><mi class="fm-mi-length-1" mathvariant="italic">k</mi><mi class="fm-mi-length-1" mathvariant="italic">C</mi></mrow></mrow></mrow></fmath>
\item  <fmath class="fm-inline"><mrow><mo class="fm-prefix-tight">−</mo><mrow><mn>680</mn><mrow><mi class="fm-mi-length-1" mathvariant="italic">k</mi><mi class="fm-mi-length-1" mathvariant="italic">C</mi></mrow></mrow></mrow></fmath>
\item  <fmath class="fm-inline"><mrow><mo class="fm-prefix-tight">+</mo><mrow><mn>680</mn><mrow><mi class="fm-mi-length-1" mathvariant="italic">k</mi><mi class="fm-mi-length-1" mathvariant="italic">C</mi></mrow></mrow></mrow></fmath>
\end{enumerate}
\newpage
\section*{Question 36}
<style>.fm-math,fmath{font-family:STIXGeneral,'DejaVu Serif','DejaVu Sans',Times,OpenSymbol,'Standard Symbols L',serif;line-height:1.2}.fm-math mtext,fmath mtext{line-height:normal}.fm-mo,.ma-sans-serif,fmath mi[mathvariant*=sans-serif],fmath mn[mathvariant*=sans-serif],fmath mo,fmath ms[mathvariant*=sans-serif],fmath mtext[mathvariant*=sans-serif]{font-family:STIXGeneral,'DejaVu Sans','DejaVu Serif','Arial Unicode MS','Lucida Grande',Times,OpenSymbol,'Standard Symbols L',sans-serif}.fm-mo-Luc{font-family:STIXGeneral,'DejaVu Sans','DejaVu Serif','Lucida Grande','Arial Unicode MS',Times,OpenSymbol,'Standard Symbols L',sans-serif}.questionsfont{font-weight:200;font-family:Arial, sans-serif, STIXGeneral,'DejaVu Sans','DejaVu Serif','Lucida Grande','Arial Unicode MS',Times,OpenSymbol,'Standard Symbols L',sans-serif!important}.fm-separator{padding:0 .56ex 0 0}.fm-infix-loose{padding:0 .56ex}.fm-infix{padding:0 .44ex}.fm-prefix{padding:0 .33ex 0 0}.fm-postfix{padding:0 0 0 .33ex}.fm-prefix-tight{padding:0 .11ex 0 0}.fm-postfix-tight{padding:0 0 0 .11ex}.fm-quantifier{padding:0 .11ex 0 .22ex}.ma-non-marking{display:none}.fm-vert,fmath menclose,menclose.fm-menclose{display:inline-block}.fm-large-op{font-size:1.3em}.fm-inline .fm-large-op{font-size:1em}fmath mrow{white-space:nowrap}.fm-vert{vertical-align:middle}fmath table,fmath tbody,fmath td,fmath tr{border:0!important;padding:0!important;margin:0!important;outline:0!important}fmath table{border-collapse:collapse!important;text-align:center!important;table-layout:auto!important;float:none!important}.fm-frac{padding:0 1px!important}td.fm-den-frac{border-top:solid thin!important}.fm-root{font-size:.6em}.fm-radicand{padding:0 1px 0 0;border-top:solid;margin-top:.1em}.fm-script{font-size:.71em}.fm-script .fm-script .fm-script{font-size:1em}td.fm-underover-base{line-height:1!important}td.fm-mtd{padding:.5ex .4em!important;vertical-align:baseline!important}fmath mphantom{visibility:hidden}fmath menclose[notation=top],menclose.fm-menclose[notation=top]{border-top:solid thin}fmath menclose[notation=right],menclose.fm-menclose[notation=right]{border-right:solid thin}fmath menclose[notation=bottom],menclose.fm-menclose[notation=bottom]{border-bottom:solid thin}fmath menclose[notation=left],menclose.fm-menclose[notation=left]{border-left:solid thin}fmath menclose[notation=box],menclose.fm-menclose[notation=box]{border:thin solid}fmath none{display:none}</style> A charge <fmath class="fm-inline"><mi class="fm-mi-length-1" mathvariant="italic">Q</mi></fmath> is placed at each of the opposite corners of a square. A charge <fmath class="fm-inline"><mi class="fm-mi-length-1" mathvariant="italic">q</mi></fmath> is placed at each of the other two corners. If the net electrical force on <fmath class="fm-inline"><mi class="fm-mi-length-1" mathvariant="italic">Q</mi></fmath> is zero, then <fmath class="fm-inline"><mrow><mi class="fm-mi-length-1" mathvariant="italic">Q</mi><mo class="fm-infix-loose">∕</mo><mi class="fm-mi-length-1" mathvariant="italic">q</mi></mrow></fmath> equals: }
\begin{enumerate}[label=(\alph*)]
\item  -1
\item  1
\item  <fmath class="fm-inline"><mrow><mo class="fm-prefix-tight">−</mo>\begin{tabular}{|c|c|}
\hline
<mn>1</mn> \\
\hline
<mrow mtagname="msqrt"><mo class="fm-radic">√</mo><mn>2</mn></mrow> \\
\hline
\end{tabular}
</mrow></fmath>
\item  <fmath class="fm-inline"><mrow><mo class="fm-prefix-tight">−</mo><mrow><mn>2</mn><mrow mtagname="msqrt"><mo class="fm-radic">√</mo><mn>2</mn></mrow></mrow></mrow></fmath>
\end{enumerate}
\newpage
\section*{Question 37}
<style>.fm-math,fmath{font-family:STIXGeneral,'DejaVu Serif','DejaVu Sans',Times,OpenSymbol,'Standard Symbols L',serif;line-height:1.2}.fm-math mtext,fmath mtext{line-height:normal}.fm-mo,.ma-sans-serif,fmath mi[mathvariant*=sans-serif],fmath mn[mathvariant*=sans-serif],fmath mo,fmath ms[mathvariant*=sans-serif],fmath mtext[mathvariant*=sans-serif]{font-family:STIXGeneral,'DejaVu Sans','DejaVu Serif','Arial Unicode MS','Lucida Grande',Times,OpenSymbol,'Standard Symbols L',sans-serif}.fm-mo-Luc{font-family:STIXGeneral,'DejaVu Sans','DejaVu Serif','Lucida Grande','Arial Unicode MS',Times,OpenSymbol,'Standard Symbols L',sans-serif}.questionsfont{font-weight:200;font-family:Arial, sans-serif, STIXGeneral,'DejaVu Sans','DejaVu Serif','Lucida Grande','Arial Unicode MS',Times,OpenSymbol,'Standard Symbols L',sans-serif!important}.fm-separator{padding:0 .56ex 0 0}.fm-infix-loose{padding:0 .56ex}.fm-infix{padding:0 .44ex}.fm-prefix{padding:0 .33ex 0 0}.fm-postfix{padding:0 0 0 .33ex}.fm-prefix-tight{padding:0 .11ex 0 0}.fm-postfix-tight{padding:0 0 0 .11ex}.fm-quantifier{padding:0 .11ex 0 .22ex}.ma-non-marking{display:none}.fm-vert,fmath menclose,menclose.fm-menclose{display:inline-block}.fm-large-op{font-size:1.3em}.fm-inline .fm-large-op{font-size:1em}fmath mrow{white-space:nowrap}.fm-vert{vertical-align:middle}fmath table,fmath tbody,fmath td,fmath tr{border:0!important;padding:0!important;margin:0!important;outline:0!important}fmath table{border-collapse:collapse!important;text-align:center!important;table-layout:auto!important;float:none!important}.fm-frac{padding:0 1px!important}td.fm-den-frac{border-top:solid thin!important}.fm-root{font-size:.6em}.fm-radicand{padding:0 1px 0 0;border-top:solid;margin-top:.1em}.fm-script{font-size:.71em}.fm-script .fm-script .fm-script{font-size:1em}td.fm-underover-base{line-height:1!important}td.fm-mtd{padding:.5ex .4em!important;vertical-align:baseline!important}fmath mphantom{visibility:hidden}fmath menclose[notation=top],menclose.fm-menclose[notation=top]{border-top:solid thin}fmath menclose[notation=right],menclose.fm-menclose[notation=right]{border-right:solid thin}fmath menclose[notation=bottom],menclose.fm-menclose[notation=bottom]{border-bottom:solid thin}fmath menclose[notation=left],menclose.fm-menclose[notation=left]{border-left:solid thin}fmath menclose[notation=box],menclose.fm-menclose[notation=box]{border:thin solid}fmath none{display:none}</style> This question contains Statement- 1 and Statement- 2 . Of the four choices given after the statements, choose the one that best describes the two statements. \newline For a charged particle moving from point <fmath class="fm-inline"><mi class="fm-mi-length-1" mathvariant="italic">P</mi></fmath> to point <fmath class="fm-inline"><mrow><mi class="fm-mi-length-1" mathvariant="italic">Q</mi><mo class="fm-postfix-tight">,</mo></mrow></fmath> the net work done by an electrostatic field on the particle is independent of the path connecting point <fmath class="fm-inline"><mi class="fm-mi-length-1" mathvariant="italic">P</mi></fmath> to point <fmath class="fm-inline"><mi class="fm-mi-length-1" mathvariant="italic">Q</mi></fmath>. \newline  The net work done by a conservative force on an object moving along a closed loop is zero. }
\begin{enumerate}[label=(\alph*)]
\item  Statement- 1 is true, Statement- 2 is true; Statement-2 is the correct explanation of Statement- 1 . 
\item  Statement- 1 is true, Statement- 2 is true; Statement-2 is not the correct explanation of Statement- 1 . 
\item  Statement- 1 is false, Statement- 2 is true. 
\item  Statement- 1 is true, Statement- 2 is false.
\end{enumerate}
\newpage
\section*{Question 38}
<style>.fm-math,fmath{font-family:STIXGeneral,'DejaVu Serif','DejaVu Sans',Times,OpenSymbol,'Standard Symbols L',serif;line-height:1.2}.fm-math mtext,fmath mtext{line-height:normal}.fm-mo,.ma-sans-serif,fmath mi[mathvariant*=sans-serif],fmath mn[mathvariant*=sans-serif],fmath mo,fmath ms[mathvariant*=sans-serif],fmath mtext[mathvariant*=sans-serif]{font-family:STIXGeneral,'DejaVu Sans','DejaVu Serif','Arial Unicode MS','Lucida Grande',Times,OpenSymbol,'Standard Symbols L',sans-serif}.fm-mo-Luc{font-family:STIXGeneral,'DejaVu Sans','DejaVu Serif','Lucida Grande','Arial Unicode MS',Times,OpenSymbol,'Standard Symbols L',sans-serif}.questionsfont{font-weight:200;font-family:Arial, sans-serif, STIXGeneral,'DejaVu Sans','DejaVu Serif','Lucida Grande','Arial Unicode MS',Times,OpenSymbol,'Standard Symbols L',sans-serif!important}.fm-separator{padding:0 .56ex 0 0}.fm-infix-loose{padding:0 .56ex}.fm-infix{padding:0 .44ex}.fm-prefix{padding:0 .33ex 0 0}.fm-postfix{padding:0 0 0 .33ex}.fm-prefix-tight{padding:0 .11ex 0 0}.fm-postfix-tight{padding:0 0 0 .11ex}.fm-quantifier{padding:0 .11ex 0 .22ex}.ma-non-marking{display:none}.fm-vert,fmath menclose,menclose.fm-menclose{display:inline-block}.fm-large-op{font-size:1.3em}.fm-inline .fm-large-op{font-size:1em}fmath mrow{white-space:nowrap}.fm-vert{vertical-align:middle}fmath table,fmath tbody,fmath td,fmath tr{border:0!important;padding:0!important;margin:0!important;outline:0!important}fmath table{border-collapse:collapse!important;text-align:center!important;table-layout:auto!important;float:none!important}.fm-frac{padding:0 1px!important}td.fm-den-frac{border-top:solid thin!important}.fm-root{font-size:.6em}.fm-radicand{padding:0 1px 0 0;border-top:solid;margin-top:.1em}.fm-script{font-size:.71em}.fm-script .fm-script .fm-script{font-size:1em}td.fm-underover-base{line-height:1!important}td.fm-mtd{padding:.5ex .4em!important;vertical-align:baseline!important}fmath mphantom{visibility:hidden}fmath menclose[notation=top],menclose.fm-menclose[notation=top]{border-top:solid thin}fmath menclose[notation=right],menclose.fm-menclose[notation=right]{border-right:solid thin}fmath menclose[notation=bottom],menclose.fm-menclose[notation=bottom]{border-bottom:solid thin}fmath menclose[notation=left],menclose.fm-menclose[notation=left]{border-left:solid thin}fmath menclose[notation=box],menclose.fm-menclose[notation=box]{border:thin solid}fmath none{display:none}</style> Two point charges <fmath class="fm-inline"><mrow><mo class="fm-prefix-tight">+</mo><mrow><mn>8</mn><mi class="fm-mi-length-1" mathvariant="italic">q</mi></mrow></mrow></fmath> and <fmath class="fm-inline"><mrow><mo class="fm-prefix-tight">−</mo><mrow><mn>2</mn><mi class="fm-mi-length-1" mathvariant="italic">q</mi></mrow></mrow></fmath> are located at <fmath class="fm-inline"><mrow><mi class="fm-mi-length-1" mathvariant="italic">x</mi><mo class="fm-infix-loose">=</mo><mn>0</mn></mrow></fmath> and <fmath class="fm-inline"><mrow><mi class="fm-mi-length-1" mathvariant="italic">x</mi><mo class="fm-infix-loose">=</mo><mi class="fm-mi-length-1" mathvariant="italic">L</mi></mrow></fmath> respectively. The location of a point on the <fmath class="fm-inline"><mi class="fm-mi-length-1" mathvariant="italic">x</mi></fmath> axis at which the net electric field due to these two point charges is zero is }
\begin{enumerate}[label=(\alph*)]
\item  <fmath class="fm-inline">\begin{tabular}{|c|c|}
\hline
<mi class="fm-mi-length-1" mathvariant="italic">L</mi> \\
\hline
<mn>4</mn> \\
\hline
\end{tabular}
</fmath>
\item  <fmath class="fm-inline"><mrow><mn>2</mn><mi class="fm-mi-length-1" mathvariant="italic">L</mi></mrow></fmath>
\item  <fmath class="fm-inline"><mrow><mn>4</mn><mi class="fm-mi-length-1" mathvariant="italic">L</mi></mrow></fmath>
\item  <fmath class="fm-inline"><mrow><mn>8</mn><mi class="fm-mi-length-1" mathvariant="italic">L</mi></mrow></fmath>
\end{enumerate}
\newpage
\section*{Question 39}
Two spherical conductors B and C having equal radii and carrying equal charges on them repel each other with a force F when kept apart at some distance. A third spherical conductor having same radius as that B but  is brought in contact with B, then brought in contact with C and finally removed away from both. The new force of repulsion between B and C is }
\begin{enumerate}[label=(\alph*)]
\item  F/8 
\item  3F/4 
\item  F/4 
\item  3F/8
\end{enumerate}
\newpage
\section*{Question 40}
A parallel plate capacitor is charged from a battery and then isolated from it. What will happen if the separation between the plates of a capacitor is increased?
\begin{enumerate}[label=(\alph*)]
\item The potential difference between the plates will not change.
\item The potential difference between the plates will decrease.
\item The field remains constant between the plates.
\item The capacitance of the plates will increase.
\end{enumerate}
\newpage
\section*{Question 41}
A parallel combination of \(0.1 \mathrm{M} \Omega\) resistor and a \(10 \mu \mathrm{F}\) capacitor is connected across a \(1.5 \mathrm{~V}\) source of negligible resistance. The time (in seconds) required for the capacitor to get charged up to \(0.75 \mathrm{~V}\) is approximately:\includegraphics[width=\textwidth]{https://testseries.edugorilla.com/static/media/wl_client/1/qdump/e55a342d057a52acc35b29ffdedd2ff4/101770b2a3d833e1bb8cb242017d2844.png}
\begin{enumerate}[label=(\alph*)]
\item \(\infty\)
\item \(\log _e 2\)
\item \(\log _{10} 2\)
\item Zero
\end{enumerate}
\newpage
\section*{Question 42}
Two charges \(5 \times 10^{-8} \mathrm{C}\) and \(-3 \times 10^{-8} \mathrm{C}\) are located \(16 \mathrm{~cm}\) apart. At what point(s) on the line joining the two charges is the electric potential zero? 
\begin{enumerate}[label=(\alph*)]
\item 10 cm
\item 5 cm
\item 2 cm
\item 9 cm
\end{enumerate}
\newpage
\section*{Question 43}
A \(12 \mathrm{pF}\) capacitor is connected to a \(50 \mathrm{~V}\) battery. How much electrostatic energy is stored in the capacitor?
\begin{enumerate}[label=(\alph*)]
\item \(3 \times 10^{-8} \mathrm{~J}\)
\item \(1.5 \times 10^{-3} \mathrm{~J}\)
\item \(2 \times 10^{-8} \mathrm{~J}\)
\item \(1.5 \times 10^{-8} \mathrm{~J}\)
\end{enumerate}
\newpage
\section*{Question 44}
The potential difference across \(3 \mu \mathrm{F}\) condenser is :\includegraphics[width=\textwidth]{static/media/wl_client/1/qdump/e55a342d057a52acc35b29ffdedd2ff4/be8eceadb49f9f4f26f86618d0adfc14.png}
\begin{enumerate}[label=(\alph*)]
\item 40 V
\item 60 V
\item 70 V
\item 80 V
\end{enumerate}
\newpage
\section*{Question 45}
\(4 \mu \mathrm{F}, 6 \mu \mathrm{F}\) and \(12 \mu \mathrm{F}\) condensers are in series across \(90 \mathrm{~V}\). The potential difference across \(12 \mu \mathrm{F}\) condenser is:\newline
\begin{enumerate}[label=(\alph*)]
\item 30 V
\item 90 V
\item 15 V
\item 45 V
\end{enumerate}
\newpage
\section*{Question 46}
In a Van de Graaff type generator a spherical metal shell is to be a \(15 \times 10^6 \mathrm{~V}\) electrode. The dielectric strength of the gas surrounding the electrode is \(5 \times 10^7 \mathrm{Vm}^{-1}\). What is the minimum radius of the spherical shell required?\newline
\begin{enumerate}[label=(\alph*)]
\item \(40 \mathrm{~cm}\)
\item \(30 \mathrm{~cm}\)
\item \(50 \mathrm{~cm}\)
\item \(60 \mathrm{~cm}\)
\end{enumerate}
\newpage
\section*{Question 47}
Capacitance of a capacitor becomes \(\frac{7}{6}\) times its original value, if a dielectric slab of thickness \(t=\frac{2}{3} d\) is introduced in between the plates. The dielectic constant of the dielectric slab is :
\begin{enumerate}[label=(\alph*)]
\item \(\frac{14}{11}\)
\item \(\frac{11}{14}\)
\item \(\frac{7}{11}\)
\item \(\frac{11}{7}\)
\end{enumerate}
\newpage
\section*{Question 48}
 When a 3 µC of charge is carried from point A to point B, the amount of work done by electric field is 81 µJ. Determine the potential difference?
\begin{enumerate}[label=(\alph*)]
\item 100 V
\item 2.7 V
\item 225 V
\item 27 V
\end{enumerate}
\newpage
\section*{Question 49}
In a hydrogen atom, the electron and proton are bound at a distance of about \(\mathrm{d}=0.53 \) Å:Estimate the potential energy of the system in \(\mathrm{eV}\), taking the zero of the potential energy at infinite separation of the electron from proton.
\begin{enumerate}[label=(\alph*)]
\item \(27.2 \mathrm{eV}\)
\item \(-26.2 \mathrm{eV}\)
\item \(-27.2 \mathrm{eV}\)
\item \(-96.2 \mathrm{eV}\)
\end{enumerate}
\newpage
\section*{Question 50}
What is the area of the plates of a 2 F parallel plate capacitor, given that the separation between the plates is 0.5 cm?
\begin{enumerate}[label=(\alph*)]
\item \(130 \mathrm{Km}^2\)
\item \(230 \mathrm{Km}^2\)
\item \(9500 \mathrm{Km}^2\)
\item \(1130 \mathrm{Km}^2\)
\end{enumerate}
\newpage
\section*{Question 51}
Two uncharged capacitors of capacitance \(8 \mu \mathrm{F}\) and \(2 \mu \mathrm{F}\) are connected in series across a 100volt d.c. supply. Now if the supply voltage is removed and the capacitors are connected such that similar terminals are connected together. Then the final charge on \(8 \mu \mathrm{F}\) capacitor is equal to (in \(\mu \mathrm{C}\) ):\newline
\begin{enumerate}[label=(\alph*)]
\item 64 \(\mu \mathrm{C}\)
\item 256 \(\mu \mathrm{C}\)
\item 192 \(\mu \mathrm{C}\)
\item 32 \(\mu \mathrm{C}\)
\end{enumerate}
\newpage
\section*{Question 52}
Three capacitors \(3 \mu \mathrm{F}, 10 \mu \mathrm{F}\) and \(15 \mu \mathrm{F}\) are connected in series to a voltage source of 100V. The charge on \(15 \mu \mathrm{F}\) is:
\begin{enumerate}[label=(\alph*)]
\item \(22 \mu \mathrm{C}\)
\item \(100\mathrm{CC}\)
\item \(2800 \mu \mathrm{C}\)
\item \(200 \mu \mathrm{C}\)
\end{enumerate}
\newpage
\section*{Question 53}
The area of the positive plate and the negative plate is \(100 \mathrm{~cm}^2\). They are parallel to each other and are separated by \(0.5 \mathrm{~cm}\). The capacity of a condenser with air as dielectric is :\(\left(\varepsilon_0=8.9 \times 10^{-12} \mathrm{C}^2 \mathrm{~N}^{-1} \mathrm{M}^{-2}\right)\)\newline
\begin{enumerate}[label=(\alph*)]
\item \(22.25 \mathrm{pF}\)
\item \( 20.02 \mathrm{pF}\)
\item \(17.8 \mu \mathrm{F}\)
\item \(17.8 \mathrm{pF}\)
\end{enumerate}
\newpage
\section*{Question 54}
What is the area of the plates of a 4 F parallel plate capacitor, given that the separation between the plates is 0.5 cm?
\begin{enumerate}[label=(\alph*)]
\item \(130  \times 10^6\)  m$^{2}$
\item \(230  \times 10^6\) m$^{2}$
\item \(9500  \times 10^6\) m$^{2}$
\item \(2258  \times 10^6\) m$^{2}$
\end{enumerate}
\newpage
\section*{Question 55}
The work done against electrostatic force gets stored in which form of energy?
\begin{enumerate}[label=(\alph*)]
\item Thermal energy\newline
\item Kinetic energy
\item Potential energy\newline
\item Solar energy
\end{enumerate}
\newpage
\section*{Question 56}
Two isolated metallic spheres, one with a radius R and another with a radius 5R, each carries a charge ‘q’ uniformly distributed over the entire surface. Which sphere stores more electric potential energy?
\begin{enumerate}[label=(\alph*)]
\item Initially it will be the sphere with radius 5R then it will shift to the sphere with radius R\newline
\item The sphere with radius R
\item Both of the spheres will have the same energy
\item The sphere with radius 5R\newline
\end{enumerate}
\newpage
\section*{Question 57}
Two identical capacitors having plate separation \({d}_{0}\) are connected parallel to each other across the points \(A\) and \(B\) as shown in figure. \(A\) charge \(Q\) is imparted to the system by connecting a battery across \(A\) and \(B\) and the battery is removed. Now the first plate of the first capacitor and the second plate of the second capacitor start moving with constant velocity \(u_{0}\) toward left. The magnitude of the current flowing in the loop during this process is given as \(\frac{{Qu}_{0}}{{xd}_{0}}\). Find \({x}\):\includegraphics[width=\textwidth]{https://testseries.edugorilla.com/static/media/wl_client/1/qdump/50272f977f0f492a38d073f6f7507847/0302e57c82b25ea92370e76e26c6f879.png}
\begin{enumerate}[label=(\alph*)]
\item \(2\)
\item \(3\)\newline
\item \(4\)\newline
\item \(5\)\newline
\end{enumerate}
\newpage
\section*{Question 58}
The device used for measuring potential difference is known as:
\begin{enumerate}[label=(\alph*)]
\item Voltmeter\newline
\item Electrometer
\item Van de Graff generator
\item Electroscope
\end{enumerate}
\newpage
\section*{Question 59}
Two charges \(5 \times 10^{-8} {C}\) and \(-3 \times 10^{-8} {C}\) are located \(0.16 {~m}\) apart. At what point(s) on the line joining the two charges in the electric potential zero? Take the potential at infinity to be zero.
\begin{enumerate}[label=(\alph*)]
\item \(24\)
\item \(28\)\newline
\item \(20\)\newline
\item \(25\)\newline
\end{enumerate}
\newpage
\section*{Question 60}
Van de Graaf generator is used for:
\begin{enumerate}[label=(\alph*)]
\item accelerating charged particles
\item generating large currents
\item generating electric field
\item generating high-frequency voltage
\end{enumerate}
\newpage
\end{document}