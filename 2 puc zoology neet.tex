\documentclass{article}
                    \usepackage{amsmath}
                    \usepackage{amssymb}
                    \usepackage{graphicx}
                    \usepackage{enumitem}
                    \usepackage{longtable}
                    \title{2 puc zoology neet}
                    \begin{document}
                    \maketitle
                    \section*{Question 1}
Which of the following is/are not associated with the human male reproductive system?
\begin{enumerate}[label=(\alph*)]
\item Prostate gland
\item Seminal vesicle
\item Cowper's gland
\item Bartholin's gland
\end{enumerate}
\newpage
\section*{Question 2}
In mammals the estrogens are secreted by the Graafian follicle from its:
\begin{enumerate}[label=(\alph*)]
\item External theca
\item Internal theca
\item Zona pellucida
\item Corona radiata
\end{enumerate}
\newpage
\section*{Question 3}
The endometrium in female is the lining of:
\begin{enumerate}[label=(\alph*)]
\item Uterus
\item Vagina
\item Ovary
\item Bladder
\end{enumerate}
\newpage
\section*{Question 4}
The part of fallopian tube closest to ovary is:
\begin{enumerate}[label=(\alph*)]
\item Ampulla
\item Isthmus
\item Cervix
\item Infundibulum
\end{enumerate}
\newpage
\section*{Question 5}
Several hormones like hCG, hPL, Oestrogen, Progesterone are produced by:
\begin{enumerate}[label=(\alph*)]
\item Ovary
\item Placenta
\item Fallopian tube
\item Pituitary
\end{enumerate}
\newpage
\section*{Question 6}
Hugo de Vries did his experiments on the plant :
\begin{enumerate}[label=(\alph*)]
\item Pisumsativum
\item Oenothera lamarkiana
\item Mirabilis jalappa
\item Oryza sativa
\end{enumerate}
\newpage
\section*{Question 7}
When is it possible for a woman to be colorblind?
\begin{enumerate}[label=(\alph*)]
\item The father has normal vision and the mother is a carrier
\item The father has normal vision and the mother is colorblind
\item The father is colorblind and the mother has a normal vision
\item The father is colorblind and the mother is a carrier
\end{enumerate}
\newpage
\section*{Question 8}
Which of the following is a cause of thalassemia?
\begin{enumerate}[label=(\alph*)]
\item RBC defects
\item WBC defects
\item Platelet defects
\item Lymphocyte defects
\end{enumerate}
\newpage
\section*{Question 9}
Which one of the following is an example for X-linked chromosome inheritance?
\begin{enumerate}[label=(\alph*)]
\item Colour Blindness
\item Albinism
\item Alkaptonuria
\item Mangolism
\end{enumerate}
\newpage
\section*{Question 10}
What was the type of pea lines used by Mendel?
\begin{enumerate}[label=(\alph*)]
\item True-breeding
\item False-breeding
\item Not breeding
\item Indefinitely breeding
\end{enumerate}
\newpage
\section*{Question 11}
<style>.fm-math,fmath{font-family:STIXGeneral,'DejaVu Serif','DejaVu Sans',Times,OpenSymbol,'Standard Symbols L',serif;line-height:1.2}.fm-math mtext,fmath mtext{line-height:normal}.fm-mo,.ma-sans-serif,fmath mi[mathvariant*=sans-serif],fmath mn[mathvariant*=sans-serif],fmath mo,fmath ms[mathvariant*=sans-serif],fmath mtext[mathvariant*=sans-serif]{font-family:STIXGeneral,'DejaVu Sans','DejaVu Serif','Arial Unicode MS','Lucida Grande',Times,OpenSymbol,'Standard Symbols L',sans-serif}.fm-mo-Luc{font-family:STIXGeneral,'DejaVu Sans','DejaVu Serif','Lucida Grande','Arial Unicode MS',Times,OpenSymbol,'Standard Symbols L',sans-serif}.questionsfont{font-weight:200;font-family:Arial, sans-serif, STIXGeneral,'DejaVu Sans','DejaVu Serif','Lucida Grande','Arial Unicode MS',Times,OpenSymbol,'Standard Symbols L',sans-serif!important}.fm-separator{padding:0 .56ex 0 0}.fm-infix-loose{padding:0 .56ex}.fm-infix{padding:0 .44ex}.fm-prefix{padding:0 .33ex 0 0}.fm-postfix{padding:0 0 0 .33ex}.fm-prefix-tight{padding:0 .11ex 0 0}.fm-postfix-tight{padding:0 0 0 .11ex}.fm-quantifier{padding:0 .11ex 0 .22ex}.ma-non-marking{display:none}.fm-vert,fmath menclose,menclose.fm-menclose{display:inline-block}.fm-large-op{font-size:1.3em}.fm-inline .fm-large-op{font-size:1em}fmath mrow{white-space:nowrap}.fm-vert{vertical-align:middle}fmath table,fmath tbody,fmath td,fmath tr{border:0!important;padding:0!important;margin:0!important;outline:0!important}fmath table{border-collapse:collapse!important;text-align:center!important;table-layout:auto!important;float:none!important}.fm-frac{padding:0 1px!important}td.fm-den-frac{border-top:solid thin!important}.fm-root{font-size:.6em}.fm-radicand{padding:0 1px 0 0;border-top:solid;margin-top:.1em}.fm-script{font-size:.71em}.fm-script .fm-script .fm-script{font-size:1em}td.fm-underover-base{line-height:1!important}td.fm-mtd{padding:.5ex .4em!important;vertical-align:baseline!important}fmath mphantom{visibility:hidden}fmath menclose[notation=top],menclose.fm-menclose[notation=top]{border-top:solid thin}fmath menclose[notation=right],menclose.fm-menclose[notation=right]{border-right:solid thin}fmath menclose[notation=bottom],menclose.fm-menclose[notation=bottom]{border-bottom:solid thin}fmath menclose[notation=left],menclose.fm-menclose[notation=left]{border-left:solid thin}fmath menclose[notation=box],menclose.fm-menclose[notation=box]{border:thin solid}fmath none{display:none}</style> At a particular locus, frequency of <fmath class="fm-inline"><mi class="fm-mi-length-1" mathvariant="italic">A</mi></fmath> allele is 0.6 and that of <fmath class="fm-inline"><mi class="fm-mi-length-1" mathvariant="italic">a</mi></fmath> is <fmath class="fm-inline"><mrow><mn>0.4</mn><mo class="fm-postfix-tight">.</mo></mrow></fmath> What would be the frequency of heterozygotes in a random mating population at equilibrium? 
\begin{enumerate}[label=(\alph*)]
\item  0.36
\item  0.16
\item  0.24 
\item  0.48
\end{enumerate}
\newpage
\section*{Question 12}
Industrial melanism is an example of 
\begin{enumerate}[label=(\alph*)]
\item  drug resistance
\item  darkening of skin due to smoke from industries
\item  protective resemblance with the surroundings
\item  defensive adaptation of skin against ultraviolet radiations.
\end{enumerate}
\newpage
\section*{Question 13}
Which is not a vestigial organ in man? 
\begin{enumerate}[label=(\alph*)]
\item  Third molar
\item  Nails
\item  Segmental muscles of abdomen
\item  Coccyx
\end{enumerate}
\newpage
\section*{Question 14}
Study of fossils is 
\begin{enumerate}[label=(\alph*)]
\item  palaeontology
\item  herpetology
\item  saurology
\item  organic evolution
\end{enumerate}
\newpage
\section*{Question 15}
"Continuity of germplasm" theory was given by 
\begin{enumerate}[label=(\alph*)]
\item  De Vries
\item  Weismann
\item  Darwin
\item  Lamarck
\end{enumerate}
\newpage
\section*{Question 16}
'Origin of Species was written by 
\begin{enumerate}[label=(\alph*)]
\item  Oparin
\item  Weismann
\item  Lamarck
\item  Darwin.
\end{enumerate}
\newpage
\section*{Question 17}
All the following interaction are mutualism, except:
\begin{enumerate}[label=(\alph*)]
\item Association of algae and fungi in lichens.
\item Association of fungi and roots of higher plants in mycorrhiza.
\item Plant and animal relation for pollination.
\item Association of cattle egret and grazing cattle.
\end{enumerate}
\newpage
\section*{Question 18}
A mycorrhiza exhibits what kind of interaction?
\begin{enumerate}[label=(\alph*)]
\item Predation
\item Parasitism
\item Mutualism
\item Commensalism
\end{enumerate}
\newpage
\section*{Question 19}
What parameters are used for tiger census in our country’s national parks and sanctuaries?
\begin{enumerate}[label=(\alph*)]
\item Pug marks only
\item Pug marks and faecal pellets
\item Faecal pellets only
\item Actual head counts
\end{enumerate}
\newpage
\section*{Question 20}
What is a relationship between organisms of different species where an organism is benefited and other is harmed called?
\begin{enumerate}[label=(\alph*)]
\item Parasitism
\item Commensalism
\item Mutualism
\item Competition
\end{enumerate}
\newpage
\section*{Question 21}
Which of the following is a S-shaped growth curve?
\begin{enumerate}[label=(\alph*)]
\item Exponential growth
\item Logistic growth curve
\item J-shaped curve
\item Geometric curve
\end{enumerate}
\newpage
\section*{Question 22}
Match List I with List II. \begin{tabular}{|c|c|c|c|c|}
\hline
\textbf{List I</th> List II</th>} \\
\hline
A. Logistic growth & I. Unlimited resource availability condition \\
\hline
B. Exponential growth & II. Limited resource availability condition \\
\hline
C. Expanding age pyramid & III. The percent individuals of pre-reproductive age is largest followed by reproductive and post reproductive age groups \\
\hline
D. Stable age pyramid & IV. The percent individuals of pre-reproductives and reproductive age group are same \\
\hline
\end{tabular}
 Choose the correct answer from the options given below:\newline 
\begin{enumerate}[label=(\alph*)]
\item  A-II, B-III, C-I, D-IV 
\item  A-II, B-IV, C-I, D-III
\item  A-II, B-IV, C-III, D-I 
\item  A-II, B-I, C-III, D-IV
\end{enumerate}
\newpage
\section*{Question 23}
 Inspite of interspecific competition in nature, which mechanism the competing species might have evolved for their survival? 
\begin{enumerate}[label=(\alph*)]
\item  Resource partitioning
\item  Competitive release
\item  Mutualism
\item  Predation
\end{enumerate}
\newpage
\section*{Question 24}
Amensalism can be represented as: \newline 
\begin{enumerate}[label=(\alph*)]
\item  Species A (–); Species B (0)
\item  Species A (+); Species B (+)
\item  Species A (–); Species B (–)
\item  Species A (+); Species B (0)
\end{enumerate}
\newpage
\section*{Question 25}
<style>.fm-math,fmath{font-family:STIXGeneral,'DejaVu Serif','DejaVu Sans',Times,OpenSymbol,'Standard Symbols L',serif;line-height:1.2}.fm-math mtext,fmath mtext{line-height:normal}.fm-mo,.ma-sans-serif,fmath mi[mathvariant*=sans-serif],fmath mn[mathvariant*=sans-serif],fmath mo,fmath ms[mathvariant*=sans-serif],fmath mtext[mathvariant*=sans-serif]{font-family:STIXGeneral,'DejaVu Sans','DejaVu Serif','Arial Unicode MS','Lucida Grande',Times,OpenSymbol,'Standard Symbols L',sans-serif}.fm-mo-Luc{font-family:STIXGeneral,'DejaVu Sans','DejaVu Serif','Lucida Grande','Arial Unicode MS',Times,OpenSymbol,'Standard Symbols L',sans-serif}.questionsfont{font-weight:200;font-family:Arial, sans-serif, STIXGeneral,'DejaVu Sans','DejaVu Serif','Lucida Grande','Arial Unicode MS',Times,OpenSymbol,'Standard Symbols L',sans-serif!important}.fm-separator{padding:0 .56ex 0 0}.fm-infix-loose{padding:0 .56ex}.fm-infix{padding:0 .44ex}.fm-prefix{padding:0 .33ex 0 0}.fm-postfix{padding:0 0 0 .33ex}.fm-prefix-tight{padding:0 .11ex 0 0}.fm-postfix-tight{padding:0 0 0 .11ex}.fm-quantifier{padding:0 .11ex 0 .22ex}.ma-non-marking{display:none}.fm-vert,fmath menclose,menclose.fm-menclose{display:inline-block}.fm-large-op{font-size:1.3em}.fm-inline .fm-large-op{font-size:1em}fmath mrow{white-space:nowrap}.fm-vert{vertical-align:middle}fmath table,fmath tbody,fmath td,fmath tr{border:0!important;padding:0!important;margin:0!important;outline:0!important}fmath table{border-collapse:collapse!important;text-align:center!important;table-layout:auto!important;float:none!important}.fm-frac{padding:0 1px!important}td.fm-den-frac{border-top:solid thin!important}.fm-root{font-size:.6em}.fm-radicand{padding:0 1px 0 0;border-top:solid;margin-top:.1em}.fm-script{font-size:.71em}.fm-script .fm-script .fm-script{font-size:1em}td.fm-underover-base{line-height:1!important}td.fm-mtd{padding:.5ex .4em!important;vertical-align:baseline!important}fmath mphantom{visibility:hidden}fmath menclose[notation=top],menclose.fm-menclose[notation=top]{border-top:solid thin}fmath menclose[notation=right],menclose.fm-menclose[notation=right]{border-right:solid thin}fmath menclose[notation=bottom],menclose.fm-menclose[notation=bottom]{border-bottom:solid thin}fmath menclose[notation=left],menclose.fm-menclose[notation=left]{border-left:solid thin}fmath menclose[notation=box],menclose.fm-menclose[notation=box]{border:thin solid}fmath none{display:none}</style> In the exponential growth equation \newline <fmath class="fm-inline"><mrow><msub><mi class="fm-mi-length-1" mathvariant="italic" style="padding-right: 0.44ex;">N</mi><mi class="fm-mi-length-1" mathvariant="italic">t</mi></msub><mo class="fm-infix-loose">=</mo><mrow><msub><mi class="fm-mi-length-1" mathvariant="italic" style="padding-right: 0.44ex;">N</mi><mn>0</mn></msub><msup><mi class="fm-mi-length-1" mathvariant="italic">e</mi><mrow><mi class="fm-mi-length-1" mathvariant="italic">r</mi><mi class="fm-mi-length-1" mathvariant="italic">t</mi></mrow></msup></mrow></mrow></fmath>, e represents
\begin{enumerate}[label=(\alph*)]
\item  The base of number logarithms
\item  The base of exponential logarithms
\item  The base of natural logarithms
\item  The base of geometric logarithms
\end{enumerate}
\newpage
\section*{Question 26}
The logistic population growth is expressed by the equation:  
\begin{enumerate}[label=(\alph*)]
\item  <fmath class="fm-inline"><mrow><mrow><mspace style="margin-right: 0.28em; padding-right: 0.001em; visibility: hidden;" width=".28em">‌</mspace><mspace style="margin-right: 0.28em; padding-right: 0.001em; visibility: hidden;" width=".28em">‌</mspace>\begin{tabular}{|c|c|}
\hline
<mrow><mi class="fm-mi-length-1" mathvariant="italic" style="padding-right: 0.44ex;">d</mi><mi class="fm-mi-length-1" mathvariant="italic">t</mi></mrow> \\
\hline
<mrow><mi class="fm-mi-length-1" mathvariant="italic" style="padding-right: 0.44ex;">d</mi><mi class="fm-mi-length-1" mathvariant="italic" style="padding-right: 0.44ex;">N</mi></mrow> \\
\hline
\end{tabular}
</mrow><mo class="fm-infix-loose">=</mo><mrow><mrow><mi class="fm-mi-length-1" mathvariant="italic" style="padding-right: 0.44ex;">N</mi><mi class="fm-mi-length-1" mathvariant="italic">r</mi></mrow><mrow><mo class="fm-mo-Luc" style="font-size: 2.05em; vertical-align: -0.128em; display: inline-block; transform: scaleX(0.5);">(</mo><mrow><mspace style="margin-right: 0.28em; padding-right: 0.001em; visibility: hidden;" width=".28em">‌</mspace>\begin{tabular}{|c|c|}
\hline
<mrow><mi class="fm-mi-length-1" mathvariant="italic" style="padding-right: 0.44ex;">K</mi><mo class="fm-infix">−</mo><mi class="fm-mi-length-1" mathvariant="italic" style="padding-right: 0.44ex;">N</mi></mrow> \\
\hline
<mi class="fm-mi-length-1" mathvariant="italic" style="padding-right: 0.44ex;">K</mi> \\
\hline
\end{tabular}
</mrow><mo class="fm-mo-Luc" style="font-size: 2.05em; vertical-align: -0.128em; display: inline-block; transform: scaleX(0.5);">)</mo></mrow></mrow></mrow></fmath>
\item  <fmath class="fm-inline"><mrow><mrow><mrow><mspace style="margin-right: 0.28em; padding-right: 0.001em; visibility: hidden;" width=".28em">‌</mspace><mspace style="margin-right: 0.28em; padding-right: 0.001em; visibility: hidden;" width=".28em">‌</mspace><mrow><mi class="fm-mi-length-1" mathvariant="italic" style="padding-right: 0.44ex;">d</mi><mi class="fm-mi-length-1" mathvariant="italic" style="padding-right: 0.44ex;">N</mi></mrow></mrow><mo class="fm-infix-loose">∕</mo><mrow><mi class="fm-mi-length-1" mathvariant="italic" style="padding-right: 0.44ex;">d</mi><mi class="fm-mi-length-1" mathvariant="italic">t</mi></mrow></mrow><mo class="fm-infix-loose">=</mo><mrow><mrow><mi class="fm-mi-length-1" mathvariant="italic">r</mi><mi class="fm-mi-length-1" mathvariant="italic" style="padding-right: 0.44ex;">N</mi></mrow><mrow><mo class="fm-mo-Luc" style="font-size: 2.05em; vertical-align: -0.128em; display: inline-block; transform: scaleX(0.5);">(</mo><mrow><mspace style="margin-right: 0.28em; padding-right: 0.001em; visibility: hidden;" width=".28em">‌</mspace>\begin{tabular}{|c|c|}
\hline
<mrow><mi class="fm-mi-length-1" mathvariant="italic" style="padding-right: 0.44ex;">K</mi><mo class="fm-infix">−</mo><mi class="fm-mi-length-1" mathvariant="italic" style="padding-right: 0.44ex;">N</mi></mrow> \\
\hline
<mi class="fm-mi-length-1" mathvariant="italic" style="padding-right: 0.44ex;">K</mi> \\
\hline
\end{tabular}
</mrow><mo class="fm-mo-Luc" style="font-size: 2.05em; vertical-align: -0.128em; display: inline-block; transform: scaleX(0.5);">)</mo></mrow></mrow></mrow></fmath>
\item  <fmath class="fm-inline"><mrow><mrow><mrow><mspace style="margin-right: 0.28em; padding-right: 0.001em; visibility: hidden;" width=".28em">‌</mspace><mspace style="margin-right: 0.28em; padding-right: 0.001em; visibility: hidden;" width=".28em">‌</mspace><mrow><mi class="fm-mi-length-1" mathvariant="italic" style="padding-right: 0.44ex;">d</mi><mi class="fm-mi-length-1" mathvariant="italic" style="padding-right: 0.44ex;">N</mi></mrow></mrow><mo class="fm-infix-loose">∕</mo><mrow><mi class="fm-mi-length-1" mathvariant="italic" style="padding-right: 0.44ex;">d</mi><mi class="fm-mi-length-1" mathvariant="italic">t</mi></mrow></mrow><mo class="fm-infix-loose">=</mo><mrow><mi class="fm-mi-length-1" mathvariant="italic">r</mi><mi class="fm-mi-length-1" mathvariant="italic" style="padding-right: 0.44ex;">N</mi></mrow></mrow></fmath>
\item  <fmath class="fm-inline"><mrow><mrow><mrow><mspace style="margin-right: 0.28em; padding-right: 0.001em; visibility: hidden;" width=".28em">‌</mspace><mspace style="margin-right: 0.28em; padding-right: 0.001em; visibility: hidden;" width=".28em">‌</mspace><mrow><mi class="fm-mi-length-1" mathvariant="italic" style="padding-right: 0.44ex;">d</mi><mi class="fm-mi-length-1" mathvariant="italic" style="padding-right: 0.44ex;">N</mi></mrow></mrow><mo class="fm-infix-loose">∕</mo><mrow><mi class="fm-mi-length-1" mathvariant="italic" style="padding-right: 0.44ex;">d</mi><mi class="fm-mi-length-1" mathvariant="italic">t</mi></mrow></mrow><mo class="fm-infix-loose">=</mo><mrow><mrow><mi class="fm-mi-length-1" mathvariant="italic">r</mi><mi class="fm-mi-length-1" mathvariant="italic" style="padding-right: 0.44ex;">N</mi></mrow><mrow><mo class="fm-mo-Luc" style="font-size: 2.05em; vertical-align: -0.128em; display: inline-block; transform: scaleX(0.5);">(</mo><mrow><mspace style="margin-right: 0.28em; padding-right: 0.001em; visibility: hidden;" width=".28em">‌</mspace>\begin{tabular}{|c|c|}
\hline
<mrow><mi class="fm-mi-length-1" mathvariant="italic" style="padding-right: 0.44ex;">N</mi><mo class="fm-infix">−</mo><mi class="fm-mi-length-1" mathvariant="italic" style="padding-right: 0.44ex;">K</mi></mrow> \\
\hline
<mi class="fm-mi-length-1" mathvariant="italic" style="padding-right: 0.44ex;">N</mi> \\
\hline
\end{tabular}
</mrow><mo class="fm-mo-Luc" style="font-size: 2.05em; vertical-align: -0.128em; display: inline-block; transform: scaleX(0.5);">)</mo></mrow></mrow></mrow></fmath>
\end{enumerate}
\newpage
\section*{Question 27}
What is the protection and conservation of species outside their natural habitat called?\newline
\begin{enumerate}[label=(\alph*)]
\item In-situ conservation
\item Ex-situ conservation
\item Off-site conservation
\item No conservation
\end{enumerate}
\newpage
\section*{Question 28}
In which approach do we protect and conserve the whole ecosystem to protect the endangered species?\newline
\begin{enumerate}[label=(\alph*)]
\item In-situ conservation
\item Ex-situ conservation
\item Off-site conservation
\item No conservation
\end{enumerate}
\newpage
\section*{Question 29}
What are the species confined to a particular region and not found anywhere else called?\newline
\begin{enumerate}[label=(\alph*)]
\item Pandemic
\item Endemic
\item Extinct
\item Vulnerable
\end{enumerate}
\newpage
\section*{Question 30}
How many total biodiversity hotspots are present throughout the world?
\begin{enumerate}[label=(\alph*)]
\item 20
\item 80
\item 55
\item 34
\end{enumerate}
\newpage
\section*{Question 31}
How much area of the Earth’s surface does the biodiversity hotspots cover?\newline
\begin{enumerate}[label=(\alph*)]
\item 8 %
\item 5 %
\item 10 %
\item 2 %
\end{enumerate}
\newpage
\section*{Question 32}
What is the number of biosphere reserves present throughout the world?\newline
\begin{enumerate}[label=(\alph*)]
\item 24
\item 44
\item 14
\item 4
\end{enumerate}
\newpage
\section*{Question 33}
What is the number of national parks India consists of?
\begin{enumerate}[label=(\alph*)]
\item 19
\item 90
\item 29
\item 120
\end{enumerate}
\newpage
\section*{Question 34}
By which of the following techniques the gametes of threatened species are preserved in viable and fertile conditions for long periods?\newline
\begin{enumerate}[label=(\alph*)]
\item Botanical gardens
\item Cryopreservation techniques
\item Zoological parks
\item Wildlife safari parks
\end{enumerate}
\newpage
\section*{Question 35}
Which phenomenon does the coevolved plant-pollinator mutualism explain?\newline
\begin{enumerate}[label=(\alph*)]
\item Co-extinction
\item Fragmentation
\item Invasion
\item Loss of habitat
\end{enumerate}
\newpage
\section*{Question 36}
Why was the African catfish <em>Clarias gariepinus</em> introduced?\newline
\begin{enumerate}[label=(\alph*)]
\item Horticulture
\item Aquaculture
\item Sericulture
\item Poultry
\end{enumerate}
\newpage
\section*{Question 37}
What is <em>Eicchornia</em> called?\newline
\begin{enumerate}[label=(\alph*)]
\item Carrot grass
\item Nile perch
\item Water hyacinth
\item Water lily
\end{enumerate}
\newpage
\section*{Question 38}
What happened when the Nile perch was introduced into Lake Victoria in East Africa?\newline
\begin{enumerate}[label=(\alph*)]
\item Extinction of trees
\item Increase in the number of trees
\item Extinction of cichlid fish
\item Increase in the number of cichlid fish
\end{enumerate}
\newpage
\section*{Question 39}
What happens when alien species are introduced unintentionally or deliberately?\newline
\begin{enumerate}[label=(\alph*)]
\item Decrease of alien species
\item Increase in habitat
\item They turn invasive and cause an increase in species
\item They turn invasive and cause the decline or extinction of indigenous species
\end{enumerate}
\newpage
\section*{Question 40}
What did David Tilman’s experimental plot show?\newline
\begin{enumerate}[label=(\alph*)]
\item More variation in productivity
\item Fewer species showed less variation
\item More variation from year to year in the total biomass
\item More species showed less variation from year to year in the total biomass
\end{enumerate}
\newpage
\section*{Question 41}
For what reason is rich biodiversity important?\newline
\begin{enumerate}[label=(\alph*)]
\item Community issues
\item Ecosystem health
\item Ecological issues
\item Community problems
\end{enumerate}
\newpage
\section*{Question 42}
Who gave the ‘Rivet popper hypothesis’?\newline
\begin{enumerate}[label=(\alph*)]
\item Karl Correns
\item Robert Hooke
\item Paul Ehrlich
\item Louis Pasteur
\end{enumerate}
\newpage
\section*{Question 43}
What is considered as the rivet in the ‘Rivet popper hypothesis’?\newline
\begin{enumerate}[label=(\alph*)]
\item Ecosystem
\item Community
\item Species
\item Individual
\end{enumerate}
\newpage
\section*{Question 44}
How is the diversity of plants and animals throughout the world?\newline
\begin{enumerate}[label=(\alph*)]
\item Uniform
\item Uneven
\item Equal
\item Even
\end{enumerate}
\newpage
\section*{Question 45}
How many species of birds are located in Colombia that is located near the equator?\newline
\begin{enumerate}[label=(\alph*)]
\item 4400
\item 3000
\item 5000
\item 1400
\end{enumerate}
\newpage
\section*{Question 46}
How many species of birds are located in Greenland?\newline
\begin{enumerate}[label=(\alph*)]
\item 5000
\item 56
\item 120
\item 100
\end{enumerate}
\newpage
\section*{Question 47}
Which place has the greatest biodiversity on Earth?\newline
\begin{enumerate}[label=(\alph*)]
\item Western Ghats
\item Australian forest
\item African forest
\item Amazonian rainforest
\end{enumerate}
\newpage
\section*{Question 48}
In which range does the value of Z lies?
\begin{enumerate}[label=(\alph*)]
\item 0.1 to 0.2
\item 1 to 2
\item 0.001 to 0.002
\item 10 to 20
\end{enumerate}
\newpage
\section*{Question 49}
How many species of fishes are present on Earth?\newline
\begin{enumerate}[label=(\alph*)]
\item 28,00,000
\item 28,000
\item 280
\item 28
\end{enumerate}
\newpage
\section*{Question 50}
How many species of orchids are present on Earth?
\begin{enumerate}[label=(\alph*)]
\item 200
\item 20
\item 2
\item 20,000
\end{enumerate}
\newpage
\end{document}