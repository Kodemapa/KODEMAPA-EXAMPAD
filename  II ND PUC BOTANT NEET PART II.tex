\documentclass{article}
                    \usepackage{amsmath}
                    \usepackage{amssymb}
                    \usepackage{graphicx}
                    \usepackage{enumitem}
                    \usepackage{longtable}
                    \title{ II ND PUC BOTANT NEET PART II}
                    \begin{document}
                    \maketitle
                    \section*{Question 1}
The building block of nucleic acid is:
\begin{enumerate}[label=(\alph*)]
\item Nitrogenous base
\item Ribose sugar
\item Phosphate group
\item Nucleotide
\end{enumerate}
\newpage
\section*{Question 2}
Which of the following base pairs in DNA are correct?
\begin{enumerate}[label=(\alph*)]
\item Adenine-Cytosine
\item Guanine-Cytosine
\item Adenine-Guanine
\item Guanine-Thymine
\end{enumerate}
\newpage
\section*{Question 3}
In the Griffith experiment, which of the following resulted in the death of the mice?
\begin{enumerate}[label=(\alph*)]
\item Live S strain bacteria → Inject into mice
\item Live R strain bacteria → Inject into mice
\item Heat-killed S strain bacteria → Inject into mice
\item Heat-killed R strain bacteria → Inject into mice
\end{enumerate}
\newpage
\section*{Question 4}
In which of the following processes does copying of genetic information from one strand of DNA into RNA take place?
\begin{enumerate}[label=(\alph*)]
\item Replication
\item Transcription
\item Mutation
\item Translation
\end{enumerate}
\newpage
\section*{Question 5}
Which of the following is/are<strong> </strong>the characteristic(s) of genes?
\begin{enumerate}[label=(\alph*)]
\item Replication and transmission to progeny
\item Mutability
\item Storage of hereditary information
\item All of the above
\end{enumerate}
\newpage
\section*{Question 6}
Which of the following does not define the transcription unit in a DNA?
\begin{enumerate}[label=(\alph*)]
\item Terminator
\item Promoter
\item The inducer
\item Structural gene
\end{enumerate}
\newpage
\section*{Question 7}
VNTR are useful in DNA profiling because they:I. Are hypervariableII. Are inheritedIII. Synthesize constitutive enzymesChoose the correct option given below?
\begin{enumerate}[label=(\alph*)]
\item I, II, III are correct
\item I and II are correct
\item I and III are correct
\item II and III are correct
\end{enumerate}
\newpage
\section*{Question 8}
In a genetic code, some codons do not code for amino acids that function as stop codons. Which of the following does not belong to this group?
\begin{enumerate}[label=(\alph*)]
\item UAG
\item UAA
\item UUU
\item UGA
\end{enumerate}
\newpage
\section*{Question 9}
Which of the following codes has a dual function?
\begin{enumerate}[label=(\alph*)]
\item AUG
\item UUU
\item UGA
\item UAA
\end{enumerate}
\newpage
\section*{Question 10}
In point mutation, how many base pairs are inserted or deleted?
\begin{enumerate}[label=(\alph*)]
\item Three
\item Multiples of three
\item Two
\item One
\end{enumerate}
\newpage
\section*{Question 11}
Match the following:\begin{tabular}{|c|c|c|}
\hline
<p align="center" class="MsoNormal">1.<o:p></o:p> & <p align="center" class="MsoNormal">Splicing<o:p></o:p> & <p align="center" class="MsoNormal">a.<o:p></o:p> & <p align="center" class="MsoNormal">Addition of methyl guanosine triphosphate at 5'–end<o:p></o:p> \\
\hline
<p align="center" class="MsoNormal">2.<o:p></o:p> & <p align="center" class="MsoNormal">Capping<o:p></o:p> & <p align="center" class="MsoNormal">b.<o:p></o:p> & <p align="center" class="MsoNormal">Addition of adenylate residues at 3'–end<o:p></o:p> \\
\hline
<p align="center" class="MsoNormal">3.<o:p></o:p> & <p align="center" class="MsoNormal">Tailing<o:p></o:p> & <p align="center" class="MsoNormal">c.<o:p></o:p> & <p align="center" class="MsoNormal">Removal of non-coding introns<o:p></o:p> \\
\hline
\end{tabular}

\begin{enumerate}[label=(\alph*)]
\item 1 - c, 2 - b, 3 - a
\item 1 - c, 2 - a, 3 - b
\item 1 - a, 2 - c, 3 - b
\item 1 - b, 2 - a, 3 - c
\end{enumerate}
\newpage
\section*{Question 12}
The tendency of offspring to differ from parents is called:
\begin{enumerate}[label=(\alph*)]
\item Resemblance
\item Variation
\item Inheritance
\item Heredity
\end{enumerate}
\newpage
\section*{Question 13}
 Which of the following is not a correct statement with respect to DNA?
\begin{enumerate}[label=(\alph*)]
\item It is a long polymer
\item It is found in the nucleus
\item First identified by Friedrich Meischer
\item It is a basic substance\newline
\end{enumerate}
\newpage
\section*{Question 14}
Which one of the following is correct regarding mature mRNA in eukaryotes?
\begin{enumerate}[label=(\alph*)]
\item Exons and introns do not appear in the mature RNA.
\item Exons appear but introns do not appear in the mature RNA.
\item Introns appear but exons do not appear in the mature RNA.
\item Both exons and introns appear in the mature RNA.
\end{enumerate}
\newpage
\section*{Question 15}
The last human chromosome that was completely sequenced is:
\begin{enumerate}[label=(\alph*)]
\item Chromosome 21
\item Chromosome X
\item Chromosome 11
\item Chromosome 1
\end{enumerate}
\newpage
\section*{Question 16}
In double helix model of DNA, how far is each base pair from the next base pair?
\begin{enumerate}[label=(\alph*)]
\item 3.4 nm
\item 0.34 nm
\item 2.0 nm
\item 34 nm
\end{enumerate}
\newpage
\section*{Question 17}
By which of the following bonds, a nitrogenous base is linked to the pentose sugar?
\begin{enumerate}[label=(\alph*)]
\item Phosphate bond
\item  Ester bond
\item Peptide bond
\item N-glycosidic bond
\end{enumerate}
\newpage
\section*{Question 18}
The location of promoter site and the terminator site for transcription is:
\begin{enumerate}[label=(\alph*)]
\item 3 (downstream) end and 5 (upstream) end, respectively of the transcription unit
\item 5 (upstream) end and 3 (downstream) end, respectively of the transcription unit
\item The 5 (upstream) end
\item The 3 (downstream) end
\end{enumerate}
\newpage
\section*{Question 19}
The chemical name for thymine is known as:
\begin{enumerate}[label=(\alph*)]
\item 5-methoxy uracil
\item 3-methoxy uracil
\item 5-methyl uracil
\item 3-methy uracil
\end{enumerate}
\newpage
\section*{Question 20}
Which of the following enzymes are used to transcript a portion of the DNA into mRNA?
\begin{enumerate}[label=(\alph*)]
\item RNA polymerase\newline
\item DNA polymerase
\item Protein polymerase
\item Hydrolase
\end{enumerate}
\newpage
\section*{Question 21}
DNA generally acts as a template for the synthesis of:
\begin{enumerate}[label=(\alph*)]
\item Protein only
\item DNA only
\item RNA only
\item RNA and DNA both
\end{enumerate}
\newpage
\section*{Question 22}
GUG specifies amino acid valine. However, when functioning as initiation codon it specifies _____________.
\begin{enumerate}[label=(\alph*)]
\item Methionine
\item Valine
\item Lysine
\item Isoleucine
\end{enumerate}
\newpage
\section*{Question 23}
Meselson and Stahl used which radio isotopes for their experiment?
\begin{enumerate}[label=(\alph*)]
\item N$^{15}$
\item O$^{18}$
\item C$^{14}$
\item All of the above
\end{enumerate}
\newpage
\section*{Question 24}
In the absence of Lactose, what is expected to happen according to lac operon model?
\begin{enumerate}[label=(\alph*)]
\item Structural genes transcribe for lactose permease
\item Repressor protein binds to the operator site
\item RNA polymerase interacts with DNA to initiate transcription
\item β-galactosidase, lactose permease and thiogalactoside transacetylase are synthesized
\end{enumerate}
\newpage
\end{document}