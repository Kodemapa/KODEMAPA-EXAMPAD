\documentclass{article}
                    \usepackage{amsmath}
                    \usepackage{amssymb}
                    \usepackage{graphicx}
                    \usepackage{enumitem}
                    \usepackage{longtable}
                    \title{check5 }
                    \begin{document}
                    \maketitle
                    \section*{Question 1}
An organic compound "A" on treatment with benzene sulphonyl chloride gives compound \(B . B\) is soluble in dil. \(\mathrm{NaOH}\) solution. Compound \(A\) is______________. 
\begin{enumerate}[label=(\alph*)]
\item \(\mathrm{C}_6 \mathrm{H}_5-\mathrm{N}-\left(\mathrm{CH}_3\right)_2\)
\item \(\mathrm{C}_6 \mathrm{H}_5-\mathrm{NHCH}_2 \mathrm{CH}_3\)
\item \(\mathrm{C}_6 \mathrm{H}_5-\mathrm{CH}_2 \mathrm{NHCH}_3\)
\item \(\mathrm{C}_6 \mathrm{H}_5-\mathrm{CH}-\mathrm{NH}_2\)
\end{enumerate}
\newpage
\section*{Question 2}
The total number of reagents from those given below, that can convert nitrobenzene into aniline is _______________ (Integer answer)\begin{tabular}{|c|c|c|}
\hline
\(I . \mathrm{Sn}-\mathrm{HCI}\) & \(\mathrm{II} \cdot \mathrm{Sn}-\mathrm{NH}_4 \mathrm{OH}\) \\
\hline
\(I I I \cdot \mathrm{Fe}-\mathrm{HCl}\) & \(I V \cdot \mathrm{Zn}-\mathrm{HCI}\) \\
\hline
\(V \cdot \mathrm{H}_2-\mathrm{Pd}\) & \(V I \cdot \mathrm{H}_2-\) Raney nickel \\
\hline
\end{tabular}

\begin{enumerate}[label=(\alph*)]
\end{enumerate}
\newpage
\section*{Question 3}
Match List I with List II.\begin{tabular}{|c|c|c|c|c|}
\hline
List-I & List-II \\
\hline
A. Benzenesulphonyl Chloride & I. Test for primary amines \\
\hline
B. Hoffmann bromamide reaction & II. Anti Saytzeff \\
\hline
C. Carbylamine reaction & III. Hinsberg reagent \\
\hline
D. Hoffmann orientation & IV. Known reaction of Isocyanates \\
\hline
\end{tabular}
Choose the correct answer from the options given below: \newline
\begin{enumerate}[label=(\alph*)]
\item A-IV, B-III, C-II, D-I
\item A-IV, B-II, C-I, D-II
\item A-III, B-IV, C-I, D-II
\item A-IV, B-III, C-I, D-II
\end{enumerate}
\newpage
\section*{Question 4}
Consider the following sequence of reaction :\includegraphics[width=\textwidth]{https://kodemapa.com/static/media/wl_client/1/qdump/dd962b43da3e663bef2c213d7dbe3f88/57dcd9d8a769c8fea41cff9d8e01d464.png}\newlineThe product ' \(\mathrm{B}\) ' is : 
\begin{enumerate}[label=(\alph*)]
\item \includegraphics[width=\textwidth]{https://kodemapa.com/static/media/wl_client/1/qdump/dd962b43da3e663bef2c213d7dbe3f88/424f41a0563c11457d451a9c8548f457.png}
\item \includegraphics[width=\textwidth]{https://kodemapa.com/static/media/wl_client/1/qdump/dd962b43da3e663bef2c213d7dbe3f88/d4b4b4ca1568a3c5ecaffa3c96777568.png}
\item \includegraphics[width=\textwidth]{https://kodemapa.com/static/media/wl_client/1/qdump/dd962b43da3e663bef2c213d7dbe3f88/514d98122f31a4308660c7cf35c5c48c.png}
\item \includegraphics[width=\textwidth]{https://kodemapa.com/static/media/wl_client/1/qdump/dd962b43da3e663bef2c213d7dbe3f88/b8f6fcf1f36be06b6950c90da3323688.png}
\end{enumerate}
\newpage
\section*{Question 5}
The correct order in aqueous medium of basic strength in case of methyl substituted amines is : 
\begin{enumerate}[label=(\alph*)]
\item \(\mathrm{Me}_2 \mathrm{NH}>\mathrm{MeNH}_2>\mathrm{Me}_3 \mathrm{~N}>\mathrm{NH}_3\)
\item \(\mathrm{Me}_2 \mathrm{NH}>\mathrm{Me}_3 \mathrm{~N}>\mathrm{MeNH}_2>\mathrm{NH}_3\)
\item \(\mathrm{NH}_3>\mathrm{Me}_3 \mathrm{~N}>\mathrm{MeNH}_2>\mathrm{Me}_2 \mathrm{NH}\)
\item \(\mathrm{Me}_3 \mathrm{~N}>\mathrm{Me}_2 \mathrm{NH}>\mathrm{MeNH}_2>\mathrm{NH}_3\)
\end{enumerate}
\newpage
\end{document}