\documentclass{article}
                    \usepackage{amsmath}
                    \usepackage{amssymb}
                    \usepackage{graphicx}
                    \usepackage{enumitem}
                    \usepackage{longtable}
                    \title{II ND PUC BOTANY NEET.}
                    \begin{document}
                    \maketitle
                    \section*{Question 1}
Flowers with both androecium and gynoecium are called:
\begin{enumerate}[label=(\alph*)]
\item Bisexual flowers
\item Anther
\item Stamens
\item Unisexual flowers
\end{enumerate}
\newpage
\section*{Question 2}
The transfer of pollen from the anther to stigma is called _________.\newline
\begin{enumerate}[label=(\alph*)]
\item Pollination
\item Fertilization
\item Adoption
\item Diffusion
\end{enumerate}
\newpage
\section*{Question 3}
The fusion of female reproductive nucleus with the male reproductive nucleus is known as _______.
\begin{enumerate}[label=(\alph*)]
\item Adoption
\item Excretion
\item Fertilization
\item Regeneration
\end{enumerate}
\newpage
\section*{Question 4}
The two nuclei at the end of the pollen tube are called _____.
\begin{enumerate}[label=(\alph*)]
\item Tube nucleus and a generative nucleus
\item Sperm and ovum
\item Generative nucleus and stigma
\item Tube nucleus and sperm
\end{enumerate}
\newpage
\section*{Question 5}
Generative nucleus divides forming:
\begin{enumerate}[label=(\alph*)]
\item 2 male nuclei
\item 3 male nuclei
\item 2 female nuclei
\item 3 female nuclei
\end{enumerate}
\newpage
\section*{Question 6}
Embryo sac is located inside the:
\begin{enumerate}[label=(\alph*)]
\item Stigma
\item Ovule
\item Micropyle
\item Style
\end{enumerate}
\newpage
\section*{Question 7}
One nucleus of the pollen tube and secondary nucleus of the ovum grow into:
\begin{enumerate}[label=(\alph*)]
\item Stigma
\item Endosperm
\item Anther
\item Stamen
\end{enumerate}
\newpage
\section*{Question 8}
The male reproductive part of a flower is known as ___________.
\begin{enumerate}[label=(\alph*)]
\item petals
\item sepals
\item stamen
\item pistil
\end{enumerate}
\newpage
\section*{Question 9}
The other name for gynoecium is:
\begin{enumerate}[label=(\alph*)]
\item Pistil
\item Stigma
\item Style\newline
\item None of these
\end{enumerate}
\newpage
\section*{Question 10}
Functional megaspore in a flowering plant develops into:
\begin{enumerate}[label=(\alph*)]
\item Endosperm
\item Ovule
\item Embryo-sac
\item Embryo
\end{enumerate}
\newpage
\section*{Question 11}
Which of the following is similar to autogamy, but requires pollinators?
\begin{enumerate}[label=(\alph*)]
\item Geitonogamy
\item Cleistogamy
\item Apogamy
\item Xenogamy
\end{enumerate}
\newpage
\section*{Question 12}
What is the function of the filiform apparatus?
\begin{enumerate}[label=(\alph*)]
\item Guide the entry of pollen tube
\item Recognize the suitable pollen at the stigma
\item Produce nectar
\item Stimulate division of the generative cell
\end{enumerate}
\newpage
\section*{Question 13}
Seed formation without fertilization in flowering plants involves the process of:\newline
\begin{enumerate}[label=(\alph*)]
\item Budding
\item Apomixis
\item Sporulation
\item Somatic hybridization
\end{enumerate}
\newpage
\section*{Question 14}
A dioecious flowering plant prevents:
\begin{enumerate}[label=(\alph*)]
\item Geitonogamy and xenogamy
\item Autogamy and xenogamy
\item Autogamy and geitonogamy
\item Cleistogamy and xenogamy
\end{enumerate}
\newpage
\section*{Question 15}
If an endosperm cell of an angiosperm contains 24 chromosomes, the number of chromosomes in each cell of the root will be:\newline<ul class="Ts_solution_list ng-scope" dir="ltr" ng-if="show_single_data[10] == 1 || show_single_data[10] == 3"></ul>
\begin{enumerate}[label=(\alph*)]
\item 8
\item 4
\item 16
\item 24
\end{enumerate}
\newpage
\section*{Question 16}
The cells of endosperm have 24 chromosomes. What will be the number of chromosomes in the gametes?\newline
\begin{enumerate}[label=(\alph*)]
\item 8
\item 16
\item 23
\item 32
\end{enumerate}
\newpage
\section*{Question 17}
The true embryo develops as a result of fusion of:
\begin{enumerate}[label=(\alph*)]
\item Two polar nuclei of embryo sac
\item Egg cell and male gamete
\item Synergid and male gamete
\item Male gamete and antipodals
\end{enumerate}
\newpage
\section*{Question 18}
The portion of embryonal axis between plumule(future shoot) and cotyledons is called ___________.
\begin{enumerate}[label=(\alph*)]
\item Hypocotyl
\item Epicotyl
\item Coleorhiza
\item Coleoptile
\end{enumerate}
\newpage
\section*{Question 19}
Coleorhiza and coleoptile are the protective sheaths covering _______ and ______ respectively.
\begin{enumerate}[label=(\alph*)]
\item Radicle, Plumule
\item Plumule, Radicle
\item Plumule, Hypocotyl
\item Epicotyle, Radicle
\end{enumerate}
\newpage
\section*{Question 20}
________ is not an endospermic seed.
\begin{enumerate}[label=(\alph*)]
\item Pea
\item Castor
\item Maize
\item Wheat
\end{enumerate}
\newpage
\section*{Question 21}
Pollen grain is a __________.
\begin{enumerate}[label=(\alph*)]
\item Megaspore
\item Microspore
\item Microsporophyll
\item Microsporangium
\end{enumerate}
\newpage
\section*{Question 22}
How many pollen mother cells should undergo meiotic division to produce 64 pollen grains?
\begin{enumerate}[label=(\alph*)]
\item 64
\item 32
\item 16
\item 8
\end{enumerate}
\newpage
\section*{Question 23}
Proximal end of the filament of stamen is attached to the:
\begin{enumerate}[label=(\alph*)]
\item Anther
\item Connective
\item Placenta
\item Thalamus or petal
\end{enumerate}
\newpage
\section*{Question 24}
The most common type of ovule in Angiosperms is:
\begin{enumerate}[label=(\alph*)]
\item Anatropous
\item Amphitropus
\item Orthotropous type
\item Antropous
\end{enumerate}
\newpage
\section*{Question 25}
The flower which does not open for pollination are called as:
\begin{enumerate}[label=(\alph*)]
\item Autogamous
\item Chasmogomous
\item Geitogamous
\item Cleistogamous
\end{enumerate}
\newpage
\section*{Question 26}
Insect pollinated flowers are:
\begin{enumerate}[label=(\alph*)]
\item Large and without fragrance
\item Small and without fragrance
\item Colorful and contain nectar
\item Colorless and without nectar
\end{enumerate}
\newpage
\section*{Question 27}
Which of the following flower contain both stamens and pistils?
\begin{enumerate}[label=(\alph*)]
\item Perfect flower
\item Incomplete flower
\item Staminate flower
\item Bracteate flower
\end{enumerate}
\newpage
\end{document}