\documentclass{article}
                    \usepackage{amsmath}
                    \usepackage{amssymb}
                    \usepackage{graphicx}
                    \usepackage{enumitem}
                    \usepackage{longtable}
                    \title{KCET 20}
                    \begin{document}
                    \maketitle
                    \section*{Question 1}
Which one of the products of the following reactions does not react with Hinsberg reagent to form sulphonamide? [25 Jul 2021]
\begin{enumerate}[label=(\alph*)]
\item \includegraphics[width=\textwidth]{https://kodemapa.com/static/media/wl_client/1/qdump/dd962b43da3e663bef2c213d7dbe3f88/2c25b5302176976e694d20326b655cbb.png}
\item \includegraphics[width=\textwidth]{https://kodemapa.com/static/media/wl_client/1/qdump/dd962b43da3e663bef2c213d7dbe3f88/77e8483bbf8a5f91aca9e2e40598b611.png}
\item \includegraphics[width=\textwidth]{https://kodemapa.com/static/media/wl_client/1/qdump/dd962b43da3e663bef2c213d7dbe3f88/335c787e1add6d2886cf53b8fe0870d7.png}
\item \includegraphics[width=\textwidth]{https://kodemapa.com/static/media/wl_client/1/qdump/dd962b43da3e663bef2c213d7dbe3f88/07ad49568681c4e55c2b1bfd60d5b384.png}
\end{enumerate}
\newpage
\section*{Question 2}
An organic compound "A" on treatment with benzene sulphonyl chloride gives compound \(B . B\) is soluble in dil. \(\mathrm{NaOH}\) solution. Compound \(A\) is______________. 
\begin{enumerate}[label=(\alph*)]
\item \(\mathrm{C}_6 \mathrm{H}_5-\mathrm{N}-\left(\mathrm{CH}_3\right)_2\)
\item \(\mathrm{C}_6 \mathrm{H}_5-\mathrm{NHCH}_2 \mathrm{CH}_3\)
\item \(\mathrm{C}_6 \mathrm{H}_5-\mathrm{CH}_2 \mathrm{NHCH}_3\)
\item \(\mathrm{C}_6 \mathrm{H}_5-\mathrm{CH}-\mathrm{NH}_2\)
\end{enumerate}
\newpage
\section*{Question 3}
The total number of reagents from those given below, that can convert nitrobenzene into aniline is _______________ (Integer answer)\begin{tabular}{|c|c|c|}
\hline
\(I . \mathrm{Sn}-\mathrm{HCI}\) & \(\mathrm{II} \cdot \mathrm{Sn}-\mathrm{NH}_4 \mathrm{OH}\) \\
\hline
\(I I I \cdot \mathrm{Fe}-\mathrm{HCl}\) & \(I V \cdot \mathrm{Zn}-\mathrm{HCI}\) \\
\hline
\(V \cdot \mathrm{H}_2-\mathrm{Pd}\) & \(V I \cdot \mathrm{H}_2-\) Raney nickel \\
\hline
\end{tabular}

\begin{enumerate}[label=(\alph*)]
\end{enumerate}
\newpage
\section*{Question 4}
Match List I with List II.\begin{tabular}{|c|c|c|c|c|}
\hline
List-I & List-II \\
\hline
A. Benzenesulphonyl Chloride & I. Test for primary amines \\
\hline
B. Hoffmann bromamide reaction & II. Anti Saytzeff \\
\hline
C. Carbylamine reaction & III. Hinsberg reagent \\
\hline
D. Hoffmann orientation & IV. Known reaction of Isocyanates \\
\hline
\end{tabular}
Choose the correct answer from the options given below: \newline
\begin{enumerate}[label=(\alph*)]
\item A-IV, B-III, C-II, D-I
\item A-IV, B-II, C-I, D-II
\item A-III, B-IV, C-I, D-II
\item A-IV, B-III, C-I, D-II
\end{enumerate}
\newpage
\section*{Question 5}
Consider the following sequence of reaction :\includegraphics[width=\textwidth]{https://kodemapa.com/static/media/wl_client/1/qdump/dd962b43da3e663bef2c213d7dbe3f88/57dcd9d8a769c8fea41cff9d8e01d464.png}\newlineThe product ' \(\mathrm{B}\) ' is : 
\begin{enumerate}[label=(\alph*)]
\item \includegraphics[width=\textwidth]{https://kodemapa.com/static/media/wl_client/1/qdump/dd962b43da3e663bef2c213d7dbe3f88/424f41a0563c11457d451a9c8548f457.png}
\item \includegraphics[width=\textwidth]{https://kodemapa.com/static/media/wl_client/1/qdump/dd962b43da3e663bef2c213d7dbe3f88/d4b4b4ca1568a3c5ecaffa3c96777568.png}
\item \includegraphics[width=\textwidth]{https://kodemapa.com/static/media/wl_client/1/qdump/dd962b43da3e663bef2c213d7dbe3f88/514d98122f31a4308660c7cf35c5c48c.png}
\item \includegraphics[width=\textwidth]{https://kodemapa.com/static/media/wl_client/1/qdump/dd962b43da3e663bef2c213d7dbe3f88/b8f6fcf1f36be06b6950c90da3323688.png}
\end{enumerate}
\newpage
\section*{Question 6}
Given below are two statements :Statement I : In Hofmann degradation reaction, the migration of only an alkyl group takes place from carbonyl carbon of the amide to the nitrogen atom.Statement II : The group is migrated in Hofmann degradation reaction to electron deficient atom.In the light of the above statements, choose the most appropriate answer from the options given below: 
\begin{enumerate}[label=(\alph*)]
\item Both Statement I and Statement II are correct.
\item Both Statement I and Statement II are incorrect.
\item Statement I is correct but Statement II is incorrect.
\item Statement I is incorrect but Statement II is correct.
\end{enumerate}
\newpage
\section*{Question 7}
During halogen test, sodium fusion extract is boiled with concentrated \(\mathrm{HNO}_3\) to 
\begin{enumerate}[label=(\alph*)]
\item remove unreacted sodium
\item decompose cyanide or sulphide of sodium
\item extract halogen from organic compound
\item maintain the \(\mathrm{pH}\) of extract.
\end{enumerate}
\newpage
\section*{Question 8}
The decreasing order of basicity of the following amines is:\includegraphics[width=\textwidth]{https://kodemapa.com/static/media/wl_client/1/qdump/dd962b43da3e663bef2c213d7dbe3f88/2752f49d87bf0fdd936512818343839f.png}\newline
\begin{enumerate}[label=(\alph*)]
\item \((A)>(C)>(D)>(B)\)
\item \((C)>(A)>(B)>(D)\)
\item \((B)>(C)>(D)>(A)\)
\item \((C)>(B)>(A)>(D)\)
\end{enumerate}
\newpage
\section*{Question 9}
\includegraphics[width=\textwidth]{https://kodemapa.com/static/media/wl_client/1/qdump/dd962b43da3e663bef2c213d7dbe3f88/e017bcaa9a7e8b73dcdedce8d8d5d33d.png}\newlineIn the chemical reactions given above \(A\) and \(B\) respectively are: 
\begin{enumerate}[label=(\alph*)]
\item \(\mathrm{H}_3 \mathrm{PO}_2\) and \(\mathrm{CH}_3 \mathrm{CH}_2 \mathrm{Cl}\)
\item \(\mathrm{CH}_3 \mathrm{CH}_2 \mathrm{OH}\) and \(\mathrm{H}_3 \mathrm{PO}_2\)
\item \(\mathrm{H}_3 \mathrm{O}_2\) and \(\mathrm{CH}_3 \mathrm{CH}_2 \mathrm{OH}\)
\item \(\mathrm{CH}_3 \mathrm{CH}_2 \mathrm{Cl}\) and \(\mathrm{H}_3 \mathrm{PO}_2\)
\end{enumerate}
\newpage
\section*{Question 10}
The products A and B formed in the following reaction scheme are respectively\includegraphics[width=\textwidth]{https://kodemapa.com/static/media/wl_client/1/qdump/dd962b43da3e663bef2c213d7dbe3f88/4d6bf3841a1af092d1876901f8f5ca11.png}\newline\newline
\begin{enumerate}[label=(\alph*)]
\item \includegraphics[width=\textwidth]{https://kodemapa.com/static/media/wl_client/1/qdump/dd962b43da3e663bef2c213d7dbe3f88/2a3246b15ff319d98764dd66e03e52ca.png}
\item \includegraphics[width=\textwidth]{https://kodemapa.com/static/media/wl_client/1/qdump/dd962b43da3e663bef2c213d7dbe3f88/eee70e6d6582dc36823640835c13ac70.png}
\item \includegraphics[width=\textwidth]{https://kodemapa.com/static/media/wl_client/1/qdump/dd962b43da3e663bef2c213d7dbe3f88/5c0894ccccc6ea4d63ce98ae8ae80869.png}
\item \includegraphics[width=\textwidth]{https://kodemapa.com/static/media/wl_client/1/qdump/dd962b43da3e663bef2c213d7dbe3f88/347277ab3c9f0b5b788ab0c182916da4.png}
\end{enumerate}
\newpage
\section*{Question 11}
Which of the following is not a correct statement for primary aliphatic amines? 
\begin{enumerate}[label=(\alph*)]
\item The intermolecular association in primary amines is less than the intermolecular association in secondary amines.
\item Primary amines on treating with nitrous acid solution form corresponding alcohols except methyl amine.
\item Primary amines are less basic than the secondary amines.
\item Primary amines can be prepared by the gabriel phthalimide synthesis.
\end{enumerate}
\newpage
\section*{Question 12}
The total number of reagents from those given below, that can convert nitrobenzene into aniline is _______________ (Integer answer)\begin{tabular}{|c|c|c|}
\hline
\(I . \mathrm{Sn}-\mathrm{HCI}\) & \(\mathrm{II} \cdot \mathrm{Sn}-\mathrm{NH}_4 \mathrm{OH}\) \\
\hline
\(I I I \cdot \mathrm{Fe}-\mathrm{HCl}\) & \(I V \cdot \mathrm{Zn}-\mathrm{HCI}\) \\
\hline
\(V \cdot \mathrm{H}_2-\mathrm{Pd}\) & \(V I \cdot \mathrm{H}_2-\) Raney nickel \\
\hline
\end{tabular}

\begin{enumerate}[label=(\alph*)]
\end{enumerate}
\newpage
\section*{Question 13}
Match List I with List II.\begin{tabular}{|c|c|c|c|c|}
\hline
List-I & List-II \\
\hline
A. Benzenesulphonyl Chloride & I. Test for primary amines \\
\hline
B. Hoffmann bromamide reaction & II. Anti Saytzeff \\
\hline
C. Carbylamine reaction & III. Hinsberg reagent \\
\hline
D. Hoffmann orientation & IV. Known reaction of Isocyanates \\
\hline
\end{tabular}
Choose the correct answer from the options given below: \newline
\begin{enumerate}[label=(\alph*)]
\item A-IV, B-III, C-II, D-I
\item A-IV, B-II, C-I, D-II
\item A-III, B-IV, C-I, D-II
\item A-IV, B-III, C-I, D-II
\end{enumerate}
\newpage
\section*{Question 14}
Which of the following is not a correct statement for primary aliphatic amines? 
\begin{enumerate}[label=(\alph*)]
\item The intermolecular association in primary amines is less than the intermolecular association in secondary amines.
\item Primary amines on treating with nitrous acid solution form corresponding alcohols except methyl amine.
\item Primary amines are less basic than the secondary amines.
\item Primary amines can be prepared by the gabriel phthalimide synthesis.
\end{enumerate}
\newpage
\section*{Question 15}
The total number of electrons around the nitrogen atom in amines are, 
\begin{enumerate}[label=(\alph*)]
\end{enumerate}
\newpage
\section*{Question 16}
What is the correct name for a molecule that has two amino groups in opposing (para) locations around a benzene ring?
\begin{enumerate}[label=(\alph*)]
\item  Benzenediamine
\item Benzene-1,4-diamine
\item p-Aminoaniline
\item 4-Aminobenzenamine
\end{enumerate}
\newpage
\section*{Question 17}
An organic compound "A" on treatment with benzene sulphonyl chloride gives compound \(B . B\) is soluble in dil. \(\mathrm{NaOH}\) solution. Compound \(A\) is______________. 
\begin{enumerate}[label=(\alph*)]
\item \(\mathrm{C}_6 \mathrm{H}_5-\mathrm{N}-\left(\mathrm{CH}_3\right)_2\)
\item \(\mathrm{C}_6 \mathrm{H}_5-\mathrm{NHCH}_2 \mathrm{CH}_3\)
\item \(\mathrm{C}_6 \mathrm{H}_5-\mathrm{CH}_2 \mathrm{NHCH}_3\)
\item \(\mathrm{C}_6 \mathrm{H}_5-\mathrm{CH}-\mathrm{NH}_2\)
\end{enumerate}
\newpage
\section*{Question 18}
The total number of reagents from those given below, that can convert nitrobenzene into aniline is _______________ (Integer answer)\begin{tabular}{|c|c|c|}
\hline
\(I . \mathrm{Sn}-\mathrm{HCI}\) & \(\mathrm{II} \cdot \mathrm{Sn}-\mathrm{NH}_4 \mathrm{OH}\) \\
\hline
\(I I I \cdot \mathrm{Fe}-\mathrm{HCl}\) & \(I V \cdot \mathrm{Zn}-\mathrm{HCI}\) \\
\hline
\(V \cdot \mathrm{H}_2-\mathrm{Pd}\) & \(V I \cdot \mathrm{H}_2-\) Raney nickel \\
\hline
\end{tabular}

\begin{enumerate}[label=(\alph*)]
\end{enumerate}
\newpage
\section*{Question 19}
A primary aliphatic amine on reaction with nitrous acid in cold ( \(273 \mathrm{~K})\) and there after raising temperature of reaction mixture to room temperature (298 K), gives.
\begin{enumerate}[label=(\alph*)]
\item nitrile
\item alcohol
\item diazonium salt
\item secondary amine
\end{enumerate}
\newpage
\section*{Question 20}
Match List I with List II.\begin{tabular}{|c|c|c|c|c|}
\hline
List-I & List-II \\
\hline
A. Benzenesulphonyl Chloride & I. Test for primary amines \\
\hline
B. Hoffmann bromamide reaction & II. Anti Saytzeff \\
\hline
C. Carbylamine reaction & III. Hinsberg reagent \\
\hline
D. Hoffmann orientation & IV. Known reaction of Isocyanates \\
\hline
\end{tabular}
Choose the correct answer from the options given below: \newline
\begin{enumerate}[label=(\alph*)]
\item A-IV, B-III, C-II, D-I
\item A-IV, B-II, C-I, D-II
\item A-III, B-IV, C-I, D-II
\item A-IV, B-III, C-I, D-II
\end{enumerate}
\newpage
\section*{Question 21}
The major product formed in the following reaction is.\includegraphics[width=\textwidth]{https://kodemapa.com/static/media/wl_client/1/qdump/dd962b43da3e663bef2c213d7dbe3f88/cd501f5bf6a7ba9867212e69f5da1961.png}\newline
\begin{enumerate}[label=(\alph*)]
\item \includegraphics[width=\textwidth]{https://kodemapa.com/static/media/wl_client/1/qdump/dd962b43da3e663bef2c213d7dbe3f88/ea66b26d652be69c42f279d0ca208c29.png}
\item \includegraphics[width=\textwidth]{https://kodemapa.com/static/media/wl_client/1/qdump/dd962b43da3e663bef2c213d7dbe3f88/2952e31a186e1acc29cc186fdfb5dd6b.png}
\item \includegraphics[width=\textwidth]{https://kodemapa.com/static/media/wl_client/1/qdump/dd962b43da3e663bef2c213d7dbe3f88/d4fa77cbadb82c63c120985ab1e82614.png}
\item \includegraphics[width=\textwidth]{https://kodemapa.com/static/media/wl_client/1/qdump/dd962b43da3e663bef2c213d7dbe3f88/59084e148fc2d739417759e72917593f.png}
\end{enumerate}
\newpage
\section*{Question 22}
\includegraphics[width=\textwidth]{https://kodemapa.com/static/media/wl_client/1/qdump/dd962b43da3e663bef2c213d7dbe3f88/4ec04ea353b6f65a5cf7b06ed0289d5e.png}\newlineConsider the given reaction, percentage yield of, 
\begin{enumerate}[label=(\alph*)]
\item \((C)>(A)>(B)\)
\item \((B)>(C)>(A)\)
\item \((A)>(C)>(B)\)
\item \((C)>(B)>(A)\)
\end{enumerate}
\newpage
\section*{Question 23}
Given below are two statements :Statement I: Aniline reacts with con. \(\mathrm{H}_2 \mathrm{SO}_4\) followed by heating at \(453-473 \mathrm{~K}\) gives p-aminobenzene sulphonic acid, which gives blood red colour in the 'Lassaigne's test'.\newlineStatement II: In Friedel - Craft's alkylation and acylation reactions, aniline forms salt with the \(\mathrm{AlCl}_3\) catalyst.\newlineDue to this, nitrogen of aniline aquires a positive charge and acts as deactivating group.\newlineIn the light of the above statements, choose the correct answer from the options given below :
\begin{enumerate}[label=(\alph*)]
\item Statement I is false but statement II is true
\item Both statement I and statement II are false
\item Statement I is true but statement II is false
\item Both statement I and statement II are true
\end{enumerate}
\newpage
\end{document}