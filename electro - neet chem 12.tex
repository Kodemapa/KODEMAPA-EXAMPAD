\documentclass{article}
                    \usepackage{amsmath}
                    \usepackage{amssymb}
                    \usepackage{graphicx}
                    \usepackage{enumitem}
                    \usepackage{longtable}
                    \title{electro - neet chem 12}
                    \begin{document}
                    \maketitle
                    \section*{Question 1}
Following reactions are taking place in a Galvanic cell,  \(Z n \rightarrow Z n^{2+}+2 e^{-} ; A g^{+}+e^{-} \rightarrow A g\)Which of the given representations is the correct method of depicting the cell?
\begin{enumerate}[label=(\alph*)]
\item \(Z n_{(s)}|Z n_{(a q)}^{2+} \| A g_{(a q)}^{+}| A g_{(s)}\)
\item \(Z n^{2+}|Z n \| A g| A g^{+}\)
\item \(Z n_{(a q)}|Z n_{(s)}^{2+} \| A g_{(s)}^{+}| A g_{(a q)}\)
\item \(Z n_{(s)}|A g_{(a q)}^{+} \| Z n_{(a q)}^{2+}| A g_{(s)}\)
\end{enumerate}
\newpage
\section*{Question 2}
Ratio of number of faradays of electricity required to deposit magnesium, aluminium and sodium in equimolar ratio an electrolysis of their respective molten salts is:
\begin{enumerate}[label=(\alph*)]
\item \(2: 3: 2\)
\item \(2: 1: 1\)
\item \(2: 3: 1\)
\item \(4: 6: 1\)
\end{enumerate}
\newpage
\section*{Question 3}
When lead storage battery is charged:
\begin{enumerate}[label=(\alph*)]
\item Lead dioxide dissolves
\item Sulphuric acid is regenerated
\item The lead electrode becomes coated with lead sulphate
\item The amount of sulphuric acid decreases
\end{enumerate}
\newpage
\section*{Question 4}
A \(0.05 \mathrm{M} ~\mathrm{NaOH}\) solution offered a resistance of \(31.6~ \mathrm{\Omega}\) in a conductivity cell is \(0.367 \mathrm{~cm}^{-1}\), find out the specific conductance of the sodium hydroxide solution.
\begin{enumerate}[label=(\alph*)]
\item \(0.0216 \mathrm{~ohm}^{-1} \mathrm{cm}^{-1}\)
\item \(0.0116 \mathrm{~ohm}^{-1} \mathrm{cm}^{-1}\)
\item \(0.156 \mathrm{~ohm}^{-1} \mathrm{cm}^{-1}\)
\item \( 0.356 \mathrm{~ohm}^{-1} \mathrm{cm}^{-1}\)
\end{enumerate}
\newpage
\section*{Question 5}
Specific conductance of \(0.1 \mathrm{ {M} {~NaCl}}\) solution is \(1.01 \times 10^{-2} \mathrm{{ohm}^{-1} {~cm}^{-1}}\). Its molar conductance in \(\mathrm{ohm ^{-1} {~cm}^{2} {~mol}^{-1}}\) is:
\begin{enumerate}[label=(\alph*)]
\item \(1.01 \times 10^{-2} \mathrm{{ohm}^{-1} {~cm}^{2} {~mol}^{-1}}\)\newline
\item \(1.01 \times 10^{3} \mathrm{{ohm}^{-1} {~cm}^{2} {~mol}^{-1}}\)
\item \(1.01 \times 10^{2} \mathrm{{ohm}^{-1} {~cm}^{2} {~mol}^{-1}}\)
\item \(1.01 \times 10^{-3} \mathrm{{ohm}^{-1} {~cm}^{2} {~mol}^{-1}}\)
\end{enumerate}
\newpage
\section*{Question 6}
\(E_{\text {cell }}^{o}\) for the reaction, \(2 {H}_{2} {O} \rightarrow {H}_{3} {O}^{+}+{OH}^{-}\)at \(25^{\circ} {C}\) is \(-0.8277 {~V}\). The equilibrium constant for the reaction is:
\begin{enumerate}[label=(\alph*)]
\item \(10^{-14}\)
\item \(10^{-23}\)
\item \(10^{-7}\)
\item \(10^{-21}\)
\end{enumerate}
\newpage
\section*{Question 7}
How much electricity in terms of Faraday is required to produce \(100 {~g}\) of Ca from molten \(C a C l_{2} ?\)
\begin{enumerate}[label=(\alph*)]
\item 1F
\item 2F
\item 3F
\item 5F
\end{enumerate}
\newpage
\section*{Question 8}
When an aqueous solution of \(A g N O_{3}\) is electrolysed between platinum electrodes, the substances liberated at anode and cathode are:
\begin{enumerate}[label=(\alph*)]
\item Silver is deposited at cathode and \(O_{2}\) is liberated at anode
\item Silver is deposited at cathode and \(H_{2}\) is liberated at anode
\item Hydrogen is liberated at cathode and \(O_{2}\) is liberated at anode
\item Silver is deposited at cathode and Pt is dissolved in electrolyte
\end{enumerate}
\newpage
\section*{Question 9}
A \(0.05 {M} {NaOH}\) solution offered a resistance of \(31.6 {Q}\) in a conductivity cell is \(0.367 {~cm}^{-1}\), find out the molar conductivity of the sodium hydroxide solution.
\begin{enumerate}[label=(\alph*)]
\item \(233 {~ohm}^{-1} {~cm}^{2} {~mol}^{-1}\)
\item \(232 {~ohm}^{-1} {~cm}^{2} {~mol}^{-1}\)
\item \(234 {~ohm}^{-1} {~cm}^{2} {~mol}^{-1}\)
\item \(235 {~ohm}^{-1} {~cm}^{2} {~mol}^{-1}\)
\end{enumerate}
\newpage
\section*{Question 10}
How much electricity must pass through acidulated water to release \(22,400 cm ^3\) of hydrogen at N.T.P.?
\begin{enumerate}[label=(\alph*)]
\item \(96500 C\)
\item \(193000 C\)
\item \(22.4 C\)
\item \(95.5 C\)
\end{enumerate}
\newpage
\section*{Question 11}
Standard electrode potentials of \(Fe ^{2+}+2 e ^{-} \rightarrow Fe\) and \(Fe ^{3+}+3 e ^{-} \rightarrow Fe\) are \(-0.440 V\) and \(-0.036 V\) respectively. The standard electrode potential \(\left( E ^{\circ}\right)\) for \(Fe ^{3+}+ e ^{-} \rightarrow Fe ^{2+}\) is:
\begin{enumerate}[label=(\alph*)]
\item \(-0.476 V\)
\item \(-0.404 V\)
\item \(+0.404 V\)
\item \(+0.772 V\)
\end{enumerate}
\newpage
\section*{Question 12}
When equilibrium is reached inside the two half-cells of the electrochemical cells, what is the net voltage across the electrodes?
\begin{enumerate}[label=(\alph*)]
\item > 1
\item < 1
\item = 0
\item Not defined
\end{enumerate}
\newpage
\section*{Question 13}
<style>.fm-math,fmath{font-family:STIXGeneral,'DejaVu Serif','DejaVu Sans',Times,OpenSymbol,'Standard Symbols L',serif;line-height:1.2}.fm-math mtext,fmath mtext{line-height:normal}.fm-mo,.ma-sans-serif,fmath mi[mathvariant*=sans-serif],fmath mn[mathvariant*=sans-serif],fmath mo,fmath ms[mathvariant*=sans-serif],fmath mtext[mathvariant*=sans-serif]{font-family:STIXGeneral,'DejaVu Sans','DejaVu Serif','Arial Unicode MS','Lucida Grande',Times,OpenSymbol,'Standard Symbols L',sans-serif}.fm-mo-Luc{font-family:STIXGeneral,'DejaVu Sans','DejaVu Serif','Lucida Grande','Arial Unicode MS',Times,OpenSymbol,'Standard Symbols L',sans-serif}.questionsfont{font-weight:200;font-family:Arial, sans-serif, STIXGeneral,'DejaVu Sans','DejaVu Serif','Lucida Grande','Arial Unicode MS',Times,OpenSymbol,'Standard Symbols L',sans-serif!important}.fm-separator{padding:0 .56ex 0 0}.fm-infix-loose{padding:0 .56ex}.fm-infix{padding:0 .44ex}.fm-prefix{padding:0 .33ex 0 0}.fm-postfix{padding:0 0 0 .33ex}.fm-prefix-tight{padding:0 .11ex 0 0}.fm-postfix-tight{padding:0 0 0 .11ex}.fm-quantifier{padding:0 .11ex 0 .22ex}.ma-non-marking{display:none}.fm-vert,fmath menclose,menclose.fm-menclose{display:inline-block}.fm-large-op{font-size:1.3em}.fm-inline .fm-large-op{font-size:1em}fmath mrow{white-space:nowrap}.fm-vert{vertical-align:middle}fmath table,fmath tbody,fmath td,fmath tr{border:0!important;padding:0!important;margin:0!important;outline:0!important}fmath table{border-collapse:collapse!important;text-align:center!important;table-layout:auto!important;float:none!important}.fm-frac{padding:0 1px!important}td.fm-den-frac{border-top:solid thin!important}.fm-root{font-size:.6em}.fm-radicand{padding:0 1px 0 0;border-top:solid;margin-top:.1em}.fm-script{font-size:.71em}.fm-script .fm-script .fm-script{font-size:1em}td.fm-underover-base{line-height:1!important}td.fm-mtd{padding:.5ex .4em!important;vertical-align:baseline!important}fmath mphantom{visibility:hidden}fmath menclose[notation=top],menclose.fm-menclose[notation=top]{border-top:solid thin}fmath menclose[notation=right],menclose.fm-menclose[notation=right]{border-right:solid thin}fmath menclose[notation=bottom],menclose.fm-menclose[notation=bottom]{border-bottom:solid thin}fmath menclose[notation=left],menclose.fm-menclose[notation=left]{border-left:solid thin}fmath menclose[notation=box],menclose.fm-menclose[notation=box]{border:thin solid}fmath none{display:none}</style> Molar conductance of an electrolyte increase with dilution according to the equation:\newline<fmath class="fm-inline"><mrow><msub><mi class="fm-mi-length-1" mathvariant="italic">Λ</mi><mi class="fm-mi-length-1 ma-upright" mathvariant="normal" style="padding-right: 0px;">m</mi></msub><mo class="fm-infix-loose">=</mo><mrow><msubsup><mi class="fm-mi-length-1" mathvariant="italic">Λ</mi><mi class="fm-mi-length-1 ma-upright" mathvariant="normal" style="padding-right: 0px;">m</mi><mo>∘</mo></msubsup><mo class="fm-infix">−</mo><mrow><mi class="fm-mi-length-1 ma-upright" mathvariant="normal" style="padding-right: 0px;">A</mi><mrow mtagname="msqrt"><mo class="fm-radic">√</mo><mi class="fm-mi-length-1 ma-upright" mathvariant="normal" style="padding-right: 0px;">c</mi></mrow></mrow></mrow></mrow></fmath>\newlineWhich of the following statements are true?\newline(A) This equation applies to both strong and weak electrolytes.\newline(B) Value of the constant A depends upon the nature of the solvent.\newline(C) Value of constant <fmath class="fm-inline"><mi class="fm-mi-length-1 ma-upright" mathvariant="normal" style="padding-right: 0px;">A</mi></fmath> is same for both <fmath class="fm-inline"><msub><mi class="ma-repel-adj" mathvariant="normal">BaCl</mi><mn>2</mn></msub></fmath> and <fmath class="fm-inline"><msub><mi class="ma-repel-adj" mathvariant="normal">MgSO</mi><mn>4</mn></msub></fmath>\newline(D) Value of constant <fmath class="fm-inline"><mi class="fm-mi-length-1 ma-upright" mathvariant="normal" style="padding-right: 0px;">A</mi></fmath> is same for both <fmath class="fm-inline"><msub><mi class="ma-repel-adj" mathvariant="normal">BaCl</mi><mn>2</mn></msub></fmath> and <fmath class="fm-inline"><mrow class="ma-repel-adj"><mi class="ma-repel-adj" mathvariant="normal">Mg</mi><msub><mrow><mo class="fm-mo-Luc">(</mo><mi class="ma-repel-adj" mathvariant="normal">OH</mi><mo class="fm-mo-Luc">)</mo></mrow><mn>2</mn></msub></mrow></fmath>\newlineChoose the most appropriate answer from the options given below:\newline 
\begin{enumerate}[label=(\alph*)]
\item  (A) and (B) only
\item  (A), (B) and (C) only
\item  (B) and (C) only
\item  (B) and (D) only
\end{enumerate}
\newpage
\section*{Question 14}
<style>.fm-math,fmath{font-family:STIXGeneral,'DejaVu Serif','DejaVu Sans',Times,OpenSymbol,'Standard Symbols L',serif;line-height:1.2}.fm-math mtext,fmath mtext{line-height:normal}.fm-mo,.ma-sans-serif,fmath mi[mathvariant*=sans-serif],fmath mn[mathvariant*=sans-serif],fmath mo,fmath ms[mathvariant*=sans-serif],fmath mtext[mathvariant*=sans-serif]{font-family:STIXGeneral,'DejaVu Sans','DejaVu Serif','Arial Unicode MS','Lucida Grande',Times,OpenSymbol,'Standard Symbols L',sans-serif}.fm-mo-Luc{font-family:STIXGeneral,'DejaVu Sans','DejaVu Serif','Lucida Grande','Arial Unicode MS',Times,OpenSymbol,'Standard Symbols L',sans-serif}.questionsfont{font-weight:200;font-family:Arial, sans-serif, STIXGeneral,'DejaVu Sans','DejaVu Serif','Lucida Grande','Arial Unicode MS',Times,OpenSymbol,'Standard Symbols L',sans-serif!important}.fm-separator{padding:0 .56ex 0 0}.fm-infix-loose{padding:0 .56ex}.fm-infix{padding:0 .44ex}.fm-prefix{padding:0 .33ex 0 0}.fm-postfix{padding:0 0 0 .33ex}.fm-prefix-tight{padding:0 .11ex 0 0}.fm-postfix-tight{padding:0 0 0 .11ex}.fm-quantifier{padding:0 .11ex 0 .22ex}.ma-non-marking{display:none}.fm-vert,fmath menclose,menclose.fm-menclose{display:inline-block}.fm-large-op{font-size:1.3em}.fm-inline .fm-large-op{font-size:1em}fmath mrow{white-space:nowrap}.fm-vert{vertical-align:middle}fmath table,fmath tbody,fmath td,fmath tr{border:0!important;padding:0!important;margin:0!important;outline:0!important}fmath table{border-collapse:collapse!important;text-align:center!important;table-layout:auto!important;float:none!important}.fm-frac{padding:0 1px!important}td.fm-den-frac{border-top:solid thin!important}.fm-root{font-size:.6em}.fm-radicand{padding:0 1px 0 0;border-top:solid;margin-top:.1em}.fm-script{font-size:.71em}.fm-script .fm-script .fm-script{font-size:1em}td.fm-underover-base{line-height:1!important}td.fm-mtd{padding:.5ex .4em!important;vertical-align:baseline!important}fmath mphantom{visibility:hidden}fmath menclose[notation=top],menclose.fm-menclose[notation=top]{border-top:solid thin}fmath menclose[notation=right],menclose.fm-menclose[notation=right]{border-right:solid thin}fmath menclose[notation=bottom],menclose.fm-menclose[notation=bottom]{border-bottom:solid thin}fmath menclose[notation=left],menclose.fm-menclose[notation=left]{border-left:solid thin}fmath menclose[notation=box],menclose.fm-menclose[notation=box]{border:thin solid}fmath none{display:none}</style> The molar conductance of <fmath class="fm-inline"><mrow><mrow><mrow><mrow><mi class="fm-mi-length-1" mathvariant="italic" style="padding-right: 0.44ex;">N</mi><mi class="fm-mi-length-1" mathvariant="italic">a</mi></mrow><mi class="fm-mi-length-1" mathvariant="italic">C</mi></mrow><mi class="fm-mi-length-1" mathvariant="italic" style="padding-right: 0.44ex;">l</mi></mrow><mo class="fm-separator">,</mo><mrow><mrow><mi class="fm-mi-length-1" mathvariant="italic" style="padding-right: 0.44ex;">H</mi><mi class="fm-mi-length-1" mathvariant="italic">C</mi></mrow><mi class="fm-mi-length-1" mathvariant="italic" style="padding-right: 0.44ex;">l</mi></mrow></mrow></fmath> and <fmath class="fm-inline"><mrow><msub><mrow><mi class="fm-mi-length-1" mathvariant="italic">C</mi><mi class="fm-mi-length-1" mathvariant="italic" style="padding-right: 0.44ex;">H</mi></mrow><mn>3</mn></msub><mrow><mrow><mrow><mrow><mi class="fm-mi-length-1" mathvariant="italic">C</mi><mi class="fm-mi-length-1" mathvariant="italic">O</mi></mrow><mi class="fm-mi-length-1" mathvariant="italic">O</mi></mrow><mi class="fm-mi-length-1" mathvariant="italic" style="padding-right: 0.44ex;">N</mi></mrow><mi class="fm-mi-length-1" mathvariant="italic">a</mi></mrow></mrow></fmath> at infinite dilution are 126.45, <fmath class="fm-inline"><mn>426.16</mn></fmath> and <fmath class="fm-inline"><mrow><mrow><mrow><mn>91.0</mn><mi class="fm-mi-length-1" mathvariant="italic">S</mi></mrow><msup><mrow><mi class="fm-mi-length-1" mathvariant="italic">c</mi><mi class="fm-mi-length-1" mathvariant="italic">m</mi></mrow><mn>2</mn></msup></mrow><msup><mrow><mrow><mi class="fm-mi-length-1" mathvariant="italic">m</mi><mi class="fm-mi-length-1" mathvariant="italic">o</mi></mrow><mi class="fm-mi-length-1" mathvariant="italic" style="padding-right: 0.44ex;">l</mi></mrow><mrow><mo class="fm-prefix-tight">−</mo><mn>1</mn></mrow></msup></mrow></fmath> respectively. The molar conductance of <fmath class="fm-inline"><mrow><msub><mrow><mi class="fm-mi-length-1" mathvariant="italic">C</mi><mi class="fm-mi-length-1" mathvariant="italic" style="padding-right: 0.44ex;">H</mi></mrow><mn>3</mn></msub><mrow><mrow><mrow><mi class="fm-mi-length-1" mathvariant="italic">C</mi><mi class="fm-mi-length-1" mathvariant="italic">O</mi></mrow><mi class="fm-mi-length-1" mathvariant="italic">O</mi></mrow><mi class="fm-mi-length-1" mathvariant="italic" style="padding-right: 0.44ex;">H</mi></mrow></mrow></fmath> at infinite dilution is. Choose the right option for your answer.
\begin{enumerate}[label=(\alph*)]
\item  <fmath class="fm-inline"><mrow><mrow><mrow><mrow><mrow><mrow><mn>201.28</mn><mi class="fm-mi-length-1 ma-upright" mathvariant="normal" style="padding-right: 0px;"> </mi></mrow><mi class="fm-mi-length-1" mathvariant="italic">S</mi></mrow><mi class="fm-mi-length-1 ma-upright" mathvariant="normal" style="padding-right: 0px;"> </mi></mrow><msup><mrow><mi class="fm-mi-length-1" mathvariant="italic">c</mi><mi class="fm-mi-length-1" mathvariant="italic">m</mi></mrow><mn>2</mn></msup></mrow><mi class="fm-mi-length-1 ma-upright" mathvariant="normal" style="padding-right: 0px;"> </mi></mrow><msup><mrow><mrow><mi class="fm-mi-length-1" mathvariant="italic">m</mi><mi class="fm-mi-length-1" mathvariant="italic">o</mi></mrow><mi class="fm-mi-length-1" mathvariant="italic" style="padding-right: 0.44ex;">l</mi></mrow><mrow><mo class="fm-prefix-tight">−</mo><mn>1</mn></mrow></msup></mrow></fmath>
\item  <fmath class="fm-inline"><mrow><mrow><mrow><mrow><mrow><mrow><mn>390.71</mn><mi class="fm-mi-length-1 ma-upright" mathvariant="normal" style="padding-right: 0px;"> </mi></mrow><mi class="fm-mi-length-1" mathvariant="italic">S</mi></mrow><mi class="fm-mi-length-1 ma-upright" mathvariant="normal" style="padding-right: 0px;"> </mi></mrow><msup><mrow><mi class="fm-mi-length-1" mathvariant="italic">c</mi><mi class="fm-mi-length-1" mathvariant="italic">m</mi></mrow><mn>2</mn></msup></mrow><mi class="fm-mi-length-1 ma-upright" mathvariant="normal" style="padding-right: 0px;"> </mi></mrow><msup><mrow><mrow><mi class="fm-mi-length-1" mathvariant="italic">m</mi><mi class="fm-mi-length-1" mathvariant="italic">o</mi></mrow><mi class="fm-mi-length-1" mathvariant="italic" style="padding-right: 0.44ex;">l</mi></mrow><mrow><mo class="fm-prefix-tight">−</mo><mn>1</mn></mrow></msup></mrow></fmath>
\item  <fmath class="fm-inline"><mrow><mrow><mrow><mrow><mrow><mrow><mn>698.28</mn><mi class="fm-mi-length-1 ma-upright" mathvariant="normal" style="padding-right: 0px;"> </mi></mrow><mi class="fm-mi-length-1" mathvariant="italic">S</mi></mrow><mi class="fm-mi-length-1 ma-upright" mathvariant="normal" style="padding-right: 0px;"> </mi></mrow><msup><mrow><mi class="fm-mi-length-1" mathvariant="italic">c</mi><mi class="fm-mi-length-1" mathvariant="italic">m</mi></mrow><mn>2</mn></msup></mrow><mi class="fm-mi-length-1 ma-upright" mathvariant="normal" style="padding-right: 0px;"> </mi></mrow><msup><mrow><mrow><mi class="fm-mi-length-1" mathvariant="italic">m</mi><mi class="fm-mi-length-1" mathvariant="italic">o</mi></mrow><mi class="fm-mi-length-1" mathvariant="italic" style="padding-right: 0.44ex;">l</mi></mrow><mrow><mo class="fm-prefix-tight">−</mo><mn>1</mn></mrow></msup></mrow></fmath>
\item  <fmath class="fm-inline"><mrow><mrow><mrow><mrow><mrow><mrow><mn>540.48</mn><mi class="fm-mi-length-1 ma-upright" mathvariant="normal" style="padding-right: 0px;"> </mi></mrow><mi class="fm-mi-length-1" mathvariant="italic">S</mi></mrow><mi class="fm-mi-length-1 ma-upright" mathvariant="normal" style="padding-right: 0px;"> </mi></mrow><msup><mrow><mi class="fm-mi-length-1" mathvariant="italic">c</mi><mi class="fm-mi-length-1" mathvariant="italic">m</mi></mrow><mn>2</mn></msup></mrow><mi class="fm-mi-length-1 ma-upright" mathvariant="normal" style="padding-right: 0px;"> </mi></mrow><msup><mrow><mrow><mi class="fm-mi-length-1" mathvariant="italic">m</mi><mi class="fm-mi-length-1" mathvariant="italic">o</mi></mrow><mi class="fm-mi-length-1" mathvariant="italic" style="padding-right: 0.44ex;">l</mi></mrow><mrow><mo class="fm-prefix-tight">−</mo><mn>1</mn></mrow></msup></mrow></fmath>
\end{enumerate}
\newpage
\section*{Question 15}
<style>.fm-math,fmath{font-family:STIXGeneral,'DejaVu Serif','DejaVu Sans',Times,OpenSymbol,'Standard Symbols L',serif;line-height:1.2}.fm-math mtext,fmath mtext{line-height:normal}.fm-mo,.ma-sans-serif,fmath mi[mathvariant*=sans-serif],fmath mn[mathvariant*=sans-serif],fmath mo,fmath ms[mathvariant*=sans-serif],fmath mtext[mathvariant*=sans-serif]{font-family:STIXGeneral,'DejaVu Sans','DejaVu Serif','Arial Unicode MS','Lucida Grande',Times,OpenSymbol,'Standard Symbols L',sans-serif}.fm-mo-Luc{font-family:STIXGeneral,'DejaVu Sans','DejaVu Serif','Lucida Grande','Arial Unicode MS',Times,OpenSymbol,'Standard Symbols L',sans-serif}.questionsfont{font-weight:200;font-family:Arial, sans-serif, STIXGeneral,'DejaVu Sans','DejaVu Serif','Lucida Grande','Arial Unicode MS',Times,OpenSymbol,'Standard Symbols L',sans-serif!important}.fm-separator{padding:0 .56ex 0 0}.fm-infix-loose{padding:0 .56ex}.fm-infix{padding:0 .44ex}.fm-prefix{padding:0 .33ex 0 0}.fm-postfix{padding:0 0 0 .33ex}.fm-prefix-tight{padding:0 .11ex 0 0}.fm-postfix-tight{padding:0 0 0 .11ex}.fm-quantifier{padding:0 .11ex 0 .22ex}.ma-non-marking{display:none}.fm-vert,fmath menclose,menclose.fm-menclose{display:inline-block}.fm-large-op{font-size:1.3em}.fm-inline .fm-large-op{font-size:1em}fmath mrow{white-space:nowrap}.fm-vert{vertical-align:middle}fmath table,fmath tbody,fmath td,fmath tr{border:0!important;padding:0!important;margin:0!important;outline:0!important}fmath table{border-collapse:collapse!important;text-align:center!important;table-layout:auto!important;float:none!important}.fm-frac{padding:0 1px!important}td.fm-den-frac{border-top:solid thin!important}.fm-root{font-size:.6em}.fm-radicand{padding:0 1px 0 0;border-top:solid;margin-top:.1em}.fm-script{font-size:.71em}.fm-script .fm-script .fm-script{font-size:1em}td.fm-underover-base{line-height:1!important}td.fm-mtd{padding:.5ex .4em!important;vertical-align:baseline!important}fmath mphantom{visibility:hidden}fmath menclose[notation=top],menclose.fm-menclose[notation=top]{border-top:solid thin}fmath menclose[notation=right],menclose.fm-menclose[notation=right]{border-right:solid thin}fmath menclose[notation=bottom],menclose.fm-menclose[notation=bottom]{border-bottom:solid thin}fmath menclose[notation=left],menclose.fm-menclose[notation=left]{border-left:solid thin}fmath menclose[notation=box],menclose.fm-menclose[notation=box]{border:thin solid}fmath none{display:none}</style> The standard electrode potential <fmath class="fm-inline"><mrow><mo class="fm-mo-Luc">(</mo><msup><mi class="fm-mi-length-1 ma-upright" mathvariant="normal" style="padding-right: 0px;">E</mi><mo>−</mo></msup><mo class="fm-mo-Luc">)</mo></mrow></fmath>values of <fmath class="fm-inline"><mrow><mrow><mrow><msup><mi class="ma-repel-adj" mathvariant="normal">Al</mi><mrow><mn>3</mn><mo class="fm-postfix-tight">+</mo></mrow></msup><mo class="fm-infix-loose">∕</mo><msup><mi class="ma-repel-adj" mathvariant="normal">Al</mi><mn>2</mn></msup></mrow><mo class="fm-separator">,</mo><mrow><msup><mi class="ma-repel-adj" mathvariant="normal">Ag</mi><mo>+</mo></msup><mo class="fm-infix-loose">∕</mo><mi class="ma-repel-adj" mathvariant="normal">Ag</mi></mrow></mrow><mo class="fm-separator">,</mo><mrow><msup><mi class="fm-mi-length-1 ma-upright" mathvariant="normal" style="padding-right: 0px;">K</mi><mo>+</mo></msup><mo class="fm-infix-loose">∕</mo><mi class="fm-mi-length-1 ma-upright" mathvariant="normal" style="padding-right: 0px;">K</mi></mrow></mrow></fmath> and <fmath class="fm-inline"><mrow><msup><mi class="ma-repel-adj" mathvariant="normal">Cr</mi><mrow><mn>3</mn><mo class="fm-postfix-tight">+</mo></mrow></msup><mo class="fm-infix-loose">∕</mo><mi class="ma-repel-adj" mathvariant="normal">Cr</mi></mrow></fmath> are <fmath class="fm-inline"><mrow><mrow><mo class="fm-prefix-tight">−</mo><mrow><mn>1.66</mn><mi class="fm-mi-length-1 ma-upright" mathvariant="normal" style="padding-right: 0px;">V</mi></mrow></mrow><mo class="fm-separator">,</mo><mrow><mn>0.80</mn><mi class="fm-mi-length-1 ma-upright" mathvariant="normal" style="padding-right: 0px;">V</mi></mrow></mrow></fmath>, <fmath class="fm-inline"><mrow><mo class="fm-prefix-tight">−</mo><mrow><mn>2.93</mn><mi class="fm-mi-length-1 ma-upright" mathvariant="normal" style="padding-right: 0px;">V</mi></mrow></mrow></fmath> and <fmath class="fm-inline"><mrow><mo class="fm-prefix-tight">−</mo><mrow><mn>0.74</mn><mi class="fm-mi-length-1 ma-upright" mathvariant="normal" style="padding-right: 0px;">V</mi></mrow></mrow></fmath>, respectively. The correct decreasing order of reducing power of the metal is :\newline 
\begin{enumerate}[label=(\alph*)]
\item  <fmath class="fm-inline"><mrow><mrow><mrow><mi class="ma-repel-adj" mathvariant="normal">Ag</mi><mo class="fm-infix-loose">></mo><mi class="ma-repel-adj" mathvariant="normal">Cr</mi></mrow><mo class="fm-infix-loose">></mo><mi class="ma-repel-adj" mathvariant="normal">Al</mi></mrow><mo class="fm-infix-loose">></mo><mi class="fm-mi-length-1 ma-upright" mathvariant="normal" style="padding-right: 0px;">K</mi></mrow></fmath>
\item  <fmath class="fm-inline"><mrow><mrow><mrow><mi class="fm-mi-length-1 ma-upright" mathvariant="normal" style="padding-right: 0px;">K</mi><mo class="fm-infix-loose">></mo><mi class="ma-repel-adj" mathvariant="normal">Al</mi></mrow><mo class="fm-infix-loose">></mo><mi class="ma-repel-adj" mathvariant="normal">Cr</mi></mrow><mo class="fm-infix-loose">></mo><mi class="ma-repel-adj" mathvariant="normal">Ag</mi></mrow></fmath>
\item  <fmath class="fm-inline"><mrow><mrow><mrow><mi class="fm-mi-length-1 ma-upright" mathvariant="normal" style="padding-right: 0px;">K</mi><mo class="fm-infix-loose">></mo><mi class="ma-repel-adj" mathvariant="normal">Al</mi></mrow><mo class="fm-infix-loose">></mo><mi class="ma-repel-adj" mathvariant="normal">Ag</mi></mrow><mo class="fm-infix-loose">></mo><mi class="ma-repel-adj" mathvariant="normal">Cr</mi></mrow></fmath>
\item  <fmath class="fm-inline"><mrow><mrow><mrow><mi class="ma-repel-adj" mathvariant="normal">Al</mi><mo class="fm-infix-loose">></mo><mi class="fm-mi-length-1 ma-upright" mathvariant="normal" style="padding-right: 0px;">K</mi></mrow><mo class="fm-infix-loose">></mo><mi class="ma-repel-adj" mathvariant="normal">Ag</mi></mrow><mo class="fm-infix-loose">></mo><mi class="ma-repel-adj" mathvariant="normal">Cr</mi></mrow></fmath>
\end{enumerate}
\newpage
\section*{Question 16}
<style>.fm-math,fmath{font-family:STIXGeneral,'DejaVu Serif','DejaVu Sans',Times,OpenSymbol,'Standard Symbols L',serif;line-height:1.2}.fm-math mtext,fmath mtext{line-height:normal}.fm-mo,.ma-sans-serif,fmath mi[mathvariant*=sans-serif],fmath mn[mathvariant*=sans-serif],fmath mo,fmath ms[mathvariant*=sans-serif],fmath mtext[mathvariant*=sans-serif]{font-family:STIXGeneral,'DejaVu Sans','DejaVu Serif','Arial Unicode MS','Lucida Grande',Times,OpenSymbol,'Standard Symbols L',sans-serif}.fm-mo-Luc{font-family:STIXGeneral,'DejaVu Sans','DejaVu Serif','Lucida Grande','Arial Unicode MS',Times,OpenSymbol,'Standard Symbols L',sans-serif}.questionsfont{font-weight:200;font-family:Arial, sans-serif, STIXGeneral,'DejaVu Sans','DejaVu Serif','Lucida Grande','Arial Unicode MS',Times,OpenSymbol,'Standard Symbols L',sans-serif!important}.fm-separator{padding:0 .56ex 0 0}.fm-infix-loose{padding:0 .56ex}.fm-infix{padding:0 .44ex}.fm-prefix{padding:0 .33ex 0 0}.fm-postfix{padding:0 0 0 .33ex}.fm-prefix-tight{padding:0 .11ex 0 0}.fm-postfix-tight{padding:0 0 0 .11ex}.fm-quantifier{padding:0 .11ex 0 .22ex}.ma-non-marking{display:none}.fm-vert,fmath menclose,menclose.fm-menclose{display:inline-block}.fm-large-op{font-size:1.3em}.fm-inline .fm-large-op{font-size:1em}fmath mrow{white-space:nowrap}.fm-vert{vertical-align:middle}fmath table,fmath tbody,fmath td,fmath tr{border:0!important;padding:0!important;margin:0!important;outline:0!important}fmath table{border-collapse:collapse!important;text-align:center!important;table-layout:auto!important;float:none!important}.fm-frac{padding:0 1px!important}td.fm-den-frac{border-top:solid thin!important}.fm-root{font-size:.6em}.fm-radicand{padding:0 1px 0 0;border-top:solid;margin-top:.1em}.fm-script{font-size:.71em}.fm-script .fm-script .fm-script{font-size:1em}td.fm-underover-base{line-height:1!important}td.fm-mtd{padding:.5ex .4em!important;vertical-align:baseline!important}fmath mphantom{visibility:hidden}fmath menclose[notation=top],menclose.fm-menclose[notation=top]{border-top:solid thin}fmath menclose[notation=right],menclose.fm-menclose[notation=right]{border-right:solid thin}fmath menclose[notation=bottom],menclose.fm-menclose[notation=bottom]{border-bottom:solid thin}fmath menclose[notation=left],menclose.fm-menclose[notation=left]{border-left:solid thin}fmath menclose[notation=box],menclose.fm-menclose[notation=box]{border:thin solid}fmath none{display:none}</style> When 0.1 mol <fmath class="fm-inline"><msubsup><mtext>MnO</mtext><mn>4</mn><mrow><mn>2</mn><mo class="fm-postfix-tight">−</mo></mrow></msubsup></fmath> is oxidised the quantity of electricity required to completely <fmath class="fm-inline"><msubsup><mtext>MnO</mtext><mn>4</mn><mrow><mn>2</mn><mo class="fm-postfix-tight">−</mo></mrow></msubsup></fmath> to <fmath class="fm-inline"><msubsup><mtext>MnO</mtext><mn>4</mn><mo>−</mo></msubsup></fmath> is 
\begin{enumerate}[label=(\alph*)]
\item <fmath class="fm-inline"><mn>96500</mn></fmath> C
\item <fmath class="fm-inline"><mrow><mn>2</mn><mo class="fm-infix" lspace=".22em" rspace=".22em">×</mo><mn>96500</mn></mrow></fmath> C
\item <fmath class="fm-inline"><mn>9650</mn></fmath> C
\item <fmath class="fm-inline"><mn>96.50</mn></fmath> C
\end{enumerate}
\newpage
\section*{Question 17}
Assertion:During electrolysis \(48250\) coulombs of electricity will deposit \(0.5\) gram equivalent of silver metal from \(\mathrm{Ag}^{+}\)ions.Reason:One Faraday of electricity will be required to deposit \(0.5\) gram equivalent of any substance.
\begin{enumerate}[label=(\alph*)]
\item Both Assertion and Reason are correct and Reason is the correct explanation for Assertion
\item Both Assertion and Reason are correct but Reason is not the correct explanation for Assertion
\item Assertion is correct but Reason is incorrect
\item Assertion is incorrect but Reason is correct
\end{enumerate}
\newpage
\end{document}