\documentclass{article}
                    \usepackage{amsmath}
                    \usepackage{amssymb}
                    \usepackage{graphicx}
                    \usepackage{enumitem}
                    \usepackage{longtable}
                    \title{Neet chem 12}
                    \begin{document}
                    \maketitle
                    \section*{Question 1}
\(12 {~g}\) of urea is present in 1 litre of solution and \(68.4 {~g}\) of sucrose is separately dissolved in 1 litre of another sample of solution. The lowering of vapour pressure of first solution is:
\begin{enumerate}[label=(\alph*)]
\item Equal to that of second solution
\item Greater than that of second solution
\item Less than that of second solution
\item Double that of seconds solution
\end{enumerate}
\newpage
\section*{Question 2}
At a particular temperature, the vapour pressures of two liquids A and B are 120 mm and 180 mm of mercury respectively. If 2 moles of A and 3 moles of B are mixed to form an ideal solution, the vapour pressure of the solution at the same temperature will be: (in mm of mercury)
\begin{enumerate}[label=(\alph*)]
\item 156
\item 145
\item 150
\item 108
\end{enumerate}
\newpage
\section*{Question 3}
Which of the following 0.10m aqueous solution will have the lowest freezing point?
\begin{enumerate}[label=(\alph*)]
\item \(\mathrm{Al}_{2}\left(\mathrm{SO}_{4}\right)_{3}\)
\item \(\mathrm{C}_{5} \mathrm{H}_{10} \mathrm{O}_{8}\)
\item \(\mathrm{KI}\)
\item \(\mathrm{C}_{12} \mathrm{H}_{22} \mathrm{O}_{11}\)
\end{enumerate}
\newpage
\section*{Question 4}
During osmosis, flow of water through a semi-permeable membrane is:
\begin{enumerate}[label=(\alph*)]
\item From a solution having higher concentration only.
\item From both sides of a semi-permeable membrane with equal flow rates.
\item From both sides of a semi-permeable membrane with unequal flow rates.
\item From a solution having lower concentration only.
\end{enumerate}
\newpage
\section*{Question 5}
If benzene in solution containing 30% by mass in carbon tetrachloride, calculate the mole fraction of benzene.
\begin{enumerate}[label=(\alph*)]
\item  0.412
\item  0.326
\item  0.529
\item  0.458
\end{enumerate}
\newpage
\section*{Question 6}
Calculate the amount of benzoic acid \(\left({C}_{6} {H}_{5} {COOH}\right)\) required for preparing 250 \({mL}\) of \(0.15 {M}\) solution in methanol.
\begin{enumerate}[label=(\alph*)]
\item 4.680 g
\item 4.790 g
\item 4.875 g
\item 4.575 g
\end{enumerate}
\newpage
\section*{Question 7}
Which of the following pair will form an ideal solution?
\begin{enumerate}[label=(\alph*)]
\item Chlorobenzen, chloro ethane
\item Benzene, Toluene
\item Acetone, chloroform
\item Water, \(\mathrm{HCl}\)
\end{enumerate}
\newpage
\section*{Question 8}
Exactly \(1 {~g}\) of urea dissolved in \(75 {~g}\) of water gives a solution that boils at \(100.114^{\circ} {C}\) at 760 torr. The molecular weight of urea is \(60.1\). The boiling point elevation constant for water is:
\begin{enumerate}[label=(\alph*)]
\item 1.02
\item 0.51
\item 3.06
\item 1.51
\end{enumerate}
\newpage
\section*{Question 9}
Calculate the molarity and molality of \(20 \%\) aqueous ethanol \(\left(\mathrm{C}_2 \mathrm{H}_5 \mathrm{OH}\right.\) ) solution by volume. (Density of solution \(=0.96\) \(\mathrm{g} / \mathrm{mL})\)
\begin{enumerate}[label=(\alph*)]
\item Molarity \(=3.47\), Molality \(=4.34\)
\item Molarity \(=4.35\), Molality \(=3.48\)
\item Molarity \(=4.48\), Molality \(=3.35\)
\item None of the above
\end{enumerate}
\newpage
\section*{Question 10}
Calculate van't Hoff factor for 0.2 m aqueous solution of KCl which freezes at −0.680\(^{\circ}\)C.(K$_{f}$=1.86 K kg mol$^{−1}$)\newline
\begin{enumerate}[label=(\alph*)]
\item \(3.72\)
\item \(1.83\)
\item \(6.8\)
\item \(1.86\)
\end{enumerate}
\newpage
\section*{Question 11}
Solubility of a gas in liquid increases with:
\begin{enumerate}[label=(\alph*)]
\item Increase of pressure and increase of temperature
\item Decrease of pressure and increase of temperature
\item Increase of pressure and decrease of temperature
\item Decrease of pressure and decrease of temperature
\end{enumerate}
\newpage
\section*{Question 12}
Value of Henry's constant \(\mathrm{k_H}\)?\newline
\begin{enumerate}[label=(\alph*)]
\item Increases with increase in temperature
\item Decreases with increase in temperature
\item Remains constant
\item First increases then decreases
\end{enumerate}
\newpage
\section*{Question 13}
The colligative properties of a solution depend on:
\begin{enumerate}[label=(\alph*)]
\item The number of solute particles present in it
\item The chemical nature of the solute particles present in it
\item The nature of the solvent used
\item None of these
\end{enumerate}
\newpage
\section*{Question 14}
Van't Hoff factor, when benzoic acid is dissolved in benzene, will be:
\begin{enumerate}[label=(\alph*)]
\item \(2\)
\item \(1\)
\item \(0.5\)
\item \(1.5\)
\end{enumerate}
\newpage
\end{document}