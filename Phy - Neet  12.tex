\documentclass{article}
                    \usepackage{amsmath}
                    \usepackage{amssymb}
                    \usepackage{graphicx}
                    \usepackage{enumitem}
                    \usepackage{longtable}
                    \title{Phy - Neet  12}
                    \begin{document}
                    \maketitle
                    \section*{Question 1}
\(1 kWh =\) ______ \(J\)
\begin{enumerate}[label=(\alph*)]
\item \(3.6 \times 10^{4} J\)
\item \(3.6 \times 10^{5} J\)
\item \(3.6 \times 10^{6} J\)
\item \(3.6 \times 10^{3} J\)
\end{enumerate}
\newpage
\section*{Question 2}
Find the equivalent resistance of the circuit.\includegraphics[width=\textwidth]{static/media/wl_client/1/qdump/e15a7351bd3aa55f4a06d4d7dd6f96fc/7d957e0d359e5c99820f2356725c120d.jpeg}
\begin{enumerate}[label=(\alph*)]
\item 2 Ω
\item 4 Ω
\item 6 Ω
\item 8 Ω
\end{enumerate}
\newpage
\section*{Question 3}
An electron is placed in an electric field of intensity \(10^{4}\) Newton per Coulomb. The electric force working on the electron is:
\begin{enumerate}[label=(\alph*)]
\item \(0.625 \times 10^{13}\) Newton
\item \(0.625 \times 10^{-15}\) Newton
\item \(1.6 \times 10^{15}\) Newton
\item \(1.6 \times 10^{-15}\) Newton
\end{enumerate}
\newpage
\section*{Question 4}
A wire of resistance \(12 \mathrm{ohm} /\) meter is bent to form a complete circle of radius \(10 \mathrm{~cm}\). The resistance between its two diametrically opposite points, \(A\) and \(B\) as shown in the figure is?\includegraphics[width=\textwidth]{https://testseries.edugorilla.com/static/media/wl_client/1/qdump/e55a342d057a52acc35b29ffdedd2ff4/fd58117906de9b3bdc0f49689317a2df.png}
\begin{enumerate}[label=(\alph*)]
\item \(3 \Omega\)
\item \(6 \pi \Omega\)
\item \(6 \Omega\)
\item \(0.6 \pi \Omega\)
\end{enumerate}
\newpage
\section*{Question 5}
Two cells of emf's \(1.25 \mathrm{~V}\) and \(0.75 \mathrm{~V}\) having equal internal resistance are connected in parallel. The effective emf is:
\begin{enumerate}[label=(\alph*)]
\item \(0.75{\mathrm{V}}\)
\item \(1.25 \mathrm{~V}\)
\item \(2.0 \mathrm{~V}\)
\item \(1.0 \mathrm{~V}\)
\end{enumerate}
\newpage
\section*{Question 6}
A current of \(2~ A\) flows in conductors as shown. The potential difference \(V_A-V_B\) will be:\newline\includegraphics[width=\textwidth]{https://testseries.edugorilla.com/static/media/wl_client/1/qdump/e15a7351bd3aa55f4a06d4d7dd6f96fc/af8244670ba631de9614f9ff5c8f691c.jpeg}
\begin{enumerate}[label=(\alph*)]
\item \(+4 V\)
\item \(-1 {~V}\)
\item \(+1 {~V}\)
\item \(+2 {~V}\)
\end{enumerate}
\newpage
\section*{Question 7}
The temperature coefficient of resistance of a wire is \(0.00125\) per degree celcius. At \(300 {~K}\) its resistance is \(1\) ohm. The resistance of the wire will be \(2\) ohms at a temperature:
\begin{enumerate}[label=(\alph*)]
\item \(1154 {~K}\)
\item \(1127 {~K}\)
\item \(600~ K\)
\item \(1400 {~K}\)
\end{enumerate}
\newpage
\section*{Question 8}
1. Three resistors \(2 \Omega, 4 \Omega\) and \(5 \Omega\) are combined in parallel. What is the total resistance of the combination?2. If the combination is connected to a battery of emf \(20 \mathrm{~V}\) and negligible internal resistance, determine the current through each resistor, and the total current drawn from the battery.
\begin{enumerate}[label=(\alph*)]
\item \(\frac{20}{19} \Omega\) and \(19A\)
\item \(\frac{25}{19} \Omega\) and \(29A\)\newline
\item \(\frac{18}{15} \Omega\) and \(22A\)\newline
\item \(\frac{16}{19} \Omega\) and \(17A\)\newline
\end{enumerate}
\newpage
\section*{Question 9}
A uniform wire of resistance \(50 \Omega\) is cut into 5 equal parts. These parts are now connected in parallel. The equivalent resistance of the combination is:
\begin{enumerate}[label=(\alph*)]
\item \(2 \Omega\)
\item \(10 \Omega\)
\item \(250 \Omega\)
\item \(6250 \Omega\)
\end{enumerate}
\newpage
\section*{Question 10}
Two resistors \(R_{1}\) and \(R_{2}\) of \(4 \Omega\) and \(6 \Omega\) are connected in parallel across a battery. The ratio of power dissipated in them, \(P_{1}\) : \(\mathrm{P}_{2}\) will be:
\begin{enumerate}[label=(\alph*)]
\item \(4: 9\)
\item \(3: 2\)
\item \(9: 4\)
\item \(2: 3\)
\end{enumerate}
\newpage
\section*{Question 11}
Meter bridge can be used to find resistance of:
\begin{enumerate}[label=(\alph*)]
\item high value
\item moderate value
\item low value
\item All of the above
\end{enumerate}
\newpage
\section*{Question 12}
<style>.fm-math,fmath{font-family:STIXGeneral,'DejaVu Serif','DejaVu Sans',Times,OpenSymbol,'Standard Symbols L',serif;line-height:1.2}.fm-math mtext,fmath mtext{line-height:normal}.fm-mo,.ma-sans-serif,fmath mi[mathvariant*=sans-serif],fmath mn[mathvariant*=sans-serif],fmath mo,fmath ms[mathvariant*=sans-serif],fmath mtext[mathvariant*=sans-serif]{font-family:STIXGeneral,'DejaVu Sans','DejaVu Serif','Arial Unicode MS','Lucida Grande',Times,OpenSymbol,'Standard Symbols L',sans-serif}.fm-mo-Luc{font-family:STIXGeneral,'DejaVu Sans','DejaVu Serif','Lucida Grande','Arial Unicode MS',Times,OpenSymbol,'Standard Symbols L',sans-serif}.questionsfont{font-weight:200;font-family:Arial, sans-serif, STIXGeneral,'DejaVu Sans','DejaVu Serif','Lucida Grande','Arial Unicode MS',Times,OpenSymbol,'Standard Symbols L',sans-serif!important}.fm-separator{padding:0 .56ex 0 0}.fm-infix-loose{padding:0 .56ex}.fm-infix{padding:0 .44ex}.fm-prefix{padding:0 .33ex 0 0}.fm-postfix{padding:0 0 0 .33ex}.fm-prefix-tight{padding:0 .11ex 0 0}.fm-postfix-tight{padding:0 0 0 .11ex}.fm-quantifier{padding:0 .11ex 0 .22ex}.ma-non-marking{display:none}.fm-vert,fmath menclose,menclose.fm-menclose{display:inline-block}.fm-large-op{font-size:1.3em}.fm-inline .fm-large-op{font-size:1em}fmath mrow{white-space:nowrap}.fm-vert{vertical-align:middle}fmath table,fmath tbody,fmath td,fmath tr{border:0!important;padding:0!important;margin:0!important;outline:0!important}fmath table{border-collapse:collapse!important;text-align:center!important;table-layout:auto!important;float:none!important}.fm-frac{padding:0 1px!important}td.fm-den-frac{border-top:solid thin!important}.fm-root{font-size:.6em}.fm-radicand{padding:0 1px 0 0;border-top:solid;margin-top:.1em}.fm-script{font-size:.71em}.fm-script .fm-script .fm-script{font-size:1em}td.fm-underover-base{line-height:1!important}td.fm-mtd{padding:.5ex .4em!important;vertical-align:baseline!important}fmath mphantom{visibility:hidden}fmath menclose[notation=top],menclose.fm-menclose[notation=top]{border-top:solid thin}fmath menclose[notation=right],menclose.fm-menclose[notation=right]{border-right:solid thin}fmath menclose[notation=bottom],menclose.fm-menclose[notation=bottom]{border-bottom:solid thin}fmath menclose[notation=left],menclose.fm-menclose[notation=left]{border-left:solid thin}fmath menclose[notation=box],menclose.fm-menclose[notation=box]{border:thin solid}fmath none{display:none}</style> A charged particle having drift velocity of <fmath class="fm-inline"><mrow><mn>7.5</mn><mo class="fm-infix" lspace=".22em" rspace=".22em">×</mo><mrow><mrow><msup><mn>10</mn><mrow><mo class="fm-prefix-tight">−</mo><mn>4</mn></mrow></msup><mi class="fm-mi-length-1" mathvariant="italic">m</mi></mrow><msup><mi class="fm-mi-length-1" mathvariant="italic">s</mi><mrow><mo class="fm-prefix-tight">−</mo><mn>1</mn></mrow></msup></mrow></mrow></fmath> in an electric field of <fmath class="fm-inline"><mrow><mn>3</mn><mo class="fm-infix" lspace=".22em" rspace=".22em">×</mo><mrow><mrow><msup><mn>10</mn><mrow><mo class="fm-prefix-tight">−</mo><mn>10</mn></mrow></msup><mi class="fm-mi-length-1" mathvariant="italic" style="padding-right: 0.44ex;">V</mi></mrow><msup><mi class="fm-mi-length-1" mathvariant="italic">m</mi><mrow><mo class="fm-prefix-tight">−</mo><mn>1</mn></mrow></msup></mrow></mrow></fmath> , has a mobility in <fmath class="fm-inline"><mrow><mrow><msup><mi class="fm-mi-length-1" mathvariant="italic">m</mi><mn>2</mn></msup><msup><mi class="fm-mi-length-1" mathvariant="italic" style="padding-right: 0.44ex;">V</mi><mrow><mo class="fm-prefix-tight">−</mo><mn>1</mn></mrow></msup></mrow><msup><mi class="fm-mi-length-1" mathvariant="italic">s</mi><mrow><mo class="fm-prefix-tight">−</mo><mn>1</mn></mrow></msup></mrow></fmath> of :  
\begin{enumerate}[label=(\alph*)]
\item  <fmath class="fm-inline"><mrow><mn>2.5</mn><mo class="fm-infix" lspace=".22em" rspace=".22em">×</mo><msup><mn>10</mn><mn>6</mn></msup></mrow></fmath> 
\item  <fmath class="fm-inline"><mrow><mn>2.5</mn><mo class="fm-infix" lspace=".22em" rspace=".22em">×</mo><msup><mn>10</mn><mrow><mo class="fm-prefix-tight">−</mo><mn>6</mn></mrow></msup></mrow></fmath> 
\item  <fmath class="fm-inline"><mrow><mn>2.25</mn><mo class="fm-infix" lspace=".22em" rspace=".22em">×</mo><msup><mn>10</mn><mrow><mo class="fm-prefix-tight">−</mo><mn>15</mn></mrow></msup></mrow></fmath> 
\item  <fmath class="fm-inline"><mrow><mn>2.25</mn><mo class="fm-infix" lspace=".22em" rspace=".22em">×</mo><msup><mn>10</mn><mn>15</mn></msup></mrow></fmath> 
\end{enumerate}
\newpage
\section*{Question 13}
A set of ‘n’ equal resistors, of value ‘R’ each,are connected in series to a battery of emf ‘E’ and internal resistance ‘R’. The current drawn is I. Now, the ‘n’ resistors are connected in parallel to the same battery. Then the current drawn from battery becomes 10 I. The value of‘n’ is 
\begin{enumerate}[label=(\alph*)]
\item 10
\item 11
\item 9
\item 20
\end{enumerate}
\newpage
\section*{Question 14}
A potentiometer wire has length 4 m and resistance 8W. The resistance that must be connected in series with the wire and an accumulator of e.m.f. 2V, so as to get a potential gradient 1 mV per cm on the wire is 
\begin{enumerate}[label=(\alph*)]
\item <fmath class="fm-inline"><mrow><mn>40</mn><mspace style="margin-right: 0.28em; padding-right: 0.001em; visibility: hidden;" width=".28em">‌</mspace><mi class="fm-mi-length-1" mathvariant="italic">Ω</mi></mrow></fmath>
\item <fmath class="fm-inline"><mrow><mn>44</mn><mspace style="margin-right: 0.28em; padding-right: 0.001em; visibility: hidden;" width=".28em">‌</mspace><mi class="fm-mi-length-1" mathvariant="italic">Ω</mi></mrow></fmath>
\item <fmath class="fm-inline"><mrow><mn>48</mn><mspace style="margin-right: 0.28em; padding-right: 0.001em; visibility: hidden;" width=".28em">‌</mspace><mi class="fm-mi-length-1" mathvariant="italic">Ω</mi></mrow></fmath>
\item <fmath class="fm-inline"><mrow><mn>32</mn><mspace style="margin-right: 0.28em; padding-right: 0.001em; visibility: hidden;" width=".28em">‌</mspace><mi class="fm-mi-length-1" mathvariant="italic">Ω</mi></mrow></fmath>
\end{enumerate}
\newpage
\section*{Question 15}
<style>.fm-math,fmath{font-family:STIXGeneral,'DejaVu Serif','DejaVu Sans',Times,OpenSymbol,'Standard Symbols L',serif;line-height:1.2}.fm-math mtext,fmath mtext{line-height:normal}.fm-mo,.ma-sans-serif,fmath mi[mathvariant*=sans-serif],fmath mn[mathvariant*=sans-serif],fmath mo,fmath ms[mathvariant*=sans-serif],fmath mtext[mathvariant*=sans-serif]{font-family:STIXGeneral,'DejaVu Sans','DejaVu Serif','Arial Unicode MS','Lucida Grande',Times,OpenSymbol,'Standard Symbols L',sans-serif}.fm-mo-Luc{font-family:STIXGeneral,'DejaVu Sans','DejaVu Serif','Lucida Grande','Arial Unicode MS',Times,OpenSymbol,'Standard Symbols L',sans-serif}.questionsfont{font-weight:200;font-family:Arial, sans-serif, STIXGeneral,'DejaVu Sans','DejaVu Serif','Lucida Grande','Arial Unicode MS',Times,OpenSymbol,'Standard Symbols L',sans-serif!important}.fm-separator{padding:0 .56ex 0 0}.fm-infix-loose{padding:0 .56ex}.fm-infix{padding:0 .44ex}.fm-prefix{padding:0 .33ex 0 0}.fm-postfix{padding:0 0 0 .33ex}.fm-prefix-tight{padding:0 .11ex 0 0}.fm-postfix-tight{padding:0 0 0 .11ex}.fm-quantifier{padding:0 .11ex 0 .22ex}.ma-non-marking{display:none}.fm-vert,fmath menclose,menclose.fm-menclose{display:inline-block}.fm-large-op{font-size:1.3em}.fm-inline .fm-large-op{font-size:1em}fmath mrow{white-space:nowrap}.fm-vert{vertical-align:middle}fmath table,fmath tbody,fmath td,fmath tr{border:0!important;padding:0!important;margin:0!important;outline:0!important}fmath table{border-collapse:collapse!important;text-align:center!important;table-layout:auto!important;float:none!important}.fm-frac{padding:0 1px!important}td.fm-den-frac{border-top:solid thin!important}.fm-root{font-size:.6em}.fm-radicand{padding:0 1px 0 0;border-top:solid;margin-top:.1em}.fm-script{font-size:.71em}.fm-script .fm-script .fm-script{font-size:1em}td.fm-underover-base{line-height:1!important}td.fm-mtd{padding:.5ex .4em!important;vertical-align:baseline!important}fmath mphantom{visibility:hidden}fmath menclose[notation=top],menclose.fm-menclose[notation=top]{border-top:solid thin}fmath menclose[notation=right],menclose.fm-menclose[notation=right]{border-right:solid thin}fmath menclose[notation=bottom],menclose.fm-menclose[notation=bottom]{border-bottom:solid thin}fmath menclose[notation=left],menclose.fm-menclose[notation=left]{border-left:solid thin}fmath menclose[notation=box],menclose.fm-menclose[notation=box]{border:thin solid}fmath none{display:none}</style> In India electricity is supplied for domestic use at <fmath class="fm-inline"><mrow><mrow><mn>220</mn><mi class="fm-mi-length-1 ma-upright" mathvariant="normal" style="padding-right: 0px;"> </mi></mrow><mi class="fm-mi-length-1" mathvariant="italic" style="padding-right: 0.44ex;">V</mi></mrow></fmath>. It is supplied at <fmath class="fm-inline"><mrow><mrow><mn>110</mn><mi class="fm-mi-length-1 ma-upright" mathvariant="normal" style="padding-right: 0px;"> </mi></mrow><mi class="fm-mi-length-1" mathvariant="italic" style="padding-right: 0.44ex;">V</mi></mrow></fmath> in USA. If the resistance of a <fmath class="fm-inline"><mrow><mrow><mn>60</mn><mi class="fm-mi-length-1 ma-upright" mathvariant="normal" style="padding-right: 0px;"> </mi></mrow><mi class="fm-mi-length-1" mathvariant="italic" style="padding-right: 0.44ex;">W</mi></mrow></fmath> bulb for use in India is <fmath class="fm-inline"><mi class="fm-mi-length-1" mathvariant="italic">R</mi></fmath>, the resistance of a <fmath class="fm-inline"><mrow><mrow><mn>60</mn><mi class="fm-mi-length-1 ma-upright" mathvariant="normal" style="padding-right: 0px;"> </mi></mrow><mi class="fm-mi-length-1" mathvariant="italic" style="padding-right: 0.44ex;">W</mi></mrow></fmath> bulb for use in USA will be }
\begin{enumerate}[label=(\alph*)]
\item  <fmath class="fm-inline"><mi class="fm-mi-length-1" mathvariant="italic">R</mi></fmath>
\item  <fmath class="fm-inline"><mrow><mn>2</mn><mi class="fm-mi-length-1" mathvariant="italic">R</mi></mrow></fmath>
\item  <fmath class="fm-inline">\begin{tabular}{|c|c|}
\hline
<mi class="fm-mi-length-1" mathvariant="italic">R</mi> \\
\hline
<mn>4</mn> \\
\hline
\end{tabular}
</fmath>
\item  <fmath class="fm-inline">\begin{tabular}{|c|c|}
\hline
<mi class="fm-mi-length-1" mathvariant="italic">R</mi> \\
\hline
<mn>2</mn> \\
\hline
\end{tabular}
</fmath>
\end{enumerate}
\newpage
\section*{Question 16}
<style>.fm-math,fmath{font-family:STIXGeneral,'DejaVu Serif','DejaVu Sans',Times,OpenSymbol,'Standard Symbols L',serif;line-height:1.2}.fm-math mtext,fmath mtext{line-height:normal}.fm-mo,.ma-sans-serif,fmath mi[mathvariant*=sans-serif],fmath mn[mathvariant*=sans-serif],fmath mo,fmath ms[mathvariant*=sans-serif],fmath mtext[mathvariant*=sans-serif]{font-family:STIXGeneral,'DejaVu Sans','DejaVu Serif','Arial Unicode MS','Lucida Grande',Times,OpenSymbol,'Standard Symbols L',sans-serif}.fm-mo-Luc{font-family:STIXGeneral,'DejaVu Sans','DejaVu Serif','Lucida Grande','Arial Unicode MS',Times,OpenSymbol,'Standard Symbols L',sans-serif}.questionsfont{font-weight:200;font-family:Arial, sans-serif, STIXGeneral,'DejaVu Sans','DejaVu Serif','Lucida Grande','Arial Unicode MS',Times,OpenSymbol,'Standard Symbols L',sans-serif!important}.fm-separator{padding:0 .56ex 0 0}.fm-infix-loose{padding:0 .56ex}.fm-infix{padding:0 .44ex}.fm-prefix{padding:0 .33ex 0 0}.fm-postfix{padding:0 0 0 .33ex}.fm-prefix-tight{padding:0 .11ex 0 0}.fm-postfix-tight{padding:0 0 0 .11ex}.fm-quantifier{padding:0 .11ex 0 .22ex}.ma-non-marking{display:none}.fm-vert,fmath menclose,menclose.fm-menclose{display:inline-block}.fm-large-op{font-size:1.3em}.fm-inline .fm-large-op{font-size:1em}fmath mrow{white-space:nowrap}.fm-vert{vertical-align:middle}fmath table,fmath tbody,fmath td,fmath tr{border:0!important;padding:0!important;margin:0!important;outline:0!important}fmath table{border-collapse:collapse!important;text-align:center!important;table-layout:auto!important;float:none!important}.fm-frac{padding:0 1px!important}td.fm-den-frac{border-top:solid thin!important}.fm-root{font-size:.6em}.fm-radicand{padding:0 1px 0 0;border-top:solid;margin-top:.1em}.fm-script{font-size:.71em}.fm-script .fm-script .fm-script{font-size:1em}td.fm-underover-base{line-height:1!important}td.fm-mtd{padding:.5ex .4em!important;vertical-align:baseline!important}fmath mphantom{visibility:hidden}fmath menclose[notation=top],menclose.fm-menclose[notation=top]{border-top:solid thin}fmath menclose[notation=right],menclose.fm-menclose[notation=right]{border-right:solid thin}fmath menclose[notation=bottom],menclose.fm-menclose[notation=bottom]{border-bottom:solid thin}fmath menclose[notation=left],menclose.fm-menclose[notation=left]{border-left:solid thin}fmath menclose[notation=box],menclose.fm-menclose[notation=box]{border:thin solid}fmath none{display:none}</style> In a meter bridge, the balancing length from the left end (standard resistance of one ohm is in the right gap) is found to be <fmath class="fm-inline"><mrow><mrow><mn>20</mn><mi class="fm-mi-length-1 ma-upright" mathvariant="normal" style="padding-right: 0px;"> </mi></mrow><mrow><mi class="fm-mi-length-1" mathvariant="italic">c</mi><mi class="fm-mi-length-1" mathvariant="italic">m</mi></mrow></mrow></fmath>. The value of the unknown resistance is }
\begin{enumerate}[label=(\alph*)]
\item  <fmath class="fm-inline"><mrow><mn>0.8</mn><mi class="fm-mi-length-1 ma-upright" mathvariant="normal" style="padding-right: 0px;">Ω</mi></mrow></fmath>
\item  <fmath class="fm-inline"><mrow><mn>0.5</mn><mi class="fm-mi-length-1 ma-upright" mathvariant="normal" style="padding-right: 0px;">Ω</mi></mrow></fmath>
\item  <fmath class="fm-inline"><mrow><mn>0.4</mn><mi class="fm-mi-length-1 ma-upright" mathvariant="normal" style="padding-right: 0px;">Ω</mi></mrow></fmath> 
\item  <fmath class="fm-inline"><mrow><mn>0.25</mn><mi class="fm-mi-length-1 ma-upright" mathvariant="normal" style="padding-right: 0px;">Ω</mi></mrow></fmath> 
\end{enumerate}
\newpage
\section*{Question 17}
Fuse wire is a wire of }
\begin{enumerate}[label=(\alph*)]
\item  high resistance and high melting point
\item  high resistance and low melting point
\item  low resistance and low melting point
\item  low resistance and high melting point
\end{enumerate}
\newpage
\section*{Question 18}
The resistance of a wire is \(R\). It is bent at the middle by \(180^{\circ}\) and both the ends are twisted together to make a shorter wire. The resistance of the new wire is
\begin{enumerate}[label=(\alph*)]
\item \(2 R\)
\item \(\frac{R }{ 2}\)
\item \(\frac{R}{4}\)
\item \(\frac{R }{ 8}\)
\end{enumerate}
\newpage
\section*{Question 19}
<style>.fm-math,fmath{font-family:STIXGeneral,'DejaVu Serif','DejaVu Sans',Times,OpenSymbol,'Standard Symbols L',serif;line-height:1.2}.fm-math mtext,fmath mtext{line-height:normal}.fm-mo,.ma-sans-serif,fmath mi[mathvariant*=sans-serif],fmath mn[mathvariant*=sans-serif],fmath mo,fmath ms[mathvariant*=sans-serif],fmath mtext[mathvariant*=sans-serif]{font-family:STIXGeneral,'DejaVu Sans','DejaVu Serif','Arial Unicode MS','Lucida Grande',Times,OpenSymbol,'Standard Symbols L',sans-serif}.fm-mo-Luc{font-family:STIXGeneral,'DejaVu Sans','DejaVu Serif','Lucida Grande','Arial Unicode MS',Times,OpenSymbol,'Standard Symbols L',sans-serif}.questionsfont{font-weight:200;font-family:Arial, sans-serif, STIXGeneral,'DejaVu Sans','DejaVu Serif','Lucida Grande','Arial Unicode MS',Times,OpenSymbol,'Standard Symbols L',sans-serif!important}.fm-separator{padding:0 .56ex 0 0}.fm-infix-loose{padding:0 .56ex}.fm-infix{padding:0 .44ex}.fm-prefix{padding:0 .33ex 0 0}.fm-postfix{padding:0 0 0 .33ex}.fm-prefix-tight{padding:0 .11ex 0 0}.fm-postfix-tight{padding:0 0 0 .11ex}.fm-quantifier{padding:0 .11ex 0 .22ex}.ma-non-marking{display:none}.fm-vert,fmath menclose,menclose.fm-menclose{display:inline-block}.fm-large-op{font-size:1.3em}.fm-inline .fm-large-op{font-size:1em}fmath mrow{white-space:nowrap}.fm-vert{vertical-align:middle}fmath table,fmath tbody,fmath td,fmath tr{border:0!important;padding:0!important;margin:0!important;outline:0!important}fmath table{border-collapse:collapse!important;text-align:center!important;table-layout:auto!important;float:none!important}.fm-frac{padding:0 1px!important}td.fm-den-frac{border-top:solid thin!important}.fm-root{font-size:.6em}.fm-radicand{padding:0 1px 0 0;border-top:solid;margin-top:.1em}.fm-script{font-size:.71em}.fm-script .fm-script .fm-script{font-size:1em}td.fm-underover-base{line-height:1!important}td.fm-mtd{padding:.5ex .4em!important;vertical-align:baseline!important}fmath mphantom{visibility:hidden}fmath menclose[notation=top],menclose.fm-menclose[notation=top]{border-top:solid thin}fmath menclose[notation=right],menclose.fm-menclose[notation=right]{border-right:solid thin}fmath menclose[notation=bottom],menclose.fm-menclose[notation=bottom]{border-bottom:solid thin}fmath menclose[notation=left],menclose.fm-menclose[notation=left]{border-left:solid thin}fmath menclose[notation=box],menclose.fm-menclose[notation=box]{border:thin solid}fmath none{display:none}</style> The number of turns of the coil of a moving coil galvanometer is increased in order to increase current sensitivity by <fmath class="fm-inline"><mrow><mn>50</mn><mo>%</mo></mrow></fmath>. The percentage change in voltage sensitivity of the galvanometer will be:\newline 
\begin{enumerate}[label=(\alph*)]
\item  <fmath class="fm-inline"><mrow><mn>100</mn><mo>%</mo></mrow></fmath>
\item  <fmath class="fm-inline"><mrow><mn>50</mn><mo>%</mo></mrow></fmath>
\item  <fmath class="fm-inline"><mrow><mn>75</mn><mo>%</mo></mrow></fmath>
\item  <fmath class="fm-inline"><mrow><mn>0</mn><mo>%</mo></mrow></fmath>
\end{enumerate}
\newpage
\section*{Question 20}
(A) The drift velocity of electrons decreases with the increase in the temperature of conductor. \newline (B) The drift velocity is inversely proportional to the area of cross-section of given conductor. \newline (C) The drift velocity does not depend on the applied potential difference to the conductor. \newline (D) The drift velocity of electron is inversely proportional to the length of the conductor. \newline (E) The drift velocity increases with the increase in the temperature of conductor. \newline Choose the correct answer from the options given below : \newline 
\begin{enumerate}[label=(\alph*)]
\item  (A) and (B) only
\item  (A) and (D) only
\item  (B) and (E) only
\item  (B) and (C) only
\end{enumerate}
\newpage
\section*{Question 21}
<style>.fm-math,fmath{font-family:STIXGeneral,'DejaVu Serif','DejaVu Sans',Times,OpenSymbol,'Standard Symbols L',serif;line-height:1.2}.fm-math mtext,fmath mtext{line-height:normal}.fm-mo,.ma-sans-serif,fmath mi[mathvariant*=sans-serif],fmath mn[mathvariant*=sans-serif],fmath mo,fmath ms[mathvariant*=sans-serif],fmath mtext[mathvariant*=sans-serif]{font-family:STIXGeneral,'DejaVu Sans','DejaVu Serif','Arial Unicode MS','Lucida Grande',Times,OpenSymbol,'Standard Symbols L',sans-serif}.fm-mo-Luc{font-family:STIXGeneral,'DejaVu Sans','DejaVu Serif','Lucida Grande','Arial Unicode MS',Times,OpenSymbol,'Standard Symbols L',sans-serif}.questionsfont{font-weight:200;font-family:Arial, sans-serif, STIXGeneral,'DejaVu Sans','DejaVu Serif','Lucida Grande','Arial Unicode MS',Times,OpenSymbol,'Standard Symbols L',sans-serif!important}.fm-separator{padding:0 .56ex 0 0}.fm-infix-loose{padding:0 .56ex}.fm-infix{padding:0 .44ex}.fm-prefix{padding:0 .33ex 0 0}.fm-postfix{padding:0 0 0 .33ex}.fm-prefix-tight{padding:0 .11ex 0 0}.fm-postfix-tight{padding:0 0 0 .11ex}.fm-quantifier{padding:0 .11ex 0 .22ex}.ma-non-marking{display:none}.fm-vert,fmath menclose,menclose.fm-menclose{display:inline-block}.fm-large-op{font-size:1.3em}.fm-inline .fm-large-op{font-size:1em}fmath mrow{white-space:nowrap}.fm-vert{vertical-align:middle}fmath table,fmath tbody,fmath td,fmath tr{border:0!important;padding:0!important;margin:0!important;outline:0!important}fmath table{border-collapse:collapse!important;text-align:center!important;table-layout:auto!important;float:none!important}.fm-frac{padding:0 1px!important}td.fm-den-frac{border-top:solid thin!important}.fm-root{font-size:.6em}.fm-radicand{padding:0 1px 0 0;border-top:solid;margin-top:.1em}.fm-script{font-size:.71em}.fm-script .fm-script .fm-script{font-size:1em}td.fm-underover-base{line-height:1!important}td.fm-mtd{padding:.5ex .4em!important;vertical-align:baseline!important}fmath mphantom{visibility:hidden}fmath menclose[notation=top],menclose.fm-menclose[notation=top]{border-top:solid thin}fmath menclose[notation=right],menclose.fm-menclose[notation=right]{border-right:solid thin}fmath menclose[notation=bottom],menclose.fm-menclose[notation=bottom]{border-bottom:solid thin}fmath menclose[notation=left],menclose.fm-menclose[notation=left]{border-left:solid thin}fmath menclose[notation=box],menclose.fm-menclose[notation=box]{border:thin solid}fmath none{display:none}</style> Given below are two statements :\newlineStatement I : A uniform wire of resistance <fmath class="fm-inline"><mrow><mn>80</mn><mi class="fm-mi-length-1" mathvariant="italic">Ω</mi></mrow></fmath> is cut into four equal parts. These parts are now connected in parallel. The equivalent resistance of the combination will be <fmath class="fm-inline"><mrow><mn>5</mn><mi class="fm-mi-length-1" mathvariant="italic">Ω</mi></mrow></fmath>.\newlineStatement II: Two resistances <fmath class="fm-inline"><mrow><mn>2</mn><mi class="fm-mi-length-1 ma-upright" mathvariant="normal" style="padding-right: 0px;">R</mi></mrow></fmath> and <fmath class="fm-inline"><mrow><mn>3</mn><mi class="fm-mi-length-1 ma-upright" mathvariant="normal" style="padding-right: 0px;">R</mi></mrow></fmath> are connected in parallel in a electric circuit. The value of thermal energy developed in <fmath class="fm-inline"><mrow><mn>3</mn><mi class="fm-mi-length-1" mathvariant="italic">R</mi></mrow></fmath> and <fmath class="fm-inline"><mrow><mn>2</mn><mi class="fm-mi-length-1" mathvariant="italic">R</mi></mrow></fmath> will be in the ratio <fmath class="fm-inline"><mrow><mn>3</mn><mo class="fm-infix-loose">:</mo><mn>2</mn></mrow></fmath>.\newlineIn the light of the above statements, choose the most appropriate answer from the option given below 
\begin{enumerate}[label=(\alph*)]
\item  Both statement I and statement II are correct 
\item  Both statement I and statement II are incorrect 
\item  Statement I is correct but statement II is incorrect 
\item  Statement I is incorrect but statement II is correct
\end{enumerate}
\newpage
\section*{Question 22}
<style>.fm-math,fmath{font-family:STIXGeneral,'DejaVu Serif','DejaVu Sans',Times,OpenSymbol,'Standard Symbols L',serif;line-height:1.2}.fm-math mtext,fmath mtext{line-height:normal}.fm-mo,.ma-sans-serif,fmath mi[mathvariant*=sans-serif],fmath mn[mathvariant*=sans-serif],fmath mo,fmath ms[mathvariant*=sans-serif],fmath mtext[mathvariant*=sans-serif]{font-family:STIXGeneral,'DejaVu Sans','DejaVu Serif','Arial Unicode MS','Lucida Grande',Times,OpenSymbol,'Standard Symbols L',sans-serif}.fm-mo-Luc{font-family:STIXGeneral,'DejaVu Sans','DejaVu Serif','Lucida Grande','Arial Unicode MS',Times,OpenSymbol,'Standard Symbols L',sans-serif}.questionsfont{font-weight:200;font-family:Arial, sans-serif, STIXGeneral,'DejaVu Sans','DejaVu Serif','Lucida Grande','Arial Unicode MS',Times,OpenSymbol,'Standard Symbols L',sans-serif!important}.fm-separator{padding:0 .56ex 0 0}.fm-infix-loose{padding:0 .56ex}.fm-infix{padding:0 .44ex}.fm-prefix{padding:0 .33ex 0 0}.fm-postfix{padding:0 0 0 .33ex}.fm-prefix-tight{padding:0 .11ex 0 0}.fm-postfix-tight{padding:0 0 0 .11ex}.fm-quantifier{padding:0 .11ex 0 .22ex}.ma-non-marking{display:none}.fm-vert,fmath menclose,menclose.fm-menclose{display:inline-block}.fm-large-op{font-size:1.3em}.fm-inline .fm-large-op{font-size:1em}fmath mrow{white-space:nowrap}.fm-vert{vertical-align:middle}fmath table,fmath tbody,fmath td,fmath tr{border:0!important;padding:0!important;margin:0!important;outline:0!important}fmath table{border-collapse:collapse!important;text-align:center!important;table-layout:auto!important;float:none!important}.fm-frac{padding:0 1px!important}td.fm-den-frac{border-top:solid thin!important}.fm-root{font-size:.6em}.fm-radicand{padding:0 1px 0 0;border-top:solid;margin-top:.1em}.fm-script{font-size:.71em}.fm-script .fm-script .fm-script{font-size:1em}td.fm-underover-base{line-height:1!important}td.fm-mtd{padding:.5ex .4em!important;vertical-align:baseline!important}fmath mphantom{visibility:hidden}fmath menclose[notation=top],menclose.fm-menclose[notation=top]{border-top:solid thin}fmath menclose[notation=right],menclose.fm-menclose[notation=right]{border-right:solid thin}fmath menclose[notation=bottom],menclose.fm-menclose[notation=bottom]{border-bottom:solid thin}fmath menclose[notation=left],menclose.fm-menclose[notation=left]{border-left:solid thin}fmath menclose[notation=box],menclose.fm-menclose[notation=box]{border:thin solid}fmath none{display:none}</style> The resistance of an electrical toaster has a temperature dependence given by <fmath class="fm-inline"><mrow><mrow><mi class="fm-mi-length-1" mathvariant="italic">R</mi><mrow><mo class="fm-mo-Luc">(</mo><mi class="fm-mi-length-1" mathvariant="italic" style="padding-right: 0.44ex;">T</mi><mo class="fm-mo-Luc">)</mo></mrow></mrow><mo class="fm-infix-loose">=</mo><mrow><msub><mi class="fm-mi-length-1" mathvariant="italic">R</mi><mn>0</mn></msub><mrow><mo class="fm-mo-Luc">[</mo><mrow><mn>1</mn><mo class="fm-infix">+</mo><mrow><mi class="fm-mi-length-1 ma-upright" mathvariant="normal" style="padding-right: 0px;">α</mi><mrow><mo class="fm-mo-Luc">(</mo><mrow><mi class="fm-mi-length-1" mathvariant="italic" style="padding-right: 0.44ex;">T</mi><mo class="fm-infix">−</mo><msub><mi class="fm-mi-length-1" mathvariant="italic" style="padding-right: 0.44ex;">T</mi><mn>0</mn></msub></mrow><mo class="fm-mo-Luc">)</mo></mrow></mrow></mrow><mo class="fm-mo-Luc">]</mo></mrow></mrow></mrow></fmath> in its range of operation. At <fmath class="fm-inline"><mrow><mrow><msub><mi class="fm-mi-length-1" mathvariant="italic" style="padding-right: 0.44ex;">T</mi><mn>0</mn></msub><mo class="fm-infix-loose">=</mo><mrow><mn>300</mn><mi class="fm-mi-length-1" mathvariant="italic" style="padding-right: 0.44ex;">K</mi></mrow></mrow><mo class="fm-separator">,</mo><mrow><mi class="fm-mi-length-1" mathvariant="italic">R</mi><mo class="fm-infix-loose">=</mo><mrow><mn>100</mn><mi class="fm-mi-length-1 ma-upright" mathvariant="normal" style="padding-right: 0px;">Ω</mi></mrow></mrow></mrow></fmath> and at <fmath class="fm-inline"><mrow><mrow><mi class="fm-mi-length-1" mathvariant="italic" style="padding-right: 0.44ex;">T</mi><mo class="fm-infix-loose">=</mo><mrow><mn>500</mn><mi class="fm-mi-length-1" mathvariant="italic" style="padding-right: 0.44ex;">K</mi></mrow></mrow><mo class="fm-separator">,</mo><mrow><mi class="fm-mi-length-1" mathvariant="italic">R</mi><mo class="fm-infix-loose">=</mo></mrow></mrow></fmath><fmath class="fm-inline"><mrow><mn>120</mn><mi class="fm-mi-length-1 ma-upright" mathvariant="normal" style="padding-right: 0px;">Ω</mi></mrow></fmath>. The toaster is connected to a voltage source at 200 <fmath class="fm-inline"><mi class="fm-mi-length-1" mathvariant="italic" style="padding-right: 0.44ex;">V</mi></fmath> and its temperature is raised at a constant rate from 300 to <fmath class="fm-inline"><mrow><mn>500</mn><mi class="fm-mi-length-1" mathvariant="italic" style="padding-right: 0.44ex;">K</mi></mrow></fmath> in <fmath class="fm-inline"><mrow><mn>30</mn><mi class="fm-mi-length-1" mathvariant="italic">s</mi></mrow></fmath>. The total work done in raising the temperature is : } 
\begin{enumerate}[label=(\alph*)]
\item  <fmath class="fm-inline"><mrow class="ma-repel-adj"><mrow class="ma-repel-adj"><mrow class="ma-repel-adj"><mn>400</mn><mspace style="margin-right: 0.17em; padding-right: 0.001em; visibility: hidden;" width=".17em">‌</mspace><mi class="ma-repel-adj" mathvariant="normal">ln</mi></mrow><mspace style="margin-right: 0.17em; padding-right: 0.001em; visibility: hidden;" width=".17em">‌</mspace>\begin{tabular}{|c|c|}
\hline
<mn>5</mn> \\
\hline
<mn>6</mn> \\
\hline
\end{tabular}
</mrow><mspace style="margin-right: 0.17em; padding-right: 0.001em; visibility: hidden;" width=".17em">‌</mspace><mi class="fm-mi-length-1" mathvariant="italic" style="padding-right: 0.44ex;">J</mi></mrow></fmath>
\item  <fmath class="fm-inline"><mrow class="ma-repel-adj"><mrow class="ma-repel-adj"><mrow class="ma-repel-adj"><mn>200</mn><mspace style="margin-right: 0.17em; padding-right: 0.001em; visibility: hidden;" width=".17em">‌</mspace><mi class="ma-repel-adj" mathvariant="normal">ln</mi></mrow><mspace style="margin-right: 0.17em; padding-right: 0.001em; visibility: hidden;" width=".17em">‌</mspace>\begin{tabular}{|c|c|}
\hline
<mn>2</mn> \\
\hline
<mn>3</mn> \\
\hline
\end{tabular}
</mrow><mspace style="margin-right: 0.17em; padding-right: 0.001em; visibility: hidden;" width=".17em">‌</mspace><mi class="fm-mi-length-1" mathvariant="italic" style="padding-right: 0.44ex;">J</mi></mrow></fmath>
\item  <fmath class="fm-inline"><mrow><mn>300</mn><mi class="fm-mi-length-1" mathvariant="italic" style="padding-right: 0.44ex;">J</mi></mrow></fmath>
\item  <fmath class="fm-inline"><mrow class="ma-repel-adj"><mrow class="ma-repel-adj"><mrow class="ma-repel-adj"><mn>400</mn><mspace style="margin-right: 0.17em; padding-right: 0.001em; visibility: hidden;" width=".17em">‌</mspace><mi class="ma-repel-adj" mathvariant="normal">ln</mi></mrow><mspace style="margin-right: 0.17em; padding-right: 0.001em; visibility: hidden;" width=".17em">‌</mspace>\begin{tabular}{|c|c|}
\hline
<mn>1.5</mn> \\
\hline
<mn>1.3</mn> \\
\hline
\end{tabular}
</mrow><mspace style="margin-right: 0.17em; padding-right: 0.001em; visibility: hidden;" width=".17em">‌</mspace><mi class="fm-mi-length-1" mathvariant="italic" style="padding-right: 0.44ex;">J</mi></mrow></fmath>
\item None
\end{enumerate}
\newpage
\section*{Question 23}
<style>.fm-math,fmath{font-family:STIXGeneral,'DejaVu Serif','DejaVu Sans',Times,OpenSymbol,'Standard Symbols L',serif;line-height:1.2}.fm-math mtext,fmath mtext{line-height:normal}.fm-mo,.ma-sans-serif,fmath mi[mathvariant*=sans-serif],fmath mn[mathvariant*=sans-serif],fmath mo,fmath ms[mathvariant*=sans-serif],fmath mtext[mathvariant*=sans-serif]{font-family:STIXGeneral,'DejaVu Sans','DejaVu Serif','Arial Unicode MS','Lucida Grande',Times,OpenSymbol,'Standard Symbols L',sans-serif}.fm-mo-Luc{font-family:STIXGeneral,'DejaVu Sans','DejaVu Serif','Lucida Grande','Arial Unicode MS',Times,OpenSymbol,'Standard Symbols L',sans-serif}.questionsfont{font-weight:200;font-family:Arial, sans-serif, STIXGeneral,'DejaVu Sans','DejaVu Serif','Lucida Grande','Arial Unicode MS',Times,OpenSymbol,'Standard Symbols L',sans-serif!important}.fm-separator{padding:0 .56ex 0 0}.fm-infix-loose{padding:0 .56ex}.fm-infix{padding:0 .44ex}.fm-prefix{padding:0 .33ex 0 0}.fm-postfix{padding:0 0 0 .33ex}.fm-prefix-tight{padding:0 .11ex 0 0}.fm-postfix-tight{padding:0 0 0 .11ex}.fm-quantifier{padding:0 .11ex 0 .22ex}.ma-non-marking{display:none}.fm-vert,fmath menclose,menclose.fm-menclose{display:inline-block}.fm-large-op{font-size:1.3em}.fm-inline .fm-large-op{font-size:1em}fmath mrow{white-space:nowrap}.fm-vert{vertical-align:middle}fmath table,fmath tbody,fmath td,fmath tr{border:0!important;padding:0!important;margin:0!important;outline:0!important}fmath table{border-collapse:collapse!important;text-align:center!important;table-layout:auto!important;float:none!important}.fm-frac{padding:0 1px!important}td.fm-den-frac{border-top:solid thin!important}.fm-root{font-size:.6em}.fm-radicand{padding:0 1px 0 0;border-top:solid;margin-top:.1em}.fm-script{font-size:.71em}.fm-script .fm-script .fm-script{font-size:1em}td.fm-underover-base{line-height:1!important}td.fm-mtd{padding:.5ex .4em!important;vertical-align:baseline!important}fmath mphantom{visibility:hidden}fmath menclose[notation=top],menclose.fm-menclose[notation=top]{border-top:solid thin}fmath menclose[notation=right],menclose.fm-menclose[notation=right]{border-right:solid thin}fmath menclose[notation=bottom],menclose.fm-menclose[notation=bottom]{border-bottom:solid thin}fmath menclose[notation=left],menclose.fm-menclose[notation=left]{border-left:solid thin}fmath menclose[notation=box],menclose.fm-menclose[notation=box]{border:thin solid}fmath none{display:none}</style>  \includegraphics[width=\textwidth]{static/media/wl_client/1/qdump/dd962b43da3e663bef2c213d7dbe3f88/0018a1c3d845810b48597111ee1bd263.png}In the circuit shown, the resistance ris a variable resistance. If for <fmath class="fm-inline"><mrow><mrow><mi class="fm-mi-length-1" mathvariant="italic">r</mi><mo class="fm-infix-loose">=</mo><mrow><mi class="fm-mi-length-1" mathvariant="italic" style="padding-right: 0.44ex;">f</mi><mi class="fm-mi-length-1" mathvariant="italic">R</mi></mrow></mrow><mo class="fm-postfix-tight">,</mo></mrow></fmath> the heat generation in <fmath class="fm-inline"><mi class="fm-mi-length-1" mathvariant="italic">r</mi></fmath> is maximum then the value of f is: } 
\begin{enumerate}[label=(\alph*)]
\item  <fmath class="fm-inline">\begin{tabular}{|c|c|}
\hline
<mn>1</mn> \\
\hline
<mn>2</mn> \\
\hline
\end{tabular}
</fmath>
\item  1
\item  <fmath class="fm-inline">\begin{tabular}{|c|c|}
\hline
<mn>1</mn> \\
\hline
<mn>4</mn> \\
\hline
\end{tabular}
</fmath>
\item  <fmath class="fm-inline">\begin{tabular}{|c|c|}
\hline
<mn>3</mn> \\
\hline
<mn>4</mn> \\
\hline
\end{tabular}
</fmath>
\end{enumerate}
\newpage
\section*{Question 24}
An electric bulb is rated 220 volt - 100 watt. The power consumed by it when operated on 110 volt will be }
\begin{enumerate}[label=(\alph*)]
\item  75 watt 
\item  40 watt 
\item  25 watt 
\item  50 watt
\end{enumerate}
\newpage
\section*{Question 25}
The current carrying rectangular loop is placed in the uniform magnetic field, the torque on the loop will be maximum when the angle between the area vector and the magnetic field is:
\begin{enumerate}[label=(\alph*)]
\item \(0^{\circ}\)
\item \(90^{\circ}\)
\item \(45^{\circ}\)
\item \(60^{\circ}\)
\end{enumerate}
\newpage
\section*{Question 26}
A rectangular coil of length \(40\) cm and width \(10\) cm consists of \(10\) turns and carries a current of \(16\) A. The coil is suspended such that the normal to the plane of the coil makes an angle of \(60^{\circ}\) with the direction of a uniform magnetic field of magnitude \(0.60\) T. Find the magnitude of the torque experienced by the coil.
\begin{enumerate}[label=(\alph*)]
\item \(1.92\) N-m
\item \(1.92 \sqrt{3}\) N-m
\item \(1.62 \sqrt{3}\) N-m
\item \(0.64 \sqrt{3}\) N-m
\end{enumerate}
\newpage
\section*{Question 27}
A circular coil A of radius ‘a’ carries current ‘I’. Another circular coil B of radius ‘2a’ also carries the same current of ‘I’. The magnetic fields at the centers of the circular coils are in the ratio of:
\begin{enumerate}[label=(\alph*)]
\item 2 ∶ 1
\item 4 ∶ 1
\item 3 ∶ 1
\item 1 ∶ 1
\end{enumerate}
\newpage
\section*{Question 28}
The force per unit length is 10-3 N on the two current-carrying wires of equal length that are separated by a distance of 2 m and placed parallel to each other. If the current in both the wires is doubled and the distance between the wires is halved, then what will be the force per unit length on the wire?
\begin{enumerate}[label=(\alph*)]
\item 2×10$^{-3}$ N
\item 4×10$^{-3}$ N
\item 8×10$^{-3}$ N
\item 16×10$^{-3}$ N
\end{enumerate}
\newpage
\section*{Question 29}
An electron is moving in a circular orbit in a magnetic field of \(2 \times 10^{-4}\) weber/m\(^{2}\). Its time period of revolution is:
\begin{enumerate}[label=(\alph*)]
\item \(1.79 \times 10^{-7}\) sec
\item \(3.5 \times 10^{-7}\) sec
\item \(7 \times 10^{-7}\) sec
\item \(2.75 \times 10^{-7}\) sec
\end{enumerate}
\newpage
\section*{Question 30}
A proton is moving in a uniform magnetic field B in a circular path of radius 'a' in a direction perpendicular to z-axis along which the field B exists. Calculate the angular momentum if charge on the proton is 'e'.
\begin{enumerate}[label=(\alph*)]
\item \(\frac{ Be }{ a ^2}\)
\item \(e^2 a\)
\item \(a^2 e B\)
\item \(aeB ^2\)
\end{enumerate}
\newpage
\section*{Question 31}
In hydrogen atom the electron is making \(6.6 \times 10^{15} rev / s\) around the nucleus of radius 0.50 \(\mathring{\mathrm{A}}\). The magnetic field produced at the centre of the orbit is nearly :
\begin{enumerate}[label=(\alph*)]
\item \(0.12 W b / m ^2\)
\item 1. \(2 Wb / m ^2\)
\item \(12 Wb / m ^2\)
\item \(120 Wb / m ^2\)
\end{enumerate}
\newpage
\section*{Question 32}
Electron of mass m and charge q is travelling with a speed v along a circular path of radius r at right angles to a uniform magnetic field of intensity B. If the speed of the electron is doubled and the magnetic field is halved the resulting path would have a radius.
\begin{enumerate}[label=(\alph*)]
\item 2r
\item 4r
\item \(\frac{r}{4}\)
\item \(\frac{r}{2}\)
\end{enumerate}
\newpage
\section*{Question 33}
On what factors does the force experienced by a current carrying conductor placed in a uniform magnetic field depend?
\begin{enumerate}[label=(\alph*)]
\item Magnetic field
\item Current
\item Length of the conductor
\item All of the above
\end{enumerate}
\newpage
\section*{Question 34}
A circular loop has radius r = 1 cm, is placed on the x-z plane with its center at the origin, and is carrying current I = 2 A in an anti-clockwise direction when seen from a point on the +y-axis. The magnetic field due to the current loop at point P(0,-1) is:
\begin{enumerate}[label=(\alph*)]
\item 1.25 × 10$^{-10}$ T
\item 2.5 × 10$^{-10 }$T
\item 5 × 10$^{-10}$ T
\item 7.5 × 10$^{-10 T}$
\end{enumerate}
\newpage
\section*{Question 35}
A circular loop has radius 'r', is placed perpendicular on the x-z plane with its center at the origin, and is carrying current 'I' in an anti-clockwise direction when seen from a point on the +x-axis. The magnetic field due to the current loop at point P(d,0) is (d >> r).
\begin{enumerate}[label=(\alph*)]
\item \(\frac{\mu_0}{4 \pi} \frac{I r^2}{d^3} \)
\item \(\frac{\mu_0}{2 \pi} \frac{I r^2}{d^3}\)
\item \(\frac{\mu_0}{4} \frac{I r^2}{d^3}\)
\item \(\frac{\mu_0}{2} \frac{I r^2}{d^3}\)
\end{enumerate}
\newpage
\section*{Question 36}
A circular loop has a radius r = 1 cm, is placed on the x-z plane with its center at the origin, and is carrying current I = 2 A in an anti-clockwise direction when seen from a point on the +y-axis. The magnitude of magnetic field due to the current loop at point P (0,1) and Q (d,0) is the same. The value of d is:
\begin{enumerate}[label=(\alph*)]
\item 0.25
\item (0.5$^{)1/3}$
\item 0.75 
\item 1
\end{enumerate}
\newpage
\section*{Question 37}
A power line lies along the East-West direction and carries a current of 10 A. The force per unit length due to the earth's magnetic field of 10$^{–4}$ T is :
\begin{enumerate}[label=(\alph*)]
\item 10$^{–5}$ Nm$^{–1}$
\item 10$^{–4}$ Nm$^{–1}$
\item 10$^{–3}$ Nm$^{–1}$
\item 10$^{–2}$ Nm$^{–1}$
\end{enumerate}
\newpage
\section*{Question 38}
A particle of mass \(m\) and charge \(q\) moves with a constant velocity \(v\) along the positive \(x\) - direction. It enters a region containing a uniform magnetic field \(B\) directed along the negative \(z\) - direction, extending from \(x = a\) to \(x = b\). The minimum value of \(v\) required so that the particle can just enter the region \(x > b\) is :
\begin{enumerate}[label=(\alph*)]
\item \(\frac{q b B}{m}\)
\item \(\frac{q(b-a) B}{m}\)
\item \(\frac{q a B}{m}\)
\item \(\frac{q(b+a) B}{2 m}\)
\end{enumerate}
\newpage
\section*{Question 39}
An proton is moving with velocity 10$^{4}$ m/s in a magnetic field of 5 tesla. The maximum force on electron is:
\begin{enumerate}[label=(\alph*)]
\item \(8 \times 10^{-15} N\)
\item \(10^4 N\)
\item \(1.6 \times 10^{-19} N\)
\item \(5 \times 10^4 N\)
\end{enumerate}
\newpage
\section*{Question 40}
The pole of a magnet cannot be separated by breaking the magnet into two pieces.Magnetic monopoles do not exist.
\begin{enumerate}[label=(\alph*)]
\item Statement A is correct
\item Statement B is correct
\item Both A and B are correct
\item Both A and B are incorrect
\end{enumerate}
\newpage
\section*{Question 41}
The presence of magnetic monopoles is ruled out by:
\begin{enumerate}[label=(\alph*)]
\item Gauss's law for electrostatics
\item Gauss's law for magnetism
\item Faraday's law
\item Ampere's circuital law with Maxwell's addition
\end{enumerate}
\newpage
\section*{Question 42}
The effective length of a magnet is \(31.4 \mathrm{~cm}\) and its pole strength is \(0.8 \mathrm{~Am}\). The magnetic moment, if it is bent in the form of a semicircle is _____________  \(\mathrm{Am}^2\).
\begin{enumerate}[label=(\alph*)]
\item \(1.2\)
\item \(1.6\)
\item \(0.16\)
\item \(0.12\)
\end{enumerate}
\newpage
\section*{Question 43}
When a bar magnet of magnetic moment \(\vec{M}\) is placed in a magnetic field \(\vec{B}\), then the torque exerted on the magnet will be:
\begin{enumerate}[label=(\alph*)]
\item \(\vec{M} \cdot \vec{B}\)
\item \(\vec{M} \cdot \vec{B}\)
\item \(\vec{M} \times \vec{B}\)
\item \(\vec{B} \times \vec{M}\)
\end{enumerate}
\newpage
\section*{Question 44}
A bar magnet having a magnetic moment of 2 × 10$^{4}$ JT$^{-1}$ is free to rotate in a horizontal plane. A horizontal magnetic field B = 6 × 10$^{-4}$ T exists in the space. The work done in taking the magnet slowly from a direction parallel to the field to a direction 60° from the field is:
\begin{enumerate}[label=(\alph*)]
\item 12 J
\item 6 J
\item 2 J
\item 0.6 J
\end{enumerate}
\newpage
\section*{Question 45}
A magnet of magnetic moment \(50 \hat{i} Am ^2\) is placed along the \(x\)-axis in a magnetic field \(\vec{B}=(0.5 \hat{i}+3.0 \hat{j}) T\). The torque acting on the magnet is:
\begin{enumerate}[label=(\alph*)]
\item \(175 \hat{k} N-m\)
\item \(150 \hat{k} N-m\)
\item \(75 \hat{k} N-m\)
\item \(25 \sqrt{37} \hat{k} N-m\)
\end{enumerate}
\newpage
\section*{Question 46}
A magnet is freely suspended in a constant magnetic field of strength B. How much work will be done if the magnet is deflected by an angle 90° from the initial direction?
\begin{enumerate}[label=(\alph*)]
\item MH (1 – sinθ)
\item MH (cosθ - 1)
\item MH
\item MH (1 – cosθ)
\end{enumerate}
\newpage
\section*{Question 47}
A magnetic needle suspended parallel to a magnetic field requires \(\sqrt{3} J\) of work to turn it through \(60^{\circ}\). The torque needed to maintain the needle in this position will be:
\begin{enumerate}[label=(\alph*)]
\item \(2 \sqrt{3} J\)
\item 3J
\item \(\sqrt{3} J\)
\item \(\frac{3 }{ 2 }J\)
\end{enumerate}
\newpage
\section*{Question 48}
What is the torque experienced by a circular loop with circumference L and 1 turn which is suspended in a magnetic field B and current i passes through it?
\begin{enumerate}[label=(\alph*)]
\item (\(\frac{1}{4}\)π) BiL
\item (\(\frac{1}{4}\)π) BiL$^{2}$
\item (\(\frac{1}{4}\)π) B$^{2}$iL
\item (\(\frac{1}{4}\)π) Bi$^{2}$L
\end{enumerate}
\newpage
\section*{Question 49}
A long, straight wire carries a current i. The magnetizing field intensity H is measured at a point P close to the wire. A long, cylindrical iron rod is brought close to the wire so that the point P is at the centre the rod. The value of H at P will:
\begin{enumerate}[label=(\alph*)]
\item Increase many times
\item Decrease many times
\item Remain almost constant
\item Become zero
\end{enumerate}
\newpage
\section*{Question 50}
Let r be the distance of a point on the axis of a bar magnet from its centre. The magnetic field at such a point is proportional to:
\begin{enumerate}[label=(\alph*)]
\item \(\frac{1}{r}\)
\item \(\frac{1}{r^{2}}\)
\item \(\frac{1}{r^{3}}\)
\item None of these
\end{enumerate}
\newpage
\end{document}