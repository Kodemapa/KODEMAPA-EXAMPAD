\documentclass{article}
                    \usepackage{amsmath}
                    \usepackage{amssymb}
                    \usepackage{graphicx}
                    \usepackage{enumitem}
                    \usepackage{longtable}
                    \usepackage{array}
                    \title{JEE 102}
                    \begin{document}
                    \maketitle
                    \section*{Question 1}
what is A in the following reaction ?\includegraphics[width=\textwidth]{https://kodemapa.com/static/media/wl_client/1/qdump/dd962b43da3e663bef2c213d7dbe3f88/236e4ffa2b286d5be56b69d34825b864.png}\newline
\begin{enumerate}[label=(\alph*)]
\item \includegraphics[width=\textwidth]{https://kodemapa.com/static/media/wl_client/1/qdump/dd962b43da3e663bef2c213d7dbe3f88/7e913335697edc435c318cd8f4c74c35.png}
\item \includegraphics[width=\textwidth]{https://kodemapa.com/static/media/wl_client/1/qdump/dd962b43da3e663bef2c213d7dbe3f88/af241ff17a69316d69e677cbe7bee714.png}
\item \includegraphics[width=\textwidth]{https://kodemapa.com/static/media/wl_client/1/qdump/dd962b43da3e663bef2c213d7dbe3f88/96390fc856177859ff1bc0820852c540.png}
\item \includegraphics[width=\textwidth]{https://kodemapa.com/static/media/wl_client/1/qdump/dd962b43da3e663bef2c213d7dbe3f88/502f4bfc33a7ea5f5446b31091aa8c80.png}
\end{enumerate}
\newpage
\section*{Question 2}
Which one of the products of the following reactions does not react with Hinsberg reagent to form sulphonamide? [25 Jul 2021]
\begin{enumerate}[label=(\alph*)]
\item \includegraphics[width=\textwidth]{https://kodemapa.com/static/media/wl_client/1/qdump/dd962b43da3e663bef2c213d7dbe3f88/2c25b5302176976e694d20326b655cbb.png}
\item \includegraphics[width=\textwidth]{https://kodemapa.com/static/media/wl_client/1/qdump/dd962b43da3e663bef2c213d7dbe3f88/77e8483bbf8a5f91aca9e2e40598b611.png}
\item \includegraphics[width=\textwidth]{https://kodemapa.com/static/media/wl_client/1/qdump/dd962b43da3e663bef2c213d7dbe3f88/335c787e1add6d2886cf53b8fe0870d7.png}
\item \includegraphics[width=\textwidth]{https://kodemapa.com/static/media/wl_client/1/qdump/dd962b43da3e663bef2c213d7dbe3f88/07ad49568681c4e55c2b1bfd60d5b384.png}
\end{enumerate}
\newpage
\section*{Question 3}
An organic compound "A" on treatment with benzene sulphonyl chloride gives compound \(B . B\) is soluble in dil. \(\mathrm{NaOH}\) solution. Compound \(A\) is______________. 
\begin{enumerate}[label=(\alph*)]
\item \(\mathrm{C}_6 \mathrm{H}_5-\mathrm{N}-\left(\mathrm{CH}_3\right)_2\)
\item \(\mathrm{C}_6 \mathrm{H}_5-\mathrm{NHCH}_2 \mathrm{CH}_3\)
\item \(\mathrm{C}_6 \mathrm{H}_5-\mathrm{CH}_2 \mathrm{NHCH}_3\)
\item \(\mathrm{C}_6 \mathrm{H}_5-\mathrm{CH}-\mathrm{NH}_2\)
\end{enumerate}
\newpage
\section*{Question 4}
The number of nitrogen atoms in a semicarbazone molecule of acetone is_____________.
\begin{enumerate}[label=(\alph*)]
\end{enumerate}
\newpage
\section*{Question 5}
The total number of reagents from those given below, that can convert nitrobenzene into aniline is _______________ (Integer answer)
    \setlength{\arrayrulewidth}{0.8mm}
    \begin{tabular}{|c|c|}
    \hline
    \(I . \mathrm{Sn}-\mathrm{HCI}\) & \(\mathrm{II} \cdot \mathrm{Sn}-\mathrm{NH}_4 \mathrm{OH}\) \\
\hline
\(I I I \cdot \mathrm{Fe}-\mathrm{HCl}\) & \(I V \cdot \mathrm{Zn}-\mathrm{HCI}\) \\
\hline
\(V \cdot \mathrm{H}_2-\mathrm{Pd}\) & \(V I \cdot \mathrm{H}_2-\) Raney nickel \\
\hline

    \end{tabular}
    \setlength{\arrayrulewidth}{0.4mm}
    
\begin{enumerate}[label=(\alph*)]
\end{enumerate}
\newpage
\section*{Question 6}
A primary aliphatic amine on reaction with nitrous acid in cold ( \(273 \mathrm{~K})\) and there after raising temperature of reaction mixture to room temperature (298 K), gives.
\begin{enumerate}[label=(\alph*)]
\item nitrile
\item alcohol
\item diazonium salt
\item secondary amine
\end{enumerate}
\newpage
\section*{Question 7}
Choose the correct colour of the product for the following reaction.\includegraphics[width=\textwidth]{https://kodemapa.com/static/media/wl_client/1/qdump/dd962b43da3e663bef2c213d7dbe3f88/40efa76b9a92d9d9f7f9a8bc33bd53c0.png}\newline
\begin{enumerate}[label=(\alph*)]
\item Yellow
\item White
\item Red
\item Blue
\end{enumerate}
\newpage
\section*{Question 8}
Number of isomeric aromatic amines with molecular formula \(\mathrm{C}_8 \mathrm{H}_{11} \mathrm{~N}\), which can be synthesized by Gabriel Phthalimide synthesis is ___________.  [6-Apr-2023]
\begin{enumerate}[label=(\alph*)]
\end{enumerate}
\newpage
\section*{Question 9}
Given below are two statements :Statement I : Aniline is less basic than acetamide.Statement II : In aniline, the lone pair of electrons on nitrogen atom is delocalised over benzene ring due to resonance and hence less available to a proton.Choose the most appropriate option ; 
\begin{enumerate}[label=(\alph*)]
\item Statement I is true but statement II is false.
\item Statement I is false but statement II is true.
\item Both statement I and statement II are true.
\item Both statement I and statement II are false.
\end{enumerate}
\newpage
\section*{Question 10}
The most appropriate reagent for conversion of \(\mathrm{C}_2 \mathrm{H}_5 \mathrm{CN}\) into \(\mathrm{CH}_3 \mathrm{CH}_2 \mathrm{CH}_2 \mathrm{NH}_2\) is:
\begin{enumerate}[label=(\alph*)]
\item \(\mathrm{NaBH}_4\)
\item \(\mathrm{CaH}_2\)
\item LiAl H\(_4\)
\item \(\mathrm{Na}(\mathrm{CN}) \mathrm{BH}_3\)
\end{enumerate}
\newpage
\section*{Question 11}
The number of primary amines of formula \(\mathrm{C}_4 \mathrm{H}_{11} \mathrm{~N}\) is ? 
\begin{enumerate}[label=(\alph*)]
\end{enumerate}
\newpage
\section*{Question 12}
what is A in the following reaction ?\includegraphics[width=\textwidth]{https://kodemapa.com/static/media/wl_client/1/qdump/dd962b43da3e663bef2c213d7dbe3f88/236e4ffa2b286d5be56b69d34825b864.png}\newline
\begin{enumerate}[label=(\alph*)]
\item \includegraphics[width=\textwidth]{https://kodemapa.com/static/media/wl_client/1/qdump/dd962b43da3e663bef2c213d7dbe3f88/7e913335697edc435c318cd8f4c74c35.png}
\item \includegraphics[width=\textwidth]{https://kodemapa.com/static/media/wl_client/1/qdump/dd962b43da3e663bef2c213d7dbe3f88/af241ff17a69316d69e677cbe7bee714.png}
\item \includegraphics[width=\textwidth]{https://kodemapa.com/static/media/wl_client/1/qdump/dd962b43da3e663bef2c213d7dbe3f88/96390fc856177859ff1bc0820852c540.png}
\item \includegraphics[width=\textwidth]{https://kodemapa.com/static/media/wl_client/1/qdump/dd962b43da3e663bef2c213d7dbe3f88/502f4bfc33a7ea5f5446b31091aa8c80.png}
\end{enumerate}
\newpage
\section*{Question 13}
Match List I with List II.
    \setlength{\arrayrulewidth}{0.8mm}
    \begin{tabular}{|c|c|}
    \hline
    List-I & List-II \\
\hline
A. Benzenesulphonyl Chloride & I. Test for primary amines \\
\hline
B. Hoffmann bromamide reaction & II. Anti Saytzeff \\
\hline
C. Carbylamine reaction & III. Hinsberg reagent \\
\hline
D. Hoffmann orientation & IV. Known reaction of Isocyanates \\
\hline

    \end{tabular}
    \setlength{\arrayrulewidth}{0.4mm}
    Choose the correct answer from the options given below: \newline
\begin{enumerate}[label=(\alph*)]
\item A-IV, B-III, C-II, D-I
\item A-IV, B-II, C-I, D-II
\item A-III, B-IV, C-I, D-II
\item A-IV, B-III, C-I, D-II
\end{enumerate}
\newpage
\section*{Question 14}
Compound \(\mathrm{A}\) is converted to \(\mathrm{B}\) on reaction with \(\mathrm{CHCl}_3\) and \(\mathrm{KOH}\). The compound \(\mathrm{B}\) is toxic and can be decomposed by C. A, B and \(C\) respectively are :
\begin{enumerate}[label=(\alph*)]
\item primary amine, nitrile compound, conc. \(\mathrm{HCl}\)
\item secondary amine, isonitrile compound, conc. \(\mathrm{NaOH}\)
\item primary amine, isonitrile compound, conc. \(\mathrm{HCl}\)
\item secondary amine, nitrile compound, conc. \(\mathrm{NaOH}\)
\end{enumerate}
\newpage
\section*{Question 15}
The major product formed in the following reaction is.\includegraphics[width=\textwidth]{https://kodemapa.com/static/media/wl_client/1/qdump/dd962b43da3e663bef2c213d7dbe3f88/cd501f5bf6a7ba9867212e69f5da1961.png}\newline
\begin{enumerate}[label=(\alph*)]
\item \includegraphics[width=\textwidth]{https://kodemapa.com/static/media/wl_client/1/qdump/dd962b43da3e663bef2c213d7dbe3f88/ea66b26d652be69c42f279d0ca208c29.png}
\item \includegraphics[width=\textwidth]{https://kodemapa.com/static/media/wl_client/1/qdump/dd962b43da3e663bef2c213d7dbe3f88/2952e31a186e1acc29cc186fdfb5dd6b.png}
\item \includegraphics[width=\textwidth]{https://kodemapa.com/static/media/wl_client/1/qdump/dd962b43da3e663bef2c213d7dbe3f88/d4fa77cbadb82c63c120985ab1e82614.png}
\item \includegraphics[width=\textwidth]{https://kodemapa.com/static/media/wl_client/1/qdump/dd962b43da3e663bef2c213d7dbe3f88/59084e148fc2d739417759e72917593f.png}
\end{enumerate}
\newpage
\section*{Question 16}
Choose the correct colour of the product for the following reaction.\includegraphics[width=\textwidth]{https://kodemapa.com/static/media/wl_client/1/qdump/dd962b43da3e663bef2c213d7dbe3f88/40efa76b9a92d9d9f7f9a8bc33bd53c0.png}\newline
\begin{enumerate}[label=(\alph*)]
\item Yellow
\item White
\item Red
\item Blue
\end{enumerate}
\newpage
\section*{Question 17}
Number of isomeric aromatic amines with molecular formula \(\mathrm{C}_8 \mathrm{H}_{11} \mathrm{~N}\), which can be synthesized by Gabriel Phthalimide synthesis is ___________.  [6-Apr-2023]
\begin{enumerate}[label=(\alph*)]
\end{enumerate}
\newpage
\section*{Question 18}
Given below are two statements :Statement I : In Hofmann degradation reaction, the migration of only an alkyl group takes place from carbonyl carbon of the amide to the nitrogen atom.Statement II : The group is migrated in Hofmann degradation reaction to electron deficient atom.In the light of the above statements, choose the most appropriate answer from the options given below: 
\begin{enumerate}[label=(\alph*)]
\item Both Statement I and Statement II are correct.
\item Both Statement I and Statement II are incorrect.
\item Statement I is correct but Statement II is incorrect.
\item Statement I is incorrect but Statement II is correct.
\end{enumerate}
\newpage
\section*{Question 19}
During halogen test, sodium fusion extract is boiled with concentrated \(\mathrm{HNO}_3\) to 
\begin{enumerate}[label=(\alph*)]
\item remove unreacted sodium
\item decompose cyanide or sulphide of sodium
\item extract halogen from organic compound
\item maintain the \(\mathrm{pH}\) of extract.
\end{enumerate}
\newpage
\section*{Question 20}
A compound with molecular mass 180 is acylated with \(\mathrm{CH}_3 \mathrm{COCl}\) to get a compound with molecular mass 390 . The number of amino groups present per molecule of the former compound is: 
\begin{enumerate}[label=(\alph*)]
\end{enumerate}
\newpage
\section*{Question 21}
The product A formed in the following reaction is:\includegraphics[width=\textwidth]{https://kodemapa.com/static/media/wl_client/1/qdump/dd962b43da3e663bef2c213d7dbe3f88/1520d487acea3e12471d7be5f928b59f.png}\newline\newline
\begin{enumerate}[label=(\alph*)]
\item \includegraphics[width=\textwidth]{https://kodemapa.com/static/media/wl_client/1/qdump/dd962b43da3e663bef2c213d7dbe3f88/2ca5365cc0784172da4f25597d9ad12a.png}
\item \includegraphics[width=\textwidth]{https://kodemapa.com/static/media/wl_client/1/qdump/dd962b43da3e663bef2c213d7dbe3f88/9fbdb1561f5c01c9175d265eb6391c64.png}
\item \includegraphics[width=\textwidth]{https://kodemapa.com/static/media/wl_client/1/qdump/dd962b43da3e663bef2c213d7dbe3f88/3dc05a6f09a27296e56e9f0ecfac5216.png}
\item \includegraphics[width=\textwidth]{https://kodemapa.com/static/media/wl_client/1/qdump/dd962b43da3e663bef2c213d7dbe3f88/567f0d4eba6072aab1e7657d9bd4afdd.png}
\end{enumerate}
\newpage
\section*{Question 22}
What is the correct name for a molecule that has two amino groups in opposing (para) locations around a benzene ring?
\begin{enumerate}[label=(\alph*)]
\item  Benzenediamine
\item Benzene-1,4-diamine
\item p-Aminoaniline
\item 4-Aminobenzenamine
\end{enumerate}
\newpage
\section*{Question 23}
Which of the following is not a correct statement for primary aliphatic amines? 
\begin{enumerate}[label=(\alph*)]
\item The intermolecular association in primary amines is less than the intermolecular association in secondary amines.
\item Primary amines on treating with nitrous acid solution form corresponding alcohols except methyl amine.
\item Primary amines are less basic than the secondary amines.
\item Primary amines can be prepared by the gabriel phthalimide synthesis.
\end{enumerate}
\newpage
\section*{Question 24}
what is A in the following reaction ?\includegraphics[width=\textwidth]{https://kodemapa.com/static/media/wl_client/1/qdump/dd962b43da3e663bef2c213d7dbe3f88/236e4ffa2b286d5be56b69d34825b864.png}\newline
\begin{enumerate}[label=(\alph*)]
\item \includegraphics[width=\textwidth]{https://kodemapa.com/static/media/wl_client/1/qdump/dd962b43da3e663bef2c213d7dbe3f88/7e913335697edc435c318cd8f4c74c35.png}
\item \includegraphics[width=\textwidth]{https://kodemapa.com/static/media/wl_client/1/qdump/dd962b43da3e663bef2c213d7dbe3f88/af241ff17a69316d69e677cbe7bee714.png}
\item \includegraphics[width=\textwidth]{https://kodemapa.com/static/media/wl_client/1/qdump/dd962b43da3e663bef2c213d7dbe3f88/96390fc856177859ff1bc0820852c540.png}
\item \includegraphics[width=\textwidth]{https://kodemapa.com/static/media/wl_client/1/qdump/dd962b43da3e663bef2c213d7dbe3f88/502f4bfc33a7ea5f5446b31091aa8c80.png}
\end{enumerate}
\newpage
\section*{Question 25}
An organic compound "A" on treatment with benzene sulphonyl chloride gives compound \(B . B\) is soluble in dil. \(\mathrm{NaOH}\) solution. Compound \(A\) is______________. 
\begin{enumerate}[label=(\alph*)]
\item \(\mathrm{C}_6 \mathrm{H}_5-\mathrm{N}-\left(\mathrm{CH}_3\right)_2\)
\item \(\mathrm{C}_6 \mathrm{H}_5-\mathrm{NHCH}_2 \mathrm{CH}_3\)
\item \(\mathrm{C}_6 \mathrm{H}_5-\mathrm{CH}_2 \mathrm{NHCH}_3\)
\item \(\mathrm{C}_6 \mathrm{H}_5-\mathrm{CH}-\mathrm{NH}_2\)
\end{enumerate}
\newpage
\section*{Question 26}
A primary aliphatic amine on reaction with nitrous acid in cold ( \(273 \mathrm{~K})\) and there after raising temperature of reaction mixture to room temperature (298 K), gives.
\begin{enumerate}[label=(\alph*)]
\item nitrile
\item alcohol
\item diazonium salt
\item secondary amine
\end{enumerate}
\newpage
\section*{Question 27}
Match List I with List II.
    \setlength{\arrayrulewidth}{0.8mm}
    \begin{tabular}{|c|c|}
    \hline
    List-I & List-II \\
\hline
A. Benzenesulphonyl Chloride & I. Test for primary amines \\
\hline
B. Hoffmann bromamide reaction & II. Anti Saytzeff \\
\hline
C. Carbylamine reaction & III. Hinsberg reagent \\
\hline
D. Hoffmann orientation & IV. Known reaction of Isocyanates \\
\hline

    \end{tabular}
    \setlength{\arrayrulewidth}{0.4mm}
    Choose the correct answer from the options given below: \newline
\begin{enumerate}[label=(\alph*)]
\item A-IV, B-III, C-II, D-I
\item A-IV, B-II, C-I, D-II
\item A-III, B-IV, C-I, D-II
\item A-IV, B-III, C-I, D-II
\end{enumerate}
\newpage
\section*{Question 28}
Compound \(\mathrm{A}\) is converted to \(\mathrm{B}\) on reaction with \(\mathrm{CHCl}_3\) and \(\mathrm{KOH}\). The compound \(\mathrm{B}\) is toxic and can be decomposed by C. A, B and \(C\) respectively are :
\begin{enumerate}[label=(\alph*)]
\item primary amine, nitrile compound, conc. \(\mathrm{HCl}\)
\item secondary amine, isonitrile compound, conc. \(\mathrm{NaOH}\)
\item primary amine, isonitrile compound, conc. \(\mathrm{HCl}\)
\item secondary amine, nitrile compound, conc. \(\mathrm{NaOH}\)
\end{enumerate}
\newpage
\section*{Question 29}
The major product formed in the following reaction is.\includegraphics[width=\textwidth]{https://kodemapa.com/static/media/wl_client/1/qdump/dd962b43da3e663bef2c213d7dbe3f88/cd501f5bf6a7ba9867212e69f5da1961.png}\newline
\begin{enumerate}[label=(\alph*)]
\item \includegraphics[width=\textwidth]{https://kodemapa.com/static/media/wl_client/1/qdump/dd962b43da3e663bef2c213d7dbe3f88/ea66b26d652be69c42f279d0ca208c29.png}
\item \includegraphics[width=\textwidth]{https://kodemapa.com/static/media/wl_client/1/qdump/dd962b43da3e663bef2c213d7dbe3f88/2952e31a186e1acc29cc186fdfb5dd6b.png}
\item \includegraphics[width=\textwidth]{https://kodemapa.com/static/media/wl_client/1/qdump/dd962b43da3e663bef2c213d7dbe3f88/d4fa77cbadb82c63c120985ab1e82614.png}
\item \includegraphics[width=\textwidth]{https://kodemapa.com/static/media/wl_client/1/qdump/dd962b43da3e663bef2c213d7dbe3f88/59084e148fc2d739417759e72917593f.png}
\end{enumerate}
\newpage
\section*{Question 30}
Given below are two statements :Statement I : In Hofmann degradation reaction, the migration of only an alkyl group takes place from carbonyl carbon of the amide to the nitrogen atom.Statement II : The group is migrated in Hofmann degradation reaction to electron deficient atom.In the light of the above statements, choose the most appropriate answer from the options given below: 
\begin{enumerate}[label=(\alph*)]
\item Both Statement I and Statement II are correct.
\item Both Statement I and Statement II are incorrect.
\item Statement I is correct but Statement II is incorrect.
\item Statement I is incorrect but Statement II is correct.
\end{enumerate}
\newpage
\section*{Question 31}
Given below are two statements :Statement I : Aniline is less basic than acetamide.Statement II : In aniline, the lone pair of electrons on nitrogen atom is delocalised over benzene ring due to resonance and hence less available to a proton.Choose the most appropriate option ; 
\begin{enumerate}[label=(\alph*)]
\item Statement I is true but statement II is false.
\item Statement I is false but statement II is true.
\item Both statement I and statement II are true.
\item Both statement I and statement II are false.
\end{enumerate}
\newpage
\section*{Question 32}
The decreasing order of basicity of the following amines is:\includegraphics[width=\textwidth]{https://kodemapa.com/static/media/wl_client/1/qdump/dd962b43da3e663bef2c213d7dbe3f88/2752f49d87bf0fdd936512818343839f.png}\newline
\begin{enumerate}[label=(\alph*)]
\item \((A)>(C)>(D)>(B)\)
\item \((C)>(A)>(B)>(D)\)
\item \((B)>(C)>(D)>(A)\)
\item \((C)>(B)>(A)>(D)\)
\end{enumerate}
\newpage
\section*{Question 33}
\includegraphics[width=\textwidth]{https://kodemapa.com/static/media/wl_client/1/qdump/dd962b43da3e663bef2c213d7dbe3f88/e017bcaa9a7e8b73dcdedce8d8d5d33d.png}\newlineIn the chemical reactions given above \(A\) and \(B\) respectively are: 
\begin{enumerate}[label=(\alph*)]
\item \(\mathrm{H}_3 \mathrm{PO}_2\) and \(\mathrm{CH}_3 \mathrm{CH}_2 \mathrm{Cl}\)
\item \(\mathrm{CH}_3 \mathrm{CH}_2 \mathrm{OH}\) and \(\mathrm{H}_3 \mathrm{PO}_2\)
\item \(\mathrm{H}_3 \mathrm{O}_2\) and \(\mathrm{CH}_3 \mathrm{CH}_2 \mathrm{OH}\)
\item \(\mathrm{CH}_3 \mathrm{CH}_2 \mathrm{Cl}\) and \(\mathrm{H}_3 \mathrm{PO}_2\)
\end{enumerate}
\newpage
\section*{Question 34}
\(C_7 H_9 N \mid\) has how many isomeric forms that contain a benzene ring? 
\begin{enumerate}[label=(\alph*)]
\end{enumerate}
\newpage
\section*{Question 35}
The total number of electrons around the nitrogen atom in amines are, 
\begin{enumerate}[label=(\alph*)]
\end{enumerate}
\newpage
\section*{Question 36}
The number of primary amines of formula \(\mathrm{C}_4 \mathrm{H}_{11} \mathrm{~N}\) is ? 
\begin{enumerate}[label=(\alph*)]
\end{enumerate}
\newpage
\section*{Question 37}
The product A formed in the following reaction is:\includegraphics[width=\textwidth]{https://kodemapa.com/static/media/wl_client/1/qdump/dd962b43da3e663bef2c213d7dbe3f88/1520d487acea3e12471d7be5f928b59f.png}\newline\newline
\begin{enumerate}[label=(\alph*)]
\item \includegraphics[width=\textwidth]{https://kodemapa.com/static/media/wl_client/1/qdump/dd962b43da3e663bef2c213d7dbe3f88/2ca5365cc0784172da4f25597d9ad12a.png}
\item \includegraphics[width=\textwidth]{https://kodemapa.com/static/media/wl_client/1/qdump/dd962b43da3e663bef2c213d7dbe3f88/9fbdb1561f5c01c9175d265eb6391c64.png}
\item \includegraphics[width=\textwidth]{https://kodemapa.com/static/media/wl_client/1/qdump/dd962b43da3e663bef2c213d7dbe3f88/3dc05a6f09a27296e56e9f0ecfac5216.png}
\item \includegraphics[width=\textwidth]{https://kodemapa.com/static/media/wl_client/1/qdump/dd962b43da3e663bef2c213d7dbe3f88/567f0d4eba6072aab1e7657d9bd4afdd.png}
\end{enumerate}
\newpage
\section*{Question 38}
In the reaction of hypobromite with amide, the carbonyl carbon is lost as 
\begin{enumerate}[label=(\alph*)]
\item \(\mathrm{CO}_3{ }^{2-}\)
\item \(\mathrm{HCO}_3{ }^{-}\)
\item \(\mathrm{CO}_2\)
\item \(\mathrm{CO}\)
\end{enumerate}
\newpage
\section*{Question 39}
An organic compound "A" on treatment with benzene sulphonyl chloride gives compound \(B . B\) is soluble in dil. \(\mathrm{NaOH}\) solution. Compound \(A\) is______________. 
\begin{enumerate}[label=(\alph*)]
\item \(\mathrm{C}_6 \mathrm{H}_5-\mathrm{N}-\left(\mathrm{CH}_3\right)_2\)
\item \(\mathrm{C}_6 \mathrm{H}_5-\mathrm{NHCH}_2 \mathrm{CH}_3\)
\item \(\mathrm{C}_6 \mathrm{H}_5-\mathrm{CH}_2 \mathrm{NHCH}_3\)
\item \(\mathrm{C}_6 \mathrm{H}_5-\mathrm{CH}-\mathrm{NH}_2\)
\end{enumerate}
\newpage
\section*{Question 40}
Match List I with List II.
    \setlength{\arrayrulewidth}{0.8mm}
    \begin{tabular}{|c|c|}
    \hline
    List-I & List-II \\
\hline
A. Benzenesulphonyl Chloride & I. Test for primary amines \\
\hline
B. Hoffmann bromamide reaction & II. Anti Saytzeff \\
\hline
C. Carbylamine reaction & III. Hinsberg reagent \\
\hline
D. Hoffmann orientation & IV. Known reaction of Isocyanates \\
\hline

    \end{tabular}
    \setlength{\arrayrulewidth}{0.4mm}
    Choose the correct answer from the options given below: \newline
\begin{enumerate}[label=(\alph*)]
\item A-IV, B-III, C-II, D-I
\item A-IV, B-II, C-I, D-II
\item A-III, B-IV, C-I, D-II
\item A-IV, B-III, C-I, D-II
\end{enumerate}
\newpage
\section*{Question 41}
Consider the following sequence of reaction :\includegraphics[width=\textwidth]{https://kodemapa.com/static/media/wl_client/1/qdump/dd962b43da3e663bef2c213d7dbe3f88/57dcd9d8a769c8fea41cff9d8e01d464.png}\newlineThe product ' \(\mathrm{B}\) ' is : 
\begin{enumerate}[label=(\alph*)]
\item \includegraphics[width=\textwidth]{https://kodemapa.com/static/media/wl_client/1/qdump/dd962b43da3e663bef2c213d7dbe3f88/424f41a0563c11457d451a9c8548f457.png}
\item \includegraphics[width=\textwidth]{https://kodemapa.com/static/media/wl_client/1/qdump/dd962b43da3e663bef2c213d7dbe3f88/d4b4b4ca1568a3c5ecaffa3c96777568.png}
\item \includegraphics[width=\textwidth]{https://kodemapa.com/static/media/wl_client/1/qdump/dd962b43da3e663bef2c213d7dbe3f88/514d98122f31a4308660c7cf35c5c48c.png}
\item \includegraphics[width=\textwidth]{https://kodemapa.com/static/media/wl_client/1/qdump/dd962b43da3e663bef2c213d7dbe3f88/b8f6fcf1f36be06b6950c90da3323688.png}
\end{enumerate}
\newpage
\section*{Question 42}
Number of isomeric aromatic amines with molecular formula \(\mathrm{C}_8 \mathrm{H}_{11} \mathrm{~N}\), which can be synthesized by Gabriel Phthalimide synthesis is ___________.  [6-Apr-2023]
\begin{enumerate}[label=(\alph*)]
\end{enumerate}
\newpage
\section*{Question 43}
Given below are two statements :Statement I : In Hofmann degradation reaction, the migration of only an alkyl group takes place from carbonyl carbon of the amide to the nitrogen atom.Statement II : The group is migrated in Hofmann degradation reaction to electron deficient atom.In the light of the above statements, choose the most appropriate answer from the options given below: 
\begin{enumerate}[label=(\alph*)]
\item Both Statement I and Statement II are correct.
\item Both Statement I and Statement II are incorrect.
\item Statement I is correct but Statement II is incorrect.
\item Statement I is incorrect but Statement II is correct.
\end{enumerate}
\newpage
\section*{Question 44}
During halogen test, sodium fusion extract is boiled with concentrated \(\mathrm{HNO}_3\) to 
\begin{enumerate}[label=(\alph*)]
\item remove unreacted sodium
\item decompose cyanide or sulphide of sodium
\item extract halogen from organic compound
\item maintain the \(\mathrm{pH}\) of extract.
\end{enumerate}
\newpage
\section*{Question 45}
Given below are two statements :Statement I : Aniline is less basic than acetamide.Statement II : In aniline, the lone pair of electrons on nitrogen atom is delocalised over benzene ring due to resonance and hence less available to a proton.Choose the most appropriate option ; 
\begin{enumerate}[label=(\alph*)]
\item Statement I is true but statement II is false.
\item Statement I is false but statement II is true.
\item Both statement I and statement II are true.
\item Both statement I and statement II are false.
\end{enumerate}
\newpage
\section*{Question 46}
The decreasing order of basicity of the following amines is:\includegraphics[width=\textwidth]{https://kodemapa.com/static/media/wl_client/1/qdump/dd962b43da3e663bef2c213d7dbe3f88/2752f49d87bf0fdd936512818343839f.png}\newline
\begin{enumerate}[label=(\alph*)]
\item \((A)>(C)>(D)>(B)\)
\item \((C)>(A)>(B)>(D)\)
\item \((B)>(C)>(D)>(A)\)
\item \((C)>(B)>(A)>(D)\)
\end{enumerate}
\newpage
\section*{Question 47}
\(C_7 H_9 N \mid\) has how many isomeric forms that contain a benzene ring? 
\begin{enumerate}[label=(\alph*)]
\end{enumerate}
\newpage
\section*{Question 48}
The number of primary amines of formula \(\mathrm{C}_4 \mathrm{H}_{11} \mathrm{~N}\) is ? 
\begin{enumerate}[label=(\alph*)]
\end{enumerate}
\newpage
\section*{Question 49}
\includegraphics[width=\textwidth]{https://kodemapa.com/static/media/wl_client/1/qdump/dd962b43da3e663bef2c213d7dbe3f88/80a0fa0979964d76f700803011f08f52.png}\newlineIn the above reactions, product \(A\) and product \(B\) respectively are:
\begin{enumerate}[label=(\alph*)]
\item \includegraphics[width=\textwidth]{https://kodemapa.com/static/media/wl_client/1/qdump/dd962b43da3e663bef2c213d7dbe3f88/c0d95ec45caccc885e7a10882ab11dcf.png}
\item \includegraphics[width=\textwidth]{https://kodemapa.com/static/media/wl_client/1/qdump/dd962b43da3e663bef2c213d7dbe3f88/edb2a59275575343522e7f636ee9ead1.png}
\item \includegraphics[width=\textwidth]{https://kodemapa.com/static/media/wl_client/1/qdump/dd962b43da3e663bef2c213d7dbe3f88/eb64f7e901eb6e6ef08159de542d0071.png}
\item \includegraphics[width=\textwidth]{https://kodemapa.com/static/media/wl_client/1/qdump/dd962b43da3e663bef2c213d7dbe3f88/f1418dc91d368959bcf7b2a892c91d36.png}
\end{enumerate}
\newpage
\section*{Question 50}
\includegraphics[width=\textwidth]{https://kodemapa.com/static/media/wl_client/1/qdump/dd962b43da3e663bef2c213d7dbe3f88/4ec04ea353b6f65a5cf7b06ed0289d5e.png}\newlineConsider the given reaction, percentage yield of, 
\begin{enumerate}[label=(\alph*)]
\item \((C)>(A)>(B)\)
\item \((B)>(C)>(A)\)
\item \((A)>(C)>(B)\)
\item \((C)>(B)>(A)\)
\end{enumerate}
\newpage
\section*{Question 51}
\includegraphics[width=\textwidth]{https://kodemapa.com/static/media/wl_client/1/qdump/dd962b43da3e663bef2c213d7dbe3f88/4db12f12a4cdcdfcd29ad6819c260159.png}\newlineIn the chemical reactions given above \(A\) and \(B\) respectively are: 
\begin{enumerate}[label=(\alph*)]
\item \(\mathrm{H}_3 \mathrm{PO}_2\) and \(\mathrm{CH}_3 \mathrm{CH}_2 \mathrm{Cl}\)
\item \(\mathrm{CH}_3 \mathrm{CH}_2 \mathrm{OH}\) and \(\mathrm{H}_3 \mathrm{PO}_2\)
\item \(\mathrm{H}_3 \mathrm{O}_2\) and \(\mathrm{CH}_3 \mathrm{CH}_2 \mathrm{OH}\)
\item \(\mathrm{CH}_3 \mathrm{CH}_2 \mathrm{Cl}\) and \(\mathrm{H}_3 \mathrm{PO}_2\)
\end{enumerate}
\newpage
\section*{Question 52}
Given below are two statements :Statement I: Aniline reacts with con. \(\mathrm{H}_2 \mathrm{SO}_4\) followed by heating at \(453-473 \mathrm{~K}\) gives p-aminobenzene sulphonic acid, which gives blood red colour in the 'Lassaigne's test'.\newlineStatement II: In Friedel - Craft's alkylation and acylation reactions, aniline forms salt with the \(\mathrm{AlCl}_3\) catalyst.\newlineDue to this, nitrogen of aniline aquires a positive charge and acts as deactivating group.\newlineIn the light of the above statements, choose the correct answer from the options given below :
\begin{enumerate}[label=(\alph*)]
\item Statement I is false but statement II is true
\item Both statement I and statement II are false
\item Statement I is true but statement II is false
\item Both statement I and statement II are true
\end{enumerate}
\newpage
\section*{Question 53}
An organic compound "A" on treatment with benzene sulphonyl chloride gives compound \(B . B\) is soluble in dil. \(\mathrm{NaOH}\) solution. Compound \(A\) is______________. 
\begin{enumerate}[label=(\alph*)]
\item \(\mathrm{C}_6 \mathrm{H}_5-\mathrm{N}-\left(\mathrm{CH}_3\right)_2\)
\item \(\mathrm{C}_6 \mathrm{H}_5-\mathrm{NHCH}_2 \mathrm{CH}_3\)
\item \(\mathrm{C}_6 \mathrm{H}_5-\mathrm{CH}_2 \mathrm{NHCH}_3\)
\item \(\mathrm{C}_6 \mathrm{H}_5-\mathrm{CH}-\mathrm{NH}_2\)
\end{enumerate}
\newpage
\section*{Question 54}
A primary aliphatic amine on reaction with nitrous acid in cold ( \(273 \mathrm{~K})\) and there after raising temperature of reaction mixture to room temperature (298 K), gives.
\begin{enumerate}[label=(\alph*)]
\item nitrile
\item alcohol
\item diazonium salt
\item secondary amine
\end{enumerate}
\newpage
\section*{Question 55}
Primary, secondary and tertiary amines can be separated using. 
\begin{enumerate}[label=(\alph*)]
\item para-toluene sulphonyl chloride
\item chloroform and \(\mathrm{KOH}\)
\item benzene sulphonic acid
\item acetyl amide
\end{enumerate}
\newpage
\section*{Question 56}
The major product formed in the following reaction is.\includegraphics[width=\textwidth]{https://kodemapa.com/static/media/wl_client/1/qdump/dd962b43da3e663bef2c213d7dbe3f88/cd501f5bf6a7ba9867212e69f5da1961.png}\newline
\begin{enumerate}[label=(\alph*)]
\item \includegraphics[width=\textwidth]{https://kodemapa.com/static/media/wl_client/1/qdump/dd962b43da3e663bef2c213d7dbe3f88/ea66b26d652be69c42f279d0ca208c29.png}
\item \includegraphics[width=\textwidth]{https://kodemapa.com/static/media/wl_client/1/qdump/dd962b43da3e663bef2c213d7dbe3f88/2952e31a186e1acc29cc186fdfb5dd6b.png}
\item \includegraphics[width=\textwidth]{https://kodemapa.com/static/media/wl_client/1/qdump/dd962b43da3e663bef2c213d7dbe3f88/d4fa77cbadb82c63c120985ab1e82614.png}
\item \includegraphics[width=\textwidth]{https://kodemapa.com/static/media/wl_client/1/qdump/dd962b43da3e663bef2c213d7dbe3f88/59084e148fc2d739417759e72917593f.png}
\end{enumerate}
\newpage
\section*{Question 57}
Consider the following sequence of reaction :\includegraphics[width=\textwidth]{https://kodemapa.com/static/media/wl_client/1/qdump/dd962b43da3e663bef2c213d7dbe3f88/57dcd9d8a769c8fea41cff9d8e01d464.png}\newlineThe product ' \(\mathrm{B}\) ' is : 
\begin{enumerate}[label=(\alph*)]
\item \includegraphics[width=\textwidth]{https://kodemapa.com/static/media/wl_client/1/qdump/dd962b43da3e663bef2c213d7dbe3f88/424f41a0563c11457d451a9c8548f457.png}
\item \includegraphics[width=\textwidth]{https://kodemapa.com/static/media/wl_client/1/qdump/dd962b43da3e663bef2c213d7dbe3f88/d4b4b4ca1568a3c5ecaffa3c96777568.png}
\item \includegraphics[width=\textwidth]{https://kodemapa.com/static/media/wl_client/1/qdump/dd962b43da3e663bef2c213d7dbe3f88/514d98122f31a4308660c7cf35c5c48c.png}
\item \includegraphics[width=\textwidth]{https://kodemapa.com/static/media/wl_client/1/qdump/dd962b43da3e663bef2c213d7dbe3f88/b8f6fcf1f36be06b6950c90da3323688.png}
\end{enumerate}
\newpage
\section*{Question 58}
The correct order in aqueous medium of basic strength in case of methyl substituted amines is : 
\begin{enumerate}[label=(\alph*)]
\item \(\mathrm{Me}_2 \mathrm{NH}>\mathrm{MeNH}_2>\mathrm{Me}_3 \mathrm{~N}>\mathrm{NH}_3\)
\item \(\mathrm{Me}_2 \mathrm{NH}>\mathrm{Me}_3 \mathrm{~N}>\mathrm{MeNH}_2>\mathrm{NH}_3\)
\item \(\mathrm{NH}_3>\mathrm{Me}_3 \mathrm{~N}>\mathrm{MeNH}_2>\mathrm{Me}_2 \mathrm{NH}\)
\item \(\mathrm{Me}_3 \mathrm{~N}>\mathrm{Me}_2 \mathrm{NH}>\mathrm{MeNH}_2>\mathrm{NH}_3\)
\end{enumerate}
\newpage
\section*{Question 59}
Given below are two statements :Statement I : In Hofmann degradation reaction, the migration of only an alkyl group takes place from carbonyl carbon of the amide to the nitrogen atom.Statement II : The group is migrated in Hofmann degradation reaction to electron deficient atom.In the light of the above statements, choose the most appropriate answer from the options given below: 
\begin{enumerate}[label=(\alph*)]
\item Both Statement I and Statement II are correct.
\item Both Statement I and Statement II are incorrect.
\item Statement I is correct but Statement II is incorrect.
\item Statement I is incorrect but Statement II is correct.
\end{enumerate}
\newpage
\section*{Question 60}
The most appropriate reagent for conversion of \(\mathrm{C}_2 \mathrm{H}_5 \mathrm{CN}\) into \(\mathrm{CH}_3 \mathrm{CH}_2 \mathrm{CH}_2 \mathrm{NH}_2\) is:
\begin{enumerate}[label=(\alph*)]
\item \(\mathrm{NaBH}_4\)
\item \(\mathrm{CaH}_2\)
\item LiAl H\(_4\)
\item \(\mathrm{Na}(\mathrm{CN}) \mathrm{BH}_3\)
\end{enumerate}
\newpage
\section*{Question 61}
\includegraphics[width=\textwidth]{https://kodemapa.com/static/media/wl_client/1/qdump/dd962b43da3e663bef2c213d7dbe3f88/4ec04ea353b6f65a5cf7b06ed0289d5e.png}\newlineConsider the given reaction, percentage yield of, 
\begin{enumerate}[label=(\alph*)]
\item \((C)>(A)>(B)\)
\item \((B)>(C)>(A)\)
\item \((A)>(C)>(B)\)
\item \((C)>(B)>(A)\)
\end{enumerate}
\newpage
\section*{Question 62}
\includegraphics[width=\textwidth]{https://kodemapa.com/static/media/wl_client/1/qdump/dd962b43da3e663bef2c213d7dbe3f88/4db12f12a4cdcdfcd29ad6819c260159.png}\newlineIn the chemical reactions given above \(A\) and \(B\) respectively are: 
\begin{enumerate}[label=(\alph*)]
\item \(\mathrm{H}_3 \mathrm{PO}_2\) and \(\mathrm{CH}_3 \mathrm{CH}_2 \mathrm{Cl}\)
\item \(\mathrm{CH}_3 \mathrm{CH}_2 \mathrm{OH}\) and \(\mathrm{H}_3 \mathrm{PO}_2\)
\item \(\mathrm{H}_3 \mathrm{O}_2\) and \(\mathrm{CH}_3 \mathrm{CH}_2 \mathrm{OH}\)
\item \(\mathrm{CH}_3 \mathrm{CH}_2 \mathrm{Cl}\) and \(\mathrm{H}_3 \mathrm{PO}_2\)
\end{enumerate}
\newpage
\end{document}