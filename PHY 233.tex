\documentclass{article}
                    \usepackage{amsmath}
                    \usepackage{amssymb}
                    \usepackage{graphicx}
                    \usepackage{enumitem}
                    \usepackage{longtable}
                    \title{PHY 233}
                    \begin{document}
                    \maketitle
                    \section*{Question 1}
The magnitude of the difference between the individual measurement and true value of the quantity is called:\newline
\begin{enumerate}[label=(\alph*)]
\item Absolute error
\item Relative error
\item Percentage error
\item None of these
\end{enumerate}
\newpage
\section*{Question 2}
Which of the following pairs has the same dimensions?
\begin{enumerate}[label=(\alph*)]
\item Specific heat and latent heat
\item Impulse and momentum
\item Surface tension and force
\item Moment of Inertia and torque
\end{enumerate}
\newpage
\section*{Question 3}
If the fundamental quantities are energy \((E)\), velocity \((v)\) and force \((F)\), then what will the dimensions of mass?
\begin{enumerate}[label=(\alph*)]
\item \({Ev}^{2}\)
\item \({Ev}^{-2}\)
\item \({Fv}^{-1}\)
\item \({Fv}^{-2}\)
\end{enumerate}
\newpage
\section*{Question 4}
Which of the following units denotes the dimensions \(\frac{\mathrm{ML}^2 }{\mathrm{Q}^2}\), where \(\mathrm{Q}\) denotes the electric charge?
\begin{enumerate}[label=(\alph*)]
\item \(\mathrm{H} / \mathrm{m}^2\)
\item Henry (H)
\item \(\mathrm{H} / \mathrm{m}^2\)
\item Weber (Wb)
\end{enumerate}
\newpage
\section*{Question 5}
The dimensional formula of Planck's constant is:
\begin{enumerate}[label=(\alph*)]
\item \(\left[M L^{2} T^{-1}\right]\)
\item \(\left[M L^{2} T^{-2}\right]\)
\item \(\left[M L^{0} T^{2}\right]\)
\item \(\left[M L T^{-2}\right]\)
\end{enumerate}
\newpage
\section*{Question 6}
An Odometer is an instrument used to measure ________ in automobiles.
\begin{enumerate}[label=(\alph*)]
\item speed
\item odour
\item direction
\item distance
\end{enumerate}
\newpage
\section*{Question 7}
In a pendulum, the time period is measured by \(0.2 \%\) accuracy and length are measured by \(0.5 \%\) accuracy. Find the percentage accuracy in the value of \(\mathrm{g}\).
\begin{enumerate}[label=(\alph*)]
\item \(0.3 \%\)
\item \(0.7 \%\)
\item \(0.1 \%\)
\item \(0.9 \%\)
\end{enumerate}
\newpage
\section*{Question 8}
If \(\mathrm{M}\) denotes angular momentum and \(\mathrm{p}\) denotes linear momentum, the dimensions of \(\frac{\mathrm M}{\mathrm p}\) is:
\begin{enumerate}[label=(\alph*)]
\item \([\mathrm L^2]\)
\item \([\mathrm L^0]\)
\item \([\mathrm L^1]\)
\item \([\mathrm L^3]\)
\end{enumerate}
\newpage
\section*{Question 9}
The unit of momentum is:
\begin{enumerate}[label=(\alph*)]
\item \(\mathrm{Kgms}^{2}\)
\item \(\mathrm{Kgms}^{-2}\)
\item \(\mathrm{Kgms}\)
\item \(\mathrm{Kgms}^{-1}\)
\end{enumerate}
\newpage
\section*{Question 10}
Which of the physics quantity has the same unit in both C.G.S and M.K.S system?
\begin{enumerate}[label=(\alph*)]
\item Velocity
\item Distance
\item Time
\item Mass
\end{enumerate}
\newpage
\section*{Question 11}
The square root of the product of inductance and capacitance has the dimension of length:
\begin{enumerate}[label=(\alph*)]
\item Mass
\item Length
\item Time
\item No dimension
\end{enumerate}
\newpage
\section*{Question 12}
Which one of the following is not a derived unit?
\begin{enumerate}[label=(\alph*)]
\item Joule
\item  Watt
\item Newton
\item Kilogram
\end{enumerate}
\newpage
\section*{Question 13}
Which of the following pair of physical quantities does not have the same dimensions?
\begin{enumerate}[label=(\alph*)]
\item Electric flux, Electric dipole moment\newline
\item Pressure, young's modulus
\item Electromotive force, Potential difference
\item Heat, Potential energy
\end{enumerate}
\newpage
\section*{Question 14}
A wattmeter reads \(25.34 \mathrm{~W}\). The absolute error in measurement is \(-0.11 \mathrm{~W}\). What is the true value of power:
\begin{enumerate}[label=(\alph*)]
\item \(25.23 \mathrm{~W}\)
\item \(25.45 \mathrm{~W}\)
\item \(-25.23 \mathrm{~W}\)
\item \(-25.45 \mathrm{~W}\)
\end{enumerate}
\newpage
\section*{Question 15}
Dimensional formula of \(\omega\) in equation \(y=a \sin (\omega t+k x)\) is:
\begin{enumerate}[label=(\alph*)]
\item \([M^0 L^0 T^{-1}]\)
\item \([M^0 L^{-1} T^0]\)
\item \([M L^0 T^0]\)
\item \([M^0 L T^{-1}]\)
\end{enumerate}
\newpage
\section*{Question 16}
Which of the following quantity has dimensional formula as that of \(\frac{\text { Energy }}{\text { Mass } \times \text { Length }}\) is:
\begin{enumerate}[label=(\alph*)]
\item Force
\item Power
\item Pressure
\item Acceleration
\end{enumerate}
\newpage
\section*{Question 17}
The dimensions of 'resistance' are same as those of __________ where \(h\) is the Planck's constant.
\begin{enumerate}[label=(\alph*)]
\item \(\frac{h}{e^2}\)
\item \(\frac{h}{e}\)
\item \(\frac{h^2}{e^2}\)\newline
\item \(\frac{h^2}{e}\)
\end{enumerate}
\newpage
\section*{Question 18}
Given below are two statements:
Statement I : Astronomical unit \((Au)\), Parsec \((Pc)\) and Light year \((ly)\) are units for measuring astronomical distances.
Statement II : \(Au <\) Parsec \(( Pc )< ly\)
In the light of the above statements, choose the most appropriate answer from the options given below: 
\begin{enumerate}[label=(\alph*)]
\item Both Statements I and Statements II are incorrect
\item Statements I is correct but Statements II is incorrect
\item Both Statements I and Statements II are correct
\item Statements I is incorrect but Statements II is correct
\end{enumerate}
\newpage
\section*{Question 19}
Identify the pair of physical quantities which have different dimensions: 
\begin{enumerate}[label=(\alph*)]
\item Wave number and Rydberg's constant
\item Stress and Coefficient of elasticity
\item Coercivity and Magnetisation
\item Specific heat capacity and Latent heat
\end{enumerate}
\newpage
\section*{Question 20}
An expression for a dimensionless quantity \(P\) is given by \(P =\frac{\alpha}{\beta} \log _{ e }\left(\frac{ kt }{\beta x }\right)\); where \(\alpha\) and \(\beta\) are constants, \(x\) is distance ; \(k\) is Boltzmann constant and \(t\) is the temperature. Then the dimensions of \(\alpha\) will be:
\begin{enumerate}[label=(\alph*)]
\end{enumerate}
\newpage
\section*{Question 21}
The SI unit of a physical quantity is pascal-second. The dimensional formula of this quantity will be: 
\begin{enumerate}[label=(\alph*)]
\end{enumerate}
\newpage
\section*{Question 22}
In Vander Waals equation \(\left\lfloor P +\frac{ a }{ V ^2}\right\rfloor[ V - b ]= RT\); \(P\) is pressure, \(V\) is volume, \(R\) is universal gas constant and \(T\) is  temperature. The ratio of constants \(\frac{ a }{ b }\) is dimensionally equal to:
\begin{enumerate}[label=(\alph*)]
\item \(\frac{P}{V}\)
\item \(\frac{ V }{ P }\)
\item \(PV\)
\item \(PV ^3\)
\end{enumerate}
\newpage
\section*{Question 23}
The pitch of the screw gauge is 1 mm and there are 100 divisions on the circular scale. When nothing is put in between the jaws, the zero of the circular scale lies 8 divisions below the reference line. When a wire is placed between the jaws, the first linear scale division is clearly visible while 72 nd division on circular scale coincides with the reference line. The radius of the wire is: 
\begin{enumerate}[label=(\alph*)]
\end{enumerate}
\newpage
\section*{Question 24}
One main scale division of a vernier callipers is \(acm\) and \(n\)th division of the vernier scale coincide with \((n-1)\) th division of the main scale. The least count of the callipers (in \(mm\) ) is: 
\begin{enumerate}[label=(\alph*)]
\end{enumerate}
\newpage
\section*{Question 25}
If \(E , L , M\) and \(G\) denote the quantities as energy, angular momentum, mass and constant of gravitation respectively, then the dimension of \(P\) in the formula \(P = EL ^2 M ^{-5} G ^{-2}\) is: 
\begin{enumerate}[label=(\alph*)]
\end{enumerate}
\newpage
\section*{Question 26}
A screw gauge has 50 divisions on its circular scale. The circular scale is 4 units ahead of the pitch scale marking, prior to use. Upon one complete rotation of the circular scale, a displacement of \(0.5 mm\) is noticed on the pitch scale. The nature of zero error involved, and the least count of the screw gauge, are respectively:
\begin{enumerate}[label=(\alph*)]
\item Negative, 2 mm
\item Positive, 10 mm
\item Positive, 0.1 mm
\item Negative, 0.1 mm
\end{enumerate}
\newpage
\section*{Question 27}
From the following combinations of physical constants (expressed through their usual symbols) the only combination, that would have the same value in different systems of units, is: 
\begin{enumerate}[label=(\alph*)]
\item \(\frac{c h}{2 \pi \varepsilon_0^2}\)
\item \(\frac{e^2}{2 \pi \varepsilon_o G m_e{ }^2}\left(m_e=\right.\) mass of electron \()\)
\item \(\frac{\mu_0 \varepsilon_0}{c^2} \frac{G}{h e^2}\)
\item \(\frac{2 \pi \sqrt{\mu_o \varepsilon_o}}{c e^2} \frac{h}{G}\)
\end{enumerate}
\newpage
\section*{Question 28}
Given below are two statements: one is labelled as Assertion(A) and the other is labelled as Reason (R).Assertion (A) : In Vernier calliper if positive zero error exists, then while taking measurements, the reading taken will be more than the actual reading.Reason (R) : The zero error in Vernier Calliper might have happened due to manufacturing defect or due to rough handling.In the light of the above statements, choose the correct answer from the options given below : 
\begin{enumerate}[label=(\alph*)]
\item Both (A) and (R) are correct and (R) is the correct explanation of (A)
\item  Both (A) and (R) are correct but (R) is not the correct explanation of (A)
\item (A) is true but (R) is false
\item (A) is false but (R) is true
\end{enumerate}
\newpage
\section*{Question 29}
If 50 Vernier divisions are equal to 49 main scale divisions of a travelling microscope and one smallest reading of main scale is 0.5mm, the Vernier constant of travelling microscope is: 
\begin{enumerate}[label=(\alph*)]
\item 0.1mm
\item 0.1cm
\item 0.01cm
\item 0.01mm
\end{enumerate}
\newpage
\section*{Question 30}
The measured value of the length of a simple pendulum is 20cm with 2mm accuracy. The time for 50 oscillations was measured to be 40 seconds with 1 second resolution. From these measurements, the accuracy in the measurement of acceleration due to gravity is N%. The value of N is: 
\begin{enumerate}[label=(\alph*)]
\item 4
\item 8
\item 6
\item 5
\end{enumerate}
\newpage
\end{document}