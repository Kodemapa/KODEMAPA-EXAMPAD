\documentclass{article}
                    \usepackage{amsmath}
                    \usepackage{amssymb}
                    \usepackage{graphicx}
                    \usepackage{enumitem}
                    \usepackage{longtable}
                    \title{tesst56}
                    \begin{document}
                    \maketitle
                    \section*{Question 1}
Which of the following physical quantity is dimensionless?\newline
\begin{enumerate}[label=(\alph*)]
\item Angle
\item Strain
\item Specific gravity
\item All of these
\end{enumerate}
\newpage
\section*{Question 2}
Express the formula of gravitational constant in terms of mass, velocity, and wavelength?
\begin{enumerate}[label=(\alph*)]
\item \(\frac{\mathrm{v} \lambda }{ M^{2}}\)
\item \(\frac{\mathrm{V}^{2} \lambda }{\mathrm{M}}\)
\item \(\frac{\mathrm{V} \lambda^{2}}{\mathrm{M}}\)
\item \(\frac{V^{3} \lambda^{3} }{ M}\)
\end{enumerate}
\newpage
\section*{Question 3}
A physical quantity Q is found to depend on observables \(\mathrm{x}, \mathrm{y}\) and \(\mathrm{z}\), obeying relation \(\mathrm{Q}=\frac{\mathrm{x}^{2 / 5} \mathrm{z}^{3}}{\mathrm{y}}\) The percentage error in the measurements of \(\mathrm{x}, \mathrm{y}\) and \(\mathrm{z}\) are \(1 \%, 2 \%\) and \(4 \%\) respectively. What is percentage error in the quantity Q will be:
\begin{enumerate}[label=(\alph*)]
\item 9 %
\item 10 %
\item 11 %
\item 12 %
\end{enumerate}
\newpage
\section*{Question 4}
An Odometer is an instrument used to measure ________ in automobiles.
\begin{enumerate}[label=(\alph*)]
\item speed
\item odour
\item direction
\item distance
\end{enumerate}
\newpage
\section*{Question 5}
In a pendulum, the time period is measured by \(0.2 \%\) accuracy and length are measured by \(0.5 \%\) accuracy. Find the percentage accuracy in the value of \(\mathrm{g}\).
\begin{enumerate}[label=(\alph*)]
\item \(0.3 \%\)
\item \(0.7 \%\)
\item \(0.1 \%\)
\item \(0.9 \%\)
\end{enumerate}
\newpage
\section*{Question 6}
The quantity that does not have mass in its dimension is:\newline
\begin{enumerate}[label=(\alph*)]
\item Electrical potential
\item Electrical resistance
\item Specific heat
\item Magnetic flux
\end{enumerate}
\newpage
\section*{Question 7}
If \(\mathrm{M}\) denotes angular momentum and \(\mathrm{p}\) denotes linear momentum, the dimensions of \(\frac{\mathrm M}{\mathrm p}\) is:
\begin{enumerate}[label=(\alph*)]
\item \([\mathrm L^2]\)
\item \([\mathrm L^0]\)
\item \([\mathrm L^1]\)
\item \([\mathrm L^3]\)
\end{enumerate}
\newpage
\section*{Question 8}
The unit of momentum is:
\begin{enumerate}[label=(\alph*)]
\item \(\mathrm{Kgms}^{2}\)
\item \(\mathrm{Kgms}^{-2}\)
\item \(\mathrm{Kgms}\)
\item \(\mathrm{Kgms}^{-1}\)
\end{enumerate}
\newpage
\section*{Question 9}
Which of following is the dimensional formula of Density?
\begin{enumerate}[label=(\alph*)]
\item \(\left[\mathrm{M}^{0} \mathrm{LT}^{-1}\right]\)
\item \(\left[\mathrm{MLT}^{-2}\right]\)
\item \(\left[\mathrm{ML}^{-3} \mathrm{~T}^{0}\right]\)
\item \(\left[\mathrm{MLT}^{-1}\right]\)
\end{enumerate}
\newpage
\section*{Question 10}
Which of the physics quantity has the same unit in both C.G.S and M.K.S system?
\begin{enumerate}[label=(\alph*)]
\item Velocity
\item Distance
\item Time
\item Mass
\end{enumerate}
\newpage
\section*{Question 11}
If \(x=a+b t+c t^{2}\), where \(x\) is in meters and \(t\) is in second. What is the unit of \(b\) and \(c\) ?
\begin{enumerate}[label=(\alph*)]
\item \(\left[\mathrm{M}^{-2} \mathrm{~L}^{1} \mathrm{~T}^{1}\right]\), \(\left[\mathrm{M}^{1} \mathrm{~L}^{1} \mathrm{~T}^{-2}\right]\)
\item \(\left[\mathrm{M}^{2} \mathrm{~L}^{1} \mathrm{~T}^{1}\right]\), \(\left[\mathrm{M}^{1} \mathrm{~L}^{1} \mathrm{~T}^{2}\right]\)
\item \(\left[\mathrm{M}^{1} \mathrm{~L}^{0} \mathrm{~T}^{-1}\right]\), \(\left[\mathrm{M}^{1} \mathrm{~L}^{0} \mathrm{~T}^{-2}\right]\)
\item \(\left[\mathrm{M}^{2} \mathrm{~L}^{0} \mathrm{~T}^{1}\right]\), \(\left[\mathrm{M}^{1} \mathrm{~L}^{0} \mathrm{~T}^{2}\right]\)
\end{enumerate}
\newpage
\section*{Question 12}
Which one of the following is not a derived unit?
\begin{enumerate}[label=(\alph*)]
\item Joule
\item  Watt
\item Newton
\item Kilogram
\end{enumerate}
\newpage
\section*{Question 13}
Which of the following pair of physical quantities does not have the same dimensions?
\begin{enumerate}[label=(\alph*)]
\item Electric flux, Electric dipole moment\newline
\item Pressure, young's modulus
\item Electromotive force, Potential difference
\item Heat, Potential energy
\end{enumerate}
\newpage
\section*{Question 14}
Which of the following quantity has dimensional formula as that of \(\frac{\text { Energy }}{\text { Mass } \times \text { Length }}\) is:
\begin{enumerate}[label=(\alph*)]
\item Force
\item Power
\item Pressure
\item Acceleration
\end{enumerate}
\newpage
\section*{Question 15}
The dimensions of 'resistance' are same as those of __________ where \(h\) is the Planck's constant.
\begin{enumerate}[label=(\alph*)]
\item \(\frac{h}{e^2}\)
\item \(\frac{h}{e}\)
\item \(\frac{h^2}{e^2}\)\newline
\item \(\frac{h^2}{e}\)
\end{enumerate}
\newpage
\section*{Question 16}
The speed of a wave produced in water is given by \(\nu=\lambda^a g^{b} \rho^{c}\). Where \(\lambda, g\) and \(\rho\) are wavelength of wave, acceleration due to gravity and density of water respectively. The values of \(a, b\) and \(c\) respectively, are: 
\begin{enumerate}[label=(\alph*)]
\end{enumerate}
\newpage
\section*{Question 17}
An expression for a dimensionless quantity \(P\) is given by \(P =\frac{\alpha}{\beta} \log _{ e }\left(\frac{ kt }{\beta x }\right)\); where \(\alpha\) and \(\beta\) are constants, \(x\) is distance ; \(k\) is Boltzmann constant and \(t\) is the temperature. Then the dimensions of \(\alpha\) will be:
\begin{enumerate}[label=(\alph*)]
\end{enumerate}
\newpage
\section*{Question 18}
The SI unit of a physical quantity is pascal-second. The dimensional formula of this quantity will be: 
\begin{enumerate}[label=(\alph*)]
\end{enumerate}
\newpage
\section*{Question 19}
In Vander Waals equation \(\left\lfloor P +\frac{ a }{ V ^2}\right\rfloor[ V - b ]= RT\); \(P\) is pressure, \(V\) is volume, \(R\) is universal gas constant and \(T\) is  temperature. The ratio of constants \(\frac{ a }{ b }\) is dimensionally equal to:
\begin{enumerate}[label=(\alph*)]
\item \(\frac{P}{V}\)
\item \(\frac{ V }{ P }\)
\item \(PV\)
\item \(PV ^3\)
\end{enumerate}
\newpage
\section*{Question 20}
The pitch of the screw gauge is 1 mm and there are 100 divisions on the circular scale. When nothing is put in between the jaws, the zero of the circular scale lies 8 divisions below the reference line. When a wire is placed between the jaws, the first linear scale division is clearly visible while 72 nd division on circular scale coincides with the reference line. The radius of the wire is: 
\begin{enumerate}[label=(\alph*)]
\end{enumerate}
\newpage
\section*{Question 21}
The entropy of any system is given by \(S=\alpha^2 \beta \ln \left[\frac{\mu k R}{J \beta^2}+3\right]\)
where \(\alpha\) and beta are the constants. \(mu , J, k\) and \(R\) are no. of moles, mechanical equivalent of heat, Boltzmann constant and gas constant respectively.
\(\left[\right.\) Take \(\left.S=\frac{d Q}{T}\right]\)
Choose the incorrect option from the following: 
\begin{enumerate}[label=(\alph*)]
\item \(\alpha\) and \(J\) have the same dimensions.
\item \(S, \beta, k\) and \(\mu R\) have the same dimensions.
\item \(S\) and \(\alpha\) have different dimensions.
\item \(\alpha\) and \(k\) have the same dimensions.
\end{enumerate}
\newpage
\section*{Question 22}
Dimensional formula for thermal conductivity is (here \(K\) denotes the temperature: 
\begin{enumerate}[label=(\alph*)]
\item \(M L T^{-2} K\)
\item \(M L T^{-2} K^{-2}\)
\item \(M L T^{-3} K\)
\item \(M L T^{-3} K^{-1}\)
\end{enumerate}
\newpage
\section*{Question 23}
A screw gauge has 50 divisions on its circular scale. The circular scale is 4 units ahead of the pitch scale marking, prior to use. Upon one complete rotation of the circular scale, a displacement of \(0.5 mm\) is noticed on the pitch scale. The nature of zero error involved, and the least count of the screw gauge, are respectively:
\begin{enumerate}[label=(\alph*)]
\item Negative, 2 mm
\item Positive, 10 mm
\item Positive, 0.1 mm
\item Negative, 0.1 mm
\end{enumerate}
\newpage
\section*{Question 24}
A physical quantity \(z\) depends on four observables \(a, b, c\) and \(d\), as \(z=\frac{a^2 b^{\frac{2}{3}}}{\sqrt{c} d^3}\). The percentages of error in the measurement of \(a, b, c\) and \(d\) are \(2 \%, 1.5 \%, 4 \%\) and \(2.5 \%\) respectively. The percentage of error in \(z\) is: 
\begin{enumerate}[label=(\alph*)]
\item \(12.25 \%\)
\item \(16.5 \%\)
\item \(13.5 \%\)
\item \(14.5 \%\)
\end{enumerate}
\newpage
\section*{Question 25}
The least count of the main scale of a vernier callipers is \(1 mm\). Its vernier scale is divided into 10 divisions and coincide with 9 divisions of the main scale. When jaws are touching each other, the 7th division of vernier scale coincides with a division of main scale and the zero of vernier scale is lying right side of the zero of main scale. When this vernier is used to measure length of a cylinder the zero of the vernier scale between \(3.1 cm\) and \(3.2 cm\) and 4 th VSD coincides with a main scale division. The length of the cylinder is : (VSD is vernier scale division) 
\begin{enumerate}[label=(\alph*)]
\item \(3.2 cm\)
\item \(3.21 cm\)
\item \(3.07 cm\)
\item \(2.99 cm\)
\end{enumerate}
\newpage
\section*{Question 26}
Using screw gauge of pitch 0.1 cm and 50 divisions on its circular scale, the thickness of an object is measured. It should correctly be recorded as: 
\begin{enumerate}[label=(\alph*)]
\item 2.121 cm
\item 2.124 cm
\item 2.125 cm
\item 2.123 cm
\end{enumerate}
\newpage
\section*{Question 27}
A simple pendulum is being used to determine the value of gravitational acceleration \(g\) at a certain place. The length of the pendulum is \(25.0 cm\) and a stop watch with \(1 s\) resolution measures the time taken for 40 oscillations to be \(50 s\). The accuracy in \(g\) is: 
\begin{enumerate}[label=(\alph*)]
\item \(5.40 \%\)
\item \(3.40 \%\)
\item \(4.40 \%\)
\item \(2.40 \%\)
\end{enumerate}
\newpage
\section*{Question 28}
The force of interaction between two atoms is given by \(F=\alpha \beta \exp \left(-\frac{x^2}{\alpha k T}\right)\); where \(x\) is the distance, \(k\) is the Boltzmann constant and \(T\) is temperature and alpha and \(\beta\) are two constants. The dimensions of beta is: 
\begin{enumerate}[label=(\alph*)]
\item \(M^0 L^2 T^{-4}\)
\item \(M^2 L T^{-4}\)
\item \(M L T^{-2}\)
\item \(M^2 L^2 T^{-2}\)
\end{enumerate}
\newpage
\section*{Question 29}
A student measuring the diameter of a pencil of circular cross-section with the help of a vernier scale records the following four readings 5.50 mm , 5.55 mm , 5.45 mm, 5.65 mm, The average of these four reading is 5.5375 mm and the standard deviation of the data is 0.07395 mm. The average diameter of the pencil should therefore be recorded as: 
\begin{enumerate}[label=(\alph*)]
\item \( (5.5375 \pm 0.0739) mm \)
\item \( (5.5375 \pm 0.0740) mm \)
\item \( (5.538 \pm 0.074) mm \)
\item \( (5.54 \pm 0.07) mm\)
\end{enumerate}
\newpage
\section*{Question 30}
From the following combinations of physical constants (expressed through their usual symbols) the only combination, that would have the same value in different systems of units, is: 
\begin{enumerate}[label=(\alph*)]
\item \(\frac{c h}{2 \pi \varepsilon_0^2}\)
\item \(\frac{e^2}{2 \pi \varepsilon_o G m_e{ }^2}\left(m_e=\right.\) mass of electron \()\)
\item \(\frac{\mu_0 \varepsilon_0}{c^2} \frac{G}{h e^2}\)
\item \(\frac{2 \pi \sqrt{\mu_o \varepsilon_o}}{c e^2} \frac{h}{G}\)
\end{enumerate}
\newpage
\section*{Question 31}
\(N\) divisions on the main scale of a vernier calliper coincide with \((N+1)\) divisions of the vernier scale. If each division of main scale is 'a' units, then the least count of the instrument is: 
\begin{enumerate}[label=(\alph*)]
\item \(a\)
\item \(\frac{a}{N}\)
\item \(\frac{N}{N+1} \times a\)
\item \(\frac{a}{N+1}\)
\end{enumerate}
\newpage
\section*{Question 32}
Identify the physical quantity that cannot be measured using spherometer : <ul class="Ts_solution_list ng-scope" dir="ltr" ng-if="show_single_data[10] == 1 || show_single_data[10] == 3"></ul>
\begin{enumerate}[label=(\alph*)]
\item Radius of curvature of concave surface
\item Specific rotation of liquids
\item Thickness of thin plates
\item Radius of curvature of convex surface
\end{enumerate}
\newpage
\section*{Question 33}
If 50 Vernier divisions are equal to 49 main scale divisions of a travelling microscope and one smallest reading of main scale is 0.5mm, the Vernier constant of travelling microscope is: 
\begin{enumerate}[label=(\alph*)]
\item 0.1mm
\item 0.1cm
\item 0.01cm
\item 0.01mm
\end{enumerate}
\newpage
\section*{Question 34}
The measured value of the length of a simple pendulum is 20cm with 2mm accuracy. The time for 50 oscillations was measured to be 40 seconds with 1 second resolution. From these measurements, the accuracy in the measurement of acceleration due to gravity is N%. The value of N is: 
\begin{enumerate}[label=(\alph*)]
\item 4
\item 8
\item 6
\item 5
\end{enumerate}
\newpage
\section*{Question 35}
10 divisions on the main scale of a Vernier calliper coincide with 11 divisions on the Vernier scale. If each division on the main scale is of 5 units, the least count of the instrument is : 
\begin{enumerate}[label=(\alph*)]
\item \(\frac{1}{2}\)
\item \(\frac{10}{11}\)
\item \(\frac{50}{11}\)
\item \(\frac{5}{11}\)
\end{enumerate}
\newpage
\end{document}