\documentclass{article}
                    \usepackage{amsmath}
                    \usepackage{amssymb}
                    \usepackage{graphicx}
                    \usepackage{enumitem}
                    \usepackage{longtable}
                    \title{Maths - KCET - 12 }
                    \begin{document}
                    \maketitle
                    \section*{Question 1}
If \(f(x)=x^{3}+3 x^{2}+3 x-7\), then find the value of \(\frac{d f(x)}{d x}\) at \(x=2\).
\begin{enumerate}[label=(\alph*)]
\item 23
\item 24
\item 27
\item 30
\end{enumerate}
\newpage
\section*{Question 2}
Let \(f(x)=\log x^{3}+2 x^{2}-3 x+100\), then find \(f'(3)\).
\begin{enumerate}[label=(\alph*)]
\item 12
\item 9
\item 15
\item 10
\end{enumerate}
\newpage
\section*{Question 3}
If \(f(x)=4 x^{3}-3 x^{2}+5\), then find \(f''(x)\).
\begin{enumerate}[label=(\alph*)]
\item \(12 x^{2}-6 x\)
\item \(24 x-6\)
\item \(24 x\)
\item None of these
\end{enumerate}
\newpage
\section*{Question 4}
If \(f(x)=\left\{\begin{array}{ll}\frac{\sin 3 x}{e^{2 x}-1}, & x \neq 0 \\ k-2, & x=0\end{array}\right.\) is continuous at \(x=0\), then \(k=\)?
\begin{enumerate}[label=(\alph*)]
\item \(\frac{3}{2}\)
\item \(\frac{9}{5}\)
\item \(\frac{1}{2}\)
\item \(\frac{7}{2}\)
\end{enumerate}
\newpage
\section*{Question 5}
If \(\frac{x^{2}}{a^{2}}+\frac{y^{2}}{b^{2}}=1\), then \(\frac{d y}{d x}=?\)\newline
\begin{enumerate}[label=(\alph*)]
\item \(\frac{b^{2} x}{a^{2} y}\)
\item \(-\frac{b^{2} x}{a^{2} y}\)
\item \(-\frac{b^{2} y}{a^{2} x}\)
\item \(\frac{b^{2} y}{a^{2} x}\)
\end{enumerate}
\newpage
\section*{Question 6}
The derivative of \(\cos ^{-1}\left(\frac{1}{2 x^{2}-1}\right)\) with respect to \(\sqrt{1-x^{2}}\) is:
\begin{enumerate}[label=(\alph*)]
\item \(-2 \cos ^{-1} x\)
\item \(\frac{-2}{x}\)
\item \(\frac{1}{\sqrt{1+2 x}}\)
\item None of these
\end{enumerate}
\newpage
\section*{Question 7}
If \(y=3 t^{2}-4 t-3\) and \(x=8 t+5\), find \(\frac{d y}{d x}\) at \(t=6\).
\begin{enumerate}[label=(\alph*)]
\item 4
\item 3
\item 2
\item 1
\end{enumerate}
\newpage
\section*{Question 8}
Find \(\frac{\mathrm{dy}}{\mathrm{dx}}\), if \(\mathrm{y}=\tan ^{-1}\left[\frac{8 \mathrm{x}}{1-15 \mathrm{x}^{2}}\right]\).
\begin{enumerate}[label=(\alph*)]
\item \(\frac{5}{1+25 x^{2}}-\frac{3}{1+9 x^{2}}\)
\item \(\frac{5}{1+25 x^{2}}+\frac{3}{1+9 x^{2}}\)
\item \(\frac{8}{1+25 x^{2}}\)
\item None of these
\end{enumerate}
\newpage
\section*{Question 9}
If \(y+\sin ^{-1}\left(1-x^{2}\right)=e^{x}\), then \(\frac{d y}{d x}=?\) 
\begin{enumerate}[label=(\alph*)]
\item \(e^{x}-\frac{2}{\sqrt{2-x^{2}}}\)
\item \(e^{x}+\frac{2}{\sqrt{2-x^{2}}}\)
\item \(e^{x}-\frac{1}{\sqrt{2+x^{2}}}\)
\item \(e^{x}+\frac{1}{\sqrt{2-x^{2}}}\)
\end{enumerate}
\newpage
\section*{Question 10}
Find \(\frac{\mathrm{d}^{2} (\cot ^{-1} x)}{\mathrm{dx}^{2}}\).
\begin{enumerate}[label=(\alph*)]
\item \(\frac{-2 x}{\left(1+x^{2}\right)^{2}}\)
\item \(\frac{-2}{\left(1+x^{2}\right)^{2}}\)
\item \(\frac{-1}{\left(1+x^{2}\right)^{2}}\)
\item \(\frac{2 x}{\left(1+x^{2}\right)^{2}}\)
\end{enumerate}
\newpage
\section*{Question 11}
If \(y=x^{x}\), what is \(\frac{d y}{d x}\) at \(x=1\) equal to?
\begin{enumerate}[label=(\alph*)]
\item 0
\item 1
\item -1
\item 2
\end{enumerate}
\newpage
\section*{Question 12}
If \(y=e^{x+e^{x+e^{x+\cdots \infty}}}\), then \(\frac{d y}{d x}\) is:
\begin{enumerate}[label=(\alph*)]
\item \(\frac{1+{y}}{{y}}\)
\item \(\frac{y}{1+y}\)
\item \(\frac{{y}}{1-{y}}\)
\item \(\frac{1-y}{y}\)
\end{enumerate}
\newpage
\section*{Question 13}
If \(y=\frac{12 \tan x-4 \tan ^{3} x}{9-27 \tan ^{2} x}\), then find \(\frac{d^{2} y}{d x^{2}}\).\newline
\begin{enumerate}[label=(\alph*)]
\item \(8 \sec ^{2} 3 x \tan 3 x\)
\item \(24 \sec ^{2} 3 x \tan 3 x\)
\item \((\frac{4}{3}) \sec ^{2} 3 x\)
\item \(4 \sec ^{2} 3 x\)
\end{enumerate}
\newpage
\section*{Question 14}
Find the value of the constant \(\lambda\) so that the function given below is continuous at \(x=-1\)\(f(x)=\left\{\begin{array}{cc} \frac{x^2-2 x-3}{x+1}, & x \neq-1 \\ \end{array}\right.\)
\begin{enumerate}[label=(\alph*)]
\item \(-2\)
\item \(4\)
\item \(-3\)
\item \(-4\)
\end{enumerate}
\newpage
\section*{Question 15}
For the function \(f\) given by \(f(x)=x^2-6 x+8\), the value of \(f^{\prime}(5)-3 f^{\prime}(2)\) will be equals to:
\begin{enumerate}[label=(\alph*)]
\item \(f^{\prime}(9)\)
\item \(f^{\prime}(5)\)
\item \(f^{\prime}(8)\)
\item \(f^{\prime}(2)\)
\end{enumerate}
\newpage
\section*{Question 16}
If \(f(x)=\left\{\begin{array}{cl}1, & \text { if } x \leq 3 \\ a x+b, & \text { if } 3<x<5 \\ 7, & \text { if } 5 \leq x\end{array}\right.\) Determine the values of \(a\) and \(b\) so that \(f(x)\) is continuous.
\begin{enumerate}[label=(\alph*)]
\item \(a=2, b=-5\)
\item \(a=4, b=-7\)
\item \(a=1, b=-3\)
\item \(a=3, b=-8\)
\end{enumerate}
\newpage
\section*{Question 17}
Let, \(R=\{(a, b): a, b \in Z\) and \((a+b)\) is even \(\}\), then \(R\) is:
\begin{enumerate}[label=(\alph*)]
\item Reflexive relation on Z
\item Equivalence relation on Z
\item Transitive relation on Z
\item None of the above
\end{enumerate}
\newpage
\section*{Question 18}
Let \(R\) be a relation defined as \(R=\left\{(a, b): a^{2} \geq b\right.\), where \(a\) and \(b \in Z\}\). Then, relation \(R\) is a/an:\newline
\begin{enumerate}[label=(\alph*)]
\item Reflexive and symmetric
\item Symmetric and transitive
\item Transitive and reflexive
\item Reflexive only
\end{enumerate}
\newpage
\section*{Question 19}
The relation 'has the same father as' over the set of children is:
\begin{enumerate}[label=(\alph*)]
\item Only reflexive
\item Only symmetric
\item Only transitive
\item An equivalence relation
\end{enumerate}
\newpage
\section*{Question 20}
Which one of the following is correct?
\begin{enumerate}[label=(\alph*)]
\item The function is one-one into
\item The function is many-one into
\item The function is one-one onto
\item The function is many-one onto
\end{enumerate}
\newpage
\section*{Question 21}
Which of the following functions, \(f: R \rightarrow R\) is one-one?
\begin{enumerate}[label=(\alph*)]
\item \(\mathrm{f}(\mathrm{x})=|\mathrm{x}|, \forall \mathrm{x} \in \mathrm{R}\)
\item \(f(x)=x^{2}, \forall x \in R\)
\item \(f(x)=-x, \forall x \in R\)
\item None of these
\end{enumerate}
\newpage
\section*{Question 22}
Consider the function \(f: R \rightarrow\{0,1\}\) such that: \(f(x)=\left\{\begin{array}{c}1 \text { if } x \text { is rational } \\ 0 \text { if } x \text { is irrational }\end{array}\right.\). Which one of the following is correct?
\begin{enumerate}[label=(\alph*)]
\item The function is one-one into
\item The function is many-one into
\item The function is one-one onto
\item The function is many-one onto
\end{enumerate}
\newpage
\section*{Question 23}
Let \(\mathrm{N}\) be the set of natural numbers and \(\mathrm{f}: \mathrm{N} \rightarrow \mathrm{N}\) be a function given by \(\mathrm{f}(\mathrm{x})=\mathrm{x}+1 \forall \mathrm{x} \in \mathrm{N}\). Which one of the following is correct?\newline
\begin{enumerate}[label=(\alph*)]
\item f is one - one and onto\newline
\item f is one - one but not onto
\item f is only onto
\item f is neither one - one nor onto
\end{enumerate}
\newpage
\section*{Question 24}
The function \(f(x)=x^{2}+4 x+4\) is:
\begin{enumerate}[label=(\alph*)]
\item Odd
\item Even
\item Neither odd nor even
\item Periodic
\end{enumerate}
\newpage
\section*{Question 25}
If \(f(x+1)=x^{2}-3 x+2\), then what is \(f(x)\) equal to?
\begin{enumerate}[label=(\alph*)]
\item \(x^{2}-5 x+4\)
\item \(x^{2}-5 x+6\)
\item \(x^{2}+3 x+3\)
\item \(x^{2}-3 x+1\)
\end{enumerate}
\newpage
\section*{Question 26}
If \(f(x)=8 x^{3}, g(x)=x^{\frac{1}{3}}\), then \(\operatorname{gof}(2)\) is?
\begin{enumerate}[label=(\alph*)]
\item 16
\item 4
\item 8
\item 2
\end{enumerate}
\newpage
\section*{Question 27}
If \(f(x)=x^{4}-\frac{1}{x^{4}}\), then \(f(x)+f\left(\frac{1}{x}\right)=?\)\newline
\begin{enumerate}[label=(\alph*)]
\item \(2\left(x^{4}-\frac{1}{x^{4}}\right)\)
\item \(\frac{1}{x^{4}}-x^{4}\)
\item \(\frac{1+x^{4}}{1-x^{4}}\)
\item \(0\)
\end{enumerate}
\newpage
\section*{Question 28}
If \(f: R \rightarrow R\) and \(g: R \rightarrow R\) are two mappings defined as \(f(x)=3 x\) and \(g(x)=3 x^{2}+9\), then the value of \((f+g)(2)\) is:
\begin{enumerate}[label=(\alph*)]
\item 21
\item 23
\item 27
\item 20
\end{enumerate}
\newpage
\section*{Question 29}
If \(f(x)=\frac{x+1}{x-1}, x \neq 1\), then \(f\{f(x)\}=?\)
\begin{enumerate}[label=(\alph*)]
\item \(-\frac{1}{\mathrm{x}}\)
\item \(\frac{1}{x^{2}}\)
\item \(x\)
\item \(\frac{2}{x}\)
\end{enumerate}
\newpage
\section*{Question 30}
Let \({f}: {R} \rightarrow {R}\) be defined by \({f}({x})=2 {x}+6\) which is a bijective mapping, then \({f}^{-1}({x})\) is given by,\newline
\begin{enumerate}[label=(\alph*)]
\item \(6 x+2\)
\item \(x-3\)
\item \(2 x+6\)
\item \(\frac{x}{2}-3\)
\end{enumerate}
\newpage
\section*{Question 31}
If A = {1, 2}, B = {1, 2, 3, 4}, C = {5, 6} and D = {5, 6, 7, 8}. find which one is true.\newline
\begin{enumerate}[label=(\alph*)]
\item A × (B ∩ C) = (A ∩  B) × C
\item (A ∩ B) × D = A × (B ∩ D)
\item (B ∩ C) × D = B × (C ∩ D)
\item A × C is a subset of B × D
\end{enumerate}
\newpage
\section*{Question 32}
If f = {(9, 5), (11, 5), (13, 7} and g = {(4, 9), (5, 11), (6, 13)}. Write the function f o g?\newline
\begin{enumerate}[label=(\alph*)]
\item {(4, 5), (5, 5), (13, 7)}
\item {(4, 5), (5, 5), (6, 7)}
\item {(4, 5), (5, 5), (6, 6)}\newline
\item {(4, 5), (6, 5), (6, 7)}
\end{enumerate}
\newpage
\section*{Question 33}
If f ∶ R → R and g ∶ R → R are two mappings defined as f(x) = 2x and g(x) = x$^{2}$ + 2, then the value of (f + g) (2) is:\newline
\begin{enumerate}[label=(\alph*)]
\item 8
\item 10
\item 12
\item 24
\end{enumerate}
\newpage
\section*{Question 34}
Let \(f: R \rightarrow R\) be defined by\(f(x)=\left\{\begin{array}{cc}2 x ; & x>3 \\ x^{2} ; & 1<x \leq 3 \\ 3 x & x \leq 1\end{array}\right.\)Then \(f(-1)+f(2)+f(4)\) is:
\begin{enumerate}[label=(\alph*)]
\item 9
\item 14
\item 5
\item 10
\end{enumerate}
\newpage
\section*{Question 35}
Find the domain and range of following function:<math xmlns="http://www.w3.org/1998/Math/MathML"><mi>f</mi><mo>(</mo><mi>x</mi><mo>)</mo><mo>=</mo><msqrt><mn>25</mn><mo>-</mo><msup><mi>x</mi><mn>2</mn></msup></msqrt></math>\newline
\begin{enumerate}[label=(\alph*)]
\item [-5,5], [0,5]
\item [- 3,3], [- 3,3]
\item [-4,4], [4,0]
\item None of the above
\end{enumerate}
\newpage
\section*{Question 36}
If \(f(x)=x\) and \(g(x)=\cos x\) then find the value of \(g\)o\(f(\frac{\pi}{3}) ?\)
\begin{enumerate}[label=(\alph*)]
\item 1
\item - 1
\item \(\frac{1}{2}\)
\item \(-\frac{1}{2}\)
\end{enumerate}
\newpage
\section*{Question 37}
Let A = {1, 2, 3}. The total number of distinct relations that can be defined over A is:\newline
\begin{enumerate}[label=(\alph*)]
\item 2$^{9}$
\item 2$^{3}$
\item 2$^{5}$\newline
\item 2$^{2}$\newline
\end{enumerate}
\newpage
\section*{Question 38}
If \(f: R \rightarrow R\) is given by \(f(x)=x^2-3\), then \(f^{-1}\) is given by:\newline
\begin{enumerate}[label=(\alph*)]
\item \(\sqrt{x+3}\)
\item \(\sqrt{x}+3\)
\item \(x+\sqrt{3}\)
\item None of these
\end{enumerate}
\newpage
\section*{Question 39}
If \(f(x)=\frac{2 x}{1+x^{2}}\), then find the value of \(f(\tan \theta)\).
\begin{enumerate}[label=(\alph*)]
\item \(sin 2 \theta\)
\item \(cos 2 \theta\)
\item \(tan 2 \theta\)
\item \(sin \theta\)
\end{enumerate}
\newpage
\section*{Question 40}
\(\tan \left(2 \tan ^{-1}(\cos x)\right)\) is equal to:
\begin{enumerate}[label=(\alph*)]
\item \(2 \tan x \operatorname{cosec} x\)
\item \(2 \cot x \operatorname{cosec} x\)
\item \(2 \cos x \operatorname{cosec} x\)
\item \(\cot x \operatorname{cosec} x\)
\end{enumerate}
\newpage
\section*{Question 41}
\(\sec ^{-1}\left[\frac{x^{2}+1}{x^{2}-1}\right]=?\)
\begin{enumerate}[label=(\alph*)]
\item \(2 \tan ^{-1} x\)
\item \(2 x^{2}\)
\item \(2 \cot ^{-1} x\)
\item \(x^{2}\)
\end{enumerate}
\newpage
\section*{Question 42}
Find the value of \(\cos ^{-1}\left(4 x^3-3 x\right), x \in[-1,1]\).
\begin{enumerate}[label=(\alph*)]
\item \(3 \cos ^{-1} x\)
\item \(\cos ^{-1} 3 x\)
\item \(\cos ^{-1} x\)
\item None of these
\end{enumerate}
\newpage
\section*{Question 43}
Find \(\cot ^{-1} \frac{1}{3}-2 \tan ^{-1} \frac{2}{3}\).
\begin{enumerate}[label=(\alph*)]
\item 1
\item \(-1\)
\item \(\cot ^{-1} \frac{41}{3}\)
\item \(\tan ^{-1} \frac{41}{3}\)
\end{enumerate}
\newpage
\section*{Question 44}
Find the value of \(\cot \left(\tan ^{-1} \alpha+\cot ^{-1} \alpha\right)\).
\begin{enumerate}[label=(\alph*)]
\item \(0\)
\item \(-1\)
\item \(\sqrt{2}\)
\item \(-\sqrt{2}\)
\end{enumerate}
\newpage
\section*{Question 45}
What is \(\cos ^{-1}\left(\frac{1-x^{2}}{1+x^{2}}\right)\) equal to?
\begin{enumerate}[label=(\alph*)]
\item \(\sin ^{-1} x\)\newline
\item \(2 \cot ^{-1} x\)
\item \(2 \tan ^{-1} x\)
\item \(\tan ^{-1} x\)
\end{enumerate}
\newpage
\section*{Question 46}
Find the value of \(2 \tan ^{-1} \frac{1}{5}+\tan ^{-1} \frac{1}{8}\).\newline
\begin{enumerate}[label=(\alph*)]
\item \(\tan ^{1} \frac{4}{7}\)
\item \(\tan ^{1} \frac{6}{7}\)
\item \(\tan ^{-1} \frac{4}{7}\)
\item \(\tan ^{-1} \frac{3}{7}\)
\end{enumerate}
\newpage
\section*{Question 47}
If \(\sin ^{-1} {x}+\cos ^{-1} {y}=\frac{2 \pi}{5}\), then \(\cos ^{-1} {x}+\sin ^{-1} {y}\) is equal to :
\begin{enumerate}[label=(\alph*)]
\item \(\frac{2 \pi}{5}\)
\item \(\frac{3 \pi}{5}\)
\item \(\frac{4 \pi}{5}\)
\item \(\frac{3 \pi}{10}\)
\end{enumerate}
\newpage
\section*{Question 48}
If \(\tan ^{-1} \frac{x-1}{x-2}+\tan ^{-1} \frac{x+1}{x+2}=\frac{\pi}{4},\) then find the value of \(x\).
\begin{enumerate}[label=(\alph*)]
\item \(\pm \frac{\sqrt{2}}{1}\)
\item \(\pm \frac{1}{\sqrt{2}}\)
\item \(\pm \frac{1}{2}\)
\item None of these
\end{enumerate}
\newpage
\section*{Question 49}
\(\tan ^{-1} x+\cot ^{-1} x=\frac{\pi}{2}\) holds, when:
\begin{enumerate}[label=(\alph*)]
\item \(x \in R\)
\item \(x \in R-(-1,1)\) only
\item \(x \in R-\{0\}\) only
\item \(x \in R-[-1,1]\) only
\end{enumerate}
\newpage
\section*{Question 50}
Write the principal value of \(\cos ^{-1}\left(\frac{1}{2}\right)-2 \sin ^{-1}\left(-\frac{1}{2}\right)\).
\begin{enumerate}[label=(\alph*)]
\item \(2 n \pi+\frac{2 \pi}{3}\)
\item \(2 n \pi+\frac{4 \pi}{3}\)
\item \(2 n \pi+\frac{7 \pi}{3}\)
\item \(2 n \pi+\frac{2 \pi}{4}\)
\end{enumerate}
\newpage
\section*{Question 51}
Find the value of \(\tan ^{-1}(1)+\cos ^{-1}\left(-\frac{1}{2}\right)+\sin ^{-1}\left(-\frac{1}{2}\right)\).
\begin{enumerate}[label=(\alph*)]
\item \(\frac{12 \pi}{15}\)
\item \(\frac{18 \pi}{12}\)
\item \(\frac{15 \pi}{12}\)
\item \(\frac{25 \pi}{12}\)
\end{enumerate}
\newpage
\section*{Question 52}
The value of the expression \(\sin ^{-1}\left(\sin \frac{22 \pi}{7}\right)+\cos ^{-1}\left(\cos \frac{5 \pi}{3}\right)+\tan ^{-1}\left(\tan \frac{5 \pi}{7}\right)+\) \(\sin ^{-1}(\cos 2)\) is:
\begin{enumerate}[label=(\alph*)]
\item \(\frac{17 \pi}{42}-2\)
\item \(-2\)
\item \(\frac{-\pi}{21}-2\)
\item None of these
\end{enumerate}
\newpage
\section*{Question 53}
Find the value of \(\sin ^{-1}\left(\frac{3}{5}\right)+\sin ^{-1}\left(\frac{4}{5}\right)+\cos ^{-1}\left(\frac{\sqrt{3}}{2}\right)\).\newline
\begin{enumerate}[label=(\alph*)]
\item \(\frac{\pi}{3}\)
\item \(\frac{2\pi}{3}\)
\item \(\frac{4\pi}{3}\)
\item \(\frac{\pi}{4}\)
\end{enumerate}
\newpage
\section*{Question 54}
If \(\tan ^{-1} x+\tan ^{-1} 3=\tan ^{-1} 8\) then \(x=\) ?
\begin{enumerate}[label=(\alph*)]
\item \(\frac{1}{3}\)
\item \(\frac{1}{5}\)
\item 3
\item 5
\end{enumerate}
\newpage
\section*{Question 55}
If \(\cos ^{-1} x+\cos ^{-1} y+\cos ^{-1} z=\pi\), then:
\begin{enumerate}[label=(\alph*)]
\item \( x^2+y^2+z^2+x y z=0\)
\item \(x ^2+ y ^2+ z ^2+2 xyz =0\)
\item  \( x^2+y^2+z^2+x y z=1\)
\item \(x ^2+ y ^2+ z ^2+2 xyz =1\)
\end{enumerate}
\newpage
\section*{Question 56}
The principal value of \(\sin ^{-1} x\) lies in the interval
\begin{enumerate}[label=(\alph*)]
\item \(\left(-\frac{\pi}{2}, \frac{\pi}{2}\right)\)
\item \(\left[-\frac{\pi}{2}, \frac{\pi}{2}\right]\)
\item \(\left[0, \frac{\pi}{2}\right]\)
\item \([0, \pi]\)
\end{enumerate}
\newpage
\section*{Question 57}
If \(\sin ^{-1} \frac{3}{x}+\sin ^{-1} \frac{9}{x}=\frac{\pi}{2}\) then what is the value of \(x\) ?\newline
\begin{enumerate}[label=(\alph*)]
\item 1\newline
\item \(3 \sqrt{10}\)\newline
\item 0\newline
\item \(\sqrt{10}\)\newline
\end{enumerate}
\newpage
\section*{Question 58}
The value of \(\tan ^{-1}\left(\frac{1}{2}\right)+\tan ^{-1}\left(\frac{1}{3}\right)\) is:
\begin{enumerate}[label=(\alph*)]
\item \(\pi\)
\item \(\frac{\pi}{2}\)
\item \(\frac{\pi}{3}\)
\item \(\frac{\pi}{4}\)
\end{enumerate}
\newpage
\section*{Question 59}
If \(\tan ^{-1} x+\tan ^{-1} y=\frac{4 \pi}{5}\), then \(\cot ^{-1} x+\cot ^{-1} y\) equals:
\begin{enumerate}[label=(\alph*)]
\item \(\frac{\pi}{5}\)
\item \(\frac{2 \pi}{5}\)
\item \(\frac{3 \pi}{5}\)
\item \(\pi\)
\end{enumerate}
\newpage
\section*{Question 60}
Find the principal value of \(\sin ^{-1}\left(\frac{-\sqrt{3}}{2}\right)\).
\begin{enumerate}[label=(\alph*)]
\item \(-45^{\circ}\)
\item \(-60^{\circ}\)
\item \(-30^{\circ}\)
\item \(120^{\circ}\)
\end{enumerate}
\newpage
\end{document}