\documentclass{article}
                    \usepackage{amsmath}
                    \usepackage{amssymb}
                    \usepackage{graphicx}
                    \usepackage{enumitem}
                    \usepackage{longtable}
                    \title{Halo alkienes - JEE - Chem - 12 }
                    \begin{document}
                    \maketitle
                    \section*{Question 1}
<style>.fm-math,fmath{font-family:STIXGeneral,'DejaVu Serif','DejaVu Sans',Times,OpenSymbol,'Standard Symbols L',serif;line-height:1.2}.fm-math mtext,fmath mtext{line-height:normal}.fm-mo,.ma-sans-serif,fmath mi[mathvariant*=sans-serif],fmath mn[mathvariant*=sans-serif],fmath mo,fmath ms[mathvariant*=sans-serif],fmath mtext[mathvariant*=sans-serif]{font-family:STIXGeneral,'DejaVu Sans','DejaVu Serif','Arial Unicode MS','Lucida Grande',Times,OpenSymbol,'Standard Symbols L',sans-serif}.fm-mo-Luc{font-family:STIXGeneral,'DejaVu Sans','DejaVu Serif','Lucida Grande','Arial Unicode MS',Times,OpenSymbol,'Standard Symbols L',sans-serif}.questionsfont{font-weight:200;font-family:Arial, sans-serif, STIXGeneral,'DejaVu Sans','DejaVu Serif','Lucida Grande','Arial Unicode MS',Times,OpenSymbol,'Standard Symbols L',sans-serif!important}.fm-separator{padding:0 .56ex 0 0}.fm-infix-loose{padding:0 .56ex}.fm-infix{padding:0 .44ex}.fm-prefix{padding:0 .33ex 0 0}.fm-postfix{padding:0 0 0 .33ex}.fm-prefix-tight{padding:0 .11ex 0 0}.fm-postfix-tight{padding:0 0 0 .11ex}.fm-quantifier{padding:0 .11ex 0 .22ex}.ma-non-marking{display:none}.fm-vert,fmath menclose,menclose.fm-menclose{display:inline-block}.fm-large-op{font-size:1.3em}.fm-inline .fm-large-op{font-size:1em}fmath mrow{white-space:nowrap}.fm-vert{vertical-align:middle}fmath table,fmath tbody,fmath td,fmath tr{border:0!important;padding:0!important;margin:0!important;outline:0!important}fmath table{border-collapse:collapse!important;text-align:center!important;table-layout:auto!important;float:none!important}.fm-frac{padding:0 1px!important}td.fm-den-frac{border-top:solid thin!important}.fm-root{font-size:.6em}.fm-radicand{padding:0 1px 0 0;border-top:solid;margin-top:.1em}.fm-script{font-size:.71em}.fm-script .fm-script .fm-script{font-size:1em}td.fm-underover-base{line-height:1!important}td.fm-mtd{padding:.5ex .4em!important;vertical-align:baseline!important}fmath mphantom{visibility:hidden}fmath menclose[notation=top],menclose.fm-menclose[notation=top]{border-top:solid thin}fmath menclose[notation=right],menclose.fm-menclose[notation=right]{border-right:solid thin}fmath menclose[notation=bottom],menclose.fm-menclose[notation=bottom]{border-bottom:solid thin}fmath menclose[notation=left],menclose.fm-menclose[notation=left]{border-left:solid thin}fmath menclose[notation=box],menclose.fm-menclose[notation=box]{border:thin solid}fmath none{display:none}</style> Molar conductance of an electrolyte increase with dilution according to the equation:\newline<fmath class="fm-inline"><mrow><msub><mi class="fm-mi-length-1" mathvariant="italic">Λ</mi><mi class="fm-mi-length-1 ma-upright" mathvariant="normal" style="padding-right: 0px;">m</mi></msub><mo class="fm-infix-loose">=</mo><mrow><msubsup><mi class="fm-mi-length-1" mathvariant="italic">Λ</mi><mi class="fm-mi-length-1 ma-upright" mathvariant="normal" style="padding-right: 0px;">m</mi><mo>∘</mo></msubsup><mo class="fm-infix">−</mo><mrow><mi class="fm-mi-length-1 ma-upright" mathvariant="normal" style="padding-right: 0px;">A</mi><mrow mtagname="msqrt"><mo class="fm-radic">√</mo><mi class="fm-mi-length-1 ma-upright" mathvariant="normal" style="padding-right: 0px;">c</mi></mrow></mrow></mrow></mrow></fmath>\newlineWhich of the following statements are true?\newline(A) This equation applies to both strong and weak electrolytes.\newline(B) Value of the constant A depends upon the nature of the solvent.\newline(C) Value of constant <fmath class="fm-inline"><mi class="fm-mi-length-1 ma-upright" mathvariant="normal" style="padding-right: 0px;">A</mi></fmath> is same for both <fmath class="fm-inline"><msub><mi class="ma-repel-adj" mathvariant="normal">BaCl</mi><mn>2</mn></msub></fmath> and <fmath class="fm-inline"><msub><mi class="ma-repel-adj" mathvariant="normal">MgSO</mi><mn>4</mn></msub></fmath>\newline(D) Value of constant <fmath class="fm-inline"><mi class="fm-mi-length-1 ma-upright" mathvariant="normal" style="padding-right: 0px;">A</mi></fmath> is same for both <fmath class="fm-inline"><msub><mi class="ma-repel-adj" mathvariant="normal">BaCl</mi><mn>2</mn></msub></fmath> and <fmath class="fm-inline"><mrow class="ma-repel-adj"><mi class="ma-repel-adj" mathvariant="normal">Mg</mi><msub><mrow><mo class="fm-mo-Luc">(</mo><mi class="ma-repel-adj" mathvariant="normal">OH</mi><mo class="fm-mo-Luc">)</mo></mrow><mn>2</mn></msub></mrow></fmath>\newlineChoose the most appropriate answer from the options given below:\newline 
\begin{enumerate}[label=(\alph*)]
\item  (A) and (B) only
\item  (A), (B) and (C) only
\item  (B) and (C) only
\item  (B) and (D) only
\end{enumerate}
\newpage
\section*{Question 2}
<style>.fm-math,fmath{font-family:STIXGeneral,'DejaVu Serif','DejaVu Sans',Times,OpenSymbol,'Standard Symbols L',serif;line-height:1.2}.fm-math mtext,fmath mtext{line-height:normal}.fm-mo,.ma-sans-serif,fmath mi[mathvariant*=sans-serif],fmath mn[mathvariant*=sans-serif],fmath mo,fmath ms[mathvariant*=sans-serif],fmath mtext[mathvariant*=sans-serif]{font-family:STIXGeneral,'DejaVu Sans','DejaVu Serif','Arial Unicode MS','Lucida Grande',Times,OpenSymbol,'Standard Symbols L',sans-serif}.fm-mo-Luc{font-family:STIXGeneral,'DejaVu Sans','DejaVu Serif','Lucida Grande','Arial Unicode MS',Times,OpenSymbol,'Standard Symbols L',sans-serif}.questionsfont{font-weight:200;font-family:Arial, sans-serif, STIXGeneral,'DejaVu Sans','DejaVu Serif','Lucida Grande','Arial Unicode MS',Times,OpenSymbol,'Standard Symbols L',sans-serif!important}.fm-separator{padding:0 .56ex 0 0}.fm-infix-loose{padding:0 .56ex}.fm-infix{padding:0 .44ex}.fm-prefix{padding:0 .33ex 0 0}.fm-postfix{padding:0 0 0 .33ex}.fm-prefix-tight{padding:0 .11ex 0 0}.fm-postfix-tight{padding:0 0 0 .11ex}.fm-quantifier{padding:0 .11ex 0 .22ex}.ma-non-marking{display:none}.fm-vert,fmath menclose,menclose.fm-menclose{display:inline-block}.fm-large-op{font-size:1.3em}.fm-inline .fm-large-op{font-size:1em}fmath mrow{white-space:nowrap}.fm-vert{vertical-align:middle}fmath table,fmath tbody,fmath td,fmath tr{border:0!important;padding:0!important;margin:0!important;outline:0!important}fmath table{border-collapse:collapse!important;text-align:center!important;table-layout:auto!important;float:none!important}.fm-frac{padding:0 1px!important}td.fm-den-frac{border-top:solid thin!important}.fm-root{font-size:.6em}.fm-radicand{padding:0 1px 0 0;border-top:solid;margin-top:.1em}.fm-script{font-size:.71em}.fm-script .fm-script .fm-script{font-size:1em}td.fm-underover-base{line-height:1!important}td.fm-mtd{padding:.5ex .4em!important;vertical-align:baseline!important}fmath mphantom{visibility:hidden}fmath menclose[notation=top],menclose.fm-menclose[notation=top]{border-top:solid thin}fmath menclose[notation=right],menclose.fm-menclose[notation=right]{border-right:solid thin}fmath menclose[notation=bottom],menclose.fm-menclose[notation=bottom]{border-bottom:solid thin}fmath menclose[notation=left],menclose.fm-menclose[notation=left]{border-left:solid thin}fmath menclose[notation=box],menclose.fm-menclose[notation=box]{border:thin solid}fmath none{display:none}</style> The conductivity of centimolar solution of <fmath class="fm-inline"><mi class="ma-repel-adj" mathvariant="normal">KCl</mi></fmath> at <fmath class="fm-inline"><mrow><msup><mn>25</mn><mo>∘</mo></msup><mi class="fm-mi-length-1 ma-upright" mathvariant="normal" style="padding-right: 0px;">C</mi></mrow></fmath> is <fmath class="fm-inline"><mrow><mrow><mn>0.0210</mn><msup><mi class="ma-repel-adj" mathvariant="normal">ohm</mi><mrow><mo class="fm-prefix-tight">−</mo><mn>1</mn></mrow></msup></mrow><msup><mi class="ma-repel-adj" mathvariant="normal">cm</mi><mrow><mo class="fm-prefix-tight">−</mo><mn>1</mn></mrow></msup></mrow></fmath> and the resistance of the cell containing the solution at <fmath class="fm-inline"><mrow><msup><mn>25</mn><mo>∘</mo></msup><mi class="fm-mi-length-1 ma-upright" mathvariant="normal" style="padding-right: 0px;">C</mi></mrow></fmath> is <fmath class="fm-inline"><mrow class="ma-repel-adj"><mn>60</mn><mspace style="margin-right: 0.17em; padding-right: 0.001em; visibility: hidden;" width=".17em">‌</mspace><mi class="ma-repel-adj" mathvariant="normal">ohm</mi></mrow></fmath>. The value of cell constant is\newline 
\begin{enumerate}[label=(\alph*)]
\item  <fmath class="fm-inline"><mrow><mn>3.28</mn><msup><mi class="ma-repel-adj" mathvariant="normal">cm</mi><mrow><mo class="fm-prefix-tight">−</mo><mn>1</mn></mrow></msup></mrow></fmath>
\item  <fmath class="fm-inline"><mrow><mn>1.26</mn><msup><mi class="ma-repel-adj" mathvariant="normal">cm</mi><mrow><mo class="fm-prefix-tight">−</mo><mn>1</mn></mrow></msup></mrow></fmath>
\item  <fmath class="fm-inline"><mrow><mn>3.34</mn><msup><mi class="ma-repel-adj" mathvariant="normal">cm</mi><mrow><mo class="fm-prefix-tight">−</mo><mn>1</mn></mrow></msup></mrow></fmath>
\item  <fmath class="fm-inline"><mrow><mn>1.34</mn><msup><mi class="ma-repel-adj" mathvariant="normal">cm</mi><mrow><mo class="fm-prefix-tight">−</mo><mn>1</mn></mrow></msup></mrow></fmath>
\end{enumerate}
\newpage
\section*{Question 3}
<style>.fm-math,fmath{font-family:STIXGeneral,'DejaVu Serif','DejaVu Sans',Times,OpenSymbol,'Standard Symbols L',serif;line-height:1.2}.fm-math mtext,fmath mtext{line-height:normal}.fm-mo,.ma-sans-serif,fmath mi[mathvariant*=sans-serif],fmath mn[mathvariant*=sans-serif],fmath mo,fmath ms[mathvariant*=sans-serif],fmath mtext[mathvariant*=sans-serif]{font-family:STIXGeneral,'DejaVu Sans','DejaVu Serif','Arial Unicode MS','Lucida Grande',Times,OpenSymbol,'Standard Symbols L',sans-serif}.fm-mo-Luc{font-family:STIXGeneral,'DejaVu Sans','DejaVu Serif','Lucida Grande','Arial Unicode MS',Times,OpenSymbol,'Standard Symbols L',sans-serif}.questionsfont{font-weight:200;font-family:Arial, sans-serif, STIXGeneral,'DejaVu Sans','DejaVu Serif','Lucida Grande','Arial Unicode MS',Times,OpenSymbol,'Standard Symbols L',sans-serif!important}.fm-separator{padding:0 .56ex 0 0}.fm-infix-loose{padding:0 .56ex}.fm-infix{padding:0 .44ex}.fm-prefix{padding:0 .33ex 0 0}.fm-postfix{padding:0 0 0 .33ex}.fm-prefix-tight{padding:0 .11ex 0 0}.fm-postfix-tight{padding:0 0 0 .11ex}.fm-quantifier{padding:0 .11ex 0 .22ex}.ma-non-marking{display:none}.fm-vert,fmath menclose,menclose.fm-menclose{display:inline-block}.fm-large-op{font-size:1.3em}.fm-inline .fm-large-op{font-size:1em}fmath mrow{white-space:nowrap}.fm-vert{vertical-align:middle}fmath table,fmath tbody,fmath td,fmath tr{border:0!important;padding:0!important;margin:0!important;outline:0!important}fmath table{border-collapse:collapse!important;text-align:center!important;table-layout:auto!important;float:none!important}.fm-frac{padding:0 1px!important}td.fm-den-frac{border-top:solid thin!important}.fm-root{font-size:.6em}.fm-radicand{padding:0 1px 0 0;border-top:solid;margin-top:.1em}.fm-script{font-size:.71em}.fm-script .fm-script .fm-script{font-size:1em}td.fm-underover-base{line-height:1!important}td.fm-mtd{padding:.5ex .4em!important;vertical-align:baseline!important}fmath mphantom{visibility:hidden}fmath menclose[notation=top],menclose.fm-menclose[notation=top]{border-top:solid thin}fmath menclose[notation=right],menclose.fm-menclose[notation=right]{border-right:solid thin}fmath menclose[notation=bottom],menclose.fm-menclose[notation=bottom]{border-bottom:solid thin}fmath menclose[notation=left],menclose.fm-menclose[notation=left]{border-left:solid thin}fmath menclose[notation=box],menclose.fm-menclose[notation=box]{border:thin solid}fmath none{display:none}</style> At <fmath class="fm-inline"><mrow><mn>298</mn><mi class="fm-mi-length-1" mathvariant="italic" style="padding-right: 0.44ex;">K</mi></mrow></fmath>, the standard electrode potentials of <fmath class="fm-inline"><mrow><mrow><mrow><msup><mrow><mi class="fm-mi-length-1" mathvariant="italic">C</mi><mi class="fm-mi-length-1" mathvariant="italic">u</mi></mrow><mrow><mn>2</mn><mo class="fm-postfix-tight">+</mo></mrow></msup><mo class="fm-infix-loose">∕</mo><mrow><mi class="fm-mi-length-1" mathvariant="italic">C</mi><mi class="fm-mi-length-1" mathvariant="italic">u</mi></mrow></mrow><mo class="fm-separator">,</mo><mrow><msup><mrow><mi class="fm-mi-length-1" mathvariant="italic" style="padding-right: 0.44ex;">Z</mi><mi class="fm-mi-length-1" mathvariant="italic">n</mi></mrow><mrow><mn>2</mn><mo class="fm-postfix-tight">+</mo></mrow></msup><mo class="fm-infix-loose">∕</mo><mrow><mi class="fm-mi-length-1" mathvariant="italic" style="padding-right: 0.44ex;">Z</mi><mi class="fm-mi-length-1" mathvariant="italic">n</mi></mrow></mrow></mrow><mo class="fm-separator">,</mo><mrow><msup><mrow><mi class="fm-mi-length-1" mathvariant="italic" style="padding-right: 0.44ex;">F</mi><mi class="fm-mi-length-1" mathvariant="italic">e</mi></mrow><mrow><mn>2</mn><mo class="fm-postfix-tight">+</mo></mrow></msup><mo class="fm-infix-loose">∕</mo><mrow><mi class="fm-mi-length-1" mathvariant="italic" style="padding-right: 0.44ex;">F</mi><mi class="fm-mi-length-1" mathvariant="italic">e</mi></mrow></mrow></mrow></fmath> and <fmath class="fm-inline"><mrow><msup><mrow><mi class="fm-mi-length-1" mathvariant="italic">A</mi><mi class="fm-mi-length-1" mathvariant="italic">g</mi></mrow><mo>+</mo></msup><mo class="fm-infix-loose">∕</mo><mrow><mi class="fm-mi-length-1" mathvariant="italic">A</mi><mi class="fm-mi-length-1" mathvariant="italic">g</mi></mrow></mrow></fmath> are <fmath class="fm-inline"><mrow><mn>0.34</mn><mi class="fm-mi-length-1" mathvariant="italic" style="padding-right: 0.44ex;">V</mi></mrow></fmath>, <fmath class="fm-inline"><mrow><mrow><mo class="fm-prefix-tight">−</mo><mrow><mn>0.76</mn><mi class="fm-mi-length-1" mathvariant="italic" style="padding-right: 0.44ex;">V</mi></mrow></mrow><mo class="fm-separator">,</mo><mrow><mo class="fm-prefix-tight">−</mo><mrow><mn>0.44</mn><mi class="fm-mi-length-1" mathvariant="italic" style="padding-right: 0.44ex;">V</mi></mrow></mrow></mrow></fmath> and <fmath class="fm-inline"><mrow><mn>0.80</mn><mi class="fm-mi-length-1" mathvariant="italic" style="padding-right: 0.44ex;">V</mi></mrow></fmath>, respectively. \newline On the basis of standard electrode potential, predict which of the following reaction cannot occur? 
\begin{enumerate}[label=(\alph*)]
\item  <fmath class="fm-inline"><mrow><mrow><mrow><msub><mrow><mrow><mrow><mi class="fm-mi-length-1" mathvariant="italic">C</mi><mi class="fm-mi-length-1" mathvariant="italic">u</mi></mrow><mi class="fm-mi-length-1" mathvariant="italic">S</mi></mrow><mi class="fm-mi-length-1" mathvariant="italic">O</mi></mrow><mn>4</mn></msub><mrow><mo class="fm-mo-Luc">(</mo><mrow><mi class="fm-mi-length-1" mathvariant="italic">a</mi><mi class="fm-mi-length-1" mathvariant="italic">q</mi></mrow><mo class="fm-mo-Luc">)</mo></mrow></mrow><mo class="fm-infix">+</mo><mrow><mrow><mi class="fm-mi-length-1" mathvariant="italic" style="padding-right: 0.44ex;">Z</mi><mi class="fm-mi-length-1" mathvariant="italic">n</mi></mrow><mrow><mo class="fm-mo-Luc">(</mo><mi class="fm-mi-length-1" mathvariant="italic">s</mi><mo class="fm-mo-Luc">)</mo></mrow></mrow></mrow><mo class="fm-infix-loose">→</mo><mrow><mrow><msub><mrow><mrow><mrow><mi class="fm-mi-length-1" mathvariant="italic" style="padding-right: 0.44ex;">Z</mi><mi class="fm-mi-length-1" mathvariant="italic">n</mi></mrow><mi class="fm-mi-length-1" mathvariant="italic">S</mi></mrow><mi class="fm-mi-length-1" mathvariant="italic">O</mi></mrow><mn>4</mn></msub><mrow><mo class="fm-mo-Luc">(</mo><mrow><mi class="fm-mi-length-1" mathvariant="italic">a</mi><mi class="fm-mi-length-1" mathvariant="italic">q</mi></mrow><mo class="fm-mo-Luc">)</mo></mrow></mrow><mo class="fm-infix">+</mo><mrow><mrow><mi class="fm-mi-length-1" mathvariant="italic">C</mi><mi class="fm-mi-length-1" mathvariant="italic">u</mi></mrow><mrow><mo class="fm-mo-Luc">(</mo><mi class="fm-mi-length-1" mathvariant="italic">s</mi><mo class="fm-mo-Luc">)</mo></mrow></mrow></mrow></mrow></fmath> 
\item  <fmath class="fm-inline"><mrow><mrow><mrow><msub><mrow><mrow><mrow><mi class="fm-mi-length-1" mathvariant="italic">C</mi><mi class="fm-mi-length-1" mathvariant="italic">u</mi></mrow><mi class="fm-mi-length-1" mathvariant="italic">S</mi></mrow><mi class="fm-mi-length-1" mathvariant="italic">O</mi></mrow><mn>4</mn></msub><mrow><mo class="fm-mo-Luc">(</mo><mrow><mi class="fm-mi-length-1" mathvariant="italic">a</mi><mi class="fm-mi-length-1" mathvariant="italic">q</mi></mrow><mo class="fm-mo-Luc">)</mo></mrow></mrow><mo class="fm-infix">+</mo><mrow><mrow><mi class="fm-mi-length-1" mathvariant="italic" style="padding-right: 0.44ex;">F</mi><mi class="fm-mi-length-1" mathvariant="italic">e</mi></mrow><mrow><mo class="fm-mo-Luc">(</mo><mi class="fm-mi-length-1" mathvariant="italic">s</mi><mo class="fm-mo-Luc">)</mo></mrow></mrow></mrow><mo class="fm-infix-loose">→</mo><mrow><mrow><msub><mrow><mrow><mrow><mi class="fm-mi-length-1" mathvariant="italic" style="padding-right: 0.44ex;">F</mi><mi class="fm-mi-length-1" mathvariant="italic">e</mi></mrow><mi class="fm-mi-length-1" mathvariant="italic">S</mi></mrow><mi class="fm-mi-length-1" mathvariant="italic">O</mi></mrow><mn>4</mn></msub><mrow><mo class="fm-mo-Luc">(</mo><mrow><mi class="fm-mi-length-1" mathvariant="italic">a</mi><mi class="fm-mi-length-1" mathvariant="italic">q</mi></mrow><mo class="fm-mo-Luc">)</mo></mrow></mrow><mo class="fm-infix">+</mo><mrow><mrow><mi class="fm-mi-length-1" mathvariant="italic">C</mi><mi class="fm-mi-length-1" mathvariant="italic">u</mi></mrow><mrow><mo class="fm-mo-Luc">(</mo><mi class="fm-mi-length-1" mathvariant="italic">s</mi><mo class="fm-mo-Luc">)</mo></mrow></mrow></mrow></mrow></fmath> 
\item  <fmath class="fm-inline"><mrow><mrow><mrow><msub><mrow><mrow><mrow><mi class="fm-mi-length-1" mathvariant="italic" style="padding-right: 0.44ex;">F</mi><mi class="fm-mi-length-1" mathvariant="italic">e</mi></mrow><mi class="fm-mi-length-1" mathvariant="italic">S</mi></mrow><mi class="fm-mi-length-1" mathvariant="italic">O</mi></mrow><mn>4</mn></msub><mrow><mo class="fm-mo-Luc">(</mo><mrow><mi class="fm-mi-length-1" mathvariant="italic">a</mi><mi class="fm-mi-length-1" mathvariant="italic">q</mi></mrow><mo class="fm-mo-Luc">)</mo></mrow></mrow><mo class="fm-infix">+</mo><mrow><mrow><mi class="fm-mi-length-1" mathvariant="italic" style="padding-right: 0.44ex;">Z</mi><mi class="fm-mi-length-1" mathvariant="italic">n</mi></mrow><mrow><mo class="fm-mo-Luc">(</mo><mi class="fm-mi-length-1" mathvariant="italic">s</mi><mo class="fm-mo-Luc">)</mo></mrow></mrow></mrow><mo class="fm-infix-loose">→</mo><mrow><mrow><msub><mrow><mrow><mrow><mi class="fm-mi-length-1" mathvariant="italic" style="padding-right: 0.44ex;">Z</mi><mi class="fm-mi-length-1" mathvariant="italic">n</mi></mrow><mi class="fm-mi-length-1" mathvariant="italic">S</mi></mrow><mi class="fm-mi-length-1" mathvariant="italic">O</mi></mrow><mn>4</mn></msub><mrow><mo class="fm-mo-Luc">(</mo><mrow><mi class="fm-mi-length-1" mathvariant="italic">a</mi><mi class="fm-mi-length-1" mathvariant="italic">q</mi></mrow><mo class="fm-mo-Luc">)</mo></mrow></mrow><mo class="fm-infix">+</mo><mrow><mrow><mi class="fm-mi-length-1" mathvariant="italic" style="padding-right: 0.44ex;">F</mi><mi class="fm-mi-length-1" mathvariant="italic">e</mi></mrow><mrow><mo class="fm-mo-Luc">(</mo><mi class="fm-mi-length-1" mathvariant="italic">s</mi><mo class="fm-mo-Luc">)</mo></mrow></mrow></mrow></mrow></fmath> 
\item  <fmath class="fm-inline"><mrow><mrow><mrow><mrow><mn>2</mn><msub><mrow><mrow><mrow><mi class="fm-mi-length-1" mathvariant="italic">C</mi><mi class="fm-mi-length-1" mathvariant="italic">u</mi></mrow><mi class="fm-mi-length-1" mathvariant="italic">S</mi></mrow><mi class="fm-mi-length-1" mathvariant="italic">O</mi></mrow><mn>4</mn></msub></mrow><mrow><mo class="fm-mo-Luc">(</mo><mrow><mi class="fm-mi-length-1" mathvariant="italic">a</mi><mi class="fm-mi-length-1" mathvariant="italic">q</mi></mrow><mo class="fm-mo-Luc">)</mo></mrow></mrow><mo class="fm-infix">+</mo><mrow><mrow><mn>2</mn><mrow><mi class="fm-mi-length-1" mathvariant="italic">A</mi><mi class="fm-mi-length-1" mathvariant="italic">g</mi></mrow></mrow><mrow><mo class="fm-mo-Luc">(</mo><mi class="fm-mi-length-1" mathvariant="italic">s</mi><mo class="fm-mo-Luc">)</mo></mrow></mrow></mrow><mo class="fm-infix-loose">→</mo><mrow><mrow><mrow><mn>2</mn><mrow><mi class="fm-mi-length-1" mathvariant="italic">C</mi><mi class="fm-mi-length-1" mathvariant="italic">u</mi></mrow></mrow><mrow><mo class="fm-mo-Luc">(</mo><mi class="fm-mi-length-1" mathvariant="italic">s</mi><mo class="fm-mo-Luc">)</mo></mrow></mrow><mo class="fm-infix">+</mo><mrow><mrow><msub><mrow><mi class="fm-mi-length-1" mathvariant="italic">A</mi><mi class="fm-mi-length-1" mathvariant="italic">g</mi></mrow><mn>2</mn></msub><msub><mrow><mi class="fm-mi-length-1" mathvariant="italic">S</mi><mi class="fm-mi-length-1" mathvariant="italic">O</mi></mrow><mn>4</mn></msub></mrow><mrow><mo class="fm-mo-Luc">(</mo><mrow><mi class="fm-mi-length-1" mathvariant="italic">a</mi><mi class="fm-mi-length-1" mathvariant="italic">q</mi></mrow><mo class="fm-mo-Luc">)</mo></mrow></mrow></mrow></mrow></fmath> 
\end{enumerate}
\newpage
\section*{Question 4}
<style>.fm-math,fmath{font-family:STIXGeneral,'DejaVu Serif','DejaVu Sans',Times,OpenSymbol,'Standard Symbols L',serif;line-height:1.2}.fm-math mtext,fmath mtext{line-height:normal}.fm-mo,.ma-sans-serif,fmath mi[mathvariant*=sans-serif],fmath mn[mathvariant*=sans-serif],fmath mo,fmath ms[mathvariant*=sans-serif],fmath mtext[mathvariant*=sans-serif]{font-family:STIXGeneral,'DejaVu Sans','DejaVu Serif','Arial Unicode MS','Lucida Grande',Times,OpenSymbol,'Standard Symbols L',sans-serif}.fm-mo-Luc{font-family:STIXGeneral,'DejaVu Sans','DejaVu Serif','Lucida Grande','Arial Unicode MS',Times,OpenSymbol,'Standard Symbols L',sans-serif}.questionsfont{font-weight:200;font-family:Arial, sans-serif, STIXGeneral,'DejaVu Sans','DejaVu Serif','Lucida Grande','Arial Unicode MS',Times,OpenSymbol,'Standard Symbols L',sans-serif!important}.fm-separator{padding:0 .56ex 0 0}.fm-infix-loose{padding:0 .56ex}.fm-infix{padding:0 .44ex}.fm-prefix{padding:0 .33ex 0 0}.fm-postfix{padding:0 0 0 .33ex}.fm-prefix-tight{padding:0 .11ex 0 0}.fm-postfix-tight{padding:0 0 0 .11ex}.fm-quantifier{padding:0 .11ex 0 .22ex}.ma-non-marking{display:none}.fm-vert,fmath menclose,menclose.fm-menclose{display:inline-block}.fm-large-op{font-size:1.3em}.fm-inline .fm-large-op{font-size:1em}fmath mrow{white-space:nowrap}.fm-vert{vertical-align:middle}fmath table,fmath tbody,fmath td,fmath tr{border:0!important;padding:0!important;margin:0!important;outline:0!important}fmath table{border-collapse:collapse!important;text-align:center!important;table-layout:auto!important;float:none!important}.fm-frac{padding:0 1px!important}td.fm-den-frac{border-top:solid thin!important}.fm-root{font-size:.6em}.fm-radicand{padding:0 1px 0 0;border-top:solid;margin-top:.1em}.fm-script{font-size:.71em}.fm-script .fm-script .fm-script{font-size:1em}td.fm-underover-base{line-height:1!important}td.fm-mtd{padding:.5ex .4em!important;vertical-align:baseline!important}fmath mphantom{visibility:hidden}fmath menclose[notation=top],menclose.fm-menclose[notation=top]{border-top:solid thin}fmath menclose[notation=right],menclose.fm-menclose[notation=right]{border-right:solid thin}fmath menclose[notation=bottom],menclose.fm-menclose[notation=bottom]{border-bottom:solid thin}fmath menclose[notation=left],menclose.fm-menclose[notation=left]{border-left:solid thin}fmath menclose[notation=box],menclose.fm-menclose[notation=box]{border:thin solid}fmath none{display:none}</style> The molar conductance of <fmath class="fm-inline"><mrow><mrow><mrow><mrow><mi class="fm-mi-length-1" mathvariant="italic" style="padding-right: 0.44ex;">N</mi><mi class="fm-mi-length-1" mathvariant="italic">a</mi></mrow><mi class="fm-mi-length-1" mathvariant="italic">C</mi></mrow><mi class="fm-mi-length-1" mathvariant="italic" style="padding-right: 0.44ex;">l</mi></mrow><mo class="fm-separator">,</mo><mrow><mrow><mi class="fm-mi-length-1" mathvariant="italic" style="padding-right: 0.44ex;">H</mi><mi class="fm-mi-length-1" mathvariant="italic">C</mi></mrow><mi class="fm-mi-length-1" mathvariant="italic" style="padding-right: 0.44ex;">l</mi></mrow></mrow></fmath> and <fmath class="fm-inline"><mrow><msub><mrow><mi class="fm-mi-length-1" mathvariant="italic">C</mi><mi class="fm-mi-length-1" mathvariant="italic" style="padding-right: 0.44ex;">H</mi></mrow><mn>3</mn></msub><mrow><mrow><mrow><mrow><mi class="fm-mi-length-1" mathvariant="italic">C</mi><mi class="fm-mi-length-1" mathvariant="italic">O</mi></mrow><mi class="fm-mi-length-1" mathvariant="italic">O</mi></mrow><mi class="fm-mi-length-1" mathvariant="italic" style="padding-right: 0.44ex;">N</mi></mrow><mi class="fm-mi-length-1" mathvariant="italic">a</mi></mrow></mrow></fmath> at infinite dilution are 126.45, <fmath class="fm-inline"><mn>426.16</mn></fmath> and <fmath class="fm-inline"><mrow><mrow><mrow><mn>91.0</mn><mi class="fm-mi-length-1" mathvariant="italic">S</mi></mrow><msup><mrow><mi class="fm-mi-length-1" mathvariant="italic">c</mi><mi class="fm-mi-length-1" mathvariant="italic">m</mi></mrow><mn>2</mn></msup></mrow><msup><mrow><mrow><mi class="fm-mi-length-1" mathvariant="italic">m</mi><mi class="fm-mi-length-1" mathvariant="italic">o</mi></mrow><mi class="fm-mi-length-1" mathvariant="italic" style="padding-right: 0.44ex;">l</mi></mrow><mrow><mo class="fm-prefix-tight">−</mo><mn>1</mn></mrow></msup></mrow></fmath> respectively. The molar conductance of <fmath class="fm-inline"><mrow><msub><mrow><mi class="fm-mi-length-1" mathvariant="italic">C</mi><mi class="fm-mi-length-1" mathvariant="italic" style="padding-right: 0.44ex;">H</mi></mrow><mn>3</mn></msub><mrow><mrow><mrow><mi class="fm-mi-length-1" mathvariant="italic">C</mi><mi class="fm-mi-length-1" mathvariant="italic">O</mi></mrow><mi class="fm-mi-length-1" mathvariant="italic">O</mi></mrow><mi class="fm-mi-length-1" mathvariant="italic" style="padding-right: 0.44ex;">H</mi></mrow></mrow></fmath> at infinite dilution is. Choose the right option for your answer.
\begin{enumerate}[label=(\alph*)]
\item  <fmath class="fm-inline"><mrow><mrow><mrow><mrow><mrow><mrow><mn>201.28</mn><mi class="fm-mi-length-1 ma-upright" mathvariant="normal" style="padding-right: 0px;"> </mi></mrow><mi class="fm-mi-length-1" mathvariant="italic">S</mi></mrow><mi class="fm-mi-length-1 ma-upright" mathvariant="normal" style="padding-right: 0px;"> </mi></mrow><msup><mrow><mi class="fm-mi-length-1" mathvariant="italic">c</mi><mi class="fm-mi-length-1" mathvariant="italic">m</mi></mrow><mn>2</mn></msup></mrow><mi class="fm-mi-length-1 ma-upright" mathvariant="normal" style="padding-right: 0px;"> </mi></mrow><msup><mrow><mrow><mi class="fm-mi-length-1" mathvariant="italic">m</mi><mi class="fm-mi-length-1" mathvariant="italic">o</mi></mrow><mi class="fm-mi-length-1" mathvariant="italic" style="padding-right: 0.44ex;">l</mi></mrow><mrow><mo class="fm-prefix-tight">−</mo><mn>1</mn></mrow></msup></mrow></fmath>
\item  <fmath class="fm-inline"><mrow><mrow><mrow><mrow><mrow><mrow><mn>390.71</mn><mi class="fm-mi-length-1 ma-upright" mathvariant="normal" style="padding-right: 0px;"> </mi></mrow><mi class="fm-mi-length-1" mathvariant="italic">S</mi></mrow><mi class="fm-mi-length-1 ma-upright" mathvariant="normal" style="padding-right: 0px;"> </mi></mrow><msup><mrow><mi class="fm-mi-length-1" mathvariant="italic">c</mi><mi class="fm-mi-length-1" mathvariant="italic">m</mi></mrow><mn>2</mn></msup></mrow><mi class="fm-mi-length-1 ma-upright" mathvariant="normal" style="padding-right: 0px;"> </mi></mrow><msup><mrow><mrow><mi class="fm-mi-length-1" mathvariant="italic">m</mi><mi class="fm-mi-length-1" mathvariant="italic">o</mi></mrow><mi class="fm-mi-length-1" mathvariant="italic" style="padding-right: 0.44ex;">l</mi></mrow><mrow><mo class="fm-prefix-tight">−</mo><mn>1</mn></mrow></msup></mrow></fmath>
\item  <fmath class="fm-inline"><mrow><mrow><mrow><mrow><mrow><mrow><mn>698.28</mn><mi class="fm-mi-length-1 ma-upright" mathvariant="normal" style="padding-right: 0px;"> </mi></mrow><mi class="fm-mi-length-1" mathvariant="italic">S</mi></mrow><mi class="fm-mi-length-1 ma-upright" mathvariant="normal" style="padding-right: 0px;"> </mi></mrow><msup><mrow><mi class="fm-mi-length-1" mathvariant="italic">c</mi><mi class="fm-mi-length-1" mathvariant="italic">m</mi></mrow><mn>2</mn></msup></mrow><mi class="fm-mi-length-1 ma-upright" mathvariant="normal" style="padding-right: 0px;"> </mi></mrow><msup><mrow><mrow><mi class="fm-mi-length-1" mathvariant="italic">m</mi><mi class="fm-mi-length-1" mathvariant="italic">o</mi></mrow><mi class="fm-mi-length-1" mathvariant="italic" style="padding-right: 0.44ex;">l</mi></mrow><mrow><mo class="fm-prefix-tight">−</mo><mn>1</mn></mrow></msup></mrow></fmath>
\item  <fmath class="fm-inline"><mrow><mrow><mrow><mrow><mrow><mrow><mn>540.48</mn><mi class="fm-mi-length-1 ma-upright" mathvariant="normal" style="padding-right: 0px;"> </mi></mrow><mi class="fm-mi-length-1" mathvariant="italic">S</mi></mrow><mi class="fm-mi-length-1 ma-upright" mathvariant="normal" style="padding-right: 0px;"> </mi></mrow><msup><mrow><mi class="fm-mi-length-1" mathvariant="italic">c</mi><mi class="fm-mi-length-1" mathvariant="italic">m</mi></mrow><mn>2</mn></msup></mrow><mi class="fm-mi-length-1 ma-upright" mathvariant="normal" style="padding-right: 0px;"> </mi></mrow><msup><mrow><mrow><mi class="fm-mi-length-1" mathvariant="italic">m</mi><mi class="fm-mi-length-1" mathvariant="italic">o</mi></mrow><mi class="fm-mi-length-1" mathvariant="italic" style="padding-right: 0.44ex;">l</mi></mrow><mrow><mo class="fm-prefix-tight">−</mo><mn>1</mn></mrow></msup></mrow></fmath>
\end{enumerate}
\newpage
\section*{Question 5}
<style>.fm-math,fmath{font-family:STIXGeneral,'DejaVu Serif','DejaVu Sans',Times,OpenSymbol,'Standard Symbols L',serif;line-height:1.2}.fm-math mtext,fmath mtext{line-height:normal}.fm-mo,.ma-sans-serif,fmath mi[mathvariant*=sans-serif],fmath mn[mathvariant*=sans-serif],fmath mo,fmath ms[mathvariant*=sans-serif],fmath mtext[mathvariant*=sans-serif]{font-family:STIXGeneral,'DejaVu Sans','DejaVu Serif','Arial Unicode MS','Lucida Grande',Times,OpenSymbol,'Standard Symbols L',sans-serif}.fm-mo-Luc{font-family:STIXGeneral,'DejaVu Sans','DejaVu Serif','Lucida Grande','Arial Unicode MS',Times,OpenSymbol,'Standard Symbols L',sans-serif}.questionsfont{font-weight:200;font-family:Arial, sans-serif, STIXGeneral,'DejaVu Sans','DejaVu Serif','Lucida Grande','Arial Unicode MS',Times,OpenSymbol,'Standard Symbols L',sans-serif!important}.fm-separator{padding:0 .56ex 0 0}.fm-infix-loose{padding:0 .56ex}.fm-infix{padding:0 .44ex}.fm-prefix{padding:0 .33ex 0 0}.fm-postfix{padding:0 0 0 .33ex}.fm-prefix-tight{padding:0 .11ex 0 0}.fm-postfix-tight{padding:0 0 0 .11ex}.fm-quantifier{padding:0 .11ex 0 .22ex}.ma-non-marking{display:none}.fm-vert,fmath menclose,menclose.fm-menclose{display:inline-block}.fm-large-op{font-size:1.3em}.fm-inline .fm-large-op{font-size:1em}fmath mrow{white-space:nowrap}.fm-vert{vertical-align:middle}fmath table,fmath tbody,fmath td,fmath tr{border:0!important;padding:0!important;margin:0!important;outline:0!important}fmath table{border-collapse:collapse!important;text-align:center!important;table-layout:auto!important;float:none!important}.fm-frac{padding:0 1px!important}td.fm-den-frac{border-top:solid thin!important}.fm-root{font-size:.6em}.fm-radicand{padding:0 1px 0 0;border-top:solid;margin-top:.1em}.fm-script{font-size:.71em}.fm-script .fm-script .fm-script{font-size:1em}td.fm-underover-base{line-height:1!important}td.fm-mtd{padding:.5ex .4em!important;vertical-align:baseline!important}fmath mphantom{visibility:hidden}fmath menclose[notation=top],menclose.fm-menclose[notation=top]{border-top:solid thin}fmath menclose[notation=right],menclose.fm-menclose[notation=right]{border-right:solid thin}fmath menclose[notation=bottom],menclose.fm-menclose[notation=bottom]{border-bottom:solid thin}fmath menclose[notation=left],menclose.fm-menclose[notation=left]{border-left:solid thin}fmath menclose[notation=box],menclose.fm-menclose[notation=box]{border:thin solid}fmath none{display:none}</style> The number of Faradays(F) required to produce 20 g of calcium from molten <fmath class="fm-inline"><mrow><mrow><mrow><mi class="fm-mi-length-1" mathvariant="italic">C</mi><mi class="fm-mi-length-1" mathvariant="italic">a</mi></mrow><mi class="fm-mi-length-1" mathvariant="italic">C</mi></mrow><msub><mi class="fm-mi-length-1" mathvariant="italic" style="padding-right: 0.44ex;">l</mi><mn>2</mn></msub></mrow></fmath> (Atomic mass of <fmath class="fm-inline"><mrow><mrow><mi class="fm-mi-length-1" mathvariant="italic">C</mi><mi class="fm-mi-length-1" mathvariant="italic">a</mi></mrow><mo class="fm-infix-loose">=</mo><mrow><mrow><mrow><mrow><mn>40</mn><mi class="fm-mi-length-1" mathvariant="italic">g</mi></mrow><mi class="fm-mi-length-1" mathvariant="italic">m</mi></mrow><mi class="fm-mi-length-1" mathvariant="italic">o</mi></mrow><msup><mi class="fm-mi-length-1" mathvariant="italic" style="padding-right: 0.44ex;">l</mi><mrow><mi class="fm-mi-length-1" mathvariant="italic">–</mi><mn>1</mn></mrow></msup></mrow></mrow></fmath> ) is  
\begin{enumerate}[label=(\alph*)]
\item 2
\item 3
\item 4
\item 1
\end{enumerate}
\newpage
\section*{Question 6}
<style>.fm-math,fmath{font-family:STIXGeneral,'DejaVu Serif','DejaVu Sans',Times,OpenSymbol,'Standard Symbols L',serif;line-height:1.2}.fm-math mtext,fmath mtext{line-height:normal}.fm-mo,.ma-sans-serif,fmath mi[mathvariant*=sans-serif],fmath mn[mathvariant*=sans-serif],fmath mo,fmath ms[mathvariant*=sans-serif],fmath mtext[mathvariant*=sans-serif]{font-family:STIXGeneral,'DejaVu Sans','DejaVu Serif','Arial Unicode MS','Lucida Grande',Times,OpenSymbol,'Standard Symbols L',sans-serif}.fm-mo-Luc{font-family:STIXGeneral,'DejaVu Sans','DejaVu Serif','Lucida Grande','Arial Unicode MS',Times,OpenSymbol,'Standard Symbols L',sans-serif}.questionsfont{font-weight:200;font-family:Arial, sans-serif, STIXGeneral,'DejaVu Sans','DejaVu Serif','Lucida Grande','Arial Unicode MS',Times,OpenSymbol,'Standard Symbols L',sans-serif!important}.fm-separator{padding:0 .56ex 0 0}.fm-infix-loose{padding:0 .56ex}.fm-infix{padding:0 .44ex}.fm-prefix{padding:0 .33ex 0 0}.fm-postfix{padding:0 0 0 .33ex}.fm-prefix-tight{padding:0 .11ex 0 0}.fm-postfix-tight{padding:0 0 0 .11ex}.fm-quantifier{padding:0 .11ex 0 .22ex}.ma-non-marking{display:none}.fm-vert,fmath menclose,menclose.fm-menclose{display:inline-block}.fm-large-op{font-size:1.3em}.fm-inline .fm-large-op{font-size:1em}fmath mrow{white-space:nowrap}.fm-vert{vertical-align:middle}fmath table,fmath tbody,fmath td,fmath tr{border:0!important;padding:0!important;margin:0!important;outline:0!important}fmath table{border-collapse:collapse!important;text-align:center!important;table-layout:auto!important;float:none!important}.fm-frac{padding:0 1px!important}td.fm-den-frac{border-top:solid thin!important}.fm-root{font-size:.6em}.fm-radicand{padding:0 1px 0 0;border-top:solid;margin-top:.1em}.fm-script{font-size:.71em}.fm-script .fm-script .fm-script{font-size:1em}td.fm-underover-base{line-height:1!important}td.fm-mtd{padding:.5ex .4em!important;vertical-align:baseline!important}fmath mphantom{visibility:hidden}fmath menclose[notation=top],menclose.fm-menclose[notation=top]{border-top:solid thin}fmath menclose[notation=right],menclose.fm-menclose[notation=right]{border-right:solid thin}fmath menclose[notation=bottom],menclose.fm-menclose[notation=bottom]{border-bottom:solid thin}fmath menclose[notation=left],menclose.fm-menclose[notation=left]{border-left:solid thin}fmath menclose[notation=box],menclose.fm-menclose[notation=box]{border:thin solid}fmath none{display:none}</style> Following limiting molar conductivities are given as\newline<fmath class="fm-inline"><mrow><msubsup><mi class="fm-mi-length-1" mathvariant="italic">λ</mi><mrow><mi class="fm-mi-length-1 ma-upright" mathvariant="normal" style="padding-right: 0px;">m</mi><mrow><mo class="fm-mo-Luc">(</mo><mrow><msub><mi class="fm-mi-length-1 ma-upright" mathvariant="normal" style="padding-right: 0px;">H</mi><mn>2</mn></msub><msub><mi class="ma-repel-adj" mathvariant="normal">SO</mi><mn>4</mn></msub></mrow><mo class="fm-mo-Luc">)</mo></mrow></mrow><mn>0</mn></msubsup><mo class="fm-infix-loose">=</mo><mrow><mrow><mi class="fm-mi-length-1 ma-upright" mathvariant="normal" style="padding-right: 0px;">x</mi><msup><mi class="ma-repel-adj" mathvariant="normal">Scm</mi><mn>2</mn></msup></mrow><msup><mi class="ma-repel-adj" mathvariant="normal">mol</mi><mrow><mo class="fm-prefix-tight">−</mo><mn>1</mn></mrow></msup></mrow></mrow></fmath>\newline<fmath class="fm-inline"><mrow><msubsup><mi class="fm-mi-length-1" mathvariant="italic">λ</mi><mrow><mi class="fm-mi-length-1 ma-upright" mathvariant="normal" style="padding-right: 0px;">m</mi><mrow><mo class="fm-mo-Luc">(</mo><mrow><msub><mi class="fm-mi-length-1 ma-upright" mathvariant="normal" style="padding-right: 0px;">K</mi><mn>2</mn></msub><msub><mi class="ma-repel-adj" mathvariant="normal">SO</mi><mn>4</mn></msub></mrow><mo class="fm-mo-Luc">)</mo></mrow></mrow><mn>0</mn></msubsup><mo class="fm-infix-loose">=</mo><mrow><mrow><mi class="fm-mi-length-1 ma-upright" mathvariant="normal" style="padding-right: 0px;">y</mi><msup><mi class="ma-repel-adj" mathvariant="normal">Scm</mi><mn>2</mn></msup></mrow><msup><mi class="ma-repel-adj" mathvariant="normal">mol</mi><mrow><mo class="fm-prefix-tight">−</mo><mn>1</mn></mrow></msup></mrow></mrow></fmath>\newline<fmath class="fm-inline"><mrow><msubsup><mi class="fm-mi-length-1" mathvariant="italic">λ</mi><mrow><mi class="fm-mi-length-1 ma-upright" mathvariant="normal" style="padding-right: 0px;">m</mi><mrow><mo class="fm-mo-Luc">(</mo><mrow class="ma-repel-adj"><msub><mi class="ma-repel-adj" mathvariant="normal">CH</mi><mn>3</mn></msub><mspace style="margin-right: 0.17em; padding-right: 0.001em; visibility: hidden;" width=".17em">‌</mspace><mi class="ma-repel-adj" mathvariant="normal">COOK</mi></mrow><mo class="fm-mo-Luc">)</mo></mrow></mrow><mn>0</mn></msubsup><mo class="fm-infix-loose">=</mo><mrow><mrow><mi class="fm-mi-length-1 ma-upright" mathvariant="normal" style="padding-right: 0px;">z</mi><msup><mi class="ma-repel-adj" mathvariant="normal">Scm</mi><mn>2</mn></msup></mrow><msup><mi class="ma-repel-adj" mathvariant="normal">mol</mi><mrow><mo class="fm-prefix-tight">−</mo><mn>1</mn></mrow></msup></mrow></mrow></fmath>\newline<fmath class="fm-inline"><mrow class="ma-repel-adj"><msubsup><mi class="fm-mi-length-1" mathvariant="italic">λ</mi><mi class="fm-mi-length-1 ma-upright" mathvariant="normal" style="padding-right: 0px;">m</mi><mn>0</mn></msubsup><mrow class="ma-repel-adj"><mo class="fm-mo-Luc fm-prefix">(</mo><mo>.</mo></mrow></mrow></fmath> in <fmath class="fm-inline"><mrow class="ma-repel-adj"><mrow><mo class="fm-prefix-tight">.</mo><mrow><mrow><mi class="fm-mi-length-1 ma-upright" mathvariant="normal" style="padding-right: 0px;">S</mi><msup><mi class="ma-repel-adj" mathvariant="normal">cm</mi><mn>2</mn></msup></mrow><msup><mi class="ma-repel-adj" mathvariant="normal">mol</mi><mrow><mo class="fm-prefix-tight">−</mo><mn>1</mn></mrow></msup></mrow></mrow><mo class="fm-mo-Luc fm-postfix">)</mo></mrow></fmath> for <fmath class="fm-inline"><mrow class="ma-repel-adj"><msub><mi class="ma-repel-adj" mathvariant="normal">CH</mi><mn>3</mn></msub><mspace style="margin-right: 0.17em; padding-right: 0.001em; visibility: hidden;" width=".17em">‌</mspace><mi class="ma-repel-adj" mathvariant="normal">COOH</mi></mrow></fmath> will be-\newline 
\begin{enumerate}[label=(\alph*)]
\item  <fmath class="fm-inline"><mrow><mrow><mi class="fm-mi-length-1" mathvariant="italic">x</mi><mo class="fm-infix">−</mo><mi class="fm-mi-length-1" mathvariant="italic">y</mi></mrow><mo class="fm-infix">+</mo><mrow><mn>2</mn><mi class="fm-mi-length-1" mathvariant="italic">z</mi></mrow></mrow></fmath>
\item  <fmath class="fm-inline"><mrow><mrow><mi class="fm-mi-length-1" mathvariant="italic">x</mi><mo class="fm-infix">+</mo><mi class="fm-mi-length-1" mathvariant="italic">y</mi></mrow><mo class="fm-infix">−</mo><mi class="fm-mi-length-1" mathvariant="italic">z</mi></mrow></fmath>
\item  <fmath class="fm-inline"><mrow><mrow><mi class="fm-mi-length-1" mathvariant="italic">x</mi><mo class="fm-infix">−</mo><mi class="fm-mi-length-1" mathvariant="italic">y</mi></mrow><mo class="fm-infix">+</mo><mi class="fm-mi-length-1" mathvariant="italic">z</mi></mrow></fmath>
\item  <fmath class="fm-inline"><mrow><mrow><mspace style="margin-right: 0.28em; padding-right: 0.001em; visibility: hidden;" width=".28em">‌</mspace>\begin{tabular}{|c|c|}
\hline
<mrow><mo class="fm-mo-Luc">(</mo><mrow><mi class="fm-mi-length-1" mathvariant="italic">x</mi><mo class="fm-infix">−</mo><mi class="fm-mi-length-1" mathvariant="italic">y</mi></mrow><mo class="fm-mo-Luc">)</mo></mrow> \\
\hline
<mn>2</mn> \\
\hline
\end{tabular}
</mrow><mo class="fm-infix">+</mo><mi class="fm-mi-length-1" mathvariant="italic">z</mi></mrow></fmath>
\end{enumerate}
\newpage
\section*{Question 7}
<style>.fm-math,fmath{font-family:STIXGeneral,'DejaVu Serif','DejaVu Sans',Times,OpenSymbol,'Standard Symbols L',serif;line-height:1.2}.fm-math mtext,fmath mtext{line-height:normal}.fm-mo,.ma-sans-serif,fmath mi[mathvariant*=sans-serif],fmath mn[mathvariant*=sans-serif],fmath mo,fmath ms[mathvariant*=sans-serif],fmath mtext[mathvariant*=sans-serif]{font-family:STIXGeneral,'DejaVu Sans','DejaVu Serif','Arial Unicode MS','Lucida Grande',Times,OpenSymbol,'Standard Symbols L',sans-serif}.fm-mo-Luc{font-family:STIXGeneral,'DejaVu Sans','DejaVu Serif','Lucida Grande','Arial Unicode MS',Times,OpenSymbol,'Standard Symbols L',sans-serif}.questionsfont{font-weight:200;font-family:Arial, sans-serif, STIXGeneral,'DejaVu Sans','DejaVu Serif','Lucida Grande','Arial Unicode MS',Times,OpenSymbol,'Standard Symbols L',sans-serif!important}.fm-separator{padding:0 .56ex 0 0}.fm-infix-loose{padding:0 .56ex}.fm-infix{padding:0 .44ex}.fm-prefix{padding:0 .33ex 0 0}.fm-postfix{padding:0 0 0 .33ex}.fm-prefix-tight{padding:0 .11ex 0 0}.fm-postfix-tight{padding:0 0 0 .11ex}.fm-quantifier{padding:0 .11ex 0 .22ex}.ma-non-marking{display:none}.fm-vert,fmath menclose,menclose.fm-menclose{display:inline-block}.fm-large-op{font-size:1.3em}.fm-inline .fm-large-op{font-size:1em}fmath mrow{white-space:nowrap}.fm-vert{vertical-align:middle}fmath table,fmath tbody,fmath td,fmath tr{border:0!important;padding:0!important;margin:0!important;outline:0!important}fmath table{border-collapse:collapse!important;text-align:center!important;table-layout:auto!important;float:none!important}.fm-frac{padding:0 1px!important}td.fm-den-frac{border-top:solid thin!important}.fm-root{font-size:.6em}.fm-radicand{padding:0 1px 0 0;border-top:solid;margin-top:.1em}.fm-script{font-size:.71em}.fm-script .fm-script .fm-script{font-size:1em}td.fm-underover-base{line-height:1!important}td.fm-mtd{padding:.5ex .4em!important;vertical-align:baseline!important}fmath mphantom{visibility:hidden}fmath menclose[notation=top],menclose.fm-menclose[notation=top]{border-top:solid thin}fmath menclose[notation=right],menclose.fm-menclose[notation=right]{border-right:solid thin}fmath menclose[notation=bottom],menclose.fm-menclose[notation=bottom]{border-bottom:solid thin}fmath menclose[notation=left],menclose.fm-menclose[notation=left]{border-left:solid thin}fmath menclose[notation=box],menclose.fm-menclose[notation=box]{border:thin solid}fmath none{display:none}</style> The standard electrode potential <fmath class="fm-inline"><mrow><mo class="fm-mo-Luc">(</mo><msup><mi class="fm-mi-length-1 ma-upright" mathvariant="normal" style="padding-right: 0px;">E</mi><mo>−</mo></msup><mo class="fm-mo-Luc">)</mo></mrow></fmath>values of <fmath class="fm-inline"><mrow><mrow><mrow><msup><mi class="ma-repel-adj" mathvariant="normal">Al</mi><mrow><mn>3</mn><mo class="fm-postfix-tight">+</mo></mrow></msup><mo class="fm-infix-loose">∕</mo><msup><mi class="ma-repel-adj" mathvariant="normal">Al</mi><mn>2</mn></msup></mrow><mo class="fm-separator">,</mo><mrow><msup><mi class="ma-repel-adj" mathvariant="normal">Ag</mi><mo>+</mo></msup><mo class="fm-infix-loose">∕</mo><mi class="ma-repel-adj" mathvariant="normal">Ag</mi></mrow></mrow><mo class="fm-separator">,</mo><mrow><msup><mi class="fm-mi-length-1 ma-upright" mathvariant="normal" style="padding-right: 0px;">K</mi><mo>+</mo></msup><mo class="fm-infix-loose">∕</mo><mi class="fm-mi-length-1 ma-upright" mathvariant="normal" style="padding-right: 0px;">K</mi></mrow></mrow></fmath> and <fmath class="fm-inline"><mrow><msup><mi class="ma-repel-adj" mathvariant="normal">Cr</mi><mrow><mn>3</mn><mo class="fm-postfix-tight">+</mo></mrow></msup><mo class="fm-infix-loose">∕</mo><mi class="ma-repel-adj" mathvariant="normal">Cr</mi></mrow></fmath> are <fmath class="fm-inline"><mrow><mrow><mo class="fm-prefix-tight">−</mo><mrow><mn>1.66</mn><mi class="fm-mi-length-1 ma-upright" mathvariant="normal" style="padding-right: 0px;">V</mi></mrow></mrow><mo class="fm-separator">,</mo><mrow><mn>0.80</mn><mi class="fm-mi-length-1 ma-upright" mathvariant="normal" style="padding-right: 0px;">V</mi></mrow></mrow></fmath>, <fmath class="fm-inline"><mrow><mo class="fm-prefix-tight">−</mo><mrow><mn>2.93</mn><mi class="fm-mi-length-1 ma-upright" mathvariant="normal" style="padding-right: 0px;">V</mi></mrow></mrow></fmath> and <fmath class="fm-inline"><mrow><mo class="fm-prefix-tight">−</mo><mrow><mn>0.74</mn><mi class="fm-mi-length-1 ma-upright" mathvariant="normal" style="padding-right: 0px;">V</mi></mrow></mrow></fmath>, respectively. The correct decreasing order of reducing power of the metal is :\newline 
\begin{enumerate}[label=(\alph*)]
\item  <fmath class="fm-inline"><mrow><mrow><mrow><mi class="ma-repel-adj" mathvariant="normal">Ag</mi><mo class="fm-infix-loose">></mo><mi class="ma-repel-adj" mathvariant="normal">Cr</mi></mrow><mo class="fm-infix-loose">></mo><mi class="ma-repel-adj" mathvariant="normal">Al</mi></mrow><mo class="fm-infix-loose">></mo><mi class="fm-mi-length-1 ma-upright" mathvariant="normal" style="padding-right: 0px;">K</mi></mrow></fmath>
\item  <fmath class="fm-inline"><mrow><mrow><mrow><mi class="fm-mi-length-1 ma-upright" mathvariant="normal" style="padding-right: 0px;">K</mi><mo class="fm-infix-loose">></mo><mi class="ma-repel-adj" mathvariant="normal">Al</mi></mrow><mo class="fm-infix-loose">></mo><mi class="ma-repel-adj" mathvariant="normal">Cr</mi></mrow><mo class="fm-infix-loose">></mo><mi class="ma-repel-adj" mathvariant="normal">Ag</mi></mrow></fmath>
\item  <fmath class="fm-inline"><mrow><mrow><mrow><mi class="fm-mi-length-1 ma-upright" mathvariant="normal" style="padding-right: 0px;">K</mi><mo class="fm-infix-loose">></mo><mi class="ma-repel-adj" mathvariant="normal">Al</mi></mrow><mo class="fm-infix-loose">></mo><mi class="ma-repel-adj" mathvariant="normal">Ag</mi></mrow><mo class="fm-infix-loose">></mo><mi class="ma-repel-adj" mathvariant="normal">Cr</mi></mrow></fmath>
\item  <fmath class="fm-inline"><mrow><mrow><mrow><mi class="ma-repel-adj" mathvariant="normal">Al</mi><mo class="fm-infix-loose">></mo><mi class="fm-mi-length-1 ma-upright" mathvariant="normal" style="padding-right: 0px;">K</mi></mrow><mo class="fm-infix-loose">></mo><mi class="ma-repel-adj" mathvariant="normal">Ag</mi></mrow><mo class="fm-infix-loose">></mo><mi class="ma-repel-adj" mathvariant="normal">Cr</mi></mrow></fmath>
\end{enumerate}
\newpage
\section*{Question 8}
<style>.fm-math,fmath{font-family:STIXGeneral,'DejaVu Serif','DejaVu Sans',Times,OpenSymbol,'Standard Symbols L',serif;line-height:1.2}.fm-math mtext,fmath mtext{line-height:normal}.fm-mo,.ma-sans-serif,fmath mi[mathvariant*=sans-serif],fmath mn[mathvariant*=sans-serif],fmath mo,fmath ms[mathvariant*=sans-serif],fmath mtext[mathvariant*=sans-serif]{font-family:STIXGeneral,'DejaVu Sans','DejaVu Serif','Arial Unicode MS','Lucida Grande',Times,OpenSymbol,'Standard Symbols L',sans-serif}.fm-mo-Luc{font-family:STIXGeneral,'DejaVu Sans','DejaVu Serif','Lucida Grande','Arial Unicode MS',Times,OpenSymbol,'Standard Symbols L',sans-serif}.questionsfont{font-weight:200;font-family:Arial, sans-serif, STIXGeneral,'DejaVu Sans','DejaVu Serif','Lucida Grande','Arial Unicode MS',Times,OpenSymbol,'Standard Symbols L',sans-serif!important}.fm-separator{padding:0 .56ex 0 0}.fm-infix-loose{padding:0 .56ex}.fm-infix{padding:0 .44ex}.fm-prefix{padding:0 .33ex 0 0}.fm-postfix{padding:0 0 0 .33ex}.fm-prefix-tight{padding:0 .11ex 0 0}.fm-postfix-tight{padding:0 0 0 .11ex}.fm-quantifier{padding:0 .11ex 0 .22ex}.ma-non-marking{display:none}.fm-vert,fmath menclose,menclose.fm-menclose{display:inline-block}.fm-large-op{font-size:1.3em}.fm-inline .fm-large-op{font-size:1em}fmath mrow{white-space:nowrap}.fm-vert{vertical-align:middle}fmath table,fmath tbody,fmath td,fmath tr{border:0!important;padding:0!important;margin:0!important;outline:0!important}fmath table{border-collapse:collapse!important;text-align:center!important;table-layout:auto!important;float:none!important}.fm-frac{padding:0 1px!important}td.fm-den-frac{border-top:solid thin!important}.fm-root{font-size:.6em}.fm-radicand{padding:0 1px 0 0;border-top:solid;margin-top:.1em}.fm-script{font-size:.71em}.fm-script .fm-script .fm-script{font-size:1em}td.fm-underover-base{line-height:1!important}td.fm-mtd{padding:.5ex .4em!important;vertical-align:baseline!important}fmath mphantom{visibility:hidden}fmath menclose[notation=top],menclose.fm-menclose[notation=top]{border-top:solid thin}fmath menclose[notation=right],menclose.fm-menclose[notation=right]{border-right:solid thin}fmath menclose[notation=bottom],menclose.fm-menclose[notation=bottom]{border-bottom:solid thin}fmath menclose[notation=left],menclose.fm-menclose[notation=left]{border-left:solid thin}fmath menclose[notation=box],menclose.fm-menclose[notation=box]{border:thin solid}fmath none{display:none}</style> The molar conductivity of a <fmath class="fm-inline"><mrow><mn>0.5</mn><mspace style="margin-right: 0.28em; padding-right: 0.001em; visibility: hidden;" width=".28em">‌</mspace><msup><mtext>mol/dm</mtext><mn>3</mn></msup></mrow></fmath> solution of <fmath class="fm-inline"><mrow><mrow><mrow><mi class="fm-mi-length-1" mathvariant="italic">A</mi><mi class="fm-mi-length-1" mathvariant="italic">g</mi></mrow><mi class="fm-mi-length-1" mathvariant="italic" style="padding-right: 0.44ex;">N</mi></mrow><msub><mi class="fm-mi-length-1" mathvariant="italic">O</mi><mn>3</mn></msub></mrow></fmath> with electrolytic conductivity of <fmath class="fm-inline"><mrow><mn>5.76</mn><mo class="fm-infix" lspace=".22em" rspace=".22em">×</mo><msup><mn>10</mn><mrow><mi class="fm-mi-length-1" mathvariant="italic">–</mi><mn>3</mn></mrow></msup></mrow></fmath> S <fmath class="fm-inline"><mrow><mi class="fm-mi-length-1" mathvariant="italic">c</mi><msup><mi class="fm-mi-length-1" mathvariant="italic">m</mi><mrow><mi class="fm-mi-length-1" mathvariant="italic">–</mi><mn>1</mn></mrow></msup></mrow></fmath> at 298 K is 
\begin{enumerate}[label=(\alph*)]
\item  <fmath class="fm-inline"><mrow><mrow><mn>28.8</mn><mspace style="margin-right: 0.28em; padding-right: 0.001em; visibility: hidden;" width=".28em">‌</mspace><mi class="fm-mi-length-1" mathvariant="italic">S</mi></mrow><mspace style="margin-right: 0.28em; padding-right: 0.001em; visibility: hidden;" width=".28em">‌</mspace>\begin{tabular}{|c|c|}
\hline
<mrow><mi class="fm-mi-length-1" mathvariant="italic">c</mi><msup><mi class="fm-mi-length-1" mathvariant="italic">m</mi><mn>2</mn></msup></mrow> \\
\hline
<mrow><mrow><mi class="fm-mi-length-1" mathvariant="italic">m</mi><mi class="fm-mi-length-1" mathvariant="italic">o</mi></mrow><mi class="fm-mi-length-1" mathvariant="italic" style="padding-right: 0.44ex;">l</mi></mrow> \\
\hline
\end{tabular}
</mrow></fmath>
\item  <fmath class="fm-inline"><mrow><mrow><mn>2.88</mn><mspace style="margin-right: 0.28em; padding-right: 0.001em; visibility: hidden;" width=".28em">‌</mspace><mi class="fm-mi-length-1" mathvariant="italic">S</mi></mrow><mspace style="margin-right: 0.28em; padding-right: 0.001em; visibility: hidden;" width=".28em">‌</mspace>\begin{tabular}{|c|c|}
\hline
<mrow><mi class="fm-mi-length-1" mathvariant="italic">c</mi><msup><mi class="fm-mi-length-1" mathvariant="italic">m</mi><mn>2</mn></msup></mrow> \\
\hline
<mrow><mrow><mi class="fm-mi-length-1" mathvariant="italic">m</mi><mi class="fm-mi-length-1" mathvariant="italic">o</mi></mrow><mi class="fm-mi-length-1" mathvariant="italic" style="padding-right: 0.44ex;">l</mi></mrow> \\
\hline
\end{tabular}
</mrow></fmath>
\item  <fmath class="fm-inline"><mrow><mrow><mn>11.52</mn><mspace style="margin-right: 0.28em; padding-right: 0.001em; visibility: hidden;" width=".28em">‌</mspace><mi class="fm-mi-length-1" mathvariant="italic">S</mi></mrow><mspace style="margin-right: 0.28em; padding-right: 0.001em; visibility: hidden;" width=".28em">‌</mspace>\begin{tabular}{|c|c|}
\hline
<mrow><mi class="fm-mi-length-1" mathvariant="italic">c</mi><msup><mi class="fm-mi-length-1" mathvariant="italic">m</mi><mn>2</mn></msup></mrow> \\
\hline
<mrow><mrow><mi class="fm-mi-length-1" mathvariant="italic">m</mi><mi class="fm-mi-length-1" mathvariant="italic">o</mi></mrow><mi class="fm-mi-length-1" mathvariant="italic" style="padding-right: 0.44ex;">l</mi></mrow> \\
\hline
\end{tabular}
</mrow></fmath>
\item  <fmath class="fm-inline"><mrow><mrow><mn>0.086</mn><mspace style="margin-right: 0.28em; padding-right: 0.001em; visibility: hidden;" width=".28em">‌</mspace><mi class="fm-mi-length-1" mathvariant="italic">S</mi></mrow><mspace style="margin-right: 0.28em; padding-right: 0.001em; visibility: hidden;" width=".28em">‌</mspace>\begin{tabular}{|c|c|}
\hline
<mrow><mi class="fm-mi-length-1" mathvariant="italic">c</mi><msup><mi class="fm-mi-length-1" mathvariant="italic">m</mi><mn>2</mn></msup></mrow> \\
\hline
<mrow><mrow><mi class="fm-mi-length-1" mathvariant="italic">m</mi><mi class="fm-mi-length-1" mathvariant="italic">o</mi></mrow><mi class="fm-mi-length-1" mathvariant="italic" style="padding-right: 0.44ex;">l</mi></mrow> \\
\hline
\end{tabular}
</mrow></fmath>
\end{enumerate}
\newpage
\section*{Question 9}
<style>.fm-math,fmath{font-family:STIXGeneral,'DejaVu Serif','DejaVu Sans',Times,OpenSymbol,'Standard Symbols L',serif;line-height:1.2}.fm-math mtext,fmath mtext{line-height:normal}.fm-mo,.ma-sans-serif,fmath mi[mathvariant*=sans-serif],fmath mn[mathvariant*=sans-serif],fmath mo,fmath ms[mathvariant*=sans-serif],fmath mtext[mathvariant*=sans-serif]{font-family:STIXGeneral,'DejaVu Sans','DejaVu Serif','Arial Unicode MS','Lucida Grande',Times,OpenSymbol,'Standard Symbols L',sans-serif}.fm-mo-Luc{font-family:STIXGeneral,'DejaVu Sans','DejaVu Serif','Lucida Grande','Arial Unicode MS',Times,OpenSymbol,'Standard Symbols L',sans-serif}.questionsfont{font-weight:200;font-family:Arial, sans-serif, STIXGeneral,'DejaVu Sans','DejaVu Serif','Lucida Grande','Arial Unicode MS',Times,OpenSymbol,'Standard Symbols L',sans-serif!important}.fm-separator{padding:0 .56ex 0 0}.fm-infix-loose{padding:0 .56ex}.fm-infix{padding:0 .44ex}.fm-prefix{padding:0 .33ex 0 0}.fm-postfix{padding:0 0 0 .33ex}.fm-prefix-tight{padding:0 .11ex 0 0}.fm-postfix-tight{padding:0 0 0 .11ex}.fm-quantifier{padding:0 .11ex 0 .22ex}.ma-non-marking{display:none}.fm-vert,fmath menclose,menclose.fm-menclose{display:inline-block}.fm-large-op{font-size:1.3em}.fm-inline .fm-large-op{font-size:1em}fmath mrow{white-space:nowrap}.fm-vert{vertical-align:middle}fmath table,fmath tbody,fmath td,fmath tr{border:0!important;padding:0!important;margin:0!important;outline:0!important}fmath table{border-collapse:collapse!important;text-align:center!important;table-layout:auto!important;float:none!important}.fm-frac{padding:0 1px!important}td.fm-den-frac{border-top:solid thin!important}.fm-root{font-size:.6em}.fm-radicand{padding:0 1px 0 0;border-top:solid;margin-top:.1em}.fm-script{font-size:.71em}.fm-script .fm-script .fm-script{font-size:1em}td.fm-underover-base{line-height:1!important}td.fm-mtd{padding:.5ex .4em!important;vertical-align:baseline!important}fmath mphantom{visibility:hidden}fmath menclose[notation=top],menclose.fm-menclose[notation=top]{border-top:solid thin}fmath menclose[notation=right],menclose.fm-menclose[notation=right]{border-right:solid thin}fmath menclose[notation=bottom],menclose.fm-menclose[notation=bottom]{border-bottom:solid thin}fmath menclose[notation=left],menclose.fm-menclose[notation=left]{border-left:solid thin}fmath menclose[notation=box],menclose.fm-menclose[notation=box]{border:thin solid}fmath none{display:none}</style> When 0.1 mol <fmath class="fm-inline"><msubsup><mtext>MnO</mtext><mn>4</mn><mrow><mn>2</mn><mo class="fm-postfix-tight">−</mo></mrow></msubsup></fmath> is oxidised the quantity of electricity required to completely <fmath class="fm-inline"><msubsup><mtext>MnO</mtext><mn>4</mn><mrow><mn>2</mn><mo class="fm-postfix-tight">−</mo></mrow></msubsup></fmath> to <fmath class="fm-inline"><msubsup><mtext>MnO</mtext><mn>4</mn><mo>−</mo></msubsup></fmath> is 
\begin{enumerate}[label=(\alph*)]
\item <fmath class="fm-inline"><mn>96500</mn></fmath> C
\item <fmath class="fm-inline"><mrow><mn>2</mn><mo class="fm-infix" lspace=".22em" rspace=".22em">×</mo><mn>96500</mn></mrow></fmath> C
\item <fmath class="fm-inline"><mn>9650</mn></fmath> C
\item <fmath class="fm-inline"><mn>96.50</mn></fmath> C
\end{enumerate}
\newpage
\section*{Question 10}
A hydrogen gas electrode is made by dipping platinum wire in a solution of HCl of pH = 10 and by passing hydrogen gas around the platinum wire at one atm pressure. The oxidation potential of electrode would be? 
\begin{enumerate}[label=(\alph*)]
\item 0.118 V
\item 1.18 V
\item 0.059 V
\item 0.59 V
\end{enumerate}
\newpage
\section*{Question 11}
Which of the following sequence of reactions is suitable to synthesize chlorobenzene? 
\begin{enumerate}[label=(\alph*)]
\item  Benzene, <fmath class="fm-inline"><msub><mrow><mi class="fm-mi-length-1" mathvariant="italic">C</mi><mi class="fm-mi-length-1" mathvariant="italic" style="padding-right: 0.44ex;">l</mi></mrow><mn>2</mn></msub></fmath>, anhydrous <fmath class="fm-inline"><msub><mrow><mrow><mrow><mi class="fm-mi-length-1" mathvariant="italic" style="padding-right: 0.44ex;">F</mi><mi class="fm-mi-length-1" mathvariant="italic">e</mi></mrow><mi class="fm-mi-length-1" mathvariant="italic">C</mi></mrow><mi class="fm-mi-length-1" mathvariant="italic" style="padding-right: 0.44ex;">l</mi></mrow><mn>3</mn></msub></fmath> 
\item  Phenol, <fmath class="fm-inline"><mrow><mrow><msub><mrow><mrow><mrow><mi class="fm-mi-length-1" mathvariant="italic" style="padding-right: 0.44ex;">N</mi><mi class="fm-mi-length-1" mathvariant="italic">a</mi></mrow><mi class="fm-mi-length-1" mathvariant="italic" style="padding-right: 0.44ex;">N</mi></mrow><mi class="fm-mi-length-1" mathvariant="italic">O</mi></mrow><mn>2</mn></msub><mo class="fm-separator">,</mo><mrow><mrow><mi class="fm-mi-length-1" mathvariant="italic" style="padding-right: 0.44ex;">H</mi><mi class="fm-mi-length-1" mathvariant="italic">C</mi></mrow><mi class="fm-mi-length-1" mathvariant="italic" style="padding-right: 0.44ex;">l</mi></mrow></mrow><mo class="fm-separator">,</mo><mrow><mrow><mrow><mi class="fm-mi-length-1" mathvariant="italic">C</mi><mi class="fm-mi-length-1" mathvariant="italic">u</mi></mrow><mi class="fm-mi-length-1" mathvariant="italic">C</mi></mrow><mi class="fm-mi-length-1" mathvariant="italic" style="padding-right: 0.44ex;">l</mi></mrow></mrow></fmath>
\item \includegraphics[width=\textwidth]{static/media/wl_client/1/qdump/50272f977f0f492a38d073f6f7507847/02d31c4bacff0f798b3a9346dbd59acf.png}
\item \includegraphics[width=\textwidth]{static/media/wl_client/1/qdump/50272f977f0f492a38d073f6f7507847/6b6d143ca6a3018dc00bd07238a2c5f0.png}
\end{enumerate}
\newpage
\section*{Question 12}
Given below are two statements : one is\newlinelabelled as Assertion (A) and the other is\newlinelabelled as Reason (R).\newlineAssertion (A) :\newlineChlorine is an electron withdrawing group but it\newlineis ortho, para directing in electrophilic aromatic\newlinesubstitution.\newlineReason (R) :\newlineInductive effect of chlorine destabilises the\newlineintermediate carbocation formed during the\newlineelectrophilic substitution, however due to the\newlinemore pronounced resonance effect, the\newlinehalogen stabilises the carbocation at ortho and\newlinepara positions.\newlineIn the light of the above statements, choose the\newlinemost appropriate answer from the options\newlinegiven below :\newline  
\begin{enumerate}[label=(\alph*)]
\item  (A) is not correct but (R) is correct.
\item  Both (A) and (R) are correct and (R) is the\newlinecorrect explanation of (A).
\item  Both (A) and (R) are correct but (R) is not\newlinethe correct explanation of (A).
\item  (A) is correct but (R) is not correct.
\end{enumerate}
\newpage
\section*{Question 13}
Elimination reaction of 2-Bromo-pentane to form pent-2-ene is \newline (a) β-Elimination reaction \newline (b) Follows Zaitsev rule \newline (c) Dehydrohalogenation reaction \newline (d) Dehydration reaction  
\begin{enumerate}[label=(\alph*)]
\item (a), (c), (d)
\item (b), (c), (d)
\item  (a), (b), (d)
\item (a), (b), (c)
\end{enumerate}
\newpage
\section*{Question 14}
<style>.fm-math,fmath{font-family:STIXGeneral,'DejaVu Serif','DejaVu Sans',Times,OpenSymbol,'Standard Symbols L',serif;line-height:1.2}.fm-math mtext,fmath mtext{line-height:normal}.fm-mo,.ma-sans-serif,fmath mi[mathvariant*=sans-serif],fmath mn[mathvariant*=sans-serif],fmath mo,fmath ms[mathvariant*=sans-serif],fmath mtext[mathvariant*=sans-serif]{font-family:STIXGeneral,'DejaVu Sans','DejaVu Serif','Arial Unicode MS','Lucida Grande',Times,OpenSymbol,'Standard Symbols L',sans-serif}.fm-mo-Luc{font-family:STIXGeneral,'DejaVu Sans','DejaVu Serif','Lucida Grande','Arial Unicode MS',Times,OpenSymbol,'Standard Symbols L',sans-serif}.questionsfont{font-weight:200;font-family:Arial, sans-serif, STIXGeneral,'DejaVu Sans','DejaVu Serif','Lucida Grande','Arial Unicode MS',Times,OpenSymbol,'Standard Symbols L',sans-serif!important}.fm-separator{padding:0 .56ex 0 0}.fm-infix-loose{padding:0 .56ex}.fm-infix{padding:0 .44ex}.fm-prefix{padding:0 .33ex 0 0}.fm-postfix{padding:0 0 0 .33ex}.fm-prefix-tight{padding:0 .11ex 0 0}.fm-postfix-tight{padding:0 0 0 .11ex}.fm-quantifier{padding:0 .11ex 0 .22ex}.ma-non-marking{display:none}.fm-vert,fmath menclose,menclose.fm-menclose{display:inline-block}.fm-large-op{font-size:1.3em}.fm-inline .fm-large-op{font-size:1em}fmath mrow{white-space:nowrap}.fm-vert{vertical-align:middle}fmath table,fmath tbody,fmath td,fmath tr{border:0!important;padding:0!important;margin:0!important;outline:0!important}fmath table{border-collapse:collapse!important;text-align:center!important;table-layout:auto!important;float:none!important}.fm-frac{padding:0 1px!important}td.fm-den-frac{border-top:solid thin!important}.fm-root{font-size:.6em}.fm-radicand{padding:0 1px 0 0;border-top:solid;margin-top:.1em}.fm-script{font-size:.71em}.fm-script .fm-script .fm-script{font-size:1em}td.fm-underover-base{line-height:1!important}td.fm-mtd{padding:.5ex .4em!important;vertical-align:baseline!important}fmath mphantom{visibility:hidden}fmath menclose[notation=top],menclose.fm-menclose[notation=top]{border-top:solid thin}fmath menclose[notation=right],menclose.fm-menclose[notation=right]{border-right:solid thin}fmath menclose[notation=bottom],menclose.fm-menclose[notation=bottom]{border-bottom:solid thin}fmath menclose[notation=left],menclose.fm-menclose[notation=left]{border-left:solid thin}fmath menclose[notation=box],menclose.fm-menclose[notation=box]{border:thin solid}fmath none{display:none}</style> Consider the reaction:\newline<fmath class="fm-inline"><mrow><mrow><mrow><mrow><mrow><mrow><mrow><mrow><mrow><mi class="fm-mi-length-1" mathvariant="italic">C</mi><msub><mi class="fm-mi-length-1" mathvariant="italic" style="padding-right: 0.44ex;">H</mi><mn>3</mn></msub></mrow><mi class="fm-mi-length-1" mathvariant="italic">C</mi></mrow><msub><mi class="fm-mi-length-1" mathvariant="italic" style="padding-right: 0.44ex;">H</mi><mn>2</mn></msub></mrow><mi class="fm-mi-length-1" mathvariant="italic">C</mi></mrow><msub><mi class="fm-mi-length-1" mathvariant="italic" style="padding-right: 0.44ex;">H</mi><mn>2</mn></msub></mrow><mi class="fm-mi-length-1" mathvariant="italic">B</mi></mrow><mi class="fm-mi-length-1" mathvariant="italic">r</mi></mrow><mo class="fm-infix">+</mo><mrow><mrow><mrow><mi class="fm-mi-length-1" mathvariant="italic" style="padding-right: 0.44ex;">N</mi><mi class="fm-mi-length-1" mathvariant="italic">a</mi></mrow><mi class="fm-mi-length-1" mathvariant="italic">C</mi></mrow><mi class="fm-mi-length-1" mathvariant="italic" style="padding-right: 0.44ex;">N</mi></mrow></mrow><mo class="fm-postfix-tight">→</mo></mrow></fmath> <fmath class="fm-inline"><mrow><mrow><mrow><mrow><mrow><mrow><mrow><mrow><mi class="fm-mi-length-1" mathvariant="italic">C</mi><msub><mi class="fm-mi-length-1" mathvariant="italic" style="padding-right: 0.44ex;">H</mi><mn>3</mn></msub></mrow><mi class="fm-mi-length-1" mathvariant="italic">C</mi></mrow><msub><mi class="fm-mi-length-1" mathvariant="italic" style="padding-right: 0.44ex;">H</mi><mn>2</mn></msub></mrow><mi class="fm-mi-length-1" mathvariant="italic">C</mi></mrow><msub><mi class="fm-mi-length-1" mathvariant="italic" style="padding-right: 0.44ex;">H</mi><mn>2</mn></msub></mrow><mi class="fm-mi-length-1" mathvariant="italic">C</mi></mrow><mi class="fm-mi-length-1" mathvariant="italic" style="padding-right: 0.44ex;">N</mi></mrow><mo class="fm-infix">+</mo><mrow><mrow><mrow><mi class="fm-mi-length-1" mathvariant="italic" style="padding-right: 0.44ex;">N</mi><mi class="fm-mi-length-1" mathvariant="italic">a</mi></mrow><mi class="fm-mi-length-1" mathvariant="italic">B</mi></mrow><mi class="fm-mi-length-1" mathvariant="italic">r</mi></mrow></mrow></fmath> \newlineThis reaction will be the fastest in : 
\begin{enumerate}[label=(\alph*)]
\item  water
\item  ethanol 
\item  methanol
\item  N,N' - dimethylformamide (DMF) 
\end{enumerate}
\newpage
\section*{Question 15}
Which of the following acids does not exhibit optical isomerism? }
\begin{enumerate}[label=(\alph*)]
\item  Maleic acid
\item  <fmath class="fm-inline"><mi class="ma-repel-adj" mathvariant="normal">alpha</mi></fmath> -amino acids
\item  Lactic acid
\item  Tartaric acid
\end{enumerate}
\newpage
\section*{Question 16}
<style>.fm-math,fmath{font-family:STIXGeneral,'DejaVu Serif','DejaVu Sans',Times,OpenSymbol,'Standard Symbols L',serif;line-height:1.2}.fm-math mtext,fmath mtext{line-height:normal}.fm-mo,.ma-sans-serif,fmath mi[mathvariant*=sans-serif],fmath mn[mathvariant*=sans-serif],fmath mo,fmath ms[mathvariant*=sans-serif],fmath mtext[mathvariant*=sans-serif]{font-family:STIXGeneral,'DejaVu Sans','DejaVu Serif','Arial Unicode MS','Lucida Grande',Times,OpenSymbol,'Standard Symbols L',sans-serif}.fm-mo-Luc{font-family:STIXGeneral,'DejaVu Sans','DejaVu Serif','Lucida Grande','Arial Unicode MS',Times,OpenSymbol,'Standard Symbols L',sans-serif}.questionsfont{font-weight:200;font-family:Arial, sans-serif, STIXGeneral,'DejaVu Sans','DejaVu Serif','Lucida Grande','Arial Unicode MS',Times,OpenSymbol,'Standard Symbols L',sans-serif!important}.fm-separator{padding:0 .56ex 0 0}.fm-infix-loose{padding:0 .56ex}.fm-infix{padding:0 .44ex}.fm-prefix{padding:0 .33ex 0 0}.fm-postfix{padding:0 0 0 .33ex}.fm-prefix-tight{padding:0 .11ex 0 0}.fm-postfix-tight{padding:0 0 0 .11ex}.fm-quantifier{padding:0 .11ex 0 .22ex}.ma-non-marking{display:none}.fm-vert,fmath menclose,menclose.fm-menclose{display:inline-block}.fm-large-op{font-size:1.3em}.fm-inline .fm-large-op{font-size:1em}fmath mrow{white-space:nowrap}.fm-vert{vertical-align:middle}fmath table,fmath tbody,fmath td,fmath tr{border:0!important;padding:0!important;margin:0!important;outline:0!important}fmath table{border-collapse:collapse!important;text-align:center!important;table-layout:auto!important;float:none!important}.fm-frac{padding:0 1px!important}td.fm-den-frac{border-top:solid thin!important}.fm-root{font-size:.6em}.fm-radicand{padding:0 1px 0 0;border-top:solid;margin-top:.1em}.fm-script{font-size:.71em}.fm-script .fm-script .fm-script{font-size:1em}td.fm-underover-base{line-height:1!important}td.fm-mtd{padding:.5ex .4em!important;vertical-align:baseline!important}fmath mphantom{visibility:hidden}fmath menclose[notation=top],menclose.fm-menclose[notation=top]{border-top:solid thin}fmath menclose[notation=right],menclose.fm-menclose[notation=right]{border-right:solid thin}fmath menclose[notation=bottom],menclose.fm-menclose[notation=bottom]{border-bottom:solid thin}fmath menclose[notation=left],menclose.fm-menclose[notation=left]{border-left:solid thin}fmath menclose[notation=box],menclose.fm-menclose[notation=box]{border:thin solid}fmath none{display:none}</style> Which one is most reactive towards <fmath class="fm-inline"><mrow><msub><mi class="fm-mi-length-1 ma-upright" mathvariant="normal" style="padding-right: 0px;">S</mi><mi class="fm-mi-length-1 ma-upright" mathvariant="normal" style="padding-right: 0px;">N</mi></msub><mn>1</mn></mrow></fmath> reaction? }
\begin{enumerate}[label=(\alph*)]
\item  <fmath class="fm-inline"><mrow class="ma-repel-adj"><mrow class="ma-repel-adj"><mrow class="ma-repel-adj"><mrow><msub><mi class="fm-mi-length-1 ma-upright" mathvariant="normal" style="padding-right: 0px;">C</mi><mn>6</mn></msub><msub><mi class="fm-mi-length-1 ma-upright" mathvariant="normal" style="padding-right: 0px;">H</mi><mn>5</mn></msub></mrow><mspace style="margin-right: 0.17em; padding-right: 0.001em; visibility: hidden;" width=".17em">‌</mspace><mi class="ma-repel-adj" mathvariant="normal">CH</mi></mrow><mrow><mo class="fm-mo-Luc">(</mo><mrow><msub><mi class="fm-mi-length-1 ma-upright" mathvariant="normal" style="padding-right: 0px;">C</mi><mn>6</mn></msub><msub><mi class="fm-mi-length-1 ma-upright" mathvariant="normal" style="padding-right: 0px;">H</mi><mn>5</mn></msub></mrow><mo class="fm-mo-Luc">)</mo></mrow></mrow><mspace style="margin-right: 0.17em; padding-right: 0.001em; visibility: hidden;" width=".17em">‌</mspace><mi class="ma-repel-adj" mathvariant="normal">Br</mi></mrow></fmath>
\item  <fmath class="fm-inline"><mrow class="ma-repel-adj"><mrow class="ma-repel-adj"><mrow class="ma-repel-adj"><mrow><msub><mi class="fm-mi-length-1 ma-upright" mathvariant="normal" style="padding-right: 0px;">C</mi><mn>6</mn></msub><msub><mi class="fm-mi-length-1 ma-upright" mathvariant="normal" style="padding-right: 0px;">H</mi><mn>5</mn></msub></mrow><mspace style="margin-right: 0.17em; padding-right: 0.001em; visibility: hidden;" width=".17em">‌</mspace><mi class="ma-repel-adj" mathvariant="normal">CH</mi></mrow><mrow><mo class="fm-mo-Luc">(</mo><msub><mi class="ma-repel-adj" mathvariant="normal">CH</mi><mn>3</mn></msub><mo class="fm-mo-Luc">)</mo></mrow></mrow><mspace style="margin-right: 0.17em; padding-right: 0.001em; visibility: hidden;" width=".17em">‌</mspace><mi class="ma-repel-adj" mathvariant="normal">Br</mi></mrow></fmath>
\item  <fmath class="fm-inline"><mrow class="ma-repel-adj"><mrow><mrow><mrow><mrow><msub><mi class="fm-mi-length-1 ma-upright" mathvariant="normal" style="padding-right: 0px;">C</mi><mn>6</mn></msub><msub><mi class="fm-mi-length-1 ma-upright" mathvariant="normal" style="padding-right: 0px;">H</mi><mn>5</mn></msub></mrow><mi class="fm-mi-length-1 ma-upright" mathvariant="normal" style="padding-right: 0px;">C</mi></mrow><mrow><mo class="fm-mo-Luc">(</mo><msub><mi class="ma-repel-adj" mathvariant="normal">CH</mi><mn>3</mn></msub><mo class="fm-mo-Luc">)</mo></mrow></mrow><mrow><mo class="fm-mo-Luc">(</mo><mrow><msub><mi class="fm-mi-length-1 ma-upright" mathvariant="normal" style="padding-right: 0px;">C</mi><mn>6</mn></msub><msub><mi class="fm-mi-length-1 ma-upright" mathvariant="normal" style="padding-right: 0px;">H</mi><mn>5</mn></msub></mrow><mo class="fm-mo-Luc">)</mo></mrow></mrow><mspace style="margin-right: 0.17em; padding-right: 0.001em; visibility: hidden;" width=".17em">‌</mspace><mi class="ma-repel-adj" mathvariant="normal">Br</mi></mrow></fmath>
\item  <fmath class="fm-inline"><mrow class="ma-repel-adj"><mrow><mrow><msub><mi class="fm-mi-length-1 ma-upright" mathvariant="normal" style="padding-right: 0px;">C</mi><mn>6</mn></msub><msub><mi class="fm-mi-length-1 ma-upright" mathvariant="normal" style="padding-right: 0px;">H</mi><mn>5</mn></msub></mrow><msub><mi class="ma-repel-adj" mathvariant="normal">CH</mi><mn>2</mn></msub></mrow><mspace style="margin-right: 0.17em; padding-right: 0.001em; visibility: hidden;" width=".17em">‌</mspace><mi class="ma-repel-adj" mathvariant="normal">Br</mi></mrow></fmath>
\end{enumerate}
\newpage
\section*{Question 17}
The correct order of increasing reactivity of \includegraphics[width=\textwidth]{static/media/wl_client/1/qdump/50272f977f0f492a38d073f6f7507847/de3239693da676e8872a9989bc1b0127.png}C—X bond towards nucleophile in the following compounds is 
\begin{enumerate}[label=(\alph*)]
\item I < II < IV < III
\item II < III < I < IV
\item IV < III < I < II
\item III < II < I < IV
\end{enumerate}
\newpage
\section*{Question 18}
<style>.fm-math,fmath{font-family:STIXGeneral,'DejaVu Serif','DejaVu Sans',Times,OpenSymbol,'Standard Symbols L',serif;line-height:1.2}.fm-math mtext,fmath mtext{line-height:normal}.fm-mo,.ma-sans-serif,fmath mi[mathvariant*=sans-serif],fmath mn[mathvariant*=sans-serif],fmath mo,fmath ms[mathvariant*=sans-serif],fmath mtext[mathvariant*=sans-serif]{font-family:STIXGeneral,'DejaVu Sans','DejaVu Serif','Arial Unicode MS','Lucida Grande',Times,OpenSymbol,'Standard Symbols L',sans-serif}.fm-mo-Luc{font-family:STIXGeneral,'DejaVu Sans','DejaVu Serif','Lucida Grande','Arial Unicode MS',Times,OpenSymbol,'Standard Symbols L',sans-serif}.questionsfont{font-weight:200;font-family:Arial, sans-serif, STIXGeneral,'DejaVu Sans','DejaVu Serif','Lucida Grande','Arial Unicode MS',Times,OpenSymbol,'Standard Symbols L',sans-serif!important}.fm-separator{padding:0 .56ex 0 0}.fm-infix-loose{padding:0 .56ex}.fm-infix{padding:0 .44ex}.fm-prefix{padding:0 .33ex 0 0}.fm-postfix{padding:0 0 0 .33ex}.fm-prefix-tight{padding:0 .11ex 0 0}.fm-postfix-tight{padding:0 0 0 .11ex}.fm-quantifier{padding:0 .11ex 0 .22ex}.ma-non-marking{display:none}.fm-vert,fmath menclose,menclose.fm-menclose{display:inline-block}.fm-large-op{font-size:1.3em}.fm-inline .fm-large-op{font-size:1em}fmath mrow{white-space:nowrap}.fm-vert{vertical-align:middle}fmath table,fmath tbody,fmath td,fmath tr{border:0!important;padding:0!important;margin:0!important;outline:0!important}fmath table{border-collapse:collapse!important;text-align:center!important;table-layout:auto!important;float:none!important}.fm-frac{padding:0 1px!important}td.fm-den-frac{border-top:solid thin!important}.fm-root{font-size:.6em}.fm-radicand{padding:0 1px 0 0;border-top:solid;margin-top:.1em}.fm-script{font-size:.71em}.fm-script .fm-script .fm-script{font-size:1em}td.fm-underover-base{line-height:1!important}td.fm-mtd{padding:.5ex .4em!important;vertical-align:baseline!important}fmath mphantom{visibility:hidden}fmath menclose[notation=top],menclose.fm-menclose[notation=top]{border-top:solid thin}fmath menclose[notation=right],menclose.fm-menclose[notation=right]{border-right:solid thin}fmath menclose[notation=bottom],menclose.fm-menclose[notation=bottom]{border-bottom:solid thin}fmath menclose[notation=left],menclose.fm-menclose[notation=left]{border-left:solid thin}fmath menclose[notation=box],menclose.fm-menclose[notation=box]{border:thin solid}fmath none{display:none}</style> In a <fmath class="fm-inline"><mrow><msub><mi class="fm-mi-length-1" mathvariant="italic">S</mi><mi class="fm-mi-length-1" mathvariant="italic" style="padding-right: 0.44ex;">N</mi></msub><mn>2</mn></mrow></fmath> substitution reaction of the type\newline<fmath class="fm-inline"><mrow><mrow><mrow><mrow><mi class="fm-mi-length-1" mathvariant="italic">R</mi><mo class="fm-infix">−</mo><mrow><mi class="fm-mi-length-1" mathvariant="italic">B</mi><mi class="fm-mi-length-1" mathvariant="italic">r</mi></mrow></mrow><mo class="fm-infix">+</mo><mrow><mrow><mrow><mi class="fm-mi-length-1" mathvariant="italic">C</mi><msup><mi class="fm-mi-length-1" mathvariant="italic" style="padding-right: 0.44ex;">l</mi><mo>−</mo></msup></mrow>\begin{tabular}{|c|c|}
\hline
<mrow><mrow><mi class="fm-mi-length-1" mathvariant="italic">D</mi><mi class="fm-mi-length-1" mathvariant="italic" style="padding-right: 0.44ex;">M</mi></mrow><mi class="fm-mi-length-1" mathvariant="italic" style="padding-right: 0.44ex;">F</mi></mrow> \\
\hline
<mrow><mrow><mrow><mrow><mrow><mrow><mrow><mrow><mi class="fm-mi-length-1" mathvariant="italic">–</mi><mi class="fm-mi-length-1" mathvariant="italic">–</mi></mrow><mi class="fm-mi-length-1" mathvariant="italic">–</mi></mrow><mi class="fm-mi-length-1" mathvariant="italic">–</mi></mrow><mi class="fm-mi-length-1" mathvariant="italic">–</mi></mrow><mi class="fm-mi-length-1" mathvariant="italic">–</mi></mrow><mi class="fm-mi-length-1" mathvariant="italic">–</mi></mrow><mi class="fm-mi-length-1" mathvariant="italic">–</mi></mrow><mi mathvariant="normal">🢖</mi></mrow> \\
\hline
\end{tabular}
</mrow><mi class="fm-mi-length-1" mathvariant="italic">R</mi></mrow></mrow><mo class="fm-infix">−</mo><mrow><mi class="fm-mi-length-1" mathvariant="italic">C</mi><mi class="fm-mi-length-1" mathvariant="italic" style="padding-right: 0.44ex;">l</mi></mrow></mrow><mo class="fm-infix">+</mo><mrow><mi class="fm-mi-length-1" mathvariant="italic">B</mi><msup><mi class="fm-mi-length-1" mathvariant="italic">r</mi><mo>−</mo></msup></mrow></mrow></fmath>\newlinewhich one of the following has the highest relative rate? 
\begin{enumerate}[label=(\alph*)]
\item <fmath class="fm-inline"><mrow><mrow><mrow><mi class="fm-mi-length-1" mathvariant="italic">C</mi><msub><mi class="fm-mi-length-1" mathvariant="italic" style="padding-right: 0.44ex;">H</mi><mn>3</mn></msub></mrow><mo class="fm-infix">−</mo>\begin{tabular}{|c|c|c|}
\hline
<mrow><mi class="fm-mi-length-1" mathvariant="italic">C</mi><msub><mi class="fm-mi-length-1" mathvariant="italic" style="padding-right: 0.44ex;">H</mi><mn>3</mn></msub></mrow> \\
\hline
<mo class="fm-mo-Luc">|</mo> \\
\hline
\end{tabular}
</td></tr><tr><td class="fm-underover-base"><mi class="fm-mi-length-1" mathvariant="italic">C</mi></td></tr><tr><td class="fm-script fm-inline">\begin{tabular}{|c|c|}
\hline
<mo class="fm-mo-Luc">|</mo> \\
\hline
<mrow><mi class="fm-mi-length-1" mathvariant="italic">C</mi><msub><mi class="fm-mi-length-1" mathvariant="italic" style="padding-right: 0.44ex;">H</mi><mn>3</mn></msub></mrow> \\
\hline
\end{tabular}
</td></tr></tbody></table></mrow><mo class="fm-infix">−</mo><mrow><mrow><mrow><mi class="fm-mi-length-1" mathvariant="italic">C</mi><msub><mi class="fm-mi-length-1" mathvariant="italic" style="padding-right: 0.44ex;">H</mi><mn>2</mn></msub></mrow><mi class="fm-mi-length-1" mathvariant="italic">B</mi></mrow><mi class="fm-mi-length-1" mathvariant="italic">r</mi></mrow></mrow></fmath>
\item <fmath class="fm-inline"><mrow><mrow><mrow><mrow><mrow><mi class="fm-mi-length-1" mathvariant="italic">C</mi><msub><mi class="fm-mi-length-1" mathvariant="italic" style="padding-right: 0.44ex;">H</mi><mn>3</mn></msub></mrow><mi class="fm-mi-length-1" mathvariant="italic">C</mi></mrow><msub><mi class="fm-mi-length-1" mathvariant="italic" style="padding-right: 0.44ex;">H</mi><mn>2</mn></msub></mrow><mi class="fm-mi-length-1" mathvariant="italic">B</mi></mrow><mi class="fm-mi-length-1" mathvariant="italic">r</mi></mrow></fmath>
\item <fmath class="fm-inline"><mrow><mrow><mrow><mrow><mrow><mrow><mrow><mi class="fm-mi-length-1" mathvariant="italic">C</mi><msub><mi class="fm-mi-length-1" mathvariant="italic" style="padding-right: 0.44ex;">H</mi><mn>3</mn></msub></mrow><mi class="fm-mi-length-1" mathvariant="italic">C</mi></mrow><msub><mi class="fm-mi-length-1" mathvariant="italic" style="padding-right: 0.44ex;">H</mi><mn>2</mn></msub></mrow><mi class="fm-mi-length-1" mathvariant="italic">C</mi></mrow><msub><mi class="fm-mi-length-1" mathvariant="italic" style="padding-right: 0.44ex;">H</mi><mn>2</mn></msub></mrow><mi class="fm-mi-length-1" mathvariant="italic">B</mi></mrow><mi class="fm-mi-length-1" mathvariant="italic">r</mi></mrow></fmath>
\item <fmath class="fm-inline"><mrow><mrow><mrow><mi class="fm-mi-length-1" mathvariant="italic">C</mi><msub><mi class="fm-mi-length-1" mathvariant="italic" style="padding-right: 0.44ex;">H</mi><mn>3</mn></msub></mrow><mo class="fm-infix">−</mo><mrow><mi class="fm-mi-length-1" mathvariant="italic">C</mi>\begin{tabular}{|c|c|c|c|}
\hline
<mi class="fm-mi-length-1" mathvariant="italic" style="padding-right: 0.44ex;">H</mi> \\
\hline
<mo class="fm-mo-Luc">|</mo> \\
\hline
<mrow><mi class="fm-mi-length-1" mathvariant="italic">C</mi><msub><mi class="fm-mi-length-1" mathvariant="italic" style="padding-right: 0.44ex;">H</mi><mn>3</mn></msub></mrow> \\
\hline
\end{tabular}
</td></tr></tbody></table></mrow></mrow><mo class="fm-infix">−</mo><mrow><mrow><mrow><mi class="fm-mi-length-1" mathvariant="italic">C</mi><msub><mi class="fm-mi-length-1" mathvariant="italic" style="padding-right: 0.44ex;">H</mi><mn>2</mn></msub></mrow><mi class="fm-mi-length-1" mathvariant="italic">B</mi></mrow><mi class="fm-mi-length-1" mathvariant="italic">r</mi></mrow></mrow></fmath>
\end{enumerate}
\newpage
\section*{Question 19}
Reactivity order of halides for dehydrohalogenation is }
\begin{enumerate}[label=(\alph*)]
\item  <fmath class="fm-inline"><mrow><mrow><mrow><mrow><mi class="fm-mi-length-1 ma-upright" mathvariant="normal" style="padding-right: 0px;">R</mi><mo class="fm-infix">−</mo><mi class="fm-mi-length-1 ma-upright" mathvariant="normal" style="padding-right: 0px;">F</mi></mrow><mo class="fm-infix-loose">></mo><mrow><mi class="fm-mi-length-1 ma-upright" mathvariant="normal" style="padding-right: 0px;">R</mi><mo class="fm-infix">−</mo><mi class="ma-repel-adj" mathvariant="normal">Cl</mi></mrow></mrow><mo class="fm-infix-loose">></mo><mrow><mi class="fm-mi-length-1 ma-upright" mathvariant="normal" style="padding-right: 0px;">R</mi><mo class="fm-infix">−</mo><mi class="ma-repel-adj" mathvariant="normal">Br</mi></mrow></mrow><mo class="fm-infix-loose">></mo><mrow><mi class="fm-mi-length-1 ma-upright" mathvariant="normal" style="padding-right: 0px;">R</mi><mo class="fm-infix">−</mo><mi class="fm-mi-length-1 ma-upright" mathvariant="normal" style="padding-right: 0px;">I</mi></mrow></mrow></fmath>
\item  <fmath class="fm-inline"><mrow><mrow><mrow><mrow><mi class="fm-mi-length-1 ma-upright" mathvariant="normal" style="padding-right: 0px;">R</mi><mo class="fm-infix">−</mo><mi class="fm-mi-length-1 ma-upright" mathvariant="normal" style="padding-right: 0px;">I</mi></mrow><mo class="fm-infix-loose">></mo><mrow><mi class="fm-mi-length-1 ma-upright" mathvariant="normal" style="padding-right: 0px;">R</mi><mo class="fm-infix">−</mo><mi class="ma-repel-adj" mathvariant="normal">Br</mi></mrow></mrow><mo class="fm-infix-loose">></mo><mrow><mi class="fm-mi-length-1 ma-upright" mathvariant="normal" style="padding-right: 0px;">R</mi><mo class="fm-infix">−</mo><mi class="ma-repel-adj" mathvariant="normal">Cl</mi></mrow></mrow><mo class="fm-infix-loose">></mo><mrow><mi class="fm-mi-length-1 ma-upright" mathvariant="normal" style="padding-right: 0px;">R</mi><mo class="fm-infix">−</mo><mi class="fm-mi-length-1 ma-upright" mathvariant="normal" style="padding-right: 0px;">F</mi></mrow></mrow></fmath>
\item  <fmath class="fm-inline"><mrow><mrow><mrow><mrow><mi class="fm-mi-length-1 ma-upright" mathvariant="normal" style="padding-right: 0px;">R</mi><mo class="fm-infix">−</mo><mi class="fm-mi-length-1 ma-upright" mathvariant="normal" style="padding-right: 0px;">I</mi></mrow><mo class="fm-infix-loose">></mo><mrow><mi class="fm-mi-length-1 ma-upright" mathvariant="normal" style="padding-right: 0px;">R</mi><mo class="fm-infix">−</mo><mi class="ma-repel-adj" mathvariant="normal">Cl</mi></mrow></mrow><mo class="fm-infix-loose">></mo><mrow><mi class="fm-mi-length-1 ma-upright" mathvariant="normal" style="padding-right: 0px;">R</mi><mo class="fm-infix">−</mo><mi class="ma-repel-adj" mathvariant="normal">Br</mi></mrow></mrow><mo class="fm-infix-loose">></mo><mrow><mi class="fm-mi-length-1 ma-upright" mathvariant="normal" style="padding-right: 0px;">R</mi><mo class="fm-infix">−</mo><mi class="fm-mi-length-1 ma-upright" mathvariant="normal" style="padding-right: 0px;">F</mi></mrow></mrow></fmath>
\item  <fmath class="fm-inline"><mrow><mrow><mrow><mrow><mi class="fm-mi-length-1 ma-upright" mathvariant="normal" style="padding-right: 0px;">R</mi><mo class="fm-infix">−</mo><mi class="fm-mi-length-1 ma-upright" mathvariant="normal" style="padding-right: 0px;">F</mi></mrow><mo class="fm-infix-loose">></mo><mrow><mi class="fm-mi-length-1 ma-upright" mathvariant="normal" style="padding-right: 0px;">R</mi><mo class="fm-infix">−</mo><mi class="fm-mi-length-1 ma-upright" mathvariant="normal" style="padding-right: 0px;">I</mi></mrow></mrow><mo class="fm-infix-loose">></mo><mrow><mi class="fm-mi-length-1 ma-upright" mathvariant="normal" style="padding-right: 0px;">R</mi><mo class="fm-infix">−</mo><mi class="ma-repel-adj" mathvariant="normal">Br</mi></mrow></mrow><mo class="fm-infix-loose">></mo><mrow><mi class="fm-mi-length-1 ma-upright" mathvariant="normal" style="padding-right: 0px;">R</mi><mo class="fm-infix">−</mo><mi class="ma-repel-adj" mathvariant="normal">Cl</mi></mrow></mrow></fmath>
\end{enumerate}
\newpage
\section*{Question 20}
Grignard reagent is prepared by the reaction between }
\begin{enumerate}[label=(\alph*)]
\item  magnesium and alkane
\item  magnesium and aromatic hydrocarbon
\item  zinc and alkyl halide
\item  magnesium and alkyl halide.
\end{enumerate}
\newpage
\end{document}