\documentclass{article}
                    \usepackage{amsmath}
                    \usepackage{amssymb}
                    \usepackage{graphicx}
                    \usepackage{enumitem}
                    \usepackage{longtable}
                    \title{JEE - 12 CHEM}
                    \begin{document}
                    \maketitle
                    \section*{Question 1}
At a particular temperature, the vapour pressures of two liquids A and B are 120 mm and 180 mm of mercury respectively. If 2 moles of A and 3 moles of B are mixed to form an ideal solution, the vapour pressure of the solution at the same temperature will be: (in mm of mercury)
\begin{enumerate}[label=(\alph*)]
\item 156
\item 145
\item 150
\item 108
\end{enumerate}
\newpage
\section*{Question 2}
An aqueous solution of 2% non-volatile solute exerts a pressure of 1.004 bar at the normal boiling point of the solvent. What is the molar mass of the solute?
\begin{enumerate}[label=(\alph*)]
\item \(44.8 {~g} {~mol}^{-1}\)
\item \(42.5 {~g} {~mol}^{-1}\)
\item \(41.35 {~g} {~mol}^{-1}\)
\item \(50.35 {~g} {~mol}^{-1}\)
\end{enumerate}
\newpage
\section*{Question 3}
Heptane and octane form an ideal solution. At 373 K, the vapour pressures of the two liquid components are 105.2 kPa and 46.8 kPa respectively. What will be the vapour pressure of a mixture of 26.0 g of heptane and 35 g of octane?
\begin{enumerate}[label=(\alph*)]
\item 74.3 kPa
\item 73.43 kPa
\item 76.42 kPa
\item 79.50 kPa
\end{enumerate}
\newpage
\section*{Question 4}
Calculate the mass of ascorbic acid ( \(\mathrm{C}_{6} \mathrm{H}_{8} \mathrm{O}_{6}\) ) to be dissolved in \(75 \mathrm{~g}\) of acetic acid to lower its melting point by \(1.5^{\circ} \mathrm{C}\). Take \( \mathrm{K}_{\mathrm{f}}=3.9 \mathrm{~K} \mathrm{~kg} \mathrm{~mol}^{-1}\).
\begin{enumerate}[label=(\alph*)]
\item 4.52 g
\item 6.10 g
\item 5.08 g
\item 7.34 g
\end{enumerate}
\newpage
\section*{Question 5}
Value of Henry's constant \(\mathrm{k_H}\)?\newline
\begin{enumerate}[label=(\alph*)]
\item Increases with increase in temperature
\item Decreases with increase in temperature
\item Remains constant
\item First increases then decreases
\end{enumerate}
\newpage
\section*{Question 6}
The volume of water to be added to \(100 \mathrm{~cm}^3\) of \(0.5 \mathrm{~N}~ \mathrm{H}_2 \mathrm{SO}_4\) to get decinormal concentration is:
\begin{enumerate}[label=(\alph*)]
\item \(100 \mathrm{~cm}^3\)
\item \(450 \mathrm{~cm}^3\)
\item \(500 \mathrm{~cm}^3\)
\item \(400 \mathrm{~cm}^3\)
\end{enumerate}
\newpage
\section*{Question 7}
On increasing temperature, vapour pressure of a substance ______________.
\begin{enumerate}[label=(\alph*)]
\item always increases.
\item decreases.
\item does not depend on temperature.
\item partially depends on temperature.
\end{enumerate}
\newpage
\section*{Question 8}
Which of the following solution in water possesses the lowest vapour pressure?
\begin{enumerate}[label=(\alph*)]
\item \(0.1(\mathrm{M}) \mathrm{NaCl}\)
\item \(0.1\)(M) \(\mathrm{BaCl}_2\)
\item \(0.1(\mathrm{M}) \mathrm{KCl}\)
\item None of these
\end{enumerate}
\newpage
\section*{Question 9}
The osmotic pressure of a solution increases if:
\begin{enumerate}[label=(\alph*)]
\item more of solute is added
\item number of solute molecules is increased
\item temperature is increased
\item any one of the change is made
\end{enumerate}
\newpage
\section*{Question 10}
If, at 298 K water is the solvent, and Henry’s law constant for CO$_{2}$ is 1.67 kbar and the constant of argon is 40.3 kbar, which of the following statements is true?
\begin{enumerate}[label=(\alph*)]
\item Argon is more soluble than CO$_{2}$
\item Argon is less soluble than CO$_{2}$
\item Argon is insoluble in water
\item Argon and CO$_{2}$ are equally soluble
\end{enumerate}
\newpage
\end{document}